\chapter{THE EXTREMAL PROBLEM}


	Most of the work that has been done on connected matchings concerns the minimum size of a largest connected matching in a graph with certain properties.  This is the {\it extremal} problem of connected matchings.  In particular, how many vertices must a member of a certain class of graphs have before the existence of a connected matching of a certain size is guaranteed?  In this chapter we discuss a connection between a certain special case of Hadwiger's conjecture and the extremal problem of connected matchings.

\section{Extremal problems}

	 Extremal graph theory\index{Extremal graph theory} is concerned with finding the maximum or minimum (by a variety of measures) graphs that have a certain property.  Typically, an extremal problem asks ``How many edges must be present in a graph with $n$ vertices to ensure $X$'', or ``How many vertices must a graph with property $P$ have to ensure $X$''.  The iconic example of an extremal problem, one which we use in Section 3.2.1, is the problem of edges and cliques.  How many edges must a graph on $n$ vertices have to ensure the existence of a clique of size $k$?
The answer is due to T\'uran in 1941.  T\'uran's answer \cite{turan} explicitly constructed the extremal graph, that is, the graph on $n$ vertices with no $k$-clique and the maximum number of edges.  Let $T(n,r)$ be the graph on $n$ vertices constructed by dividing the vertices as evenly as possible into $r$ parts and adding all edges among the parts.  This is the {\it complete balanced $r$-partite graph} on $n$ vertices, also called a {\it T\'uran graph}.\index{T\'uran graph}
\begin{theorem}
	The $n$ vertex graph with no complete subgraph on $r$ vertices and the maximum number of edges is $T(n, r-1)$.
\end{theorem}
Many proofs of this theorem can be found in various graph theory texts, see \cite{PFTB} for several interesting proofs.

\section{A special case of Hadwiger's conjecture}

%\item Restricted versions of HC
	In Chapter 1 we discussed some partial results on Hadwiger's conjecture, primarily for the special cases arising from restricting the chromatic number.  Now we turn our attention to a different sort of special case.  A proper vertex coloring can be thought of as a partitioning of the vertex set into independent sets.  This gives us an easy lower bound on the chromatic number in terms of the size of a largest independent set.  The least number of colors needed would be realized when all color classes are the same size, so $\chi(G) \geq n/\alpha(G)$.  This leads to the following weakening of Hadwiger's conjecture.
\bconj{[Weaker version of Hadwiger's conjecture] For all graphs $G$, 
\[\eta(G) \geq \frac{n}{\alpha(G)}.\]}
At the present time, this conjecture is open for any particular value of $\alpha$.  However, it was recently shown by Fradkin \cite{fradkin} to hold for claw-free graphs with $\alpha \geq 3$. 
An examination of this problem by Duchet and Meyniel \cite{DandM} yielded the following bound.
\begin{theorem}
	For any graph $G$, 
\[\eta(G) \geq \frac{n}{2\alpha(G) -1}\]\label{dm}
\end{theorem}
This is turn was improved by Kawarabayashi et. al  \cite{Kawa} for almost all values of $\alpha$
\begin{theorem}
	For any graph $G$ on $n$ vertices with $\alpha(G) \geq 3$
\[\eta(G) \geq \frac{n(4\alpha(G)-2)}{(4\alpha(G)-3)(2\alpha(G)-1)}\]
\end{theorem}
The first improvement by an absolute constant factor comes from Fox \cite{fox} who shows that 
\begin{theorem}
	Let $c = \frac{29-\sqrt{813}}{28}$.  Then for any graph $G$,
\[\eta(G) \geq \frac{n}{(2-c)\alpha}\]\label{fox}
\end{theorem}
The specific case of $\alpha(G) = 2$ has attracted much attention.  Plummer, Stiebitz and Toft gave this case a thorough treatment in \cite{PST}.  Despite many partial and tangential results on this case, the bound in Theorem \ref{dm} is still the best known for $\alpha = 2$ (Note that for small values of $\alpha$, the bound in Theorem \ref{dm} is better than the bound in Theorem \ref{fox}).  In addition to their work on Hadwiger's conjecture, Plummer et. al introduce the  idea of a connected matching.  This led to the following extended conjecture.
\begin{conj}[PST extension of Hadwiger's conjecture] Every graph $G$ with $\alpha(G) = 2$ and $n$ vertices has a connected matching $M$ such that the contractions of the edges in $M$ to $|M|$ single vertices result in a graph containing $K_{\lceil n/c \rceil} $ 
\end{conj}
This was also conjectured by Seymour and is sometimes referred to as {\it Seymour's strengthening of Hadwiger's conjecture.}  Plummer, Stiebitz and Toft prove this conjecture for all inflations\footnote{An {\it inflation} of a graph is obtained by replacing some vertices with complete graphs of any size all of whose vertices are adjacent to the neighbors of the replaced vertex.} of graphs with independence number 2 and fewer than 12 vertices, as well as inflations of an infinite family of $\alpha = 2$ graphs.

\section{Connected matchings in graphs with $\alpha = 2$}

The strengthened version of Hadwiger's conjecture for graphs with independence number 2 placed connected matchings front and center.  Seymour is credited with presenting the problem of improving the bound of Duchet and Meyniel in the case of independence number 2.
\bconj{There exists $\epsilon > 0$ so that every graph $G$ with $n$ vertices and $\alpha(G) = 2$ contains $K_{\lceil(\frac{1}{3}+\epsilon)\rceil n}$ as a minor.\label{seym}}
One of the results of Kawarabayashi et. al in \cite{Kawa} effectively reduces this problem to an extremal problems on connected matchings.
\begin{theorem} If a graph $G$ on $n$ vertices with $\alpha(G) \leq 2$ contains a connected matching of size greater than or equal to $kn>0$, then $G$ has $K_{\lceil (n/3)(1+k/3)\rceil}$ as a minor.

Conversely, if $G$ has $K_{\lceil cn\rceil}$ as a minor for $c> \frac{1}{3}$, then $G$ contains a connected matching of size at least $(3c-1)n/4 -\frac{1}{2}$.\label{ramsey_flavor}
\end{theorem}
Thus, if there is some $k$ for which {\it every} graph with independence number two on $n$ vertices has a connected matching of size $kn$, then Conjecture \ref{seym} must be true.  Gy\'arf\'as, F\"uredi, and Simonyi presented this extremal conjecture explicitly in  \cite{GFS}
\bconj{There exists some constant $c$ such that every graph $G$ with $n$ vertices and $\alpha(G) = 2$ has a connected matching of size $cn$.\label{GFSconj}}
Furthermore, they conjecture on the value of the constant $c$
\bconj{Every graph $G$ with $4t-1$ vertices and $\alpha(G) = 2$ has a connected matching of size $t$.\label{GFSconj2}}
They prove this for values of $t$ up to 17, and show that it is sharp by exhibiting the example of $G$ consisting of two disjoint and disconnected cliques.

Another result found in \cite{PST} is that if $H$ is a 4-vertex graph with $\alpha(H) \leq 2$, and $G$ is an $n$ vertex graph with $\alpha(G) =2$ and no copy of $H$ as an induced subgraph, then $G$ has $K_{\lceil n/2 \rceil}$ as a minor.  Kriesell has recently \cite{Kries_1} improved this result by adding the 5 vertex graphs to this list.
\begin{figure}
	\begin{center}
	\input{B}\hspace{1.5cm}
	\begin{tikzpicture}[thick,scale=0.5]

	\coordinate (a) at (0:2.5);
	\coordinate (b) at (0:5); 
	\coordinate (c) at (0:0);
	\coordinate (d) at (150:2.5);
	\coordinate (e) at (-150:2.5);
	%\coordinate (f) at (-125:3);

	%\draw (a)--(b);
	\draw (c)--(a);
	\draw (c)--(d);
	\draw (c)--(e);
	\draw (b)--(a);
	\draw (d)--(e);
	
	\draw (a) node {};
	\draw (b) node {};
	\draw (c) node {};
	\draw (d) node {};
	\draw (e) node {};
	%\draw (f) node {};
	\draw (-90:4) node[words] {$B^-$};
	%\draw (-90:5) node[words] {but not a connected matching};
\end{tikzpicture}

	\end{center}
	\caption{The graphs $B$ and $B^-$ referred to in Theorem \ref{Kriesell2}.}
	\label{B_B}
\end{figure}
\begin{theorem}
	Let $H$ be any graph with $\alpha(H) \leq 2$ on at most 5 vertices.  Then every $\{\kfree, H\}$-free graph on $n$ vertices has a collection of $\lceil n/2 \rceil$ pairwise adjacent edges and vertices.
	\end{theorem}
From this and the second part of theorem  \ref{ramsey_flavor} we conclude that every $\{\kfree, H\}$-free graph on $n$ vertices has a connected matching of size $\lfloor \frac{n}{8} \rfloor$. However when $H = \overline{K_{2,3}}$, Kriesell has found that we can say even more.
\begin{theorem}
	Every connected, $\{\kfree, \overline{K_{2,3}}\}$-free graph on $n$ vertices nonisomorphic to $B$ or $B^-$ in Figure \ref{B_B} has a connected dominating matching of size $\lfloor \frac{n}{2} \rfloor$.\label{Kriesell2}
\end{theorem}


