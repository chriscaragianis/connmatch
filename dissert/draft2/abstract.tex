\begin{center}
~\\
\vspace{1in}
ABSTRACT\\
\vspace{-0.2in}
Connected Matchings in Special Families of Graphs\\

Christopher J. Caragianis\\

November 15, 2012\\
\end{center}

A {\em connected matching} in a graph is a set of disjoint edges such that, for any pair of these edges,
there is another edge of the graph incident to both of them.
This dissertation investigates two problems related to finding large connected matchings in graphs.

The first problem is motivated by a famous and still open conjecture made by Hadwiger stating
that every $k$-chromatic graph contains a minor of the complete graph $K_k$.  If true, Hadwiger's conjecture
would imply that every graph $G$ has a minor of the complete graph $K_{n/\alpha(G)}$, where $\alpha(G)$
denotes the independence number of $G$.
For a graph $G$ with $\alpha(G)=2$, Thomass\'{e} first noted the connection between connected matchings
and large complete graph minors: there exists an $\epsilon>0$ such that every graph $G$ with $\alpha(G)=2$
contains $K_{\frac{1}{3}+\epsilon}$ as a minor if and only if there exists a positive constant $c$ such that 
every graph $G$ with $\alpha(G)=2$ contains a connected matching of size $c n$.  
In Chapter 3 we prove several structural properties of a vertex-minimal counterexample to these statements, 
extending work by Blasiak.
We also prove the existence of large connected matchings in graphs with clique size close to the Ramsey bound by proving:
for any positive constants $b$ and $c$ with $c<\frac{1}{4}$, there exists a positive integer $N$ such that,
if $G$ is a graph with $n\geq N$ vertices, $\alpha(G)=2$, and clique size at most $b \sqrt{n \log(n)}$,
then $G$ contains a connected matching of size $cn$.

The second problem concerns the computational complexity of finding the size of a
maximum connected matching in a graph.  This problem has many applications including, when
the underlying graph is chordal bipartite, applications to
the bipartite margin shop problem.  For general graphs, this problem is NP-complete.
Cameron has shown the problem is polynomial-time solvable for chordal graphs. Inspired by this and applications to the margin shop problem,
in Chapter 4 we 
focus on the class of chordal bipartite graphs and one of its subclasses, the convex bipartite graphs.   
We show that a polynomial-time algorithm to find the size of a
maximum connected matching in a chordal bipartite graph reduces to finding a polynomial-time
algorithm to recognize chordal bipartite graphs that have a perfect connected matching.  We
also prove that, in chordal bipartite graphs, a connected matching of size $k$ is equivalent
to several other statements about the graph and its biadjacency matrix, including for example,
the statement that the complement of the latter contains a $k \times k$ submatrix that is 
permutation equivalent to strictly upper triangular matrix.   \pagebreak