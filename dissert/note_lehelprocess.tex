\documentclass[12pt]{article}

\usepackage{amsthm, amssymb}
\newtheorem{theorem}{Theorem}


\begin{document}

A (simple) {\it enveloping} of a graph $G = (V, E(G))$ into a graph $H = (V, E(H))$ is a one-to-one mapping $\phi: E(G) \rightarrow E(H)$ with the property that for every $uv \in E(G)$, $\phi(uv)$ is on a $u,v$ path in $H$.

\begin{theorem}
For any pair of trees $T_1$ and $T_2$ on a vertex set $V$, there is a simple enveloping $\phi$ of $T_1$ into $T_2$.
\end{theorem}

\begin{proof}
We will work by induction on the number of vertices in $V$.  For the case of $|V| = 2$, an identity map is an enveloping. 	

Now suppose we have an eveloping between trees on up to $k$ vertices.  Let $T_1$ and $T_2$ be a pair of trees on $k+1$ vertices.  Start by taking any pendant edge $uv$ of $T_1$ with leaf vertex $u$, and mapping it to $uv'$ where $v'$ is the next vertex on the $u,v$ path in $T_2$.  Now remove $u$ from $T_1$ and identify $u$ and $v'$ as $v'$ in $T_2$.  We now have two trees $T_1 - u$ and $T_2'$ on a vertex set $V' = V-\{u\}$ with $k$ vertices.  Hence there is an enveloping $\phi$ of $T_1-\{u\}$ into $T_2'$.  Furthermore, $\phi$ can be extended  as follows to an enveloping $\phi '$ of $T_1$ into $T_2$:
\begin{enumerate}
	\item $uv$ maps to $uv'$ as above. 
	\item If $\phi(x,y)  = wv'$ then either $wu$ or $wv'$ is an edge of $T_2$.  Set $\phi'(x,y) = wu$ or $wv'$ as appropriate.
\end{enumerate} 
\end{proof}

\begin{theorem}
	If $T_1$ and $T_2$ are basic trees of a tree hypergraph $H$, then \[H/T_1 \simeq H/T_2\].
\end{theorem}

\begin{proof}
	We will show that an enveloping $\phi$ of $T_1$ into $T_2$ will induce an isomorphism between $H/T_1$ and $H/T_2$.

Suppose first that $uv \in N(e_i)$ in $H/T_1$.  Then $\{u,v\}\subseteq e_i$ in $H$.  Since $e_i$ then induces a subtree of $T_2$, the vertices of $u,v$ path of $T_2$ must be in $e_i$.  Hence, the endpoints of $\phi(uv)$ are in $e_i$, and $\phi(uv) \in N(e_i)$ in $H/T_2$.

Now we need to show that $|N(e_i)|$ in $H/T_1$ is the same as  $|N(e_i)|$ in $H/T_2$.  This is as simple as noting that the neighborhood of $e_i$ induces a subtree in both $T_1$ and $T_2$, and both of these subtrees have $|e_i|-1$ edges.  Hence $|N(e_i)| = |e_i|-1$ in both $H/T_1$ and $H/T_2$.
\end{proof}

\end{document}
