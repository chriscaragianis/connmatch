\chapter{The GFS conjecture}

In this chapter, we examine the progress that has been made on conjecture REF, and present some new partial results.  We will see that for $\kfree$-free graphs with very large cliques, as well as sufficiently large $\kfree$-graphs with only the smallest possible cliques, the conjecture holds.

\section{The properties of a counterexample}

Gy\'{a}rf\'{a}s, F\"{u}redi and Simonyi prove their conjecture for graphs up to 67 vertices.  In so doing, they implicitly introduce an important lemma concerning the relationship between cliques and connected matchings.  

\blem{[GFS] Let $0<c<1/4$.  If $G$ is a $\kfree$-free graph with $\omega(G) \geq cn$, then $G$ has a $\lfloor cn\rfloor$-connected matching.
\label{spider}}
\begin{proof}
	Let $S$ be a set of $\lceil cn\rceil$ vertices inducing a clique.  For any subset of $S'\subseteq S$, the intersection $\displaystyle I = \bigcap_{s\in S'} \{v \in V(G): sv \notin E(G)\}$ induces a clique.  If for any $S'\subseteq S$, $|I| > n/2$, then there is a $n/2$-connected matching in the clique induced by $I$.  Otherwise, 
\begin{align*}
|N(S')| &= n- |S'| - |I|\\
	&\geq n - |S'| - \frac{n}{2}\\
	&\geq \frac{n}{4} \\
	&> |S'| 
\end{align*} for all $S'\subseteq S$.  Hence, by Hall's condition (see {\it e.g.}, \cite{West}),  a matching from the vertices of $S$ to the vertices of $V(G)-S$.  This matching must be connected, since $S$ induces a clique.
\end{proof}

We might evocatively call this the ``spider lemma'' as it exhibits a connected matching with a ``head'' (the clique) and many ``legs'' (the matching with one side inducing the clique).  Using the spider lemma, we can deduce some structural qualities that a counterexample to conjecture REF must possess.  Chief among these will be connectivity, as disconnected vertex sets in $\kfree$-free graphs induce cliques.  First, we will make explicit the relationship between the GFS conjecture and Seymour's strengthening of Hadwiger's conjecture (hereafter SSH).  The following proposition shows that SSH implies the GFS conjecture.  

\bprop{If SSH holds for a $K_3$-free graph $G$ with $n$ vertices, then $\nu_c(G) \geq n/4$.}

\begin{proof}
Let $\mathcal{M}$ be the collection of branch sets of an $n/2$ SSH-minor of $G$. Let $M_1$ be the collection of elements of $\mathcal{M}$ consisting of single vertices and $M_2$ the collection of elements of $\mathcal{M}$ consisting of edges.  Obviously $M_2$  is a connected matching.  Furthermore, any matching of the clique induced by $M_1$ forms a connected matching that extends the connected matching formed by $M_2$.  Hence, 
\[\nu_c(G) \geq \lfloor\frac{|M1|}{2}\rfloor + |M2| \geq \lfloor\frac{|M1 | + |M2 |}{2}\rfloor = \lfloor n/4\rfloor\]
\end{proof}

In Lemma 2.1 of \cite{Blas}, Blasiak shows that any $\kfree$-free graph with connectivity less
than $n/2$ satisfies SSH. We can show the following for higher connectivity.  

\blem{ If G is a $\kfree$-free graph on $n$ vertices, then  $\nu_c(G) \geq \frac{n-\kappa(G)}{4}$. If $\kappa(G) \geq n/2$, then $\nu_c(G) \geq
n -\kappa(G)$. }% Furthermore, if $n/4 < \kappa(G) < n/2$, then $\nu_c(G) \geq \kappa(G)$ and if $\kappa(G) < n/4$ then $\nu_c(G) \geq n/2 - \kappa(G)$}

\begin{proof} Let $G$ be a $\kfree$-free graph on $n$ vertices. A minimum cut set separates two cliques, at least one of which has at least  $\frac{n-\kappa(G)}{2}$ vertices.  Any matching in this clique is connected, so we can find a  $\frac{n-\kappa(G)}{4}$ connected matching. 

The proof of the second claim follows the strategy of Lemma 2.1 of \cite{Blas}. 
%
From this point on, we will assume that $\kappa(G) \geq n/2$.
%
Let $S$ be a minimum cut set of $G$. 
%
Then let $L$, $R$ be a partition of $V(G)-S$ so that $L$ and $R$ do not touch. 
%
Since $G$ is $\kfree$-free, $L$ and $R$ are cliques, and every vertex of $S$ is complete to $L$ or complete to $R$. 
%
Let $S_L$ be the set of vertices complete to $L$ and $S_R$ be the set of vertices complete to $R$. 
%
We claim that between any $A \subseteq S_L$ with $|A| \leq |R|$ and $R$ ($S_R$ and $L$ resp.) there is a matching that saturates $A$. 
%
Suppose there is no matching from $R$ that saturates $A$. 
%
Hall’s condition then implies that there is a subset $T$ of $A$ such that $|N(T) \cap R| < |T |$. 
%
But then $(S -T) \cup (N(T)\cap R)$ is a cut set separating $L \cup T$ and $R - N (T)$. 
%
This set is smaller than $S$, yielding a contradiction.

Let $M$ be the largest possible matching obtained with edges between $S_L$ and $R$ (temporarily dubbed {\it type 1 edges}) and edges between $S_R$ and $L$ ({\it type 2 edges}). 
%
This matching is connected.
%
To see this, note that both types of edge form ``spiders'' as $R$ and $L$ are cliques.
%
Furthermore, without loss of generality, the $S_L$ ends of the type 1 edges are complete to $L$, and hence adjacent to an endpoint of every type 2 edge.
%

If both $|R| \leq |S_L|$ and $|L| \leq |S_R|$ (and $\kappa(G) \geq n/2$) , then we can find $T_L \subseteq S_L$ and $T_R \subseteq S_R$ so that $T_L$ and $T_R$ are disjoint, $|T_R| = |L|$, and $|T_L| = |R|$. 
%
Thus by the above claim we can construct a connected matching saturating $V(G)-S$.  
%

Now suppose, without loss of generality, that $|R| > S_L$.
%
Note that it cannot also be the case that $|L| > S_R$, as $S \geq n/2$ and $S \subseteq S_L \cup S_R$.
%
Let $R^u$ denote the set of vertices of $R$ unmatched by $M$.
%
Since $|S| \geq n/2$, we can assume that we have matched all the vertices of $S_L$ to vertices in $R$.  
%
This also means that there are at least $|R^u|$ unmatched vertices of $S_R$ (denoted $S_R^u$ ).
%
 Augment $M$ with any matching from the biclique between $R^u$ and $S_R^u$ saturating $R^u$ to yield $M'$ . 
%
These new edges are mutually connected, and connected to any type 1 or type 2 edges via edges of $R$ in the case of type 1, or edges from $S_R$ to $R$ in the case of type 2. 
%
Now $M'$ is a connected matching saturating $R \cup L$, and $|R \cup L| = n- \kappa(G)$.
\end{proof}

This lemma, together with the spider lemma, allow us to collect the properties of what we might call a ``large-clique'' counterexample to the GFS conjecture.   The following must hold in order that the largest clique is not so large that we may apply the spider lemma to find a large connected matching.
  
\bprop{
	If $G_c$ is a counterexample to conjecture \cite{GFS} with the constant $c$, then the following conditions must hold
	\begin{enumerate}
		\item $\omega(G_c) < cn$.
		\item $\delta(G_c) \geq (1 - c )n$.
		\item $G_c$ has diameter 2.
		\item  $G_c$ is $(1 - c)n$-connected.
	\end{enumerate}
}

Items 1 and 4 come directly from the two lemmas we have just presented.  That the collection of non-neighbors of a vertex in a $\kfree$-free graph form a clique leads to item 2.  Applying the pigeonhole principle to item 2 we see that every pair of vertices share a neighbor, so $G_c$ has diameter 2.  Thinking inductively, we can add to our list by considering a {\it vertex-minimal} counterexample.
\bprop{If $G_c$ is a vertex-minimal counterexample to \cite{GFS} with the constant $c$, then $G_c$ has no edge dominating $n-1/c$ vertices.}
If $G$ has an edge domminating $n-1/c$ vertices, then by an induction hypothesis we can assume it dominates a $\lfloor cn\rfloor-1$ connected matching and extends it by one.   

Taken altogether, the results of this section show that a counterexample to \ref{GFS} must be highly connected, yet lack large cliques.  In the next section we will eliminate from consideration a class of $\kfree$-free graphs that are highly connected and possess only the {\it smallest} possible cliques. (BLASIAK QUOTE ABOUT THE ``INTERESTING CASES'')

\section{Graphs with no large cliques}

In \cite{MR1369063}, Kim famously proved that the magnitude of the Ramsey number $R(3,k)$ is $\Theta(k^2/\log k)$.
%
This means that for any fixed $c$ there are $\kfree$-free graphs whose largest cliques are on the order $\sqrt{n\log n}$. 
%
These graphs are also very highly connected, so the results of the previous section are of no help.
%
CITE DEF The so-called \textit{Ramsey graphs}...
%
In this section, we will show the GFS conjecture holds for sufficiently large Ramsey graphs, witha value of $c$ arbitrarily close to $1/4$.
%
We will also discuss a natural random $\kfree$-free graph model which almost certainly satisfies the GFS conjecture. 

\begin{theorem}
Let $c < 1/4$ be a constant.  For any constant $b$ and sufficently large $n$, every $\overline{K_3}$-free graph $G$ on $n$ vertices with $\omega(G) < b\sqrt{n\log n}$ has a $cn$-connected matching.
\label{sm_cli}
\end{theorem}

First, we will prove a lemma that will place a bound on the number of pairs of separable edges in a $\kfree$-graph with a given clique number.  In order to simplify the notation in the proof, we will work on the complementary notion of {\it cycles of four vertices} in {\it $K_3$-free} graphs with a given {\it independence} number.  

\begin{lem}
For every pair of positive constants $\epsilon, d$ there is $n_{\epsilon, d}$ such that every triangle-free graph $G$ with $n > n_{\epsilon, d}$ vertices and $\alpha(G) < d\sqrt{n\log n}$ has fewer than $\epsilon n^3$ copies of $C_4$.
\end{lem}
\begin{proof}
Fix $\epsilon, d> 0$ and let $G$ be a triangle free graph on $n$ vertices with $\alpha(G) < d\sqrt{n\log n}$.  
%
Let $X_{C_4}$ be the number of copies of $C_4$ in $G$.  
%
Then
%
\begin{equation}
X_{C_4} = \frac{1}{2}\sum_{\{u,v\}\notin E(G)} {|N(u) \cap N(v)| \choose 2} 
\label{c4counteq}
\end{equation}
%
\begin{figure}
	\begin{center}
		\begin{tikzpicture}[thick,scale=0.6]

\draw (-4,4) node[lblvertex]{x}
	edge[dashed] (4,4)
	edge[] (-.8,-1.2)
	edge[] (.8, -2.3);
	
\draw(4,4) node[lblvertex]{y}
	edge[] (-.8,-1.2)
	edge[] (.8, -2.3); 
	
\draw[fill = red, opacity = 0.25](-2,-2) ellipse (4 cm  and 2.5 cm);
\draw[fill = blue, opacity = 0.25](2,-2) ellipse (4 cm and 2.5 cm);

\draw (-3.8,-2) node[words]{$N(x)$};
\draw (3.8,-2) node[words]{$N(y)$};

\draw (-.8,-1.2) node[]{}
	edge[dashed] (.8,-2.3);
\draw (.8,-2.3) node[]{};

\end{tikzpicture}	
	\end{center}
	\label{c4count}
	\caption{Illustration of the count from Eq. \ref{c4counteq}}
\end{figure}
%
For each nonadjacent vertex pair $\{u,v\}$, we count the number of distinct pairs of vertices in the intersection of the neighborhoods of $u$ and $v$.  This counts each $C_4$ twice, so we divide by two. 
%
Fix $\epsilon_1 < \sqrt{8\epsilon}$.

\noindent\textit{Claim. For sufficiently large $n$, fewer than $n^2(\log n)^{-2}$ pairs of vertices $u,v$ have neighborhood intersection larger than $\epsilon_1\sqrt{n}$.}

Suppose the contrary is true, and there are more than $n^2(\log n)^{-2}$ pairs $u,v$ so that $|N(u)\cap N(v)| \geq \epsilon_1\sqrt{n}$.
%
When we count the total number of vertices in these intersections, the count is at least $\epsilon_1n^{5/2}(\log n)^{-2}$, meaning some vertex is counted at least $\epsilon_1n^{3/2}(\log n)^{-2}$ times.  However, $\Delta(G) \leq \alpha(G) < d\sqrt{n\log n}$, so each vertex is in at most \[{d\sqrt{n\log n}\choose 2 } < \frac{d^2}{2}n\log n\] neighborhood intersections.  Thus, for sufficiently large $n$,  the claim holds.

Now we can bound $X_{C_4}$.  We will overestimate by supposing that there are precisely $n^2(\log n)^{-2}$ pairs of vertices with the largest possible vertex intersection, and the remainder have neighborhood intersection of size $\epsilon_1\sqrt{n}$.
\begin{equation}X_{C_4} < \frac{1}{2}\left[\frac{n^2}{(\log n)^2}{d\sqrt{n\log n}\choose 2}+ \left(|E(\overline{G})|- \frac{n^2}{(\log n)^2}\right){\epsilon_1\sqrt{n}\choose 2}\right]
\end{equation}
Evaluating the right hand side asymptotically we have	
\begin{equation}
\sim \frac{\epsilon_1^2-2\epsilon_1}{8}n^3 
\end{equation}
This is strictly greater than $\epsilon n^3$, so for sufficiently large $n$, the desired bound on $X_{C_4}$ holds.
\end{proof}
 
Now we can prove theorem \ref{sm_cli}.  We will use the language of RGB proximity colorings introduced in chapter 1. 

\begin{proof}[Proof (of Theorem \ref{sm_cli})]
Fix constants $d$ and $c < 1/4$, and let $G$ be a $\overline{K_3}$-free graph with $n$ vertices, $m$ edges, and $\omega(G) < b\sqrt{n\log n}$.  
%
Consider the $RGB$ graph $\mathcal{G}$ induced by $L(G)$ (recalling that green $k$-cliques correspond to $k$-connected matchings in $G$ and red edges correspond to induced $\overline{C_4}$s in $G$).
%
If $R, G,$ and $B$ denote the number of red, green and blue edges respectively, we would like to show that 
\begin{equation}
G = {m\choose 2} - R - B \geq {m\choose 2} - cn{m/cn\choose 2}
\end{equation} equivalently
\begin{equation}
	R + B \leq cn{m/cn\choose 2}\label{goal}
\end{equation}
guaranteeing by T\'{u}ran's theorem (see, \textit{e.g.}, \cite{dwest}) a green clique on $cn$ vertices in $\mathcal{G}$, and a $cn$-connected matching in $G$.

We obtain a crude upper bound on $B$ by taking the number of edges in the line graph of $K_n$.
\begin{equation}
	B < \frac{n^3}{2} - \frac{3n^2}{2} + n
\end{equation}
We can bound $R$ using Lemma 1.  For any $\epsilon > 0$ and sufficiently large $n$, 
\begin{equation}
R < \epsilon n^3\
\end{equation}
Thus for sufficiently large $n$,
\begin{equation}
	B+R < \frac{n^3}{2} + \epsilon n^3
\end{equation}
We compare this with the right hand side of (\ref{goal})
\begin{eqnarray}
	cn{m/cn\choose 2} =&\displaystyle \frac{cn}{2}\left(\frac{m^2}{c^2n^2} - \frac{m}{cn}\right)\\
	=& \displaystyle \frac{1}{2c}m^2n^{-1} - \frac{m}{2}\\
	\sim&   \displaystyle \frac{n^3}{8c}
\end{eqnarray}
Since $c < 1/4$, and we can take $\epsilon < \frac{1-4c}{8c}$, for sufficiently large $n$ (\ref{goal}) holds and $G$ has a $cn$-connected matching.
\end{proof}

The \textit{triangle free process} is a method of stochastically constructing maximal triangle free graphs.  Let $G_0$ be the empty graph on $n$ vertices and let $O_i$ be the set of edges of $K_n- G_i$ that will not create a triangle when added to $G_i$.  Then for each $G_i$, construct $G_{i+1}$ by adding an edge chosen uniformly at random from $O_i$ until some step $k$ at which $O_k$ is empty.  The complementary version of this process is a natural source of $\overline{K_3}$-free graphs.  It is worth noting, therefore, that Bohman has shown in CITE that the triangle free process asymptotically almost surely produces graphs which satisfy the hypotheses of Theorem \ref{sm_cli}.  This falls somewhat short of a proof that the GFS conjecture holds for almost all $\kfree$-free graphs because the triangle-free process does not produce a uniform distribution. ADD NOTE ABOUT APPEARANCE OF SUBGRAPHS

In conclusion, we can see that in $\kfree$-free graphs that are highly ``spread-out'' (Theorem \ref{sm_cli}) and in ones that are ``bunched-up'' (spider lemma) the GFS conjecture succeeds.   It remains to be seen if further work in tuning and sharpening these techniques can close the gap, if new approaches are needed, or if indeed there lurks a counterexample somewhere in the middle ground.

