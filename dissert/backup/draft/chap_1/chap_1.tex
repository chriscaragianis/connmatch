\chapter{Introduction}

Matching is one of the grand old topics in the theory of graphs.  Driven by a horde of useful applications, the world of matching has been as well explored, mapped, and conquered as any in combinatorics.  MATCHING IS EASY. New Paragraph CLIQUES ARE HARD new paragraph WHAT IF WE LOOKED AT THEM TOGETHER? 

As interesting as connected matchings may be in their own right, their most pivotal role in the theory of graphs is in relation to Hadwiger's conjecture, thought by some to be the most important outstanding problems in graph theory.  We will see that it is in this context that connected matchings first received substantial attention, and we will present some forward progress on the extremal connected matching problems closely related to a special case of Hadwiger's conjecture. 

\section{Preliminary ideas}
All graphs in the following are presumed to be simple and undirected unless noted otherwise.  For basic graph theory definitions and references, see \cite{West}.  

Central to this invesitgation is the idea of {\it separability} in graphs.  We say that a pair of disjoint edges $e = uv$ and $f = xy$ in a graph $G$ are separable if and only if none of the edges $ux, uy, vx, vy$ are present in $G$.  A pair of edges that are not separable will sometimes be called {\it neighborly}.  A graph is called separable if and only if it has a pair of separable edges.  

	\begin{figure}
	\begin{center}
		\begin{tikzpicture}[thick,scale=0.75]

	\coordinate (a0) at (0:0);
	\coordinate (a1) at (180:1);
	\coordinate (a3) at (-109:3);
	\coordinate (a2) at (0:1);
	\coordinate (a4) at (-71:3);
	\coordinate (text) at (-90:4);
	\coordinate (b2) at (30:3);

	\draw (a0)-- node[lblvertex] {e}(a3);
	\draw (a0)-- node[lblvertex] {f}(a4);

	\draw (a0) node {};	
	\draw (text) node[words] {neighborly};
	\draw (a3) node {};
	\draw (a4) node {};
	
\end{tikzpicture}
\hspace{2cm}
		\input{nonsep2}\hspace{2cm}
		\begin{tikzpicture}[thick,scale=0.75]

	\coordinate (a0) at (0:0);
	\coordinate (a1) at (180:1);
	\coordinate (a3) at (-109:3);
	\coordinate (a2) at (0:1);
	\coordinate (a4) at (-71:3);
	\coordinate (b1) at (150:3);
	\coordinate (text) at (-90:4);

	\draw (a1)-- node[lblvertex] {e}(a3);
	\draw (a2)-- node[lblvertex] {f}(a4);
	\draw (a1) node[]{};
	\draw (a2) node[]{};
	\draw (a3) node[]{};
	\draw (a4) node[]{};
	\draw (text) node[words] {separable};
	
\end{tikzpicture}

		\label{separable}
		\caption{Neighborly and separable pairs of edges.}
	\end{center}
	\end{figure}


A {\it matching} in a graph $G$ is a collection of disjoint edges.  A {\it connected matching} is a matching $M$ with the additional property that no two edges in $M$ are separable.  
\begin{figure}
\begin{center}
\begin{tikzpicture}[thick,scale=0.5]

	\coordinate (a) at (90:2.5);
	\coordinate (b) at (25:2.5); 
	\coordinate (c) at (0:0);
	\coordinate (d) at (155:2.5);
	\coordinate (e) at (-55:3);
	\coordinate (f) at (-125:3);

	\draw (a)--(b);
	%\draw (b)--(c);
	\draw[red] (d)--(a);
	\draw (a)--(c);
	%\draw (c)--(d);
	\draw (b)--(e);
	\draw[red] (e)--(c);
	\draw (d)--(f);
	\draw (f)--(c);
	\draw (d)--(b);
	\draw (e)--(f);
	%\draw (f) arc {-125:-33:3};
	
	\draw (a) node {};
	\draw (b) node {};
	\draw (c) node {};
	\draw (d) node {};
	\draw (e) node {};
	\draw (f) node {};
	\draw (-90:4) node[words] {Red edges form a matching,};
	\draw (-90:5) node[words] {but not a connected matching};
\end{tikzpicture}
\hspace{0.5cm}
\begin{tikzpicture}[thick,scale=0.5]

	\coordinate (a) at (90:2.5);
	\coordinate (b) at (25:2.5); 
	\coordinate (c) at (0:0);
	\coordinate (d) at (155:2.5);
	\coordinate (e) at (-55:3);
	\coordinate (f) at (-125:3);

	\draw (a)--(b);
	%\draw (b)--(c);
	\draw (d)--(a);
	\draw[green] (a)--(c);
	%\draw (c)--(d);
	\draw[green] (b)--(e);
	\draw (e)--(c);
	\draw[green] (d)--(f);
	\draw (f)--(c);
	\draw (d)--(b);
	\draw (e)--(f);
	%\draw (f) arc {-125:-33:3};
	
	\draw (a) node {};
	\draw (b) node {};
	\draw (c) node {};
	\draw (d) node {};
	\draw (e) node {};
	\draw (f) node {};
	\draw (-90:4) node[words] {Green edges form a maximum};
	\draw (-90:5) node[words] {(perfect) connected matching};
\end{tikzpicture}

\label{CMex}
\caption{Distinction between matching and connected matching}
\end{center}
\end{figure}
We will use $\nu(G)$ to denote the largest cardinality of a matching in a graph $G$, and $\nu_c(G)$ the cardinality of the largest {\it connected} matching.

If a vertex $v$ is adjacent to every vertex in a set $S$, then we say that $v$ is {\it complete} to $S$.  If a matching $M$ touches every vertex of $S$ and $S$ induces no edges of $M$, then we say $M$ {\it saturates} $S$.  If a set of vertices $T$ has a vertex adjacent to a vertex of $S$, then we say $S$ {\it touches} $T$.

HYPERGRAPHS, INTERVAL HYPER GRAPHS, BIGRAPH REPRESENTATION, INTERVAL ORDER

\section{Proximity colorings}

When a graph is used as a ``map'' of some real-world network of interconnected nodes, we are naturally curious about how ``far away'' two nodes are from each other.  Often the geographical distance between the physical nodes is much less important than the number of nodes or connections on the shortest network path between them.  This is captured in the graph theoretic notion of distance, and in this section we will look at a scheme for ``mapping'' this information.  This scheme will simplify proofs in the following chapters that depend on distance information.

The {\it distance} between two (connected) vertices in a graph $G$ refers to the length of the shortest path in $G$ starting at $u$ and terminating with $v$.  For any graph $G$ on $n$ vertices, there is a natural partition of the edges of a complete graph on $n$ vertices constructed by collecting each edge $uv$ into a distinct class according to the distance in $G$ between vertices $u$ and $v$.  Specifically, $P_i = \{uv : d_G(u,v) = i\}$ .  We call this the \textit{proximity partition} $\mathcal{P}_G$ induced by $G$.  Sometimes we may only be interested in distances that fall below a certain threshhold.  In this case, we can consider the \textit{proximity $k$-partition} $\mathcal{P}^k_G$ induced by $G$, where for $1 < i < k$, $P_i = \{uv : d_G(u,v) = i\}$ and $P_k = \{uv : d_G(u,v) \geq k\}$. 

Occasionally, for small values of $k$, we may think of this partiton as an edge coloring of $K_n$ and refer to the \textit{proximity $k$-coloring} of $K_n$. 
 
\bprop{For any fixed $k$, determining whether a given partition of the edges $K_n$ is a proximity $k$-partition can be done in time polynomial in $n$.}

\begin{proof}
Suppose we have a partition $\mathcal{P}$ of the edges of $K_n$. If the $\mathcal{P}$ has more than $k$ classes, it clearly cannot be a proximity $k$-partition. Therefore there are at most $k!$ possible indexings of $\mathcal{P}$.  Checking whether a given indexing gives rise to a proximity $k$-partition can be accomplished in no more than $k$ matrix multiplications and comparisons.  
\end{proof}

The proximity $3$-coloring will be helpful for us in studying connected matchings, so we will introduce a further shorthand.  The RGB graph induced by a graph $G$ is the proximity $3$-coloring induced by $G$ with $P_1$ colored blue, $P_2$ colored green, and $P_3$ colored red.
\begin{prop}
Connected matchings in a graph $G$ correspond to green cliques in the RGB graph induced by $L(G)$.  
\end{prop} 

\begin{proof}
	Incident edges of $G$ correspond to adjacent vertices in $L(G)$.  If we have a pair of edges $e, f \in E(G)$ that are disjoint and neighborly, then there is some edge $g \in E(G)$ between an endpoint of $e$ and an endpoint of $f$.  Hence, $eg, fg \in E(L(G))$ and $ef \notin E(L(G))$.  This means that $d_{L(G)}(e,f) = 2$, and $ef$ is green in the RGB graph induced by $L(G)$.  Since connected matchings are collections of pairwise non-incident and neighborly edges, the green cliques in   the RGB graph induced by $L(G)$ correspond to connected matchings in $G$.  
\end{proof}
\begin{figure}
\begin{center}
\begin{tikzpicture}[thick,scale=0.5]

	\coordinate (a0) at (0:0);
	\coordinate (a1) at (180:1);
	\coordinate (a3) at (-109:3);
	\coordinate (a2) at (0:1);
	\coordinate (a4) at (-71:3);
	\coordinate (b1) at (150:3);
	\coordinate (b2) at (30:3);

	\draw (a0)-- node[lblvertex] {a}(a3);
	\draw (a0)-- node[lblvertex] {b}(a4);
	\draw (b1)[blue]--(b2);
	\draw (a0) node {};	
	%\draw (a1) node {};
	%\draw (a2) node {};
	\draw (a3) node {};
	\draw (a4) node {};
	\draw (b1) node[lblvertex, draw = blue] {a};
	\draw (b2) node[lblvertex, draw = blue] {b};
\end{tikzpicture}
\hspace{1cm}
\begin{tikzpicture}[thick,scale=0.5]

	\coordinate (a0) at (0:0);
	\coordinate (a1) at (180:1);
	\coordinate (a3) at (-109:3);
	\coordinate (a2) at (0:1);
	\coordinate (a4) at (-71:3);
	\coordinate (b1) at (150:3);
	\coordinate (b2) at (30:3);

	\draw (a1)-- node[lblvertex] {a}(a3);
	\draw (a2)-- node[lblvertex] {b}(a4);
	\draw (a1)--(a4);
	\draw (b1)[green]--(b2);
	%\draw (a0) node {};	
	\draw (a1) node {};
	\draw (a2) node {};
	\draw (a3) node {};
	\draw (a4) node {};
	\draw (b1) node [lblvertex, draw = green]{a};
	\draw (b2) node [lblvertex, draw = green]{b};
\end{tikzpicture}
\hspace{1cm}
\input{coresp3}
\label{coresp}
\caption{Correspondence between pairs of edges in $G$ and the RGB edges of $L(G)$.}
\end{center}
\end{figure}
Predicatbly, some complexity results on connected matching problems essentially rest on clique problems in RGB graphs of line graphs.  While clique problems are in general computationally difficult, there are special classes of graphs for which they can be solved efficiently.  We will see that some graphs retain this quality in the green graph of their line graph.
