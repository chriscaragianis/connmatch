\chapter{Extremal problems and conjectures}

The extremal problem of connected matchings turns out to have implications for one of the most important unsolved problems in graph theory; Hadwiger's conjectured upper bound on chromatic number given in terms of graph minors.  

\section{Hadwiger's conjecture}

A proper vertex coloring of a graph is an assignment of colors to the vertices so that no two adjacent vertices have the same color.  The {\it chromatic number} of a graph $G$ (denoted $\chi(G)$) is the fewest colors with which one can properly color the vertices of $G$. It is clear that the size of a largest clique in a graph is a lower bound on the chromatic number of the graph.  If a graph contains a clique on $n$ vertices, no two vertices of that clique may be assigned the same color.  It is equally apparent by considering as small an example as a cycle on five vertices that the chromatic number of a graph may exceed the clique number.    

But if the clique number of a graph is not an upper bound on the chromatic number, what is?  One possibility to look at some type of ``topological generalization'' of a clique whose size would bound the chromatics number from above.  SEE TOFT.  

When we {\it contract} an edge $uv$ of a graph $G$, we identify the vertices $u$ and $v$ as a new vertex adjacent to all vertices in the union of the neighborhoods of $u$ and $v$ We say that $H$ is a {\it minor} of a graph $G$, or that $G$ has $H$ as a minor (denoted $H \leq G$) is there is some sequence of vertex deletions of edge contractions that transform $G$ into $H$.  
\begin{figure}
	\begin{center}
	\begin{tikzpicture}[thick,scale=0.5]
\draw[gray] 
    {
        (72:2.5) node[lblvertex] {a}  -- (144:2.5)
		(144:2.5) node[lblvertex] {b} -- (216:2.5)
		(216:2.5) node[lblvertex] {c} -- (288:2.5)
		(288:2.5) node[lblvertex] {d}  -- (0:2.5)
		(0:2.5) node[lblvertex] {e} -- (72:2.5)
		
	
    };
	\draw (62:3)node[words] {}
		edge[->] (10:3);
	
\end{tikzpicture}
\hspace{0.5cm}
	\begin{tikzpicture}[thick,scale=0.5]
\draw[gray] 
    {
        
		(144:2.5) node[lblvertex] {b} -- (216:2.5)
		(216:2.5) node[lblvertex] {c} -- (288:2.5)
		(288:2.5) node[lblvertex] {d}  -- (0:2.5)
		(0:2.5) node[lblvertex] {a,e} -- (144:2.5)
	
    };
	\draw (134:3.3)node[words] {}
		edge[->] (10:3.3);
	
\end{tikzpicture}
\hspace{0.2cm}
	\begin{tikzpicture}[thick,scale=0.5]
\draw[gray] 
    {
        
		
		(216:2.5) node[lblvertex] {c} -- (288:2.5)
		(288:2.5) node[lblvertex] {d}  -- (0:2.5)
		(0:2.5) node[lblvertex] {a,b,e} -- (216:2.5)
	
    };
	
	
\end{tikzpicture}

	\end{center}
	\caption{A sequence of edge contractions that show $K_3 \leq C_5$.}
	\label{minorex}
\end{figure}
The famous conjecture of Hadwiger suggests that the generalization of cliques needed to give an upper bound on the chromatic number is that of a clique minor.
\bconj{[Hadwiger, 1947] The number of colors needed to properly color the vertices of a graph $G$ is no more than $n$ where $K_n$ is the largest clique contained in $G$ as a minor.  In other words, for all graphs $G$, 
\[\eta(G) \geq \chi(G) \]
}
  HERE PUT IN PARTIAL RESULTS.

Another restriction of Hadwiger's  conjecture arises from the upper bound on the chromatic number coming from the independence number of the graph.  A proper coloring can be thought of as a partitioning of the vertex set into independent sets, so the following bound is quite clear.
\bprop{For all graphs $G$, \[\chi(G) \geq \frac{n}{\alpha(G)}\]}
This leads to the following weakening of Hadwiger's conjecture
\bconj{[Weaker version of Hadwiger's conjecture] For all graphs $G$, 
\[\eta(G) \geq \frac{n}{\alpha(G)}.\]}
At the present time, this conjecture remains open for any particular independence number.  The first examination of this problem by Duchet and Meyniel yielded the following bound.
\begin{theorem}[\cite{DandM}]
	For any graph $G$, 
\[\eta(G) \geq \frac{n}{2\alpha(G) -1}\]
\end{theorem}
This is turn was improved by Kawarabayashi et. al for almost all values of $\alpha$
\begin{theorem}[\cite{Kawara}]
	For any graph $G$ on $n$ vertices with $\alpha(G) \geq 3$
\[\eta(G) \geq \frac{n(4\alpha(G)-2)}{(4\alpha(G)-3)(2\alpha(G)-1)}\]
\end{theorem}
The first improvement by an absolute constant factor comes from Fox (\cite{fox}) who shows that 
\begin{theorem}
	Let $c = \frac{29-\sqrt{813}}{28}$.  Then for any graph $G$,
\[\eta(G) \geq \frac{n}{(2-c)\alpha}\]
\end{theorem}

The specific case of $\alpha(G) = 2$ has attracted much attention.  Plummer, Stiebitz and Totf gave this case a thorough treatment in \cite{PST}.  In addition to their work on Hadwiger's conjecture, it is in \cite{PST} that the idea of a connected matching is introduced.  Their work led to the following extended conjecture.

\begin{conj}[PST extention of Hadwiger's conjecture] Every graph $G$ with $\alpha(G) = 2$ and $n$ vertices has a connected matching $M$ such that the contractions of the edges in $M$ to $|M|$ single vertices results in a graph containing $K_{\lceil n/c \rceil} $ 
\end{conj}

This was also conjectured by Seymour CITE and is sometimes referred to as {\it Seymour's strengthening of Hadwiger's conjecture.}  Plummer, Stiebitz and Toft prove this conjecture for all inflations of graphs with independence number 2 and fewer than 12 vertices, as well as inflations of an infinite family of $\alpha = 2$ graphs.

\section{Connected matchings in graphs with independence number 2}

The strengthened version of Hadwiger's conjecture for graphs with independence number 2 placed connected matchings front and center.  Seymour is credited with presenting the problem of improving the bound of Duchet and Meyniel in the case of independence number 2.
\bconj{There exists $\epsilon > 0$ so that every graph $G$ with $n$ vertices and $\alpha(G) = 2$ contains $K_{\lceil(\frac{1}{3}+\epsilon)n}$ as a minor.}
One of the results of Kawarabayashi et. al in \cite{Kawara} effectively reduces this problem to an extremal problems on connected matchings.
\begin{theorem}[\cite{Kawara}] If $G$ is a graph on $n$ vertices with $\alpha(G) \leq 2$ containing a connected matching of size greater than or equal to $kn>0$, then $G$ has $K_{\lceil (n/3)(1+k/3)\rceil}$ as a minor.

Conversely, if $G$ is a graph on $n$ vertices with $\alpha(G) \leq 2$ having $K_{\lceil cn\rceil}$ as a minor for $c> \frac{1}{3}$, then $G$ contains a connected matching of size at least $(3c-1)n/4 -\frac{1}{2}$.\label{ramsey_flavor}
\end{theorem}

Gy\'arf\'as, F\:uredi, and Simonyi present the extremal conjecture in \cite{GFS}
\bconj{[\cite{GFS}]There exists some constant $c$ such that every graph $G$ with $n$ vertices and $\alpha(G) = 2$ has a connected matching of size $cn$.\label{GFSconj}}
Furthermore, they conjecture on the value of the constant $c$
\bconj{\cite{GFS}Every graph $G$ with $4t-1$ vertices and $\alpha(G) = 2$ has a connected matching of size $t$.\label{GFSconj2}}
They are able to prove this for values of $t$ up to 17, and show that it is sharp by exhibiting the example of $G$ consisting of two disjoint and disconnected cliques.

Another result found in \cite{PST} is that if $H$ is a 4-vertex graph with $\alpha(H) \leq 2$, and $G$ is an $n$ vertex graph with $\alpha(G) =2$ and no copy of $H$ as an induced subgraph, then $G$ has $K_{\lceil n/2 \rceil}$ as a minor.  Kriesell has recently improved this result by adding the 5 vertex graphs to this list.
\begin{figure}
	\begin{center}
	\input{B}\hspace{1.5cm}
	\begin{tikzpicture}[thick,scale=0.5]

	\coordinate (a) at (0:2.5);
	\coordinate (b) at (0:5); 
	\coordinate (c) at (0:0);
	\coordinate (d) at (150:2.5);
	\coordinate (e) at (-150:2.5);
	%\coordinate (f) at (-125:3);

	%\draw (a)--(b);
	\draw (c)--(a);
	\draw (c)--(d);
	\draw (c)--(e);
	\draw (b)--(a);
	\draw (d)--(e);
	
	\draw (a) node {};
	\draw (b) node {};
	\draw (c) node {};
	\draw (d) node {};
	\draw (e) node {};
	%\draw (f) node {};
	\draw (-90:4) node[words] {$B^-$};
	%\draw (-90:5) node[words] {but not a connected matching};
\end{tikzpicture}

	\end{center}
	\caption{The graphs $B$ and $B^-$ referred to in Theorem \ref{Kriesell2}.}
	\label{B_B}
\begin{theorem}[Kriesell, \cite{Kries}]
	Let $H$ be any $\kfree$-free graph on at most 5 vertices.  Then every $\{\kfree, H\}$-free graph on $n$ vertices has a collection of $\lceil n/2 \rceil$ pairwise adjacent edges and vertices.
	\end{theorem}
From this and the second part of theorem  \ref{ramsey_flavor} we can conclude that every $\{\kfree, H\}$-free graph on $n$ vertices has a connected matching of size $\lfloor \frac{n}{8} \rfloor$. However when $H = \overline{K_{2,3}}$, Kriesell has found that we can say even more.
\begin{theorem}[Kriesell, \cite{Kries2}]
	Every connected, $\{\kfree, \overline{K_{2,3}}\}$-free graph on $n$ vertices nonisomorphic to $B$ or $B^-$ in figure \ref{B_B} has a connected dominating matching of size $\lfloor \frac{n}{2} \rfloor$.
\end{theorem}
OTHER KRIESELL THEOREMS???
\end{figure}
