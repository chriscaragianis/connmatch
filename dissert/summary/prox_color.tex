%\documentclass[12pt]{article}




\usepackage{amsrefs,array,amsthm,amsmath,setspace}
\usepackage[top=1 in, bottom=1in, left=1.5 in, right=1in]{geometry}

\renewcommand{\arraystretch}{1.5}
\doublespacing
\linespread{2}

\newtheorem{conj}{Conjecture}
\newtheorem{mconj}{Metaconjecture}
\newtheorem{prop}{Proposition}
	\newcommand{\bprop}[1]{\begin{prop}#1\end{prop}}
\newtheorem{lem}{Lemma}
	\newcommand{\blem}[1]{\begin{lem}#1\end{lem}}
\newtheorem{theorem}{Theorem}
	\newcommand{\bthm}[1]{\begin{theorem}#1\end{theorem}}

\newtheorem{guess}{Guess}
	\newcommand{\bguess}[1]{\begin{guess}#1\end{guess}}

\theoremstyle{definition}
\newtheorem{mydef}{Definition}
%\begin{document}


\section{Proximity partitions}

For any graph $G$ on $n$ vertices, there is a natural partition of the edges of a complete graph on $n$ vertices constructed by collecting each edge $uv$ into a distinct class $P_i = \{uv : d_G(u,v) = i\}$ .  We will call this a \textit{proximity partition} $\mathcal{P}_G$ induced by $G$.  For an integer $k$, we can also consider the \textit{proximity $k$-partition} $\mathcal{P}^k_G$ induced by $G$, where for $1 < i < k$, $P_i = \{uv : d_G(u,v) = i\}$ and $P_k = \{uv : d_G(u,v) \geq k\}$. Occasionally, for small values of $k$, we may think of this partiton as an edge coloring of $K_n$ and refer to the \textit{proximity $k$-coloring} of $K_n$ 

\bprop{For any fixed $k$, determining whether a given partition of the edges $K_n$ is a proximity $k$-partition can be done in time polynomial in $n$.}

\begin{proof}
Suppose we have a partition $\mathcal{P}$ of the edges of $K_n$. If the $\mathcal{P}$ has more than $k$ classes, it clearly cannot be a proximity $k$-partition. Therefore there are at most $k!$ possible indexings of $\mathcal{P}$.  Checking whether a given indexing gives rise to a proximity $k$-partition can be accomplished in no more than $k$ matrix multiplications and comparisons.  
\end{proof}

\subsection{Proximity coloring and connected matchings}

Proximity coloring provides a different prespective on the problem of finding maximum connected matchings.  If we consider a proximity $k$-coloring of $L(G)$ with $k \geq 3$, then monochromatic cliques in color 2 correspond to connected matchings in $G$.  We can now phrase the integer linear program formulation more concisely.  In what follows, give $K_m$ the proximity $3$-coloring induced by the line graph of an input graph $G$ ($|E(G)| = m$). 

\noindent\textbf{MCM}

\noindent Maximize $\displaystyle \sum_{v_i \in V(K_m)} x(v_i)$ subject to: 
\begin{align}
	x(v_i) \in \{0,1\} \qquad &\forall v_i \in V(K_m)\\
	\sum_{v_i \in I} x(v_i) \leq 1 \qquad &\forall I\subset V(K_m)\makebox{, $I$ induces no color 2 edges}
\end{align} 
It's clear that an optimal solution to \textbf{MCM} corresponds to a maximum monochromatic color 2 clique, and consequently to a maximum connected matching in $G$. Since we can quickly compute the color 2 graph from the input graph $G$, we can quicky find a 2-approximation of connected matching number of $G$.

%If $G$ is has independence number 2, and the color 3 graph is bipartite, the half integer optimum is the same as the integer optimum.   

Excluding a collection of small graphs we can find maximum connected matchings quickly.

\bprop{If the input graph $G \in \mathcal{H_c}$ has no induced subgraph from the list in PICTURE (bowtie, triangled hexagon, twinned 5-cycle) then $K_m$ has no red edges.}
\begin{proof}
 We will apply this new perspective to graphs in the class of ``hard'' cases with respect to Conjecture 7, $\mathcal{H}_c$.  Let $\mathbf{P}_c = \{\mathcal{P}: \: \mathcal{P} \makebox{ is the proximity 3-coloring induced by a graph in } \mathcal{H}_c\}$.

\bprop{For any $\mathcal{P} \in \mathbf{P}$ 
	\begin{enumerate}
		\item $C_1$ induces a line graph 
		\item $C_3$ induces a triangle-free graph with diameter 3
		\item No triangle has two blue edeges and one red edge
		\item No set of four vertices induces the coloring in figure PICTURE 
	\end{enumerate}}






%\end{document}
