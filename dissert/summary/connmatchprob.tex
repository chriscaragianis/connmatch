%\documentclass[12pt]{article}




\usepackage{amsrefs,array,amsthm,amsmath,setspace}
\usepackage[top=1 in, bottom=1in, left=1.5 in, right=1in]{geometry}

\renewcommand{\arraystretch}{1.5}
\doublespacing
\linespread{2}

\newtheorem{conj}{Conjecture}
\newtheorem{mconj}{Metaconjecture}
\newtheorem{prop}{Proposition}
	\newcommand{\bprop}[1]{\begin{prop}#1\end{prop}}
\newtheorem{lem}{Lemma}
	\newcommand{\blem}[1]{\begin{lem}#1\end{lem}}
\newtheorem{theorem}{Theorem}
	\newcommand{\bthm}[1]{\begin{theorem}#1\end{theorem}}

\newtheorem{guess}{Guess}
	\newcommand{\bguess}[1]{\begin{guess}#1\end{guess}}

\theoremstyle{definition}
\newtheorem{mydef}{Definition}
%\begin{document}

\section{The connected matching problem}



We can formulate the problem of finding the size of the largest connected matching in a graph $G$ as an integer program.  It is known from \cite{MR2070161} that this problem is NP hard in general, and from \cite{MR2163948} that it remains NP hard when $G$ is bipartite, but polytime solvable when $G$ is chordal or has no cycle of size 4.  

Let $G$ be a simple graph, and define the \textit{bad pair graph} ($BPG$) to be the graph with vertex set $V(BPG) = E(G)$, and edge $e_ie_j \in E(BPG)$ if and only if $e_i$ and $e_j$ are either incident or disconnected (possesing no adjacent endpoints) in $G$.  The problem of finding a maximum connected matching in $G$ is then equivalent to the following integer linear program.

\noindent\textbf{MCM}

\noindent Maximize $\displaystyle \sum_{e_i \in E(G)} x(e_i)$ subject to: 
\begin{align}
	x(e_i) \in \{0,1\} \qquad &\forall e_i \in E(G)\\
	\sum_{e_i \ni v} x(e_i) \leq 1 \qquad &\forall v\in V(G)\\
	\sum_{e_i \in D} x(e_i) \leq 1 \qquad &\forall D\subset E(G)\makebox{, $S$ mutually disconnected}\\
	\sum_{e_i \in C} x(e_i) \leq \lfloor \frac{1}{2}k\rfloor \qquad& \forall \makebox{$k$-cycles in $BPG$}
\end{align} 
\bprop{An optimal solution to \textbf{MCM} corresponds to a maximum connected matching in $G$.}
\begin{proof}
	First we show that a connected matching $M$ is a feasible point in \textbf{MCM}.  Constraint (2) is satisfied because $M$ is a matching.  No two edges of $M$ are disconnected, so (3) is satisfied.  Finally, the edges of $M$ correspond to a set of vertices in $BPG$ that induce an empty graph, so (4) is satisfied as well.

Next we show that any feasible point of \textbf{MCM} corresponds to a connected matching.  Constraint (2) guarantees that the non-zero edges of a feasible assignment form a 1-regular graph, hence, a matching.  Constraint (3) requires that no two non-zero edges are disconnected.  Therefore the non-zero edges of a feasible assignment correspond to a connected matching.    
\end{proof}

We define the \textit{fractional connected matching number of $G$}, denoted $cm_f(G)$, as the optimum of the linear relaxation of \textbf{MCM} obtained by replacing (1) with
\begin{equation*}\tag{$1'$}x(e_i) \in (0,1) \qquad \forall e_i \in E(G)\end{equation*}
\bprop{If $G$ with $|V(G)| \geq 3$ is 2-connected and $\alpha(G) \leq 2$, then $cm_f(G) \geq n/2$}

\begin{proof}
	In \cite{MR0297600}, Erd\H{o}s and Chv{\'a}tal show that if $|V(G)| \geq 3$, $\alpha(G) \leq \kappa(G)$, then $G$ is Hamiltonian.  Assigning each edge of a Hamiltonian cycle a weight of 1/2 gives a total weight of $n/2$ with neither a vertex weight nor a mutually dsiconnected edge set weight greater than 1.
\end{proof}

Note that if $G$ is not 2-connected, then there is a clique of size $\lceil \frac{n}{2} \rceil$, and $\eta(G) \geq n/2$.

Now we consider the half-integer relaxation \textbf{MCM$'$} of \textbf{MCM} formed by replacing (1) with
\begin{equation*}\tag{$1'$}x(e_i) \in \{0, \frac{1}{2}, 1\} \qquad \forall e_i \in E(G)\end{equation*}
\bprop{$opt(\textbf{MCM$'$}) = opt(\textbf{MCM})$}
\begin{proof}
	That $opt(\textbf{MCM$'$}) \geq opt(\textbf{MCM})$ is clear.  We must show that $opt(\textbf{MCM$'$}) \leq opt(\textbf{MCM})$

	Take an optimal assignment for \textbf{MCM$'$}, and collect the edges valued 1 into a set $S$, and the edges valued $\frac{1}{2}$ into a set $T$.  Note that constraint (4) forbids the edges in $T$ from inducing an odd cycle in $BPG$.  Hence, $T$ induces a bipartite graph in $BPG$, and there is a set $T'\subset T$ such that $|T'| \geq |T-T'|$ and $T'$ induces an empty graph in $BPG$.  Furthermore, $S \cup T'$ induces an empty graph in $BPG$ (any pair $e_i, e_j \in S\cup T'$ with $e_i \in S$ has $e_i + e_j \geq \frac{3}{2}$, so no such pair is incident or disconnected in $G$.)  This means no pair of edges in $S\cup T'$ is incident or disconnected, so we may reassign $x(e_i) = 1$ for $e_i \in T'$, $x(e_j) = 0$ for $e_j \in T-T'$ and obtain an assignment valid for \textbf{MCM$'$} and \textbf{MCM} that has a total value no less than the assignmnet we started with.
\end{proof}
%\end{document}
