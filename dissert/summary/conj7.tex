%\documentclass[12pt]{article}




\usepackage{amsrefs,array,amsthm,amsmath,setspace,tikz}
%\usepackage[top=1 in, bottom=1in, left=1.5 in, right=1in]{geometry}



\renewcommand{\arraystretch}{1.5}
\doublespacing
\linespread{2}

\newtheorem{conj}{Conjecture}
	\newcommand{\bconj}[1]{\begin{conj}#1\end{conj}}
\newtheorem{mconj}{Metaconjecture}

\newtheorem{prop}{Proposition}
	\newcommand{\bprop}[1]{\begin{prop}#1\end{prop}}
\newtheorem{lem}{Lemma}
	\newcommand{\blem}[1]{\begin{lem}#1\end{lem}}
\newtheorem{theorem}{Theorem}
	\newcommand{\bthm}[1]{\begin{theorem}#1\end{theorem}}


\newtheorem{guess}{Guess}
	\newcommand{\bguess}[1]{\begin{guess}#1\end{guess}}

\theoremstyle{definition}
\newtheorem{mydef}{Definition}



%\begin{document}

\section{Conjecture 7}
\subsection{Characteristics of a (vertex minimal, $\alpha$-critical) counterexample to Conjecture \ref{fgs}}

In this section, let $G_{c'}$ be the the graph on the fewest vertices ($|V(G_{c'})| = n$) that is a counterexample to Conjecture 3 with $c = c'$.  Further, assume that removing any edge of $G_{c'}$ increases the independence number.

\bprop{$G_{c'}$ has no edge dominating $n-c'$ vertices} 

The proof of this is clear by induction, as the neighborhood of such an edge will neccessarily contain a connected matching of size $cn-1$.

\bprop{$G_{c'}$ and $\overline{G_{c'}}$ have diameter 2}

\begin{proof}If $G_{c'}$ has a pair of vertices $\{u,v\}$ with $d(u,v) > 2$, then every other vertex is either a non-neighbor of $u$ or a non-neighbor of $v$.  Thus, there must be a clique of size $\lceil (n-1)/2 \rceil$.  If $\overline{G_{c'}}$ has a pair of vertices $\{u,v\}$ with $d(u,v) > 2$, then the edge $uv$ can be deleted without increasing the independence number.  (Note that this result does not require $G_{c'}$ to be vertex minimal.)
\end{proof}

\subsection{The "hard cases", when $G$ has no large cliques.} 

\blem{Let $0<c<1/4$.  If $G$ ($\alpha(G) \leq 2$) has $\omega(G) \geq cn$, then $G$ has a $(cn)$-connected matching}
\begin{proof}
	Let $S$ be a set of $cn$ vertices inducung a clique.  For any subset of $S'\subseteq S$, the intersection $\displaystyle I = \bigcup_{s\in S'} \{v \in V(G): sv \notin E(G)\}$ induces a clique.  If for any $S'\subseteq S$, $|I| > n/2$, then there is a $cn$-connected matching in the clique induced by $I$.  Otherwise, $|N(S')| \geq |S'|$ for all $S'\subseteq S$, and there is an $S-(V(G)/S)$ matching saturating $S$.  This matching must be connected, since $S$ induces a clique. (This lemma appears implicitly in \cite{FGS})
\end{proof}

In \cite{MR1369063}, Kim famously proved that the magnitude of the Ramsey number $R(3,k)$ is $\Theta(k^2/\log k)$.  This means that for any fixed $c$ there are graphs that do not meet the hypotheses of the above lemma.  

%However, graphs with independence number two and no large ($\Omega(n)$) cliques have some properties that may be helpful in finding large connected matchings.
In what follows, let $\mathcal{P}_c$ be the class of graphs with independence number 2 and no clique of size $cn$

\bprop{If $G \in \mathcal{P}_c$, then $\delta(G) \geq (1-c)n$.}
\bprop{If $G \in \mathcal{P}_c$, then $G$ is $(1-2c)n$-connected.}

%Either of the above propositions can be used to show that $G$ is Hamiltonian.  In fact, $\delta(G) > n/2$ implies at least $\lfloor 5n/225\rfloor$ edge-disjoint Hamiltonian cycles in $G$ (\cite{MR0284366}).   Furthermore, in \cite{edhc} it is shown that for every $\gamma > 0$ there is an integer $n_0$ so that every graph on $n \geq n_0$ vertices of minimum degree at least $(1/2 + \gamma)n$ contains at least $n/8$ edge-disjoint Hamilton cycles.  

%The second proposition is interesting because of the result that a $2$-connected graph with $\alpha(G) \leq \kappa(G)$ has a Hamilton cycle.  Again the graphs of $\mathcal{P}_c$ vastly exceed these hypotheses.  It remains to be seen whether an abundance of edge-disjoint Hamilton cycle will be helpful in finding connected matchings.  Intuitively, however, the "spread-outness" of these (increasingly, with diminishing $c$) dense graphs \textit{sounds} helpful.  

Finally, for any $0 < c < 1/4$ we have a collection of ``hard'' cases $\mathcal{H}_c$ where $G \in \mathcal{H}_c$ if and only if
\begin{itemize}
	\item $\alpha(G) = 2$
	\item $\omega(G) < cn$
	\item $diam(G) = diam(\overline{G}) = 2$
\end{itemize}

%\end{document}
