\documentclass[12 pt]{article}

\usepackage{amsthm, amsmath, amsrefs,xspace}
\newtheorem{conjecture}{Conjecture}
\newtheorem{theorem}{Theorem}
\newcommand{\kfree}{$\overline{K_3}$-free \xspace}
\newcommand{\AG}{$\alpha(G) = 2$\xspace}


\begin{document}

{\linespread{1}
\begin{bibsection}[References]\vspace{-\parskip} % This is the start of the bibliography. 
	\begin{biblist}[\normalsize] % Replace the \bib entries with ones relevant to your problem.
							% The bulk of each entry can be copied and pasted from MathSciNet
							% ( http://www.ams.org/mathscinet/ ). When viewing the review of an
							% item you want to site, open the "Select alternative format" pull-down
							% and select AMSrefs.
							% Most likely, the only part you will need to change is the first parameter
							% after \bib. This is the internal name you use to cite the reference
							% with \cite. By default it will be the Mathematical Reviews number
							% (for example MR1375315). To make my life easier when I merge these all
							% into the summary document, please choose a name that begins with your
							% initials, followed by the number of problems you have submitted
							% (including this one).
							% For example, since this problem was submitted by Leonhard Euler and
							% since this is the first problem he is presenting, all citation names
							% begin with "le1". If this was his third problem, they would begin with
							% "le3".
%Z. Füredi, A. Gyárfás, G. Simonyi,  Connected matchings and Hadwiger's conjecture,  Combin. Probab. Comput., Problem Section, 14 (2005), 435--438.
%M. Kriesell, On seymour's strengthening of Hadwidger's conjecture for graphs with certain forbidden subgraphs 



%\bib{dwest}{book}{
   %author={West, Douglas B.},
   %title={Introduction to graph theory},
   %publisher={Prentice Hall Inc.},
   %place={Upper Saddle River, NJ},
   %date={1996},
   %pages={xvi+512},
   %isbn={0-13-227828-6},
   %review={\MR{1367739 (96i:05001)}},
%}

\bib{FGS}{article}{
   author={F\"{u}redi, Zolt\'{a}n},
   author={Gy{\'a}rf{\'a}s, Andr{\'a}s},
   author={Simonyi, G\'{a}bor},
   title={Connected matchings and Hadwiger's conjecture},
   journal={Combin. Probab. Comput.},
   part={Problem Section},
   volume={14},
   date={2005},
   pages={435--438},
}

\vskip 0.5 cm

Puts into print Seymour's conjectured improvement of Theorem \ref{DucAndMey} (in \cite{DandM})
\begin{conjecture}\label{SeyEps}
	There exists $\epsilon > 0$ such that every graph $G$ with $n$ vertices and with \AG contains $K_{\lceil(\frac{1}{3}+\epsilon)n}$ as a minor.
\end{conjecture}

Conjecture \ref{SeyEps} is equivalent to Conjecture \ref{cconj} SEE \cite{DandMimprove}??? CANT FIND IT!!!.  This note has a stronger conjecture
\begin{conjecture}\label{FGSconj}
Every graph $G$ with $4t-1$ vertices and with $\alpha{G} = 2$ contains a connected matching of size $t$.
\end{conjecture}
This is sharp (consider $G$ consisting of disjoint cliques), and is true for some values of $t$.  Let $f(t)$ the minimum $n$ such that every graph $G$ with $n$ vertices and \AG has a connected matching of size $t$. 
\begin{theorem}\label{FGSresult}
$f(t) = 4t-1$ for $1\leq t\leq17$
\end{theorem}
\begin{proof}
	Assume $G$ is a graph with $4t-1$
 vertices and with \AG.  Suppose first that the maximum degree of $\overline{G}$ is at least $t-1$ and let $v$ be a maximum degree vertex in $\overline{G}$.  Let $A\subset V(G)$ consist of $t$ (or all if there are only $t-1$) non-neighbors of $v$ (in $G$), thus $t-1 \leq |A|\leq t$.  Consider the bipartite subgraph $H = [A,B]$ of $G$, where $B= V(G)-(A\cup \{v\})$. If $H$ contains a matching of size $t$ then it is a connected matching, since $A$ induces a clique in $G$.  Also, if $|A| = t-1$ and $H$ contains a matching of size $t-1$m it can be extended by an edge incident to $v$ to a connected matching of size $t$.  If the required matching does not exist, by K\"onig's theorem, there is a $T\subset V(G)$ with $|T| \leq t-1$ (or $|T| \leq t-2$ if $|A| = t-1$) meeting all edges of $H$. As $|B| \geq 3t-2$, this implies that there exists a vertex in $A-T$ nonadjacent to at least $2t$ vertices of $G$.  Thus $K_{2t}\subset G$ which clearly contains a connected matching of size $t$. [{\em Note: this is essentially a different proof of the spider lemma, using minmax instead of Hall's theorem.  Determine the exact consequences of each version, esp. in the quantitative version.}] 

\vskip 0.25 cm

Therefore the maximum degree of $\overline{G}$ is at most $t-2$.  Now let $A_v$ denote the set of non-neighbors and $B_v$ the set of neighbors of $v$ in $G$.  Some vertex $w\in B_v$ is nonadjacent to at most 
\begin{equation}\label{FGS1}
\frac{|A_v|(t-3)}{B_v}\leq \frac{(t-2)(t-3)}{3t}
\end{equation}
vertices of $A_v$.  The right hand side of (\ref{FGS1}) is less that 4 if $t\leq 16$.  If $t = 17$ then, as all vertices cannot have odd degree, $v$ can be selected as a vertex nonadjacent to at most 14 vertices and the estimate (1) still gives a $w \in B_v$ nonadjacent to at most $14^2/51 < 4$ vertices of $A_v$.  Thus we have found an edge $vw$ in $G$ such that the set $C\subset V(G)$ nonadjacent to both $v$ and $w$ satisfies $|C|\leq 3$.   This allows us to carry out the inductive proof: removing $v,w$ and two further vertices (as many from $C$ as possible) the remaining graph has a connected matching of size $t-1$ and the edge $vw$ extends it to a connected matching of size $t$.  (Of course it is trivial to start the induction with $f(1) = 3$).
[{\em Note: We can bring more information about a minimum counterexample and attempt to improve this section and extend this result to larger $t$.}]
\end{proof}
A 2-connected matching of size $t$ is a matching in which each pair of edges have at least 2 edges beteween their endpoint sets.  The following follows the proof method of Theorem \ref{FGSresult} and shows that even 2-connected matchings beat the obvious Ramsey bound of $\frac{t^2}{\log t}$.  
\begin{theorem}\label{2connmatch}
Every graph $G$ with \AG and with at least $2^{3/4}t^{3/2} + 2t + 1$ vertices contains a 2-connected matching of size $t$.  
\end{theorem}
\begin{proof}
Set $c = 2^{5/4}$ which is the positive root of $\frac{4}{c} = \frac{c\sqrt{2}}{2}$.  We want to establish the recursive bound $g(t) \leq g(t-1) + ct^{1/2}+2$, for the function $g(t)(t\geq 2, g(1) = 3)$.  Then (using the inequality between arithmetic and quadratic means)
\[g(t) \leq c(\sum_{i=2}^{t}i^{1/2}) + 2(t-1) + g(1) \leq c\frac{\sqrt{2}}{2}t^{3/2}+2t+1 = 2^{3/4}t^{3/2} + 2t + 1,\]
the theorem follows (for $t=1$ it holds vacuously).
\vskip 0.25 cm
Using the argument of Theorem \ref{FGSresult}, let $N$ be the smallest integer satisfying $N\geq 2^{3/4}t^{3/2} + 2t + 1$, let $G$ be a graph with $N$ vertices and \AG.  Assuming $G$ has no 2-connected matching of size $t$, any $v\in V(G)$ is nonadjacent to at most $2t-1$ vertices of $G$.  Using the argument from the proof of Theorem \ref{FGSresult}, for any $v\in V(G)$ there is a $w\in B_v$ such that there are at most $M=\frac{(2t-1)(2t-2)}{N-2t}$ vertices of $G$ nonadjacent to both $v$ and $w$.  Therefore, it is possible to remove at most $M+2$ vertices of $G$ so that the remaining graph does not contain 2-connected matchings of size $t-1$.  Thus,
\begin{equation}\label{FGS2}
g(t) < g(t-1) + \frac{(2t-1)(2t-2)}{N-2t} + 2.
\end{equation}
Notice  that $\frac{(2t-1)(2t-2)}{N-2t} < ct^{1/2}$ because otherwise we get 
\[N < (\frac{4}{c})t^{3/2}+2t= 2^{3/4}t^{3/2} + 2t \]
implying 
\[2^{3/4}t^{3/2} + 2t +1 \leq N < 2^{3/4}t^{3/2} + 2t, \]
a contradiction.  Thus (\ref{FGS2}) gives the claimed recursive bound for $g(t)$.
\end{proof}
\vskip 0.5cm 

\bib{Spec_case}{article}{
   author={Plummer, Michael D.},
   author={Stiebitz, Michael},
   author={Toft, Bjarne},
   title={On a special case of Hadwiger's conjecture},
   journal={Discuss. Math. Graph Theory},
   volume={23},
   date={2003},
   number={2},
   pages={333--363},
   issn={1234-3099},
   review={\MR{2070161 (2005e:05055)}},
}

Here we see two equivalent statements of Hadwiger's conjecture for \kfree graphs.
\begin{conjecture}\label{H2}
For all $G$, $\alpha(G) = 2 \implies G \succeq K_{\chi(G)}$
\end{conjecture}
\begin{conjecture}\label{H3}
For all $G$, $\alpha(G) = 2 \implies G \succeq K_{\lceil |V(G)|/2\rceil}$
\end{conjecture}
\begin{theorem}\label{H2equivH3}
The conjectures \ref{H2} and \ref{H3} are equivalent
\end{theorem}
We begin with a list of properties of a smallest possible counterexample to Conjecture \ref{H2} and will prove Theorem \ref{H2equivH3} along the way.  Let $G$ be such a counterexample.
\begin{enumerate}
	\item {\em For each $x \in V(G)$, $\chi(G-x) < \chi(G)$. ($G$ is vertex critical.)}  \newline
  If $|V(G)| = 2\chi(G)$ then for arbitrary $x \in V(G)$ we have that $2\chi(G) -1 = |V(G-x)| \leq 2\chi{G-x}$ hence $\chi(G)-\frac{1}{2} \leq \chi(G-x) \leq \chi(G)$, i.e., $\chi(G-x) = \chi(G)$.  But $G\succeq G-x$ and $G-x\succeq K_{\chi(G-x)} = K_{\chi(G)}$ (by the minimality of $G$).  Hence $G\succeq K_{\chi(G)}$ and is not a counterexample.  Therefore $|V(G)| \leq 2\chi(G)-1$. \newline 
If $\chi(G-x) = \chi(G)$ for a vertex $x\in V(G)$ then by the minimality of $G$ we get a contradiction as above.
	\item {\em $\overline{G}$ is connected.}
	\item {\em $|V(G)| = 2\chi(G) -1$}
	\item {\em For all $x \in V(G)$, $\overline{G} -x$ has a perfect matching.}
	\item {\em $G$ is $\alpha$-critical.}
	\item {\em $G$ is contraction $\chi$-critical}
	\item {\em $\chi{G} \geq 7$}
	\item {\em $G$ is 7-vertex-connected}
	\item {\em $\delta(G) \geq \chi(G)$ and $G$ is $\chi(G)$-edge-connected}
	\item {\em $G$ is hamiltonian}
	\item {\em $G$ is factor critical}
	\item {\em $G$ does not contain a non-empty connected dominating matching}
	\item {\em $\omega(G)\leq \chi(G)-3$}
	\item {\em $\delta(G) \geq \chi(G)+1$}
	\item {\em $\kappa(G) \geq \chi(G)$}
	\item {\em }

\end{enumerate}

\bib{sqrtofgraph}{article}{
   author={Mukhopadhyay, A.},
   title={The square root of a graph},
   journal={J. Combinatorial Theory},
   volume={2},
   date={1967},
   pages={290--295},
   review={\MR{0210616 (35 \#1502)}},
}
\bib{DandM}{article}{
   author={Duchet, P.},
   author={Meyniel, H.},
   title={On Hadwiger's number and the stability number},
   conference={
      title={Graph theory},
      address={Cambridge},
      date={1981},
   },
   book={
      series={North-Holland Math. Stud.},
      volume={62},
      publisher={North-Holland},
      place={Amsterdam},
   },
   date={1982},
   pages={71--73},
   review={\MR{671905 (84h:05074)}},
}

\bib{MR882610}{article}{
   author={Maffray, F.},
   author={Meyniel, H.},
   title={On a relationship between Hadwiger and stability numbers},
   journal={Discrete Math.},
   volume={64},
   date={1987},
   number={1},
   pages={39--42},
   issn={0012-365X},
   review={\MR{882610 (88g:05076)}},
   doi={10.1016/0012-365X(87)90238-X},
}


\bib{MR1411244}{article}{
   author={Toft, Bjarne},
   title={A survey of Hadwiger's conjecture},
   note={Surveys in graph theory (San Francisco, CA, 1995)},
   journal={Congr. Numer.},
   volume={115},
   date={1996},
   pages={249--283},
   issn={0384-9864},
   review={\MR{1411244 (97i:05048)}},
}
\bib{MR1654153}{article}{
   author={Reed, Bruce},
   author={Seymour, Paul},
   title={Fractional colouring and Hadwiger's conjecture},
   journal={J. Combin. Theory Ser. B},
   volume={74},
   date={1998},
   number={2},
   pages={147--152},
   issn={0095-8956},
   review={\MR{1654153 (99k:05079)}},
   doi={10.1006/jctb.1998.1835},
}
\bib{MR1844036}{article}{
   author={Kotlov, Andre{\u\i}},
   title={Matchings and Hadwiger's conjecture},
   note={Algebraic and topological methods in graph theory (Lake Bled,
   1999)},
   journal={Discrete Math.},
   volume={244},
   date={2002},
   number={1-3},
   pages={241--252},
   issn={0012-365X},
   review={\MR{1844036 (2002k:05087)}},
   doi={10.1016/S0012-365X(01)00087-5},
}



\bib{K_Cam}{article}{
   author={Cameron, Kathie},
   title={Connected matchings},
   conference={
      title={Combinatorial optimization---Eureka, you shrink!},
   },
   book={
      series={Lecture Notes in Comput. Sci.},
      volume={2570},
      publisher={Springer},
      place={Berlin},
   },
   date={2003},
   pages={34--38},
   review={\MR{2163948 (2006c:90072)}},
   %doi={10.1007/3-540-36478-1_5},
}

\bib{MR1979786}{article}{
   author={Klazar, Martin},
   title={Non-$P$-recursiveness of numbers of matchings or linear chord
   diagrams with many crossings},
   note={Formal power series and algebraic combinatorics (Scottsdale, AZ,
   2001)},
   journal={Adv. in Appl. Math.},
   volume={30},
   date={2003},
   number={1-2},
   pages={126--136},
   issn={0196-8858},
   review={\MR{1979786 (2004h:05006)}},
   doi={10.1016/S0196-8858(02)00528-6},
}







\bib{DandMimprove}{article}{
   author={Kawarabayashi, Ken-ichi},
   author={Plummer, Michael D.},
   author={Toft, Bjarne},
   title={Improvements of the theorem of Duchet and Meyniel on Hadwiger's
   conjecture},
   journal={J. Combin. Theory Ser. B},
   volume={95},
   date={2005},
   number={1},
   pages={152--167},
   issn={0095-8956},
   review={\MR{2156345 (2006b:05118)}},
   doi={10.1016/j.jctb.2005.04.001},
}


\bib{MR2249267}{article}{
   author={Gy{\'a}rf{\'a}s, Andr{\'a}s},
   author={Ruszink{\'o}, Mikl{\'o}s},
   author={S{\'a}rk{\"o}zy, G{\'a}bor N.},
   author={Szemer{\'e}di, Endre},
   title={One-sided coverings of colored complete bipartite graphs},
   conference={
      title={Topics in discrete mathematics},
   },
   book={
      series={Algorithms Combin.},
      volume={26},
      publisher={Springer},
      place={Berlin},
   },
   date={2006},
   pages={133--144},
   review={\MR{2249267 (2008c:05120)}},
   %doi={10.1007/3-540-33700-8_8},
}

\bib{M_Krie}{article}{
   author={Kriesell, Matthias},
   title={On Seymour's strengthening of Hadwiger's conjecture for graphs with certain forbidden subgraphs},
   journal={Discrete Mathematics},
   %volume={},
   date={2010},
   pages={435--438},
}

\bib{MR0297600}{article}{
   author={Chv{\'a}tal, V.},
   author={Erd{\H{o}}s, P.},
   title={A note on Hamiltonian circuits},
   journal={Discrete Math.},
   volume={2},
   date={1972},
   pages={111--113},
   issn={0012-365X},
   review={\MR{0297600 (45 \#6654)}},
}

\bib{MR1369063}{article}{
   author={Kim, Jeong Han},
   title={The Ramsey number $R(3,t)$ has order of magnitude $t\sp 2/\log t$},
   journal={Random Structures Algorithms},
   volume={7},
   date={1995},
   number={3},
   pages={173--207},
   issn={1042-9832},
   review={\MR{1369063 (96m:05140)}},
   doi={10.1002/rsa.3240070302},
}

\bib{MR0284366}{article}{
   author={Nash-Williams, C. St. J. A.},
   title={Edge-disjoint Hamiltonian circuits in graphs with vertices of
   large valency},
   conference={
      title={Studies in Pure Mathematics (Presented to Richard Rado)},
   },
   book={
      publisher={Academic Press},
      place={London},
   },
   date={1971},
   pages={157--183},
   review={\MR{0284366 (44 \#1594)}},
}

\bib{blas}{article}{
   author={Blasiak, Jonah},
   title={A special case of Hadwiger's conjecture},
   journal={J. Combin. Theory Ser. B},
   volume={97},
   date={2007},
   number={6},
   pages={1056--1073},
   issn={0095-8956},
   review={\MR{2354718 (2009a:05064)}},
   doi={10.1016/j.jctb.2007.04.003},
}

\bib{edhc}{article}{
	author={Christofides, Demetres},
	author={K\"{u}hn, Daniela},
	author={Osthus, Deryk},
	title={Edge-disjoint Hamilton cycles in graphs},
	date={31 Aug 2009},
	eprint={arXiv:0908.4572v1 [math.CO]},
	url={http://arxiv.org/abs/0908.4572},
}


	\end{biblist}
\end{bibsection}}

\end{document}