
In the following, all graphs are assumed to be finite, simple, and undirected. The \textit{complete} graph on $n$ vertices, denoted $K_n$, is the $n$-vertex graph with all possible edges, and the \textit{empty} graph on $n$ vertices is the $n$-vertex graph with no edges.  A complete subgraph of a graph is called a \textit{clique}, and the \textit{clique number} of $G$, denoted $\omega(G)$, is the number of vertices in the largest clique found in $G$.  Similarly, the \textit{independence number} of $G$, denoted $\alpha(G)$, is the number of vertices in the largest empty subgraph of $G$.  The \textit{distance in $G$} between two vertices $u$ and $v$, denoted $d_G(u,v)$ is the number of edges in the shortest $u-v$ path in $G$. 

The \textit{chromatic number} of a graph $G$, denoted $\chi(G)$, is the minimum number of colors needed to color the vertices of $G$ so that no pair of adjacent vertices recieves the same color. We say that a graph $G$ is \textit{perfect} if and only if for every induced subgraph $H$ of $G$, $\chi(H) = \omega(H)$. 

A \textit{matching} in a graph $G$ is a collection of disjoint edges.  A \textit{connected matching} is a matching $M$ in $G$ with the additional property that any two edges $e_1, e_2 \in M$ have some endpoint of $e_1$ adjacent to some endpoint of $e_2$.  A connected matching $M_c$ is \textit{dominating} if each of its edges are adjacent to every vertex in $V(G)- V(M_c)$.  The maximum size of a (connected) matching in $G$ is called the \textit{(connected) matching number} of $G$, and is denoted $(\nu_c(G)) \nu(G)$. Further definitions and terminology can be found in \cite{dwest}.

