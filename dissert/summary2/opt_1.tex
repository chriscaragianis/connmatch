\subsection{Modeling with connected matchings}
\begin{figure}[h]
\begin{center}\begin{tikzpicture}[thick,scale=0.5]

	\coordinate (a1) at (22:5);
	\coordinate (a2) at (44.5:5);
	\coordinate (a3) at (67:5);
	\coordinate (a4) at (89.5:5);
	\coordinate (a5) at (112:5);
	\coordinate (a6) at (134.5:5);
	\coordinate (a7) at (157:5);
	\coordinate (a8) at (179.5:5);
	\coordinate (a9) at (202:5);
	\coordinate (a10) at (224.5:5);
	\coordinate (a11) at (247:5);
	\coordinate (a12) at (269.5:5);
	\coordinate (a13) at (292:5);
	\coordinate (a14) at (314.5:5);
	\coordinate (a15) at (337:5);
	\coordinate (a16) at (359.5:5);

	% the K_5
	\draw (a1) -- (a2);
	\draw (a1) -- (a3);
	\draw (a1) -- (a4);
	\draw (a1) -- (a5);
	\draw (a2) -- (a3);
	\draw (a2) -- (a4);
	\draw (a2) -- (a5);
	\draw (a3) -- (a4);
	\draw (a3) -- (a5);
	\draw (a4) -- (a5);

	% the connmatch
	\draw (a1) -- (a16);
	\draw (a2) -- (a13);
	\draw (a3) -- (a11);	
	\draw (a6) -- (a10); %new	
	\draw (a8) -- (a4);
	\draw (a5) -- (a14);
	
	% connect it up
	\draw (a6) -- (a4); 
	\draw (a6) -- (a8);
	\draw (a10) -- (a11);
	\draw (a10) -- (a13);
	\draw (a10) -- (a16);

	% the K_{1,3}
	\draw (a7) -- (a9);
	\draw (a7) -- (a15);
	\draw (a7) -- (a12);
	
	% connect THAT up
	\draw (a7) -- (a6);
	\draw (a7) -- (a3);
	\draw (a9) -- (a16);
	\draw (a15) -- (a2);
	\draw (a12) -- (a14);
	\draw (a15) -- (a8);
	% the nodes
	\draw (a1) node[lblvertex2] {a};
	\draw (a2) node[lblvertex2] {b};
	\draw (a3) node[lblvertex2] {c};
	\draw (a4) node[lblvertex2] {d};
	\draw (a5) node[lblvertex2] {e};
	\draw (a6) node[lblvertex2] {f};
	\draw (a7) node[lblvertex2] {g};
	\draw (a8) node[lblvertex2] {h};
	\draw (a9) node[lblvertex2] {i};
	\draw (a10) node[lblvertex2] {j};
	\draw (a11) node[lblvertex2] {k};
	\draw (a12) node[lblvertex2] {l};
	\draw (a13) node[lblvertex2] {m};
	\draw (a14) node[lblvertex2] {n};
	\draw (a15) node[lblvertex2] {o};
	\draw (a16) node[lblvertex2] {p};
	
\end{tikzpicture}
\end{center}
\caption{A graphical model of a social networking application}
\label{soc_net}
\end{figure}
When presented with any graph structure, one can immediately consider what systems the structure can model.  A graph model abstracts away the ``connectedness'' relationships of a complex system and views them out of context.  By determining the specific connectedness properties of the structure in question, one discovers the sort of systemic properties optimized by the presence of that structure. As a motivating hypothetical example, let us consider a social networking application with a mutual information sharing funtion between users (``friends'').  We derive a graphical model by assigning a vertex to represent each user, with an edge between vertices representing users that share information (figure \ref{soc_net}).

The presence of a clique in this model indicates a collection of users that pairwise share information.  The maximum clique problem has as its solution the largest such collection, illustrated in figure \ref{soc_net_clique}.
\begin{figure}[h]
\begin{center}\begin{tikzpicture}[thick,scale=0.5]

	\coordinate (a1) at (22:5);
	\coordinate (a2) at (44.5:5);
	\coordinate (a3) at (67:5);
	\coordinate (a4) at (89.5:5);
	\coordinate (a5) at (112:5);
	\coordinate (a6) at (134.5:5);
	\coordinate (a7) at (157:5);
	\coordinate (a8) at (179.5:5);
	\coordinate (a9) at (202:5);
	\coordinate (a10) at (224.5:5);
	\coordinate (a11) at (247:5);
	\coordinate (a12) at (269.5:5);
	\coordinate (a13) at (292:5);
	\coordinate (a14) at (314.5:5);
	\coordinate (a15) at (337:5);
	\coordinate (a16) at (359.5:5);

	% the K_5
	\draw (a1)[red] -- (a2);
	\draw (a1)[red] -- (a3);
	\draw (a1)[red] -- (a4);
	\draw (a1)[red] -- (a5);
	\draw (a2)[red] -- (a3);
	\draw (a2)[red] -- (a4);
	\draw (a2)[red] -- (a5);
	\draw (a3)[red] -- (a4);
	\draw (a3)[red] -- (a5);
	\draw (a4)[red] -- (a5);

	% the connmatch
	\draw (a1)[black!25] -- (a16);
	\draw (a2)[black!25] -- (a13);
	\draw (a3)[black!25] -- (a11);	
	\draw (a6)[black!25] -- (a10); %new	
	\draw (a8)[black!25] -- (a4);
	\draw (a5)[black!25] -- (a14);
	
	% connect it up
	\draw (a6)[black!25] -- (a4); 
	\draw (a6)[black!25] -- (a8);
	\draw (a10)[black!25] -- (a11);
	\draw (a10)[black!25] -- (a13);
	\draw (a10)[black!25] -- (a16);

	% the K_{1,3}
	\draw (a7)[black!25] -- (a9);
	\draw (a7)[black!25] -- (a15);
	\draw (a7)[black!25] -- (a12);
	
	% connect THAT up
	\draw (a7)[black!25] -- (a6);
	\draw (a7)[black!25] -- (a3);
	\draw (a9)[black!25] -- (a16);
	\draw (a15)[black!25] -- (a2);
	\draw (a12)[black!25] -- (a14);
	\draw (a15)[black!25] -- (a8);

	% the nodes
	\draw (a1) node[lblvertex3, draw = red] {a};
	\draw (a2) node[lblvertex3, draw = red] {b};
	\draw (a3) node[lblvertex3, draw = red] {c};
	\draw (a4) node[lblvertex3, draw = red] {d};
	\draw (a5) node[lblvertex3, draw = red] {e};
	\draw (a6) node[lblvertex3] {f};
	\draw (a7) node[lblvertex3] {g};
	\draw (a8) node[lblvertex3] {h};
	\draw (a9) node[lblvertex3] {i};
	\draw (a10) node[lblvertex3] {j};
	\draw (a11) node[lblvertex3] {k};
	\draw (a12) node[lblvertex3] {l};
	\draw (a13) node[lblvertex3] {m};
	\draw (a14) node[lblvertex3] {n};
	\draw (a15) node[lblvertex3] {o};
	\draw (a16) node[lblvertex3] {p};

	
\end{tikzpicture}
\end{center}
\caption{Maximum clique, $\omega(G) = 5$}
\label{soc_net_clique}
\end{figure}  

Now suppose that our social networking application supports a ``group'' function wherein users may form groups with the property that group members expose information (reciprocally) with any users sharing information with any other member of the group.  If one can join a group only on the invite of a ``friend'', then the group formation process can be modeling by the sequential contraction of edges in the modeling graph.  Now information-sharing among groups can be modeled by minors found in the graph in the same way that the original graph models such sharing among individual users.

In particular, we can ask for the maximum number of groups that can be formed with the property that they pairwise share information.  In the model, this is tantamount to asking for the largest \textit{clique minor}.  In figure \ref{soc_net_minor}, each color indicated the edges to contract to form a collection of seven groups mutually sharing information.
\begin{figure}[h]
\begin{center}\input{party4}\end{center}
\caption{Maximum clique minor, $\eta(G) = 7$}
\label{soc_net_minor}
\end{figure} 

Connected matchings are a special case of clique minors.  In particular, every edge of a $k$-connected matching can contracted to form a copy of $K_k$.  In this application, asking for the maximum connected matching in out graphical model tells us the size of the largest collection of groups with \textit{exactly two members} that mutually share information. 
\begin{figure}[h]
\begin{center}\begin{tikzpicture}[thick,scale=0.5]

	\coordinate (a1) at (22:5);
	\coordinate (a2) at (44.5:5);
	\coordinate (a3) at (67:5);
	\coordinate (a4) at (89.5:5);
	\coordinate (a5) at (112:5);
	\coordinate (a6) at (134.5:5);
	\coordinate (a7) at (157:5);
	\coordinate (a8) at (179.5:5);
	\coordinate (a9) at (202:5);
	\coordinate (a10) at (224.5:5);
	\coordinate (a11) at (247:5);
	\coordinate (a12) at (269.5:5);
	\coordinate (a13) at (292:5);
	\coordinate (a14) at (314.5:5);
	\coordinate (a15) at (337:5);
	\coordinate (a16) at (359.5:5);

	% the K_5
	\draw (a1)[black!25] -- (a2);
	\draw (a1)[black!25] -- (a3);
	\draw (a1)[black!25] -- (a4);
	\draw (a1)[black!25] -- (a5);
	\draw (a2)[black!25] -- (a3);
	\draw (a2)[black!25] -- (a4);
	\draw (a2)[black!25] -- (a5);
	\draw (a3)[black!25] -- (a4);
	\draw (a3)[black!25] -- (a5);
	\draw (a4)[black!25] -- (a5);

	% the connmatch
	\draw (a1)[green] -- (a16);
	\draw (a2)[green] -- (a13);
	\draw (a3)[green] -- (a11);	
	\draw (a6)[green] -- (a10); %new	
	\draw (a8)[green] -- (a4);
	\draw (a5)[green] -- (a14);
	
	% connect it up
	\draw (a6)[black!25] -- (a4); 
	\draw (a6)[black!25] -- (a8);
	\draw (a10)[black!25] -- (a11);
	\draw (a10)[black!25] -- (a13);
	\draw (a10)[black!25] -- (a16);

	% the K_{1,3}
	\draw (a7)[black!25] -- (a9);
	\draw (a7)[black!25] -- (a15);
	\draw (a7)[black!25] -- (a12);
	
	% connect THAT up
	\draw (a7)[black!25] -- (a6);
	\draw (a7)[black!25] -- (a3);
	\draw (a9)[black!25] -- (a16);
	\draw (a15)[black!25] -- (a2);
	\draw (a12)[black!25] -- (a14);
	\draw (a15)[black!25] -- (a8);

	% the nodes
	\draw (a1) node[lblvertex3, draw = green] {a};
	\draw (a2) node[lblvertex3, draw = green] {b};
	\draw (a3) node[lblvertex3, draw = green] {c};
	\draw (a4) node[lblvertex3, draw = green] {d};
	\draw (a5) node[lblvertex3, draw = green] {e};
	\draw (a6) node[lblvertex3, draw = green] {f};
	\draw (a7) node[lblvertex3] {g};
	\draw (a8) node[lblvertex3, draw = green] {h};
	\draw (a9) node[lblvertex3] {i};
	\draw (a10) node[lblvertex3, draw = green] {j};
	\draw (a11) node[lblvertex3, draw = green] {k};
	\draw (a12) node[lblvertex3] {l};
	\draw (a13) node[lblvertex3, draw = green] {m};
	\draw (a14) node[lblvertex3, draw = green] {n};
	\draw (a15) node[lblvertex3] {o};
	\draw (a16) node[lblvertex3, draw = green] {p};

	
\end{tikzpicture}
\end{center}
\caption{Maximum connected matching, $cm(G) = 6$}
\label{soc_net_conmatch}
\end{figure} 
It is worth noting that while $\omega(G)$ and $cm(G)$ are always less than or equal to $\eta(G)$ (as cliques and connected matchings are particular cases of clique minors), there is no such clear relationship between $\omega(G)$ and $cm(G)$.

\subsection{Computing connected matchings}
Now we take up the complexity of computing the largest connected matching in a graph.  We can define the computational problem as follows
{\linespread{1.2}\begin{framed}
  \noindent \textbf{Maximum Connected Matching (MCM)}
  \vskip 0.25 cm 
  \noindent Input: Graph $G$
  \newline Output: Maximum connected matching of $G$
 \end{framed}
We also may be interested in the largest connected portion of a given matching in $G$
 \begin{framed}
  \noindent \textbf{Maximum Connected Portion (MCP)}
  \vskip 0.25 cm 
  \noindent Input: A matching $M$ from a graph $G$
  \newline Output: Maximum connected portion of $M$
\end{framed}}
Both MCM and MCP are NP-hard in general.  Plummer et al. consider MCM in \cite{Spec_case}, and we will show a reduction of the maximum clique problem to MCP, verifying its hardness.  Cameron \cite{K_Cam} has shown that MCM remains hard even for bipartite graphs, while it is in $\mathbf{P}$ for chordal graphs.  Upon developing some machinery in the following section. we shall exhibit some results and further conjectures on classes of graphs with efficient solutions to these problems.

