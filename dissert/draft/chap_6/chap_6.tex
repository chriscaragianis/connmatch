\chapter{Future work}
In investigating connected matchings, many tangential problems arise that are quite interesting in their own right.  In this chapter, we discuss some of these, along with any progress made and possible areas of deeper investigation in the future.

\section{Characterizing the ``connected matching graph''}

The proximity partition induced by a graph $G$ can be thought of as the collection of \textit{distance-$k$ graphs} of $G$.
%
We may ask if there is a characterization of $\mathcal{H}_k$ where
%
\[\mathcal{H}_k = \{H :\: H \makebox{ is the distance-$k$ graph of some graph } G\}\]
%
In the case of $\mathcal{H}_2$, we can do so.
%
Let $A(G)$ denote the adjacency matrix of a graph $G$.  
%
In \cite{sqrtofgraph} Mukhopdhyay characterizes graphs that have a \textit{square root}, which is to say graphs $H$ such that for some graph $G$, $A(H) = A(G)^2$.
%  
This is equivalent to $H$ possesing an edge between any pair of vertices $u,v$ that satisfy $d_G(u,v) \leq 2$.  
%
\bthm{[Mudkhopdhyay] A connected graph $G$ with $n$ vertices $v_1, v_2, \ldots, v_n$ has a square root if and only if some set of $n$ complete subgraphs of $G$ whose union is $G$ can be labeled $C_1, C_2, \ldots, C_n$ so that, for all $i,j = 1, 2, \ldots, n$ the following conditions hold:
\begin{enumerate}
	\item $C_i$ contains $v_i$,
	\item $C_i$ contains $v_j$ if and only if $C_ij$ contains $v_i$.
\end{enumerate}}
\noindent An alternate definition of $\mathcal{H}_2$ is 
\[\mathcal{H}_2 = \{H : \: A(H) = A(G)^2-A(G)\}\] and we have the following characterization in the spirit of Mukhopdhyay's theorem.  
\bthm{A graph $G$ with $n$ vertices $v_1, v_2, \ldots , v_n$ is the distance 2 graph of some graph $H$ if and only if some set of $n$ subgraphs of $G$ whose union is $G$ can be labeled $C_1, C_2, \ldots, C_n$ so that 
\begin{enumerate}
	\item $v_a \notin C_a$ 
	\item For every pair of vertices $v_i, v_j \in C_k$, exactly one of the following holds:
	\begin{enumerate}
		\item $v_iv_j \in E(C_k)$ 
		\item If $v_i \in V(C_j)$, then $v_j \in V(C_i)$
		\item $v_i \in C_j$ and $v_j \in C_i$
	\end{enumerate}
	\item If $C_i \cap C_j \neq \emptyset$, then $v_i, v_j \in V(C_k)$ for some $k$.
\end{enumerate}}
The proof of this theorem rests on the same central realization as Mudkhopdhay's theorem, which is that the enumerated subgraphs correspond to the neighborhoods (closed neighborhoods in Mudkhopdhay's theorem, open ones in theorem REF) in the underlying graph.
\begin{proof}
Suppose we have a graph $G$ on vertices $v_1, v_2, \ldots, v_n$ and subgraphs $C_1, C_2, \ldots, C_n$ that satisfy the above hypotheses.  We construct a graph $H$ on the same vertex set by adding edges in two steps for each $C_i$.
\begin{description}
	\item[Step 1.] Add the complement of $C_i$.
	\item[Step 2.] Add all edges from $v_i$ to $C_i$.
\end{description}
	We claim that $G$ is now the distance 2 graph of $H$.  Suppose that $d_H(v_i, v_j) = 2$.  We want to show that $v_iv_j \in E(G)$.  Since $d_H(v_i,v_j) \leq 2$, there is some vertex $v_k$ so that $v_iv_k, v_jv_k \in E(H)$.  If both edges were added in step 2, then one of the following occurs
\begin{description}
	\item[Case 1.] $v_i, v_j \in C_k$
	\item[Case 2.] $v_k \in C_i$, $v_k \in C_j$
	\item[Case 3.] $v_k \in C_i, C_j$
\end{description}    
In case 1, distance 2 implies $v_iv_j \notin E(H)$. In particular, this edge was not added in step 2, so $v_i \notin C_j$ and $v_j \notin C_i$.  Condition 2 then implies that if $v_i, v_j \in C_k$ for some $C_k$, then $v_iv_j \in E(G)$.  In case 2, $V(C_i)$ and $V(C_j)$ intersect, implying (by condition 3) again that there is a $V(C_l)$ containing both $v_i$ and $v_j$.  In case 3, condition 4 requires that $v_i, v_j \in C_k$.  In any event, $v_iv_j \in E(G)$.

Now we may assume WLOG that $v_iv_k$ was added to $H$ in step 1.  Suppose $v_jv_k$ was added in step 2. Then either $v_k \in C_j$, implying $C_i \cap C_j \neq \emptyset$, or $v_j \in C_k$ implying $v_i, v_j \in C_k$.  The only remaining possibility is that both $v_iv_k$ and $v_jv_k$ were added in step 1.  Following from two applications of condition 2(b), both $v_i$ and $v_j$ are then in $V(C_k)$.  Sufficiency complete.

For the proof of necessity, we take a graph $H$ and show that its distance two graph $D_2(H)$ has a collection of subgraphs with the necessary properties.  For each vertex $v_i$, let $V(C_i) = N(v_i)$.  That condition 1 holds is immediate.  The vertices of any given $C_i$ are at most distance two apart.  Whenever there is a nonedge in a particular $C_i$ and 2(a) fails, the vertices must be adjacent in $H$, and condition 2(b) holds.  The symmetric property of the neighbor relation shows that conditions 3 and 4 hold as well.  Finally, all distance 2 edges occur between vertices with a common neighbor, so $\bigcup C_i = G$. 
\end{proof}

In a distance 2 graph, the enumerated subgraphs are actually the {\it complements} of the graphs induced by the open neighborhoods of the underlying graphs.  This allows us to characterize the distance 2 graphs of graphs with local characterizations.  The following is an example.
\bprop{If every enumerated subgraph $C_i$ of a distance 2 graph $H$ is triangle free, then it is the distance 2 graph of a claw-free graph.}
In the case of connected matchings, we are most interested in the distance 2 graphs of line graphs.  Unfortunately, line graphs do not have a local characterization.  The best local characterization that approximates them is the {\it locally co-bipartite} graphs.  A graph is locally co-bipartite if each open neighborhood induces a graph whose complement is bipartite.  Hence, the following is the best we have in the direction of characterizing the distance 2 graphs of line graphs.
\bprop{If every enumerated subgraph $C_i$ of a distance 2 graph $H$ is bipartite, then it is the distance 2 graph of a locally co-bipartite graph.}

\section{Tree Hypergraphs}

Chordal bipartite graphs, which figure heavily in chapter 4 of this dissertation, can be characterized as bipartite graph representations of a particular type of hypergraph.  
