% preamble

%\newcolumntype{R}{>{$}r<{$}} % came after \usepackage{array}

%\newcommand*{\vect}[1]{\ensuremath{\vv{\boldsymbol{\mathrm{#1}}}}}
%\newcommand{\vect}[1]{\ensuremath{\vv{#1}}}%{\vec{#1}}

%\newcommand*{\mbbone}{\ensuremath{\text{\fontspec{XITS Math}𝟙}}}
%\newcommand*{\mbbzero}{\ensuremath{\text{\fontspec{XITS Math}𝟘}}}

%\newcommand{\nsp}{\mskip-\thinmuskip}

% charge/potential notation
\begin{definition}
	\index{charge}
	\index{potential}
	Let $G$ be a simple graph. At each vertex $v$ of $G$ we will introduce a \textbf{charge}, $q_v$. We can calculate the \textbf{potential} $\phi$ at $v$ by summing the charges of the neighbors of $v$: \[ \phi_v = \sum_{u \in N\left(v\right)} q_u\]
\end{definition}

Any linear combination of the rows of the adjacency matrix of $G$ can be viewed in these terms. Let $q_v$ be the coefficient on the row vector corresponding to the vertex $v$; then the linear combination will be \[ \vect{s} = \sum_{i=1}^{n} q_{v_i} \vect{r}_{v_i} = \left\langle \phi_{v_1}, \phi_{v_2}, \dotsc, \phi_{v_n} \right\rangle \]

% monoplanes
The subspace of $\mbbRn$ orthogonal to a single vector has dimension $n-1$. Recall that a $\left(n-1\right)$-dimensional subspace of $\mbbRn$ is called a hyperplane\index{hyperplane}, and let $Q_{n}$ be the $n$-dimensional hypercube, i.e. the collection of all $\left\{0,1\right\}$-vectors of $\mbbRn$.\index{hypercube}

\begin{definition}[$n$-monoplane]
	\index{monoplane}
	\index{n-monoplane@$n$-monoplane|see{monoplane}}
	\index{size!of a monoplane}
	We say that $\mathcal{M}\subset\mbbRn$ is an \defn{$n$-monoplane} if there exists a hyperplane $H$ such that $\mathcal{M}=H\cap Q_{n}$. We will call $H$ a hyperplane \defn{associated with} $\mathcal{M}$. We may sometimes refer to the number of elements in $\mathcal{M}$ as the \defn{size} of $\mathcal{M}$.
\end{definition}

The definition forces $\mathcal{M} \subset Q_{n}$. We can think of $\mathcal{M}$ as a binary matrix $M$ with $n$ columns such that $\mathcal{M} = \Row{M}\cap Q_{n}$.

\begin{proposition}\label{prop:Monoplanes are closed}
	Let $\mathcal{M}$ be an $n$-monoplane. The subspace of $\mbbRn$ generated by $\mathcal{M}$ contains no hypercube vectors except those already in $\mathcal{M}$.
\end{proposition}
\begin{proof}
	Let $H$ be a hyperplane associated with $\mathcal{M}$. The subspace generated by $\mathcal{M}$ is a subspace of $H$, and $H$ contains no hypercube vectors except those already in $\mathcal{M}$ by definition.
\end{proof}

\begin{definition}[exact $n$-monoplane]
	\index{exact $n$-monoplane}
	\index{monoplane!exact}
	We say a monoplane $\mathcal{M}$ is an \defn{exact $n$-monoplane} if $\size(\mathcal{M})=n+1$.
\end{definition}

Since $\vect{0}\in\mathcal{M}$ for any $n$-monoplane $\mathcal{M}$, an exact $n$-monoplane will contain exactly $n$ more hypercube vectors. Thus we can write $\mathcal{M}\setminus\{\vect{0}\}$ as an $n\times n$ matrix with no hood vectors.

\begin{definition}[exact symmetric $n$-monoplane]%
	\index{exact symmetric $n$-monoplane}%
	\index{monoplane!exact symmetric}%
	We say an exact $n$-monoplane $\mathcal{M}$ is an \defn{exact symmetric $n$-monoplane} if $\mathcal{M}$ is  such that $\mathcal{M}\setminus\left\{\vect{0}\right\}$ can be written as a symmetric matrix.
\end{definition}

\begin{definition}[monopole]
	\index{monopole}
	We say that $\vect{v}\in\mbbRn$ is a \defn{monopole} for an $n$-monoplane $\mathcal{M}$ if there is a hyperplane $H$ associated with $\mathcal{M}$ such that $\vect{v}$ is orthogonal to $H$. We say $\vect{v}$ is a \defn{monopole for a matrix} $M$ if $\vect{v}$ is a monopole for the monoplane formed by $\Row M \cap Q_n$. We say $\vect{v}$ is a \defn{monopole for a graph} $G$ if $\vect{v}$ is a monopole for $\adjm{G}$. 
\end{definition}

% Prior versions of monoplole nonzero proof:

\comments{Without loss of generality we may assume $v_n=0$. Let $\left\langle b_1,\ldots, b_{n-1}, b_n\right\rangle$ be a row of $M$ other than the $n$th row. Since $\vect{v}\cdot\left\langle b_1, \ldots, b_{n-1}, 0\right\rangle = 0$ and $\vect{v}\cdot\left\langle b_1, \ldots, b_{n-1}, 1\right\rangle = 0$, any column of $M$ except the $n$th that contains one $1$ must contain at least two $1$s or else $M$ has a hood vector. Thus $M$ has at most one column (the $n$th) with weight $1$. Furthermore since $\zerovect$ is not a row of $M$ then at least one of $b_1$, $\dotsc$, $b_{n-1}$ is a $1$.

Since $\vect{v}\cdot \left\langle 0,0,\ldots, 0,1\right\rangle = 0$, $M$ has a row of weight $1$. Therefore the $n$th column of $M$ must have weight $1$ since $M$ is symmetric. Thus $\left\langle 0,0,\ldots, 0,1\right\rangle$ is the $n$th row of $M$.

Returning to $\left\langle b_1,\ldots, b_{n-1}, b_n\right\rangle$ we see that $\vect{b} = \left\langle b_1,\ldots, b_{n-1}, 1\right\rangle$ is either a hood vector or a row of $M$. Since at least one of $b_1$, $\dotsc$, $b_{n-1}$ is a $1$, $\vect{b}$ cannot be the $n$th row of $M$. But then the $n$th column of $M$ contains at least two $1$s, contradicting that $\left\langle 0,0,\ldots, 0,1\right\rangle$ is the $n$th row of $M$.}
	\comments{Without loss of generality we may assume $v_n=0$. Since $\vect{v}\cdot \left\langle 0,0,\ldots, 0,1\right\rangle = 0$, then $\left\langle 0,0,\ldots, 0,1\right\rangle$ is either a hood vector or a row of $M$. If it is a row of $M$, let $\left\langle b_1,\ldots, b_{n-1}, b_n\right\rangle$ be any other row of $M$. Since $\vect{v}\cdot\left\langle b_1, \ldots, b_{n-1}, 1\right\rangle = 0$, the $n$th column of $M$ will contain at least two $1$s. On the other hand, since $\vect{v}\cdot\left\langle b_1, \ldots, b_{n-1}, 0\right\rangle = 0$, for $1\leq i \leq n-1$ if $b_i = 1$ then the $i$th column will contain at least two $1$s. Therefore any column that contains one $1$ must contain at least two $1$s. Thus $M$ has no column containing exactly one one, and so $M$ cannot be symmetrizable

 On the other hand, if $b_i$ is always zero, then the $i$th column will not contain any ones. Regardless, the first $n-1$ columns cannot contain exactly one $1$. From our construction above, if the matrix contains at least three rows, the final column must also contain at least two $1$s. Thus no column sum will be equal to $1$ but there is a row whose sum is $1$, meaning that the matrix is not symmetrizable since the row and column sums are not compatible.}%
 
 \begin{center}\linespread{1}\selectfont
	\begin{tabular}[t]{r|r}
		\multicolumn{2}{c}{Start}\\\hline
		sum	& ways\\\hline
		$0$	& $1$
	\end{tabular}
	\begin{tabular}[t]{r|r}
		\multicolumn{2}{c}{After $8$}\\\hline
		sum	& ways\\\hline
		$0$	& $1$\\
		$8$	& $1$
	\end{tabular}
	\begin{tabular}[t]{r|r}
		\multicolumn{2}{c}{After $2$}\\\hline
		sum	& ways\\\hline
		$0$	& $1$\\
		$2$	& $1$\\
		$8$	& $1$\\
		$10$	& $1$
	\end{tabular}
	\begin{tabular}[t]{r|r}
		\multicolumn{2}{c}{After $1$}\\\hline
		sum	& ways\\\hline
		$0$	& $1$\\
		$1$	& $1$\\
		$2$	& $1$\\
		$3$	& $1$\\
		$8$	& $1$\\
		$9$	& $1$\\
		$10$	& $1$\\
		$11$	& $1$
	\end{tabular}
\end{center}