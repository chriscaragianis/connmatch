\documentclass[dvips,joc]{ipart}

\RequirePackage[OT1]{fontenc}
\RequirePackage{amsthm,amsmath}
\RequirePackage[numbers,square]{natbib}
\RequirePackage[colorlinks]{hyperref}
\RequirePackage{hypernat}

% settings
\pubyear{0000}
\volume{0}
\issue{0}
\firstpage{1}
\lastpage{8}
% \arxiv{http://arxiv.org/abs/0000.0000}

\startlocaldefs
%\numberwithin{equation}{section}
\theoremstyle{plain}
\newtheorem{thm}{Theorem}[section]
\endlocaldefs

\begin{document}

\begin{frontmatter}
\title{Ramsey Functions for Quasi-Progressions \\
with Large Diameter}
\runtitle{Ramsey Functions for Quasi-Progressions}


\begin{aug}
\author{\fnms{Adam}
\snm{Jobson}\ead[label=e1]{asjobs01@louisville.edu}},
\address{Department of Mathematics\\
University of Louisville\\
Louisville, KY 40292 USA\\
\printead{e1}}
\author{\fnms{Andr\'e} 
\snm{K\'ezdy}\ead[label=e2]{kezdy@louisville.edu}},
\address{Department of Mathematics\\
University of Louisville\\
Louisville, KY 40292 USA\\
\printead{e2}}
\author{\fnms{Hunter} 
\snm{Snevily}\ead[label=e3]{snevily@uidaho.edu}},
\address{Department of Mathematics\\
University of Idaho\\
Moscow, ID 83844 USA\\
\printead{e3}}
\and
\author{\fnms{Susan}
\snm{White}\ead[label=e4]{susan.white@louisville.edu}}
\address{Department of Mathematics\\
University of Louisville\\
Louisville, KY 40292 USA\\
\printead{e4}}

% \thankstext{t1}{Corresponding author.}
\runauthor{Jobson et al.}

\affiliation{University of Louisville and University of Idaho}
\end{aug}

% Author-defined commands
\newcommand{\Mod}[3]{$#1 \equiv #2\ (\mbox{mod }#3)$}

\begin{abstract}
A $k$-term quasi-progression of diameter $d$ is a sequence $$x_1 < \cdots < x_k$$ of positive integers for which there exists a positive integer $l$ such that $l \leq x_{j} - x_{j-1} \leq l+d$, for all $j=2,\ldots,k$. Let $Q\left(d,k\right)$ be the least positive integer such that every $2$-coloring of $\left\{1,\ldots,Q\left(d,k\right)\right\}$ contains a monochromatic $k$-term quasi-progression of diameter $d$. 
We prove that 
$$Q(k-i,k) = 2ik-4i+2r-1,$$ if $k=mi+r$ for integers $m,r$ such
that $3 \le r < \frac{i}{2}$ and $r-1 \le m$.  We also prove that,
if $k\geq 2i \geq 1$, then 
\[ Q\left(k-i,k\right) =
	\begin{cases}
		2ik-4i+3	&\mbox{if } $\Mod{k}{0\mbox{ or }2}{i}$\\
		2ik-2i+1	&\mbox{if } $\Mod{k}{1}{i}$
	\end{cases}
\]
These results partially settle several conjectures
due to Landman [Ramsey Functions for Quasi-Progressions, Graphs and Combinatorics 14 (1998) 131-142].
\end{abstract}

\received{\smonth{1} \sday{1}, \syear{0000}}

%\tableofcontents


\end{frontmatter}

% Author-defined commands
\newcommand{\Mod}[3]{$#1 \equiv #2\ (\mbox{mod }#3)$}


\section{Introduction}
Several renowned open conjectures in combinatorics and number theory involve arithmetic progressions.
Van der Waerden famously proved in 1927 that for each positive integer $k$ there exists a least positive
integer $w(k)$ such that any $2$-coloring of $1,\ldots,w(k)$ produces a monochromatic $k$-term
arithmetic progression.  The best known upper bound for $w(k)$ is due to Gowers and is quite large.  Ron Graham \cite{G}
conjectures $w(k) \le 2^{k^2}$, for all $k$.  The best known lower bound is due to Berlekamp:
$w(p+1) \ge p2^p$, for primes $p$.  Another conjecture involving arithmetic progressions is one of Erd\H{o}s' most famous (still open) conjectures:
if the sum of the reciprocals of the members of a set of positive integers $S$ diverges, then $S$
contains arbitrarily long arithmetic progressions.  Many researchers have worked on this or special cases
of this conjecture, including the recent and now famous theorem by Green and Tao which proves that the primes
contain arbitrarily long arithmetic progressions.

Motivated by Erd\H{o}s' conjecture, Brown, Erd\H{o}s, and Freedman \cite{B} introduced quasi-progressions:
a $k$-term {\em quasi-progression of diameter $d$}
is a sequence $x_1 < \cdots < x_k$ of positive integers for which there exists a positive integer $l$ such that $l \leq x_{j} - x_{j-1} \leq l+d$, for all $j=2,\ldots,k$.  Arithmetic progressions are quasi-progressions with diameter zero.
Brown et al. consider the question of when
a set of positive integers contains arbitrarily long quasi-progressions of a given diameter. A main result of theirs
is that quasi-progressions behave similarly to arithmetic progressions, at least with respect to Erd\H{o}s' conjecture,
in that Erd\H{o}s' conjecture is equivalent to the statement:
if the sum of the reciprocals of the members of a set of positive integers $S$ diverges, then there exists a diameter $d$
such that $S$ contains arbitrarily long quasi-progressions of diameter $d$.

Analogous to the van der Waerden function $w(k)$, Landman \cite{L} introduced a Ramsey function for quasi-progressions.
Let $Q\left(d,k\right)$ be the least positive integer such that every $2$-coloring of $\left\{1,\ldots,Q\left(d,k\right)\right\}$ contains a monochromatic $k$-term quasi-progression of diameter $d$. 
This function produces a lower bound for $w(k)$ since
$$
w(k) = Q(0,k) \geq Q(1,k) \geq Q(2,k) \geq Q(3,k) \geq \cdots
$$
So, it is of great interest to find bounds on $Q(d,k)$ for various $d$, especially small values of $d$.
Of particular interest is the rate of growth of $Q(1,k)$. Is it merely polynomial or is it at least exponential in $k$?  Vijay \cite{V} has recently established an exponential lower bound
for $Q(1,k)$,so quasi-progressions of small diameter behave similarly to arithmetic progressions, at least 
with respect to these Ramsey functions.  An interesting open problem is to determine the largest
diameter $d$ for which $Q(d,k)$ is at least exponential (in $k$).

Landman established several bounds on $Q(d,k)$ and made several conjectures which we resolve in this paper. 
Our results, like Landman's, focus on large diameter; that is, $d = k-i$, for some positive integer $i$ satisfying $k\geq 2i \geq 1$.
The main difficulty is the upper bound on $Q(k-i,k)$. Specifically, how can we tailor an argument
that handles the large number of extremal $2$-colorings
which seem to defy uniform description (cf. Landman's data at the end of his paper)?
In section 2 we introduce super blocks, an equivalence relation that imposes sufficient structure
on extremal $2$-colorings to extract long monochromatic quasi-progression fragments via a greedy strategy.
Through the super block lens, the extremal $2$-colorings coalesce.
The super block argument is used in Section 3 where we show how to splice monochromatic fragments together to produce 
long monochromatic quasi-progressions.  This yields an
upper bound on $Q(k-i,k)$; the bound is often sharp.  
One consequence is that, if $k \ge 2i$ and $k=mi+r$ for integers $m,r$ such
that $1 < r < i$, then $$Q(k-i,k) \le 2ik-4i+2r-1.$$
This improves bounds given by Landman.

Our main result is proven is Section 4:  
$$Q(k-i,k) = 2ik-4i+2r-1,$$ if $k=mi+r$ for integers $m,r$ such
that $3 \le r < \frac{i}{2}$ and $r-1 \le m$.  
This disproves Conjecture 1 of Landman's paper for these values of $r$.  Residues $r\ge i/2$ that are not
considered in this result appear to be more difficult. The techniques here are inadequate to resolve those
values of $Q(k-i,k)$.  However, we also prove that, 
if $k\geq 2i \geq 1$, then 
\[ Q\left(k-i,k\right) =
	\begin{cases}
		2ik-4i+3	&\mbox{if } $\Mod{k}{0\mbox{ or }2}{i}$\\
		2ik-2i+1	&\mbox{if } $\Mod{k}{1}{i}$
	\end{cases}
\]
thus proving Landman's conjecture in these cases.

\section{Super blocks}
In this section we develop notation, concepts and tools to describe the structure
of extremal strings which avoid long monochromatic quasi-progressions.

A {\em $k$-term progression} is an increasing sequence of $k$ positive
integers $x_1 < x_2 < \cdots < x_k$.  Given a $k$-term progression $P=\{x_j\}_{j=1}^k$,
the {\em differences} in $P$ are the elements in the set  
$$D(P) = \{ x_j - x_{j-1} : j=2,\ldots,k\}.$$
The {\em low-difference} of $P$, denoted $\delta(P)$ (or simply $\delta$), is the minimum
element in $D(P)$;  the {\em high-difference} of $P$, denoted $\Delta(P)$ (or simply $\Delta$), is the maximum.
The diameter of $P$ is $d = \Delta - \delta$.
Observe that arithmetic progressions are quasi-progressions with diameter $d = 0$.


Consider now $2$-colorings of the positive integers.
A quasi-progression is a {\em good progression} if it is a monochromatic 
quasi-progression with length $k$ and diameter at most $d$.
Define $Q(d,k)$ to be the least positive integer such that every $2$-coloring of
$\{1,\ldots,Q(d,k)\}$ contains a good progression.
Motivated by conjectures of Landman \cite{L}, we 
consider $Q(d,k)$ for values of $d$ and $k$
satisfying $d = k - i$ and $k \ge 2i$, where $i$ is some fixed positive integer.  So, for the rest of
this paper $d = k-i$ and a {\bf good progression} means a monochromatic 
quasi-progression with length $k$ and diameter at most $k-i$.
To understand the structure of extremal $2$-colorings avoiding good progressions,
we now introduce two important substructures: blocks and super blocks.

Let $C=c_1c_2\ldots c_\ell$ be a binary string of length $\ell$.
A {\em substring} of $C$ is a string of the form $c_p c_{p+1}\cdots c_q$, for some positive integers $1 \le p \le q \le \ell$. 
A {\em block} of $C$ is a maximal monochromatic substring of $C$.
We employ the usual shorthand notation in which, for $x \in \{0,1\}$, the shorthand $x^n$
represents the string $\underbrace{x \cdots x}_{n}$.
There is a natural partition of $C$ into blocks: 
without loss of generality, the first
block of $C$ is a block of $1$'s so $$C = 1^{\alpha_1} 0^{\alpha_2} \cdots 1^{\alpha_{b-1}} 0^{\alpha_b},$$
where $\alpha_j$'s are positive integers, except possibly $\alpha_b$ which may be zero (in which
case the final block of $C$ is actually a block of $1$'s).  Note that $\sum_{j=1}^b \alpha_j = \ell$.

Now consider an extremal coloring $C$; that is, suppose that $\ell = Q(k-i,k) - 1$ and $C$
represents a $2$-coloring of the integers $1,\ldots,\ell$ with no good progression.
For convenience, blocks of $C$ of length at most $k-i$ are {\em minor} blocks; longer blocks are {\em major}.
There are two important facts that motivate this dichotomy: 1) quasi-progressions with low-difference $1$ and diameter
$k-i$ can jump over any intermediate minor blocks, 
and 2) ``greedy monochromatic jumping'' (in which jumps of length at least $\delta$
but at most $\delta + k -i$ are taken, for some choice of $\delta$) can not get stuck in
substrings that avoid major blocks of one color (see later Theorems \ref{GreedyMajorJump} and \ref{GreedyMinorJump}).
A consequence of observation 1) is that, in a substring in which only minor blocks of one color appear,
all of the integers with the other color in this substring form a monochromatic progression with low-difference $1$ and
high-difference $k-i+1$ (that is, a monochromatic quasi-progression with diameter $k-i$).

Now we turn to the task of identifying monochromatic substructures of $C$ with the property that
a greedy strategy can guarantee dense monochromatic progressions beginning and ending at endpoints of the substructure.
To make this precise, we define the equivalence relation $\sim$ on the integers $1,\ldots,\ell$ so that
$x \sim y$ if and only if $x$ and $y$ are contained in a monochromatic quasi-progression $P$
of $C$ such that $D(P) \subseteq \{1,\ldots,k-i+1\}$ (the transitivity of $\sim$ follows from
the fact that the union of two intersecting monochromatic quasi-progressions with differences
in $\{1,\ldots,k-i+1\}$ is another such quasi-progression). The equivalence classes
under $\sim$ are called {\em {\bf super blocks}}.  Super blocks are not necessarily substrings.
Suppose that $C$ has $t$ super blocks $B_1,\ldots,B_t$.
We naturally order super blocks this way: $B_p < B_q$ if and only if $\min B_p < \min B_q$, where $\min B_p$ denotes
the smallest integer in $B_p$ (that is, the left-most one).
A super block is {\em major} if it contains all of the elements from a major block of $C$; otherwise it is {\em minor}.
The {\em extremes} of a super block are its minimum and maximum elements.

\vspace{0.5cm}
\noindent
{\sc Example 1}: Consider $k = 12, i = 6$. Because $Q(6,12) = 123$, an extremal string in this case has length $122$.
There are several extremal strings, one is shown below:
$$C =  1^{6} 0^{10} 1^{1} 0^{1} 1^{10} 0^{11} 1^{11} 0^{10} 1^{1} 0^{1} 1^{10} 0^{10} 1^{1} 0^{1} 1^{10} 0^{11} 1^{11} 0^{6}.$$
This string has $18$ blocks, but only $12$ super blocks of which exactly two are minor.  The cardinalities
of the super blocks are (in this order) 
$$6, 11, 11,11, 11,11, 11,11, 11,11, 11, 6.$$

\begin{thm}[Super Block Upper Bound]\label{SuperBlockUpperBound} If $C$ is a binary string that represents a $2$-coloring with no good progression, 
then every super block of $C$ has cardinality at most $k-1$.
\end{thm}
\begin{proof}
The union of two intersecting monochromatic quasi-progression of $C$ using differences from $\{1,\ldots,k-i+1\}$
is also a monochromatic quasi-progression of $C$ using differences from $\{1,\ldots,k-i+1\}$.
It follows that all of the elements of a super block are contained in a single 
monochromatic quasi-progression of $C$ using differences from $\{1,\ldots,k-i+1\}$.
Because $C$ contains no good quasi-progression, each super block contains fewer than $k$ elements.
\end{proof}

The following theorem lists basic facts about super blocks.

\begin{thm} [Super Block Fundamentals] 
\label{Fundamentals} Suppose $C$ is a binary string that represents a $2$-coloring with no good progression.  
If $C$ has $t$ super blocks $B_1 <\cdots < B_t$, then
\begin{itemize}
\item[(i)] all elements in a super block have the same color (that is, super blocks are monochromatic),
\item[(ii)] $B_1,\ldots,B_t$ form a partition of $C$, 
\item[(iii)] consecutive super blocks have opposite color, 
\item[(iv)] super blocks $B_2,\ldots,B_{t-1}$ contain exactly one major block (in particular they are major super blocks), 
\item[(v)] an extreme element of a super block is adjacent to either the end of the string, a minor super block, or the major block of neighboring
major super block, and
\item[(vi)] a substring of $C$ consisting of all characters between (and including) the extreme elements of a super block contains exactly one major block.
\end{itemize}
\end{thm}
\begin{proof} i) two elements are in relation $\sim$ if they are in a common monochromatic quasi-progression.  Therefore
their color is identical.  ii) the equivalence classes of an equivalence relation form a partition.  iii)-vi)  the boundary of a super
block is reached at the end of the string or at an obstructing major block of the opposite color.  Thus each super block with a neighbor,
must contain a major block that defines the boundary of that neighbor.  So super blocks alternate color.
If a super block contained two major blocks, then its size
would exceed $k-1$, contradicting Theorem \ref{SuperBlockUpperBound}.
\end{proof}

Note that, by (ii), the length of $C$ is $\ell = \sum_{j=1}^t |B_j|$.  The basic approach for an upper bound
on the length of $C$ is based on this partition -- we seek to bound the cardinality of each of the super blocks.
To accomplish this, we need to argue that segments of a long monochromatic quasi-progression can be strung together using fragments
from each super block.  The technique relies on the following two fundamental theorems.

\begin{thm}[Greedy Major Super Block Jumping]\label{GreedyMajorJump} 
Assume $k \ge 2i$.
If $B$ is a major super block of $C$ and $1 \le \delta \le i$, then
$B$ contains a monochromatic quasi-progression $P$ such that 
\begin{itemize}
   \item[i)] $P$ has length at least $\left\lceil \frac{|B|}{\delta} \right\rceil$,
   \item[ii)] the low difference of $P$ is at least $\delta$,
   \item[iii)] the diameter of $P$ is at most $k-i$, and
   \item[iv)] both extremes of $B$ are in $P$.
\end{itemize}
\end{thm}
\begin{proof}
Without loss of generality, $B$ has color $1$.
Let $S$ denote the binary substring of $C$ consisting of all characters between the minimum and maximum
elements of $B$.  By definition, $S$ begins and ends with a $1$.  
Let us suppose that $p \delta < |B| \le (p+1)\delta$, for some
positive integer $p$.
We must show that there is a monochromatic quasi-progression of length at least $p+1$ (that is, a progression that satisfies (i) above) with
the additional
properties (ii)-(iv).  First, create a monochromatic quasi-progression this way:
start with the left-most $1$ of $S$ and repeatedly jump right to the
first available $1$ that is distance at least $\delta$, but no more than $\delta + k-i$ from the last chosen $1$.
Note that there can never be an obstruction
to jumping to the next available $1$ unless we reach the end of $S$ because, if a jump to the 
next $1$ required a length more than $\delta + k-i$, then the last $k-i+1$ skipped elements would be
a major block of $0$'s in $S$ which is impossible by Theorem \ref{Fundamentals} part (vi).
This means that when this greedy jumping reaches the end of $S$, it must lands on a $1$ that is distance
at most $\delta -1$ of the rightmost $1$ of $S$.
Also notice that each jump can pass over at most $\delta-1$ ones.  Thus each $1$ in our constructed
progression ``consumes'' at most $\delta$ ones, itself plus the at most $\delta-1$ ones that are skipped by the next jump.  
But, since there are more than $p \delta$ ones and we have not wasted
any $1$'s because we started at the beginning of $S$, our progression must have at least $p+1$ ones.
The only problem is that this progression may not end at the maximum element of $B$.
We now address this problem.  

Let $x$ denoted the leftmost $1$ of $S$ and $y$ the rightmost $1$ of $S$.
Suppose that our currently constructed progression from $S$ is $x=x_1<\cdots<x_{q}$, for some $q \ge p+1$.  
In a manner similar to the construction of this sequence, construct a new 
progression starting at $y$ and greedily jumping leftward toward $x$.
Suppose that this second progression is $y_h < \cdots < y_2 < y_1$;
that is, this progression begins at the rightmost element $y_1=y$ of $S$,
jumps leftwards greedily until it reaches $y_h$ and no further jumps are possible.
A consequence of the next claim is that the $x$-progression and the $y$-progression
have the same length (i.e. $h=q$).

\vspace{0.25cm}
\noindent
{\sc Claim}: {\bf $y_j - x_{q+1-j} < \delta$, for $j=1,\ldots,q$}.
\begin{quote}
We prove this by induction on $j$.  The basis case is true because $y_1=y$ and as noted in the paragraph above,
the progression of $x$'s must end within $\delta$ of $y$.  Now suppose that $y_{j} - x_{q+1-j} < \delta$, for some $j$.
Because the progression of $y$'s must end within $\delta$ of $x$, if $q+1-j > 1$, the element
$y_{j+1}$ must exist.
The distance $y_{j} - y_{j+1}$ must be at least $\delta$ so $y_{j+1} < x_{q+1-j}$.  In particular, $x_{q-j} \le y_{j+1} < x_{q+1-j}$.
Because the $x$-sequence did not jump from $x_{q-j}$ to $ y_{j+1}$, it follows that $y_{j+1}- x_{q-j} \le \delta - 1$, as desired.
\end{quote}
Observe that if $y_j = x_{q+1-j}$, for some $j \in \{1,\ldots,q\}$ then the
progression $$x_1,\ldots,x_{q+1-j},y_{j-1},\ldots,y_1$$ satisfies the conclusion of the theorem.

So, we have proven that we may assume that these two sequences interlace:
$$
x_1 < y_q < x_2 < y_{q-1} < \cdots < x_{q-1} < y_2 < x_{q} < y_1,
$$
and $y_j - x_{q+1-j} < \delta$, for $j=1,\ldots,q$.
 
Now let $T$ denote the substring of $S$ corresponding to the major block of $1$'s in $B$.  Because $T$ is a major block,
$T$ is a substring of $1$'s with length at least $k-i+1$.  In particular, since $k \ge 2i$, the length of $T$ is
at least $i+1$, which is larger than $\delta$.  Now observe that among the differences between consecutive elements of
the progression $x_1,\ldots,x_q$,
there must be a difference of exactly $\delta$ because the first greedy jump that this progression makes into $T$
must either have length exactly $\delta$ or it hits the first element of $T$. In the latter event,
the following jump must have length $\delta$ because $T$ contains at least $\delta + 1$ ones.

So, there is some $j \in \{1,\ldots,q-1\}$ such that $x_{q+1-j} - x_{q-j} = \delta$.  Because $y_j - x_{q+1-j} < \delta$,
it follows that $y_j - x_{q-j} \le 2\delta \le k - i + \delta$.  Therefore, the progression
$$x_1,\ldots,x_{q-j},y_{j},\ldots,y_1$$ satisfies the conclusion of the theorem.
\end{proof}

\begin{thm}[Greedy Minor Super Block Jumping]\label{GreedyMinorJump} 
Assume $k \ge 2i$.
If $B$ is a minor super block of $C$, $x$ is an extreme of $B$,
 and $1 \le \delta \le i$, then
$B$ contains a monochromatic quasi-progression $P$ such that 
\begin{itemize}
   \item[i)] $P$ has length at least $\left\lceil \frac{|B|}{\delta} \right\rceil$,
   \item[ii)] the low difference of $P$ is at least $\delta$,
   \item[iii)] the diameter of $P$ is at most $k-i$, and
   \item[iv)] $x \in P$.
\end{itemize}
\end{thm}
\begin{proof} We argue essentially the same way as in the proof of Theorem \ref{GreedyMajorJump}.
Without loss of generality, $B$ has color $1$ and $x$ is the leftmost element of $B$ (that is, $x = \min B$).
Let $S$ denote the binary substring of $C$ consisting of all characters between the minimum and maximum
elements of $B$.  By definition, $S$ begins and ends with a $1$. 
Let us suppose that $p \delta < |B| \le (p+1)\delta$, for some
positive integer $p$.
We must show that there is monochromatic quasi-progression of length at least $p+1$ (that is, a progression that satisfies (i) above) with
the additional
properties (ii)-(iv).  Create such a monochromatic quasi-progression this way:
start with $x$ and repeatedly jump right to the
first available $1$ that is distance at least $\delta$, but no more than $\delta + k-i$ from the last chosen $1$.
Note that there can never be an obstruction
to jumping to the next available $1$ unless we reach the end of $S$ because, if a jump to the 
next $1$ required a length more than $\delta + k-i$, then the last $k-i+1$ skipped elements would be
a major block of $0$'s in $S$ which is impossible by Theorem \ref{Fundamentals} part (vi).
This means that when this greedy jumping reaches the end of $S$, it must lands on a $1$ that is distance
at most $\delta -1$ of the right most $1$ of $S$.
Also notice that each jump can pass over at most $\delta-1$ ones.  Thus each $1$ in our constructed
progression ``consumes'' at most $\delta$ ones, itself plus the at most $\delta-1$ ones that are skipped by the next jump.  
But, since there are more than $p \delta$ ones and we have not wasted
any $1$'s because we started at the beginning of $S$, our progression must have at least $p+1$ ones.
\end{proof}

The next theorem brings together the previous two theorems and is a significant tool in later proofs.

\begin{thm} \label{maintool}
Assume $k \ge 2i$.
Suppose $C$ is a binary string that represents a $\{\mbox{red, blue}\}$-coloring of positive integers
with no good progression.  
If $C$ has no red major blocks of cardinality at least $k-i+\delta$, for some $1 \le \delta \le i$, and $B_1,\ldots,B_h$
are the blue super blocks of $C$, then $C$ contains a monochromatic quasi-progression $P$ with
diameter at most $k-i$, low-difference at least $\delta$, and length at least $\sum_{j=1}^h \left\lceil \frac{|B_j|}{\delta} \right\rceil$.
\end{thm}
\begin{proof} Apply Theorem \ref{GreedyMajorJump} to super blocks $B_2,\ldots,B_{h-1}$ to obtain monochromatic
quasi-progression fragments $P_2,\ldots,P_{h-1}$ with length at least $\left\lceil \frac{|B_j|}{\delta} \right\rceil$, low-difference $\delta$, diameter $k-i$ and that
contain both extremes of each of their major super blocks.
Similarly, apply Theorem \ref{GreedyMinorJump} to super blocks $B_1$ and $B_h$ to obtain monochromatic
quasi-progression fragments $P_1$ and $P_h$ with length at least 
$\left\lceil \frac{|B_j|}{\delta} \right\rceil$, low-difference $\delta$, diameter $k-i$ and that
contain the maximum and minimum, respectively, of $B_1$ and $B_h$.  Because $C$ has no red major blocks of cardinality 
at least $k-i+\delta$, jumps of length at most $k-i+\delta$ (and at least $\delta$) can be made to join
the extremes of these fragments into the desired monochromatic quasi-progression $P$.
\end{proof}

\section{Some upper bounds}

This section establishes upper bounds on $Q(k-i,k)$.  
In many cases the bounds are sharp.
The proofs rely heavily on the super block results from the previous section.

\begin{thm} If $k \ge 2i$ and \Mod{k}{0}{i}, then $Q(k-i,k) = 2ik-4i+3$.
\end{thm}
\begin{proof} 
The lower bound $Q(k-i,k) \geq 2ik-4i+3$ follows from Corollary 1 of Landman's paper
\cite{L}; so it suffices to prove the upper bound.
Suppose that $\ell = Q(k-i,k) - 1$ and $C$ is a binary string of length $\ell$ that
represents a $2$-coloring of the integers $1,\ldots,\ell$ with no good progression.
We must prove that $\ell \le 2ik-4i+2$. Assume that $k = m i$, for some $m \ge 2$.

We claim that there can not be $i$ major super blocks of the same color.  To see this,
suppose to the contrary, that $B_1,\ldots,B_p$ ($p \ge i$) 
are all of the blue major super blocks of $C$.  Apply Theorem \ref{maintool} with
$\delta = i$ to the substring of $C$ between the $\min B_1$ and the $\max B_i$.
This theorem guarantees a monochromatic quasi-progression of length at least 
$$\sum_{j=1}^p \left\lceil \frac{|B_j|}{i} \right\rceil \ge \sum_{j=1}^p \left\lceil \frac{k-i+1}{i} \right\rceil \ge \sum_{j=1}^p m \ge mp \ge k,$$
a contradiction.
So $C$ has at most $i-1$ major super blocks of each color.

Suppose that $C$ has $i-1$ major super blocks of the same color.  We now claim that the
total number of elements in minor blocks of that color is at most $k-i$.  To prove this,
suppose that $C$ has $i-1$ blue major super blocks $B_1,\ldots,B_{i-1}$ and two
minor super blocks $B_0$ and $B_{i}$ (it is clear that there are most two blue minor
super blocks, since there is at most one at each end).  Again, apply Theorem \ref{maintool} with
$\delta = i$ to the substring of $C$ between the $\min B_0$ and the $\max B_i$.
This theorem guarantees a monochromatic quasi-progression of length at least
$$\sum_{j=0}^i \left\lceil \frac{|B_j|}{i} \right\rceil = (i-1)m + \left\lceil \frac{|B_0|}{i} \right\rceil + \left\lceil \frac{|B_i|}{i} \right\rceil.$$
Since this sum is at most $k-1 = mi-1$, it follows that $|B_0| + |B_i| \le k - i$.
Consequently, the blue super blocks have cardinalities that sum to at most $(i-1)(k-1) + (k-i)$.
The same argument applies to the red super blocks.
Therefore, the length of $C$ is at most
$$2(i-1)(k-1) + 2(k-i) = 2ik-4i+2,$$
as desired.
\end{proof}

\begin{thm} If $k \ge 2i + 1$ and \Mod{k}{1}{i}, then $Q(k-i,k) = 2ik-2i+1$.
\end{thm}
\begin{proof} 
The lower bound $Q(k-i,k) \geq 2ik-2i+1$ follows from Corollary 1 of Landman's paper
\cite{L}; so it suffices to prove the upper bound.
Let $C$ be binary string realizing an extremal $2$-coloring with no good progression.  We
must prove that the length of $C$ is at most $2i(k-1)$.
Let $B_1,\ldots,B_p$ be the blue super blocks of $C$ and $R_1,\ldots,R_q$ the red super blocks.
Because the super blocks form a partition, the length of $C$ is $\sum_{j=1}^p |B_j| +\sum_{j=1}^q |R_j|$.
However, applying Theorem \ref{maintool} with
$\delta = i$ to the substring of $C$ between the $\min B_1$ and the $\max B_p$, we find that $C$ contains
a monochromatic quasi-progression of length $\sum_{j=0}^p \left\lceil \frac{|B_j|}{i} \right\rceil$.  Since this can not exceed
$k-1$, it follows that $\sum_{j=1}^p |B_j| \le i (k-1)$.  A symmetric argument shows $\sum_{j=1}^q |R_j| \le i (k-1)$.
Therefore the length of $C$ is at most $2i(k-1)$, as desired.
\end{proof}

The next theorem gives a general upper bound that we shall show in the next section
is sharp when
$k=mi+r$ for integers $m,r$ such
that $3 \le r < \frac{i}{2}$ and $r-1 \le m$.

\begin{thm} \label{GeneralUpperBound} If $k \ge 2i$ and $k=mi+r$ for integers $m,r$ such
that $1 < r < i$, then $$Q(k-i,k) \le 2ik-4i+2r-1.$$
\end{thm}
\begin{proof} 
Suppose that $\ell = Q(k-i,k) - 1$ and $C$ is a binary string of length $\ell$ that
represents a $2$-coloring of the integers $1,\ldots,\ell$ with no good progression.
We must prove that $\ell \le 2ik-4i+2r-2$.  We argue by contradiction:
assume that $\ell > 2ik-4i+2r-2$.

For $j=0,1$, let $\alpha_j$ denoted the number of super blocks of $C$ that
have size greater than $mi$. Apply Theorem \ref{maintool} with
$\delta = i$ to the shortest substring containing all super blocks of
color $j$:
$$
\alpha_j (m+1) + \sum_{{\tiny \begin{array}{c} B \mbox{ has color } j \\ |B| \le mi \end{array}}} \left\lceil \frac{|B|}{i} \right\rceil \le k-1.
$$
It follows that, for $j=0,1$,
$$
\sum_{{\tiny \begin{array}{c} B \mbox{ has color } j \\ |B| \le mi \end{array}}} |B| \le i(k - 1 - \alpha_j (m+1)).
$$
Therefore, the length of $C$ can be bounded as follows:
\begin{eqnarray*}
\ell & = & \sum_{j=0}^1 \left( \sum_{{\tiny \begin{array}{c} B \mbox{ has color } j \\ |B| > mi \end{array}}} |B| \right) + \sum_{j=0}^1 \left( \sum_{{\tiny \begin{array}{c} B \mbox{ has color } j \\ |B| \le mi \end{array}}} |B| \right) \\
 & \le & (\alpha_0 + \alpha_1)(k-1) + i(k - 1 - \alpha_0 (m+1)) + i(k - 1 - \alpha_1 (m+1)) \\
 & = & 2ik - 2i - \alpha(i+1 - r),
\end{eqnarray*}
where $\alpha = \alpha_0 + \alpha_1.$
Because we are assuming that $\ell > 2ik-4i+2r-2$, we may conclude that
$\alpha < 2$.
Without loss of generality, $\alpha_0=0$ and $\alpha_1 \le 1$.

Because $\alpha_0=0$, the coloring contains no super blocks of $0$'s with cardinality larger
than $mi$.  Now apply Theorem \ref{maintool} with
$\delta = i-r+1$ to the substring of $C$ containing all super blocks
of color $1$:
$$
\sum_{{\tiny \begin{array}{c} B \mbox{ has color } 1 \end{array}}} \left\lceil \frac{|B|}{i-r+1} \right\rceil \le k-1.
$$
Consequently the number of $1$'s in $C$ is at most $(k-1)(i-r+1)$.  Applying 
Theorem \ref{maintool} with
$\delta = i$ to the substring of $C$ containing all super blocks
of color $0$ we find
$$
\sum_{{\tiny \begin{array}{c} B \mbox{ has color } 0 \end{array}}} \left\lceil \frac{|B|}{i} \right\rceil \le k-1.
$$
Therefore the number of $0$'s in $C$ is at most $(k-1)i$.
It follows that the length of $C$ can be bounded as follows:
\begin{eqnarray*}
\ell & = & \left(\# 1\mbox{'s in }C \right) + \left( \# 0\mbox{'s in }C \right) \\
 & \le & (k-1)(i-r+1) +  (k-1)i\\
 & = & 2ik - 2i +(k-1)(1 - r),
\end{eqnarray*}
which is at most $2ik-4i+2r-2$ since $r > 1$ and $k > 2i$.
This contradicts that $\ell > 2ik-4i+2r-2$. 
\end{proof}

\section*{Section 4: A general lower bound}

In this section we exhibit an extremal $2$-coloring of positive integers that
avoids monochromatic $k$-term quasi-progressions of diameter $k-i$ for many values of $k$ and $i$.
To describe this coloring we first introduce some notation.

Recall that a block of a binary string is a maximal length monochromatic substring.
A {\em segment} of a binary string is a maximal length substring in which all blocks have the same length.
Its segments naturally partition a binary string.
Therefore a binary string $C$ can be abbreviated by an expression involving positive integers of the form
$a_1^{b_1}\cdots a_s^{b_s}$, which indicates that the $j$th segment consists of $b_j$ blocks of length $a_j$
(we assume that the first block is a block of $1$'s).  We adopt this notation in this section.
Note that $C$ has length $\sum_{j=1}^s a_j b_j$. 

\begin{thm} \label{GeneralLowerBound} Suppose that $k=mi+r$ for integers $m,r$ such
that $3 \le r < \frac{i}{2}$.   If $r-1 \le m$, then the following $2$-coloring of the integers from $1$
to $2ik-4i + 2r - 2$ contains no monochromatic $k$-term quasi-progression of diameter $k-i$:
$$
(i(r-2))^1 \ (mi)^{i-1}\ (k-1)^{2}\ (mi)^{i-1}\ (i(r-2))^1.
$$
\end{thm}
\begin{proof} Let $C$ denote the binary string corresponding to this coloring.  We assume that $C$ 
begins with a block of $1$'s.  Observe that $C$ is a palindrome and, because $r-1 \le m$, the initial
block of $1$'s is shorter than the others.  For a positive integer $\delta$,
let $Q_\delta$ be a longest monochromatic quasi-progression
in $C$ with low-difference $\delta$, and let $\ell$ be the length of $Q_\delta$. 
We must show $\ell \le k-1$.
Because $C$ is a palindrome, we may assume that $Q_\delta$ consists
of elements of color $1$.  If $\delta < i - r + 1$, then $Q_\delta$ can not jump across blocks
of length $mi$ or longer, so $Q_\delta$ has length at most $k-1$ in this case. So it suffices to consider
values of $\delta \ge i-r+1$.  However, because $C$ has only interior blocks of length $mi$ or $k-1$, we can
restrict our attention of $\delta = i - r + 1$ (which permits leaps over blocks size $mi$ but not $k-1$)
or $\delta = i$ (which permits leaps over all block sizes).  Upper bounds on $\ell$ for other values
of $\delta$ follow immediately from the upper bounds on these two values of $\delta$.

Assume first that $\delta = i$.  The quasi-progression 
$Q_\delta$ can use at most $\left\lceil \frac{|B|}{i} \right\rceil$ elements from a block $B$.
Consequently, the length $\ell$ can be bounded this way:
\begin{eqnarray*}
\ell & \le & \left\lceil \frac{i(r-2)}{i} \right\rceil + (i-1) \left\lceil \frac{mi}{i} \right\rceil + \left\lceil \frac{k-1}{i} \right\rceil \\
\ & = & (r - 2) + (i-1)m + (m + 1) \\
\ & = & k - 1,
\end{eqnarray*}
as desired.

Assume now that $\delta = i - r + 1$.  Clearly the quasi-progression 
$Q_\delta$ can use at most $\left\lceil \frac{|B|}{ i - r + 1} \right\rceil$ elements from a block $B$.
There are two cases according to the parity of $i$. In each case the length of $Q_\delta$ is bounded from above:

\vspace{0.25cm}
\noindent
{\sc Case 1: $i$ is even}.  In this case the extremal length occurs when
$Q_\delta$ uses $1$'s from the small beginning block,
$(i-2)/2$ intermediate blocks of size $mi$, and the large block of $k-1$ ones, so
\begin{eqnarray} \label{Case1Bound}
\ell & \le & \left\lceil \frac{i(r-2)}{i - r + 1} \right\rceil + \frac{i-2}{2} \left\lceil \frac{mi}{i - r + 1} \right\rceil + \left\lceil \frac{k-1}{i - r + 1} \right\rceil. 
\end{eqnarray}

\vspace{0.25cm}
\noindent
{\sc Case 2: $i$ is odd}.  In this case the extremal length occurs when
$Q_\delta$ uses $1$'s from
$(i-1)/2$ intermediate blocks of size $mi$, and the large block of $k-1$ ones, so
\begin{eqnarray} \label{Case2Bound}
\ell & \le & \frac{i-1}{2} \left\lceil \frac{mi}{i - r + 1} \right\rceil + \left\lceil \frac{k-1}{i - r + 1} \right\rceil. 
\end{eqnarray}

Consider now the following upper bounds which, when substituted
into inequalities (\ref{Case1Bound}) and (\ref{Case2Bound}), will show that $\ell \le k-1$.

\vspace{0.25cm}
\noindent
{\sc Bound 1: $\left\lceil \frac{mi}{i - r + 1} \right\rceil \le \min\{2m - 1,2m + \frac{2(r-1-m)}{i-1}\}$}.
\begin{quote}
   It suffices to prove that $\frac{mi}{i - r + 1} \le 2m - 1 + \frac{2(r-1-m)}{i-1}$.
   Assume the contrary: $\frac{mi}{i - r + 1} > 2m - 1 + \frac{2(r-1-m)}{i-1}$. Applying the assumption
   that $2r + 1 \le i$ while solving for $m$, we find that $$\frac{(i-r+1)(i-2r+1)}{i(i-2r-1)+4(r-1)} > m.$$
   Because we have assumed that $m \ge r-1$, it follows that
   $$\frac{(i-r+1)(i-2r+1)}{i(i-2r-1)+4(r-1)} > r-1.$$  Consequently,
   \begin{eqnarray*}
   0 & > & (r-1)\left[i(i-2r-1)+4(r-1)\right] - (i-r+1)(i-2r+1) \\
     & = & (r-2)(i-1)\left(i - \frac{2r^2 - 5r + 3}{r-2}\right)
   \end{eqnarray*}
   Because $r > 2$ and $i > 1$, we conclude that $i < \frac{2r^2 - 5r + 3}{r-2}$.
   However, since $i > 2r$ we find that
   $$
   2r < \frac{2r^2 - 5r + 3}{r-2}
   $$
   which implies that $r < 3$, a contradiction.
\end{quote}

\vspace{0.25cm}
\noindent
{\sc Bound 2: $\left\lceil \frac{k-1}{i - r + 1} \right\rceil \le 2m$}.
\begin{quote}
   Assume, to the contrary, that $\frac{k-1}{i - r + 1} > 2m$.  Substituting $k=mi+r$ and solving
   for $m$ produces the inequality $\frac{r-1}{i-2r+2} > m$.
   Because we have assumed that $m \ge r-1$, it follows that
   $\frac{r-1}{i-2r+2} > r-1$
   which implies that $i < 2r-1$, a contradiction.  
\end{quote}

\vspace{0.25cm}
\noindent
{\sc Bound 3: $\left\lceil \frac{i(r-2)}{i - r + 1} \right\rceil \le \frac{i}{2} + r - 2$}.
\begin{quote}
   It suffices to prove that  $\frac{i(r-2)}{i - r + 1} \le \frac{i}{2} + r - 3$. 
   Assume, to the contrary, that $ \frac{i(r-2)}{i - r + 1} > \frac{i}{2} + r - 3$.
   Equivalently,
   $$
   0  >  2\left(\frac{i}{2}+ r- 3\right)(i-r+1) - 2i (r-2) 
      =  (i+r-3)(i-2r+2),
   $$
   a contradiction.
\end{quote}
To complete the proof we show, using these bounds applied to inequalities (\ref{Case1Bound}) and (\ref{Case2Bound}),
that $\ell \le k-1$.  Consider inequality (\ref{Case1Bound}):
  \begin{eqnarray*}
   \ell  & \le & \left\lceil \frac{i(r-2)}{i - r + 1} \right\rceil + \frac{i-2}{2} \left\lceil \frac{mi}{i - r + 1} \right\rceil + \left\lceil \frac{k-1}{i - r + 1} \right\rceil \\
   & \le & \left(\frac{i}{2} + r - 2\right) + \left( \left( \frac{i-2}{2}\right) \left( 2m-1 \right)\right) + \left( 2m \right)\\
   & = & k-1
   \end{eqnarray*}
Now consider inequality (\ref{Case2Bound}):
  \begin{eqnarray*}
   \ell  & \le & \frac{i-1}{2} \left\lceil \frac{mi}{i - r + 1} \right\rceil + \left\lceil \frac{k-1}{i - r + 1} \right\rceil \\
   & \le &  \left( \left( \frac{i-1}{2}\right) \left( 2m + \frac{2(r-1-m)}{i-1} \right)\right) + \left( 2m \right)\\
   & = & k-1
   \end{eqnarray*}
\end{proof}

\begin{thm} \label{SharpGeneralBound} If $k=mi+r$ for integers $m,r$ such
that $3 \le r < \frac{i}{2}$ and $r-1 \le m$, then $$Q(k-i,k) = 2ik-4i+2r-1.$$
\end{thm}
\begin{proof}  Theorem \ref{GeneralUpperBound} proves the upper bound and 
Theorem \ref{GeneralLowerBound} proves the lower bound.
\end{proof}

\begin{thm} If $k \ge 2i$ and \Mod{k}{2}{i}, then $Q(k-i,k) = 2ik-4i+3$.
\end{thm}
\begin{proof} Theorem \ref{GeneralUpperBound} gives the upper bound $Q(k-i,k) \le 2ik-4i+3$.
Arguing in a manner similar to the proof of Theorem \ref{GeneralLowerBound}, one can show that
$$
(k-2)^{i-1}\ (k-1)^{2}\ (k-2)^{i-1}.
$$
is a $2$-coloring of $1,\ldots, 2ik-4i+2$ that avoids monochromatic $k$-term quasi-progressions of diameter $k-i$.
\end{proof}  

\section*{Acknowledgements}
We thank Bruce Landman for bringing our attention to Vijay's paper \cite{V}.

\begin{thebibliography}{9} 
 
\bibitem{B}
   \textsc{Brown, T.C.}, \textsc{Erd\H{o}s, P.} and \textsc{Freedman, A. R.} (1990).
   Quasi-progressions and descending waves \textit{J. Comb. Theory Ser. A}
   \textbf{53} 81--95.
   \MR{1031614}

\bibitem{G}
   \textsc{Graham, R.}  (2007).
   Some of my favorite problems in Ramsey theory
   \textit{Combinatorial number theory} pp. 229�-236
   de Gruyter, Berlin.
   \MR{2337049}
   
\bibitem{L} \textsc{Landman, Bruce M.} (1998).
   Ramsey functions for quasi-progressions
   \textit{Graphs Combin.} \textbf{14} no. 2, 131--142
   \MR{1628121}

\bibitem{LR}
   \textsc{Landman, Bruce M.} and \textsc{Robertson, Aaron} (2004). 
   \textit{Ramsey theory on the integers}  
   Student Mathematical Library, \textbf{24}. American Mathematical Society, Providence, RI.
   \MR{2020361}
   
\bibitem{V}
   \textsc{Vijay, Sujith} (2010).
   On a variant of Van der Waerden's Theorem
   \textit{Integers} \textbf{10} 223-227

\end{thebibliography}

\end{document}
