%\RequirePackage[l2tabu, orthodox]{nag}
\RequirePackage{xkvltxp} % needed if passing in a key with space in it
%\documentclass[fancyproof,fontloader=unicode-math,mathfont=XITS Math]{ul-thesis}
%\documentclass[debug,fancyproof,textfont=Linux Libertine O]{ul-thesis}
%\documentclass[final]{ul-thesis}
\documentclass{ul-thesis}

\usepackage{custommacros}
\usetikzlibrary{shapes.misc}

\title{Avoiding, Pretending, and Querying: Three Combinatorial Problems}
\author{Adam S. Jobson}
\degrees{B.S., University of Louisville, 2004\\ M.A., University of Louisville, 2006}
\graduationdate{December 2011}
\defensedate{18 November 2011}
\email{ajobson@gmail.com}
\phone{502-608-1521}
\address{3917 Druid Hills Road}{Louisville, KY 40207}

\usepackage{array}
\newcolumntype{x}[1]{>{\raggedleft\arraybackslash\hspace{0pt}$}p{#1}<{$}}%
\newcolumntype{w}[1]{>{\centering\arraybackslash\hspace{0pt}$}p{#1}<{$}}%

\newcommand*{\mbbRn}[1][n]{\ensuremath{\mathbb{R}^{#1}}}
\DeclareMathOperator{\Aut}{Aut}
\DeclareMathOperator{\size}{size}
\DeclareMathOperator{\rank}{rank}
\DeclareMathOperator{\Row}{Row}
\DeclareMathOperator{\Nul}{Nul}
\DeclareMathOperator{\diam}{diam}
\DeclareMathOperator{\Spanop}{Span}
\DeclareMathOperator{\bigO}{O}
\newcommand*{\Span}[1]{\ensuremath{\Spanop\left\{#1\right\}}}
\newcommand*{\adjm}[1]{\ensuremath{A\left(#1\right)}}%
\newcommand*{\indegree}[1]{\ensuremath{d^{-}\left(#1\right)}}
\newcommand*{\outdegree}[1]{\ensuremath{d^{+}\left(#1\right)}}

\usepackage{esvect}
\def\bsrmnxt#1#2{\boldsymbol{\mathrm{#1}}#2}
\newcommand*{\vect}[1]{\ensuremath{\vv{\bsrmnxt#1}}}

\newcommand*{\vectent}[1]{\ensuremath{\boldsymbol{\mathrm{#1}}}}
\newcommand*{\nvect}[1]{\ensuremath{N\vv{\left(#1\right)}}}%


\newcommand*{\zerovect}{\ensuremath{\vv{\text{\bfseries 0}}}}
\newcommand*{\onevect}{\ensuremath{\vv{\text{\bfseries 1}}}}

\newcommand*{\defn}[1]{{\itshape #1}}
\def\AJbf#1{\text{\bfseries{#1}}}%

\newcommand*{\extendbottombound}{\node[vlab,anchor=north] at (current bounding box.south){\vphantom{X}};}
\newcommand*{\extendtopbound}{\node[vlab,anchor=south] at (current bounding box.north){\vphantom{X}};}

\usepackage[section]{placeins}

\addbibresource{Introduction/Introduction.bib}
\addbibresource{QP/QP.bib}
\addbibresource{Adjacency/Adjacency.bib}
\addbibresource{Yao/Yao.bib}
%\addbibresource{Conclusion/Conclusion.bib}

\begin{document}
\frontmatter
\maketitle

\begin{copyrightinfo}%Copyright Page
	Copyright 2011 by Adam S. Jobson

	\ccIconSmall{BY-SA}

	This work is licensed under the Creative Commons Attribution-ShareAlike 3.0 Unported License. To view a copy of this license, visit 
\\\url{http://creativecommons.org/licenses/by-sa/3.0/}\\ or send a letter to Creative Commons, 444 Castro Street, Suite 900, Mountain View, California, 94041, USA.
\end{copyrightinfo}

\signaturepage
% optional arguments: [number of blanks][first name][second name]...
% examples \signaturepage[5][Andr\'e K\'ezdy]
%	\signaturepage[5][Andr\'e K\'ezdy][Csaba Biro][Manav Das][Jafar Hadizadeh][Thomas Riedel]

\begin{dedication}
	To my parents.
\end{dedication}

\begin{acknowledgements}
	I wish to thank my advisor, Professor Andr\'e K\'ezdy, for his patience and encouragement. I have learned more from him than I can ever repay.

	To my family, for their unwavering belief in me.

	To Marshall Lagani, Ben Allgeier, Tim Brauch, and Lesley Wellington Wiglesworth, for countless hours of collaboration, both in mathematics and in life.

	To Professors Udayan Darji and Manav Das, for introducing me to higher mathematics and encouraging me to pursue graduate work.

	To Professor Thomas Riedel and everybody in the University of Louisville Department of Mathematics, for their years of support. 

	To Jim Conrey, for introducing me to programming. I cannot imagine where I would be without it.

	To Donald Knuth for creating \TeX, Leslie Lamport for creating \LaTeX, Till Tantau for creating Ti\emph{k}Z, and the thousands of others who have contributed to these projects. Mathematics is even more beautiful thanks to their efforts.
\end{acknowledgements}

\begin{abstract}
	A $k$-term quasi-progression of diameter $d$ is a sequence $\left\{x_1, \ldots, x_k\right\}$ for which there exists a positive integer $l$ such that $l \leq x_{i} - x_{i-1} \leq l+d$, for all $i=2,\ldots,k$. %A quasi-progression with diameter $0$ is simply an arithmetic progression so, in general, 
Quasi-progressions may be thought of as arithmetic progressions with a certain amount of `wiggle-room' allowed. Let $Q\left(d,k\right)$ be the least positive integer such that every $2$-coloring of $\left\{1,\ldots,Q\left(d,k\right)\right\}$ contains a monochromatic $k$-term quasi-progression of diameter $d$.
% Landman [Ramsey Functions for Quasi-Progressions, Graphs and Combinatorics 14 (1998) 131-142] established several results and conjectured that, if $k\geq 2i \geq 1$, then \[ Q_{k-i}\left(k\right) =
%	\begin{cases}
%		2ik-4i+3	&k\not\equiv 1 \pmod{i}\\
%		2ik-2i+1	&k \equiv 1 \pmod{i}
%	\end{cases}\]
We prove a conjecture of Landman for certain values of $k$ and $d$, provide counterexamples for some other cases, and determine that the conjecture always has the correct order of growth.% This is joint work with Andr\'e K\'ezdy, Hunter Snevily, and Susan White.

%\begin{center}\large Pretending: The Row Space of an Adjacency Matrix\end{center}
Let $A$ be the adjacency matrix of a nonempty graph. Is there always a nonzero $\left\{0,1\right\}$-vector in the row space of $A$ that is not a row of $A$? 
%We call such a vector a \emph{hood vector} for the graph, since the vector `looks like' a neighborhood of a vertex in the graph. 
Akbari, Cameron, and Khosrovshahi have shown that an affirmative answer to this question would imply bounds on many graph parameters as a function of the rank of the adjacency matrix. We demonstrate the existence of such vectors for certain families of graphs, examine techniques to find and verify the existence of such vectors, and show that if you generalize the problem to allow asymmetry in the matrices then some $\left\{0,1\right\}$-matrices do not have such vectors.

%\begin{center}\large Querying: Yao's Problem\end{center}
In 1981, Andrew Yao asked ``Should tables be sorted?''. % and provided an answer under some conditions. 
When the table has $n$ cells that are filled with entries taken from a \emph{key space} of $m$ possibilities, he showed that it is possible to decide whether any member of the key space is present in the table by inspecting (querying) only one cell of the table if and only if $m \leq 2n-2$. We make steps toward extending his result to the case where you are permitted two queries by considering several variations of the problem.
\end{abstract}

\tableofcontents*
%\listoftables
%\listoffigures

\mainmatter

%\input{Introduction/Introduction}
	\newglossaryentry{degree}{%
		name={\ensuremath{d_G\left(v\right)}, \ensuremath{d\left(v\right)}},%
		description={the degree of $v$ in $G$},%
		sort={d(v)}%
	}%
	\newglossaryentry{mindegree}{%
		name={\ensuremath{\delta\left(G\right)}},%
		description={the minimum degree of $G$},%
		sort={delta(G)}%
	}%
	\newglossaryentry{maxdegree}{%
		name={\ensuremath{\Delta\left(G\right)}},%
		description={the maximum degree of $G$},%
		sort={Delta(G)}%
	}%
	\newglossaryentry{outdegree}{%
		name={\ensuremath{d^+_G\left(v\right)}, \ensuremath{d^+\left(v\right)}},%
		description={the outdegree of $v$ in $G$},%
		sort={d+(v)}%
	}%
	\newglossaryentry{indegree}{%
		name={\ensuremath{d^-_G\left(v\right)}, \ensuremath{d^-\left(v\right)}},%
		description={the indegree of $v$ in $G$},%
		sort={d-(v)}%
	}%
	\newglossaryentry{AutG}{%
		name={\ensuremath{\Aut G}},%
		description={automorphism group of $G$},%
		sort=Aut%
	}%
	\newglossaryentry{cliquenumber}{%
		name={\ensuremath{\omega\left(G\right)}},%
		description={clique number of $G$},%
		sort=omega%
	}%
	\newglossaryentry{Kn}{%
		name={\ensuremath{K_{n}}},%
		description={complete graph on $n$ vertices},%
		sort=Kn%
	}%
	\newglossaryentry{Kmn}{%
		name={\ensuremath{K_{m,n}}},%
		description={complete bipartite graph},%
		sort=Knn%
	}%
	\newglossaryentry{Pn}{%
		name={\ensuremath{P_{n}}},%
		description={path on $n$ vertices},%
		sort=Pn%
	}%
	\newglossaryentry{duv}{%
		name={\ensuremath{d_{G}\left(u,v\right)}, \ensuremath{d\left(u,v\right)}},%
		description={distance from $u$ to $v$ in $G$},%
		sort={d(uv)}%
	}%
	\newglossaryentry{graphdiameter}{%
		name={\ensuremath{\diam G}},%
		description={diameter of $G$},%
		sort=diam%
	}%
	\newglossaryentry{opennbd}{%
		name={\ensuremath{N_G\left(v\right)}, \ensuremath{N\left(v\right)}},%
		description={the (open) neighborhood of $v$ in $G$},%
		sort={N(v)}%
	}%
	\newglossaryentry{closednbd}{%
		name={\ensuremath{N_G\left[v\right]}, \ensuremath{N\left[v\right]}},%
		description={the closed neighborhood of $v$ in $G$},%
		sort={N[v]}%
	}%
	\newglossaryentry{dominationnumber}{%
		name={\ensuremath{\gamma\left(G\right)}},%
		description={domination number of $G$},%
		sort=gamma%
	}%
	\newglossaryentry{totaldominationnumber}{%
		name={\ensuremath{\gamma_t\left(G\right)}},%
		description={total domination number of $G$},%
		sort=gammat%
	}%
	\newglossaryentry{chromaticnumber}{%
		name={\ensuremath{\chi\left(G\right)}},%
		description={chromatic number of $G$},%
		sort=chi%
	}%
%!TEX root=../Dissertation.tex
\chapter{AVOIDING: COLORING QUASI-PROGRESSIONS}

Several\footnote{This chapter contains joint work with Andr\'e K\'ezdy, Hunter Snevily, and Susan White, and has been submitted to a journal for publication.} renowned open conjectures in combinatorics and number theory involve arithmetic progressions. Van der Waerden famously proved in 1927 that for each positive integer $k$ there exists a least positive integer \index{van der Waerden number}$w(k)$ such that any $2$-coloring of $1,\ldots,w(k)$ produces a monochromatic $k$-term arithmetic progression. The best known upper bound for $w(k)$ is due to Gowers and is quite large. Ron Graham \cite{G} conjectures $w(k) \leq 2^{k^2}$, for all $k$. \comments{The best known lower bound is due to Berlekamp: $w(p+1) \geq p2^p$, for primes $p$. Another conjecture involving arithmetic progressions is one of Erd\H{o}s' most famous (still open) conjectures: if the sum of the reciprocals of the members of a set of positive integers $S$ diverges, then $S$ contains arbitrarily long arithmetic progressions. Many researchers have worked on this or special cases of this conjecture, including the recent and now famous theorem by Green and Tao which proves that the primes contain arbitrarily long arithmetic progressions.}

\section{Introduction}
Brown, Erd\H{o}s, and Freedman \cite{B} introduced quasi-progressions as a way of generalizing arithmetic progressions: 
\begin{definition}[Quasi-progression]\index{Quasi-progression}A $k$-term quasi-progression (QP) of diameter $d$\index{diameter!of a quasi-progression} is a sequence $\left\{x_1, \ldots, x_k\right\}$ for which there exists a positive integer $l$ such that $l \leq x_{i} - x_{i-1} \leq l+d$, for all $i=2,\ldots,k$.

$l$ is called the \defn{low-difference}\index{low-difference} of the QP.
\end{definition}

Arithmetic progressions are quasi-progressions with diameter zero.\comments{ Brown et al. consider the question of when a set of positive integers contains arbitrarily long quasi-progressions of a given diameter. A main result of theirs is that quasi-progressions behave similarly to arithmetic progressions, at least with respect to Erd\H{o}s' conjecture, in that Erd\H{o}s' conjecture is equivalent to the statement: if the sum of the reciprocals of the members of a set of positive integers $S$ diverges, then there exists a diameter $d$ such that $S$ contains arbitrarily long quasi-progressions of diameter $d$.
}
Analogous to the van~der Waerden function $w(k)$, Landman \cite{L,LR} introduced a Ramsey function for quasi-progressions.
\begin{definition}\index{Q(d,k)@$Q(d,k)$} $Q(d,k)$ is the least positive integer such that every two-coloring of the set $\left\{1, \ldots, Q(d,k)\right\}$ contains a monochromatic $k$-term quasi-progression of diameter $d$.
\end{definition}

This function produces a lower bound for $w(k)$ since \[w(k) = Q(0,k) \geq Q(1,k) \geq Q(2,k) \geq Q(3,k) \geq \cdots\] So, it is of great interest to find bounds on $Q(d,k)$ for various $d$, especially small values of $d$. Of particular interest is the rate of growth of $Q(1,k)$. Is it merely polynomial or is it at least exponential in $k$?  Vijay \cite{V}\footnote{We thank Bruce Landman for bringing our attention to Vijay's paper.} has recently established an exponential lower bound for $Q(1,k)$, so quasi-progressions of small diameter behave similarly to arithmetic progressions, at least with respect to these Ramsey functions. An interesting open problem is to determine the largest diameter $d$ for which $Q(d,k)$ is at least exponential (in $k$).

Landman established several bounds on $Q(d,k)$ and made several conjectures which we resolve in this paper. Our results, like Landman's, focus on large diameter; that is, $d = k-i$, for some positive integer $i$ satisfying $k\geq 2i \geq 1$. The main difficulty is the upper bound on $Q(k-i,k)$. Specifically, how can we tailor an argument that handles the large number of extremal $2$-colorings which seem to defy uniform description (cf. Landman's data at the end of his paper)? In \autoref{sec:superblocks} we introduce superblocks, an equivalence relation that imposes sufficient structure on extremal $2$-colorings to extract long monochromatic quasi-progression fragments via a greedy strategy. Through the superblock lens, the extremal $2$-colorings coalesce. The superblock argument is used in \autoref{sec:QP upper bounds} where we show how to splice monochromatic fragments together to produce long monochromatic quasi-progressions. This yields an upper bound on $Q(k-i,k)$; the bound is often sharp. One consequence is that, if $k \geq 2i$ and $k=mi+r$ for integers $m,r$ such that $1 < r < i$, then \[Q(k-i,k) \leq 2ik-4i+2r-1.\] This improves bounds given by Landman.

Our main result is proven is \autoref{sec:QP lower bound}: \[ Q(k-i,k) = 2ik-4i+2r-1,\] if $k=mi+r$ for integers $m,r$ such that $3 \leq r < \frac{i}{2}$ and $r-1 \leq m$. This disproves Conjecture 1 of Landman's paper for these values of $r$. Residues $r\geq i/2$ that are not considered in this result appear to be more difficult. The techniques here are inadequate to resolve those values of $Q(k-i,k)$. However, we also prove that, if $k\geq 2i \geq 1$, then 
\[ Q\left(k-i,k\right) =
	\begin{cases}
		2ik-4i+3	&k \equiv 0\text{ or }2 \pmod{i}\\
		2ik-2i+1	&k \equiv 1 \pmod{i}
	\end{cases}
\] thus proving Landman's conjecture in these cases.

\section{Superblocks\label{sec:superblocks}}

In this section we develop notation, concepts and tools to describe the structure of extremal strings which avoid long monochromatic quasi-progressions.

A \defn{$k$-term progression} is an increasing sequence of $k$ positive integers $x_1 < x_2 < \cdots < x_k$. Given a $k$-term progression $P=\{x_j\}_{j=1}^k$, the \defn{differences} in $P$ are the elements in the set  \[ D(P) = \{ x_j - x_{j-1} : j=2,\ldots,k\}.\] The \defn{low-difference}\index{low-difference} of $P$, denoted $\delta(P)$ (or simply $\delta$), is the minimum element in $D(P)$;  the \defn{high-difference} of $P$, denoted $\Delta(P)$ (or simply $\Delta$), is the maximum. The diameter of $P$ is $d = \Delta - \delta$. Observe that arithmetic progressions are quasi-progressions with diameter $d = 0$.

Consider now $2$-colorings of the positive integers. A quasi-progression is a \defn{good progression} if it is a monochromatic quasi-progression with length $k$ and diameter at most $d$. Define $Q(d,k)$\index{Q(d,k)@$Q(d,k)$} to be the least positive integer such that every $2$-coloring of $\{1,\ldots,Q(d,k)\}$ contains a good progression. Motivated by conjectures of Landman \cite{L}, we consider $Q(d,k)$ for values of $d$ and $k$ satisfying $d = k - i$ and $k \geq 2i$, where $i$ is some fixed positive integer. So, for the rest of this paper $d = k-i$ and a \defn{good progression} means a monochromatic quasi-progression with length $k$ and diameter at most $k-i$. To understand the structure of extremal $2$-colorings avoiding good progressions, we now introduce two important substructures: blocks and superblocks.

Let $C=c_1c_2\ldots c_\ell$ be a binary string of length $\ell$. A \defn{substring} of $C$ is a string of the form $c_p c_{p+1}\cdots c_q$, for some positive integers $1 \leq p \leq q \leq \ell$. A \index{block}\defn{block} of $C$ is a maximal monochromatic substring of $C$. We employ the usual shorthand notation in which, for $x \in \{0,1\}$, the shorthand $x^n$ represents the string $\underbrace{x \cdots x}_{n}$. There is a natural partition of $C$ into blocks: without loss of generality, the first block of $C$ is a block of $1$s so \[C = 1^{\alpha_1}\ 0^{\alpha_2}\ \cdots\ 1^{\alpha_{b-1}}\ 0^{\alpha_b},\] where the $\alpha_j$ are positive integers, except possibly $\alpha_b$ which may be zero (in which case the final block of $C$ is actually a block of $1$s). Note that $\sum_{j=1}^b \alpha_j = \ell$.

Now consider an extremal coloring $C$; that is, suppose that $\ell = Q(k-i,k) - 1$ and $C$ represents a $2$-coloring of the integers $1,\ldots,\ell$ with no good progression. For convenience, blocks of $C$ of length at most $k-i$ are \index{minor block}\index{block!minor}\defn{minor} blocks; longer blocks are \index{major block}\index{block!major}\defn{major}. There are two important facts that motivate this dichotomy:
\begin{enumerate}
	\item\label{en:QP observation 1} Quasi-progressions with low-difference $1$ and diameter $k-i$ can jump over any intermediate minor blocks.
	\item\label{en:QP observation 2} ``Greedy monochromatic jumping'' (in which jumps of length at least $\delta$ but at most $\delta + k -i$ are taken, for some choice of $\delta$) can not get stuck in substrings that avoid major blocks of one color (see later \autoref{thm:GreedyMajorJump} and \autoref{thm:GreedyMinorJump}).
\end{enumerate}

A consequence of \altref[observation]{en:QP observation 1} is that, in a substring in which only minor blocks of one color appear, all of the integers with the other color in this substring form a monochromatic progression with low-difference $1$ and high-difference $k-i+1$ (that is, a monochromatic quasi-progression with diameter $k-i$).

Now we turn to the task of identifying monochromatic substructures of $C$ with the property that a greedy strategy can guarantee dense monochromatic progressions beginning and ending at endpoints of the substructure. To make this precise, we define the equivalence relation $\sim$ on the integers $1,\ldots,\ell$ so that $x \sim y$ if and only if $x$ and $y$ are contained in a monochromatic quasi-progression $P$ of $C$ such that $D(P) \subseteq \{1,\ldots,k-i+1\}$ (the transitivity of $\sim$ follows from the fact that the union of two intersecting monochromatic quasi-progressions with differences in $\{1,\ldots,k-i+1\}$ is another such quasi-progression). The equivalence classes under $\sim$ are called \defn{\textbf{superblocks}}\index{superblock}. Superblocks are not necessarily substrings. Suppose that $C$ has $t$ superblocks $B_1,\ldots,B_t$. We naturally order superblocks this way: $B_p < B_q$ if and only if $\min B_p < \min B_q$, where $\min B_p$ denotes the smallest integer in $B_p$ (that is, the left-most one). A superblock is \index{major superblock}\index{superblock!major}\defn{major} if it contains all of the elements from a major block of $C$; otherwise it is \index{minor superblock}\index{superblock!minor}\defn{minor}. The \index{superblock!extremes}\defn{extremes} of a superblock are its minimum and maximum elements.

\begin{example} Consider $k = 12, i = 6$. Because $Q(6,12) = 123$, an extremal string in this case has length $122$. There are several extremal strings, one is shown below: \[C =  1^{6}\ 0^{10}\ 1^{1}\ 0^{1}\ 1^{10}\ 0^{11}\ 1^{11}\ 0^{10}\ 1^{1}\ 0^{1}\ 1^{10}\ 0^{10}\ 1^{1}\ 0^{1}\ 1^{10}\ 0^{11}\ 1^{11}\ 0^{6}\] This string has $18$ blocks, but only $12$ superblocks of which exactly two are minor. The cardinalities of the superblocks are (in this order) \[6, 11, 11,11, 11,11, 11,11, 11,11, 11, 6.\]
\end{example}

\begin{theorem}[Superblock Upper Bound]\label{thm:SuperBlockUpperBound} If $C$ is a $2$-coloring with no good progression, then every superblock of $C$ has cardinality at most $k-1$.
\end{theorem}
\begin{proof}
The union of two intersecting monochromatic quasi-progression of $C$ using differences from $\{1,\ldots,k-i+1\}$ is also a monochromatic quasi-progression of $C$ using differences from $\{1,\ldots,k-i+1\}$. It follows that all of the elements of a superblock are contained in a single monochromatic quasi-progression of $C$ using differences from $\{1,\ldots,k-i+1\}$. Because $C$ contains no good quasi-progression, each superblock contains fewer than $k$ elements.
\end{proof}

The following theorem lists basic facts about superblocks.

\begin{theorem}[Superblock Fundamentals]\label{thm:Fundamentals} Suppose $C$ is a binary string that represents a $2$-coloring with no good progression. If $C$ has $t$ superblocks $B_1 <\cdots < B_t$, then
\begin{enumerate}
\item\label{en:SBfund1} all elements in a superblock have the same color (that is, superblocks are monochromatic),
\item\label{en:SBfund2} $B_1,\ldots,B_t$ form a partition of $C$, 
\item\label{en:SBfund3} consecutive superblocks have opposite color, 
\item\label{en:SBfund4} superblocks $B_2,\ldots,B_{t-1}$ contain exactly one major block (in particular they are major superblocks), 
\item\label{en:SBfund5} an extreme element of a superblock is adjacent to either the end of the string, a minor superblock, or the major block of neighboring major superblock, and
\item\label{en:SBfund6} a substring of $C$ consisting of all characters between (and including) the extreme elements of a superblock contains exactly one major block.
\end{enumerate}
\end{theorem}
\begin{proof} \ref{en:SBfund1} two elements are in relation $\sim$ if they are in a common monochromatic quasi-progression. Therefore their color is identical. \ref{en:SBfund2} the equivalence classes of an equivalence relation form a partition. \ref{en:SBfund3}--\ref{en:SBfund6} the boundary of a superblock is reached at the end of the string or at an obstructing major block of the opposite color. Thus each superblock with a neighbor, must contain a major block that defines the boundary of that neighbor. So superblocks alternate color. If a superblock contained two major blocks, then its size would exceed $k-1$, contradicting \autoref{thm:SuperBlockUpperBound}.
\end{proof}

Note that, by \ref{en:SBfund2}, the length of $C$ is $\ell = \sum_{j=1}^t |B_j|$. The basic approach for an upper bound on the length of $C$ is based on this partition -- we seek to bound the cardinality of each of the superblocks. To accomplish this, we need to argue that segments of a long monochromatic quasi-progression can be strung together using fragments from each superblock. The technique relies on the following two fundamental theorems.

\begin{theorem}[Greedy Major Superblock Jumping]\label{thm:GreedyMajorJump} Assume $k \geq 2i$. If $B$ is a major superblock of $C$ and $1 \leq \delta \leq i$, then $B$ contains a monochromatic quasi-progression $P$ such that 
\begin{enumerate}
   \item\label{en:GreedyMaSBJ1} $P$ has length at least $\left\lceil \frac{|B|}{\delta} \right\rceil$,
   \item\label{en:GreedyMaSBJ2} the low difference of $P$ is at least $\delta$,
   \item\label{en:GreedyMaSBJ3} the diameter of $P$ is at most $k-i$, and
   \item\label{en:GreedyMaSBJ4} both extremes of $B$ are in $P$.
\end{enumerate}
\end{theorem}
\begin{proof}
Without loss of generality, $B$ has color $1$. Let $S$ denote the binary substring of $C$ consisting of all characters between the minimum and maximum elements of $B$. By definition, $S$ begins and ends with a $1$. Let us suppose that $p \delta < |B| \leq (p+1)\delta$, for some positive integer $p$. We must show that there is a monochromatic quasi-progression of length at least $p+1$ (that is, a progression that satisfies \ref{en:GreedyMaSBJ1} above) with the additional properties \ref{en:GreedyMaSBJ2}--\ref{en:GreedyMaSBJ4}. First, create a monochromatic quasi-progression this way: start with the left-most $1$ of $S$ and repeatedly jump right to the first available $1$ that is distance at least $\delta$, but no more than $\delta + k-i$ from the last chosen $1$. Note that there can never be an obstruction to jumping to the next available $1$ unless we reach the end of $S$ because, if a jump to the next $1$ required a length more than $\delta + k-i$, then the last $k-i+1$ skipped elements would be a major block of $0$s in $S$ which is impossible by \autoref{thm:Fundamentals}~\ref{en:SBfund6}. This means that when this greedy jumping reaches the end of $S$, it must land on a $1$ that is distance at most $\delta -1$ of the rightmost $1$ of $S$. Also notice that each jump can pass over at most $\delta-1$ ones. Thus each $1$ in our constructed progression ``consumes'' at most $\delta$ ones: itself plus the at most $\delta-1$ ones that are skipped by the next jump. But, since there are more than $p \delta$ ones and we have not wasted any $1$s because we started at the beginning of $S$, our progression must have at least $p+1$ ones. The only problem is that this progression may not end at the maximum element of $B$. We now address this problem.

Let $x$ denote the leftmost $1$ of $S$ and $y$ the rightmost $1$ of $S$. Suppose that our currently constructed progression from $S$ is $x=x_1<\cdots<x_{q}$, for some $q \geq p+1$. In a manner similar to the construction of this sequence, construct a new progression starting at $y$ and greedily jumping leftward toward $x$. Suppose that this second progression is $y_h < \cdots < y_2 < y_1$; that is, this progression begins at the rightmost element $y_1=y$ of $S$, jumps leftwards greedily until it reaches $y_h$ and no further jumps are possible. A consequence of the next claim is that the $x$-progression and the $y$-progression have the same length (i.e. $h=q$).

\begin{claim} $y_j - x_{q+1-j} < \delta$, for $j=1,\ldots,q$.

We prove this by induction on $j$. The basis case is true because $y_1=y$ and as noted in the paragraph above, the progression of $x$s must end within $\delta$ of $y$. Now suppose that $y_{j} - x_{q+1-j} < \delta$, for some $j$. Because the progression of $y$s must end within $\delta$ of $x$, if $q+1-j > 1$, the element $y_{j+1}$ must exist. The distance $y_{j} - y_{j+1}$ must be at least $\delta$ so $y_{j+1} < x_{q+1-j}$. In particular, $x_{q-j} \leq y_{j+1} < x_{q+1-j}$. Because the $x$-sequence did not jump from $x_{q-j}$ to $ y_{j+1}$, it follows that $y_{j+1}- x_{q-j} \leq \delta - 1$, as desired.
\end{claim}
Observe that if $y_j = x_{q+1-j}$, for some $j \in \{1,\ldots,q\}$ then the progression \[ x_1, \dotsc, x_{q+1-j}, y_{j-1}, \dotsc, y_1 \] satisfies the conclusion of the theorem.

So, we have proven that we may assume that these two sequences interlace: \[ x_1 < y_q < x_2 < y_{q-1} < \dotsb < x_{q-1} < y_2 < x_{q} < y_1,\] and $y_j - x_{q+1-j} < \delta$, for $j=1, \dotsc, q$.
 
Now let $T$ denote the substring of $S$ corresponding to the major block of $1$s in $B$. Because $T$ is a major block, $T$ is a substring of $1$s with length at least $k-i+1$. In particular, since $k \geq 2i$, the length of $T$ is at least $i+1$, which is larger than $\delta$. Now observe that among the differences between consecutive elements of the progression $x_1,\ldots,x_q$, there must be a difference of exactly $\delta$ because the first greedy jump that this progression makes into $T$ must either have length exactly $\delta$ or it hits the first element of $T$. In the latter event, the following jump must have length $\delta$ because $T$ contains at least $\delta + 1$ ones.

So, there is some $j \in \{1,\ldots,q-1\}$ such that $x_{q+1-j} - x_{q-j} = \delta$. Because $y_j - x_{q+1-j} < \delta$, it follows that $y_j - x_{q-j} \leq 2\delta \leq k - i + \delta$. Therefore, the progression \[ x_1,\dotsc, x_{q-j}, y_{j}, \dotsc, y_1 \] satisfies the conclusion of the theorem.
\end{proof}

\begin{theorem}[Greedy Minor Superblock Jumping]\label{thm:GreedyMinorJump} 
Assume $k \geq 2i$.
If $B$ is a minor superblock of $C$, $x$ is an extreme of $B$, and $1 \leq \delta \leq i$, then $B$ contains a monochromatic quasi-progression $P$ such that 
\begin{enumerate}
   \item\label{en:GreedyMiSBJ1} $P$ has length at least $\left\lceil \frac{|B|}{\delta} \right\rceil$,
   \item\label{en:GreedyMiSBJ2} the low difference of $P$ is at least $\delta$,
   \item\label{en:GreedyMiSBJ3} the diameter of $P$ is at most $k-i$, and
   \item\label{en:GreedyMiSBJ4} $x \in P$.
\end{enumerate}
\end{theorem}
\begin{proof} We argue essentially the same way as in the proof of \autoref{thm:GreedyMajorJump}. Without loss of generality, $B$ has color $1$ and $x$ is the leftmost element of $B$ (that is, $x = \min B$). Let $S$ denote the binary substring of $C$ consisting of all characters between the minimum and maximum elements of $B$. By definition, $S$ begins and ends with a $1$. Let us suppose that $p \delta < |B| \leq (p+1)\delta$, for some positive integer $p$. We must show that there is monochromatic quasi-progression of length at least $p+1$ (that is, a progression that satisfies \ref{en:GreedyMiSBJ1} above) with the additional properties \ref{en:GreedyMiSBJ2}--\ref{en:GreedyMiSBJ4}. Create such a monochromatic quasi-progression this way: start with $x$ and repeatedly jump right to the first available $1$ that is distance at least $\delta$, but no more than $\delta + k-i$ from the last chosen $1$. Note that there can never be an obstruction to jumping to the next available $1$ unless we reach the end of $S$ because, if a jump to the next $1$ required a length more than $\delta + k-i$, then the last $k-i+1$ skipped elements would be a major block of $0$s in $S$ which is impossible by \autoref{thm:Fundamentals}~\ref{en:SBfund6}. This means that when this greedy jumping reaches the end of $S$, it must lands on a $1$ that is distance at most $\delta - 1$ of the right most $1$ of $S$. Also notice that each jump can pass over at most $\delta - 1$ ones. Thus each $1$ in our constructed progression ``consumes'' at most $\delta$ ones, itself plus the at most $\delta-1$ ones that are skipped by the next jump. But, since there are more than $p \delta$ ones and we have not wasted any $1$s because we started at the beginning of $S$, our progression must have at least $p + 1$ ones.
\end{proof}

The next theorem brings together the previous two theorems and is a significant tool in later proofs.

\begin{theorem}\label{thm:QP main tool}
Assume $k \geq 2i$. Suppose $C$ is a binary string that represents a $\{\mbox{red, blue}\}$-coloring of positive integers with no good progression. If $C$ has no red major blocks of cardinality at least $k-i+\delta$, for some $1 \leq \delta \leq i$, and $B_1,\ldots,B_h$ are the blue superblocks of $C$, then $C$ contains a monochromatic quasi-progression $P$ with diameter at most $k-i$, low-difference at least $\delta$, and length at least $\sum_{j=1}^h \left\lceil \frac{|B_j|}{\delta} \right\rceil$.
\end{theorem}
\begin{proof} Apply \autoref{thm:GreedyMajorJump} to superblocks $B_2,\ldots,B_{h-1}$ to obtain monochromatic quasi-progression fragments $P_2,\ldots,P_{h-1}$ with length at least $\left\lceil \frac{|B_j|}{\delta} \right\rceil$, low-difference $\delta$, diameter $k-i$ and that contain both extremes of each of their major superblocks. Similarly, apply \autoref{thm:GreedyMinorJump} to superblocks $B_1$ and $B_h$ to obtain monochromatic quasi-progression fragments $P_1$ and $P_h$ with length at least $\left\lceil \frac{|B_j|}{\delta} \right\rceil$, low-difference $\delta$, diameter $k-i$ and that contain the maximum and minimum, respectively, of $B_1$ and $B_h$. Because $C$ has no red major blocks of cardinality at least $k-i+\delta$, jumps of length at most $k-i+\delta$ (and at least $\delta$) can be made to join the extremes of these fragments into the desired monochromatic quasi-progression $P$.
\end{proof}

\section{Some upper bounds\label{sec:QP upper bounds}}

This section establishes upper bounds on $Q(k-i,k)$. In many cases the bounds are sharp. The proofs rely heavily on the superblock results from the previous section.

\begin{theorem} If $k \geq 2i$ and $k \equiv 0 \pmod{i}$, then $Q(k-i,k) = 2ik-4i+3$.
\end{theorem}
\begin{proof} 
The lower bound $Q(k-i,k) \geq 2ik-4i+3$ follows from Corollary 1 of Landman's paper \cite{L}; so it suffices to prove the upper bound. Suppose that $\ell = Q(k-i,k) - 1$ and $C$ is a binary string of length $\ell$ that represents a $2$-coloring of the integers $1,\ldots,\ell$ with no good progression. We must prove that $\ell \leq 2ik-4i+2$. Assume that $k = m i$, for some $m \geq 2$.

We claim that there can not be $i$ major superblocks of the same color. To see this, suppose to the contrary, that $B_1,\ldots,B_p$ ($p \geq i$) are all of the blue major superblocks of $C$. Apply \autoref{thm:QP main tool} with $\delta = i$ to the substring of $C$ between the $\min B_1$ and the $\max B_i$. This theorem guarantees a monochromatic quasi-progression of length at least \[\sum_{j=1}^p \left\lceil \frac{|B_j|}{i} \right\rceil \geq \sum_{j=1}^p \left\lceil \frac{k-i+1}{i} \right\rceil \geq \sum_{j=1}^p m \geq mp \geq k,\] a contradiction. So $C$ has at most $i-1$ major superblocks of each color.

Suppose that $C$ has $i-1$ major superblocks of the same color. We now claim that the total number of elements in minor blocks of that color is at most $k-i$. To prove this, suppose that $C$ has $i-1$ blue major superblocks $B_1,\ldots,B_{i-1}$ and two minor superblocks $B_0$ and $B_{i}$ (it is clear that there are most two blue minor superblocks, since there is at most one at each end). Again, apply \autoref{thm:QP main tool} with $\delta = i$ to the substring of $C$ between the $\min B_0$ and the $\max B_i$. This theorem guarantees a monochromatic quasi-progression of length at least \[\sum_{j=0}^i \left\lceil \frac{|B_j|}{i} \right\rceil = (i-1)m + \left\lceil \frac{|B_0|}{i} \right\rceil + \left\lceil \frac{|B_i|}{i} \right\rceil.\] Since this sum is at most $k-1 = mi-1$, it follows that $|B_0| + |B_i| \leq k - i$. Consequently, the blue superblocks have cardinalities that sum to at most $(i-1)(k-1) + (k-i)$. The same argument applies to the red superblocks. Therefore, the length of $C$ is at most \[2(i-1)(k-1) + 2(k-i) = 2ik-4i+2,\] as desired.
\end{proof}

\begin{theorem} If $k \geq 2i + 1$ and $k \equiv 1 \pmod{i}$, then $Q(k-i,k) = 2ik-2i+1$.
\end{theorem}
\begin{proof} 
The lower bound $Q(k-i,k) \geq 2ik-2i+1$ follows from Corollary 1 of Landman's paper \cite{L}; so it suffices to prove the upper bound. Let $C$ be binary string realizing an extremal $2$-coloring with no good progression. We must prove that the length of $C$ is at most $2i(k-1)$. Let $B_1,\ldots,B_p$ be the blue superblocks of $C$ and $R_1,\ldots,R_q$ the red superblocks. Because the superblocks form a partition, the length of $C$ is $\sum_{j=1}^p |B_j| +\sum_{j=1}^q |R_j|$. However, applying \autoref{thm:QP main tool} with $\delta = i$ to the substring of $C$ between the $\min B_1$ and the $\max B_p$, we find that $C$ contains a monochromatic quasi-progression of length $\sum_{j=0}^p \left\lceil \frac{|B_j|}{i} \right\rceil$. Since this can not exceed $k-1$, it follows that $\sum_{j=1}^p |B_j| \leq i (k-1)$. A symmetric argument shows $\sum_{j=1}^q |R_j| \leq i (k-1)$. Therefore the length of $C$ is at most $2i(k-1)$, as desired.
\end{proof}

The next theorem gives a general upper bound that we shall show in the next section is sharp when $k=mi+r$ for integers $m,r$ such that $3 \leq r < \frac{i}{2}$ and $r-1 \leq m$.

\begin{theorem} \label{thm:GeneralUpperBound} If $k \geq 2i$ and $k=mi+r$ for integers $m,r$ such that $1 < r < i$, then \[Q(k-i,k) \leq 2ik-4i+2r-1.\]
\end{theorem}
\begin{proof} 
Suppose that $\ell = Q(k-i,k) - 1$ and $C$ is a binary string of length $\ell$ that represents a $2$-coloring of the integers $1,\ldots,\ell$ with no good progression. We must prove that $\ell \leq 2ik-4i+2r-2$. We argue by contradiction: assume that $\ell > 2ik-4i+2r-2$.

For $j=0,1$, let $\alpha_j$ denoted the number of superblocks of $C$ that have size greater than $mi$. Apply \autoref{thm:QP main tool} with $\delta = i$ to the shortest substring containing all superblocks of color $j$: \[\alpha_j (m+1) + \sum_{\substack{\text{$B$ has color $j$}\\ |B| \leq mi}} \left\lceil \frac{|B|}{i} \right\rceil \leq k-1.\] It follows that, for $j=0,1$, \[\sum_{\substack{\text{$B$ has color $j$}\\ |B| \leq mi}} |B| \leq i(k - 1 - \alpha_j (m+1)).\] Therefore, the length of $C$ can be bounded as follows:
\begin{align*}
\ell &= \sum_{j=0}^1 \left( \sum_{\substack{\text{$B$ has color $j$}\\ |B| > mi}} |B| \right) + \sum_{j=0}^1 \left( \sum_{\substack{\text{$B$ has color $j$}\\ |B| \leq mi}}  |B| \right) \\
 &\leq (\alpha_0 + \alpha_1)(k-1) + i(k - 1 - \alpha_0 (m+1)) + i(k - 1 - \alpha_1 (m+1)) \\
 &= 2ik - 2i - \alpha(i+1 - r),
\end{align*}
where $\alpha = \alpha_0 + \alpha_1$. Because we are assuming that $\ell > 2ik-4i+2r-2$, we may conclude that $\alpha < 2$. Without loss of generality, $\alpha_0=0$ and $\alpha_1 \leq 1$.

Because $\alpha_0=0$, the coloring contains no superblocks of $0$s with cardinality larger than $mi$. Now apply \autoref{thm:QP main tool} with $\delta = i-r+1$ to the substring of $C$ containing all superblocks of color $1$: \[\sum_{\text{$B$ has color $1$}} \left\lceil \frac{|B|}{i-r+1} \right\rceil \leq k-1.\] Consequently the number of $1$s in $C$ is at most $(k-1)(i-r+1)$. Applying \autoref{thm:QP main tool} with $\delta = i$ to the substring of $C$ containing all superblocks of color $0$ we find \[\sum_{\text{$B$ has color $0$}} \left\lceil \frac{|B|}{i} \right\rceil \leq k-1.\] Therefore the number of $0$s in $C$ is at most $(k-1)i$. It follows that the length of $C$ can be bounded as follows:
\begin{align*}
	\ell	&= \left(\# 1\mbox{s in }C \right) + \left( \# 0\mbox{s in }C \right) \\
		&\leq (k-1)(i-r+1) +  (k-1)i\\
		&= 2ik - 2i +(k-1)(1 - r),
\end{align*}
which is at most $2ik-4i+2r-2$ since $r > 1$ and $k > 2i$. This contradicts that $\ell > 2ik-4i+2r-2$. 
\end{proof}

\section{A general lower bound\label{sec:QP lower bound}}

In this section we exhibit an extremal $2$-coloring of positive integers avoiding monochromatic $k$-term quasi-progressions of diameter $k-i$ for many values of $k$ and $i$. To describe this coloring we first introduce some notation.

Recall that a block of a binary string is a maximal length monochromatic substring. A \defn{segment} of a binary string is a maximal length substring in which all blocks have the same length. Its segments naturally partition a binary string. Therefore a binary string $C$ can be abbreviated by an expression involving positive integers of the form $a_1^{b_1}\cdots a_s^{b_s}$, which indicates that the $j$th segment consists of $b_j$ blocks of length $a_j$ (we assume that the first block is a block of $1$s). We adopt this notation in this section. Note that $C$ has length $\sum_{j=1}^s a_j b_j$. 

\begin{theorem} \label{thm:GeneralLowerBound} Suppose that $k=mi+r$ for integers $m,r$ such that $3 \leq r < \frac{i}{2}$. If $r-1 \leq m$, then the following $2$-coloring of the integers from $1$ to $2ik-4i + 2r - 2$ contains no monochromatic $k$-term quasi-progression of diameter $k-i$: \[(i(r-2))^1 \ (mi)^{i-1}\ (k-1)^{2}\ (mi)^{i-1}\ (i(r-2))^1.\]
\end{theorem}
\begin{proof} Let $C$ denote the binary string corresponding to this coloring. We assume that $C$ begins with a block of $1$s. Observe that $C$ is a palindrome and, because $r-1 \leq m$, the initial block of $1$s is shorter than the others. For a positive integer $\delta$, let $Q_\delta$ be a longest monochromatic quasi-progression in $C$ with low-difference $\delta$, and let $\ell$ be the length of $Q_\delta$. We must show $\ell \leq k-1$. Because $C$ is a palindrome, we may assume that $Q_\delta$ consists of elements of color $1$. If $\delta < i - r + 1$, then $Q_\delta$ can not jump across blocks of length $mi$ or longer, so $Q_\delta$ has length at most $k-1$ in this case. So it suffices to consider values of $\delta \geq i-r+1$. However, because $C$ has only interior blocks of length $mi$ or $k-1$, we can restrict our attention of $\delta = i - r + 1$ (which permits leaps over blocks size $mi$ but not $k-1$) or $\delta = i$ (which permits leaps over all block sizes). Upper bounds on $\ell$ for other values of $\delta$ follow immediately from the upper bounds on these two values of $\delta$.

Assume first that $\delta = i$. The quasi-progression $Q_\delta$ can use at most $\left\lceil \frac{|B|}{i} \right\rceil$ elements from a block $B$. Consequently, the length $\ell$ can be bounded this way:
\begin{align*}
\ell &\leq \left\lceil \frac{i(r-2)}{i} \right\rceil + (i-1) \left\lceil \frac{mi}{i} \right\rceil + \left\lceil \frac{k-1}{i} \right\rceil \\
	&= (r - 2) + (i-1)m + (m + 1) \\
	&= k - 1
\end{align*}
as desired.

Assume now that $\delta = i - r + 1$. Clearly the quasi-progression $Q_\delta$ can use at most $\left\lceil \frac{|B|}{ i - r + 1} \right\rceil$ elements from a block $B$. There are two cases according to the parity of $i$. In each case the length of $Q_\delta$ is bounded from above:

If $i$ is even the extremal length occurs when $Q_\delta$ uses $1$s from the small beginning block, $(i-2)/2$ intermediate blocks of size $mi$, and the large block of $k-1$ ones, so
\begin{equation} \label{eq:Case1Bound}
\ell \leq \left\lceil \frac{i(r-2)}{i - r + 1} \right\rceil + \frac{i-2}{2} \left\lceil \frac{mi}{i - r + 1} \right\rceil + \left\lceil \frac{k-1}{i - r + 1} \right\rceil. 
\end{equation}

If $i$ is odd the extremal length occurs when $Q_\delta$ uses $1$s from $(i-1)/2$ intermediate blocks of size $mi$, and the large block of $k-1$ ones, so
\begin{equation} \label{eq:Case2Bound}
\ell \leq \frac{i-1}{2} \left\lceil \frac{mi}{i - r + 1} \right\rceil + \left\lceil \frac{k-1}{i - r + 1} \right\rceil. 
\end{equation}

Consider now the following upper bounds which, when substituted into inequalities (\ref{eq:Case1Bound}) and (\ref{eq:Case2Bound}), will show that $\ell \leq k-1$.

\begin{claim}[Bound 1]$\left\lceil \frac{mi}{i - r + 1} \right\rceil \leq \min\{2m - 1,2m + \frac{2(r-1-m)}{i-1}\}$.

It suffices to prove that $\frac{mi}{i - r + 1} \leq 2m - 1 + \frac{2(r-1-m)}{i-1}$. Assume the contrary: $\frac{mi}{i - r + 1} > 2m - 1 + \frac{2(r-1-m)}{i-1}$. Applying the assumption that $2r + 1 \leq i$ while solving for $m$, we find that \[\frac{(i-r+1)(i-2r+1)}{i(i-2r-1)+4(r-1)} > m.\] Because we have assumed that $m \geq r-1$, it follows that \[\frac{(i-r+1)(i-2r+1)}{i(i-2r-1)+4(r-1)} > r-1.\] Consequently,
	\begin{align*}
		0	&> \left(r-1\right)\left[i(i-2r-1)+4(r-1)\right] - (i-r+1)(i-2r+1)\\
     		&= \left(r-2\right)\left(i-1\right)\left(i - \frac{2r^2 - 5r + 3}{r-2}\right)
	\end{align*}
Because $r > 2$ and $i > 1$, we conclude that $i < \frac{2r^2 - 5r + 3}{r-2}$. However, since $i > 2r$ we find that \[ 2r < \frac{2r^2 - 5r + 3}{r-2} \] which implies that $r < 3$, a contradiction.
\end{claim}

\begin{claim}[Bound 2]$\left\lceil \frac{k-1}{i - r + 1} \right\rceil \leq 2m$.

Assume, to the contrary, that $\frac{k-1}{i - r + 1} > 2m$. Substituting $k=mi+r$ and solving for $m$ produces the inequality $\frac{r-1}{i-2r+2} > m$. Because we have assumed that $m \geq r-1$, it follows that $\frac{r-1}{i-2r+2} > r-1$ which implies that $i < 2r-1$, a contradiction.
\end{claim}

\begin{claim}[Bound 3]$\left\lceil \frac{i(r-2)}{i - r + 1} \right\rceil \leq \frac{i}{2} + r - 2$.

It suffices to prove that  $\frac{i(r-2)}{i - r + 1} \leq \frac{i}{2} + r - 3$. Assume, to the contrary, that $ \frac{i(r-2)}{i - r + 1} > \frac{i}{2} + r - 3$. Equivalently, \[ 0  >  2\left(\frac{i}{2}+ r- 3\right)(i-r+1) - 2i (r-2) = (i+r-3)(i-2r+2),\] a contradiction.
\end{claim}

To complete the proof we show, using these bounds applied to inequalities (\ref{eq:Case1Bound}) and (\ref{eq:Case2Bound}), that $\ell \leq k-1$. Consider inequality (\ref{eq:Case1Bound}):
	\begin{align*}
		\ell	&\leq \left\lceil \frac{i(r-2)}{i - r + 1} \right\rceil + \frac{i-2}{2} \left\lceil \frac{mi}{i - r + 1} \right\rceil + \left\lceil \frac{k-1}{i - r + 1} \right\rceil\\
   			&\leq \left(\frac{i}{2} + r - 2\right) + \left( \left( \frac{i-2}{2}\right) \left( 2m-1 \right)\right) + \left( 2m \right)\\
			&= k-1
   \end{align*}
Now consider inequality (\ref{eq:Case2Bound}):
	\begin{align*}
		\ell	&\leq \frac{i-1}{2} \left\lceil \frac{mi}{i - r + 1} \right\rceil + \left\lceil \frac{k-1}{i - r + 1} \right\rceil\\
			&\leq \left( \left( \frac{i-1}{2}\right) \left( 2m + \frac{2(r-1-m)}{i-1} \right)\right) + \left( 2m \right)\\
			&= k-1
   \end{align*}
\end{proof}

\begin{theorem} \label{thm:SharpGeneralBound} If $k=mi+r$ for integers $m,r$ such that $3 \leq r < \frac{i}{2}$ and $r-1 \leq m$, then \[Q(k-i,k) = 2ik-4i+2r-1.\]
\end{theorem}
\begin{proof}  \Autoref{thm:GeneralUpperBound} proves the upper bound and \autoref{thm:GeneralLowerBound} proves the lower bound.
\end{proof}

\begin{theorem} If $k \geq 2i$ and $k \equiv 2 \pmod{i}$, then $Q(k-i,k) = 2ik-4i+3$.
\end{theorem}
\begin{proof} \Autoref{thm:GeneralUpperBound} gives the upper bound $Q(k-i,k) \leq 2ik-4i+3$. Arguing in a manner similar to the proof of \autoref{thm:GeneralLowerBound}, one can show that \[ (k-2)^{i-1}\ (k-1)^{2}\ (k-2)^{i-1}.\] is a $2$-coloring of $1,\dotsc, 2ik-4i+2$ that avoids monochromatic $k$-term quasi-progressions of diameter $k-i$.
\end{proof} 

\section{Computational Results}

Landman \cite{L} gave a table of maximal known $2$-colorings avoiding quasi-progressions of length $k$ and diameter $k-i$ for various values of $k$ and $i$. Here we present extended versions of that table obtained using our own program. Results not included in the original table are indicated in bold; if the $k$ and $i$ values are bolded then the row did not appear in the original at all. A horizontal line separates the entries with $k < 2i$ from the entries with $k \geq 2i$. The fact that $Q\left(13-5,5\right) = 115$ is of particular interest; this is the easiest-to-compute value for which Landman's conjecture fails.

\newcommand*{\updated}[1]{{\bfseries #1}}%

\begin{table}\caption{Verified and \updated{updated} version of the table from \cite{L} $\left(i=3\right)$}\centering\linespread{1}\selectfont\begin{tabular}{rrrl}\hline\\[-8pt]
$k$	&$i$	&\multicolumn{2}{l}{$Q\left(k-i,k\right)$\hfill Maximal Valid Colorings\hfill\null}\\[2pt]\hline\hline\\[-8pt]
$3$	&$3$	&\hphantom{$9999$}$9$		&$2^4$, $1^3 2^1 1^3$, $1^1 2^3 1^1$\\
$4$	&$3$	&$19$	&$3^6$\\
$5$	&$3$	&$29$	&$4^2 3^2 4^2 3^2$, $3^2 4^2 3^2 4^2$, $4^1 3^2 4^2 3^2 4^1$, $3^1 4^2 3^2 4^2 3^1$\\
\cline{1-3}
$6$	&$3$	&$27$	&$3^1 5^4 3^1$\\
$7$	&$3$	&$37$	&$6^6$, $3^1 6^5 3^1$\\
$8$	&$3$	&$39$	&$6^2 7^2 6^2$ 
\updated{and 5 others}\\

$9$	&$3$	&$45$	&$6^1 8^4 6^1$\\
$10$ &$3$	&$55$	&$9^6$, $6^1 9^5 3^1$, $3^1 9^5 6^1$\\
$11$ &$3$	&$57$	&$9^2 10^2 9^2$ \updated{and 7 others}\\
$12$ &$3$	&$63$	&$9^1 11^4 9^1$\\
$13$ &$3$	&$73$	&$12^6$, $9^1 12^5 3^1$, $6^1 12^5 6^1$, $3^1 12^5 9^1$\\
$14$ &$3$	&$75$	&$12^2 13^2 12^2$ \updated{and 9 others}	\\
$15$ &$3$	&$81$	&$12^1 14^4 12^1$\\
$16$ &$3$	&$91$	&$15^6$, $12^1 15^5 3^1$, $9^1 15^5 6^1$, $6^1 15^5 6^1$, $3^1 15^5 12^1$	\\
$17$ &$3$	&$93$	&$15^2 16^2 15^2$ \updated{and 11 others}\\
\updated{18} &\updated{3}	&$99$	&$15^1 17^4 15^1$\\
\updated{19} &\updated{3}	&$109$	&6 examples\\
\updated{20} &\updated{3}	&$111$	&14 examples\\
\updated{21} &\updated{3}	&$117$	&$18^1 20^4 18^1$
\end{tabular}\label{tab:i = 3}\end{table}

\begin{table}\caption{Verified and \updated{updated} version of the table from \cite{L} $\left(i=4\right)$}\centering\linespread{1}\selectfont\begin{tabular}{rrrl}\hline\\[-8pt]
$k$	&$i$	&\multicolumn{2}{l}{$Q\left(k-i,k\right)$\hfill Maximal Valid Colorings\hfill\null}\\[2pt]\hline\hline\\[-8pt]
$4$ &$4$	&\hphantom{$999$}$35$	&\updated{14 examples}\\
$5$	&$4$	&$33$	&$4^8$ \updated{and 43 others}\\
$6$	&$4$	&$49$	&$5^4 4^2 5^4$ \updated{and 8 others}\\
$7$	&$4$	&$65$	&$6^2 4^1 6^2 4^1 6^2 4^1 6^2 4^1$, $6^1 4^1 6^2 4^1 6^2 4^1 6^2 4^1 6^1$, $4^1 6^2 4^1 6^2 4^1 6^2 4^1 6^2$\\
\cline{1-3}
$8$	&$4$	&$51$	&$4^1 7^6 4^1$ \updated{and 6 others}\\
$9$ &$4$	&$65$	&$8^8$, $4^1 8^7 4^1$\\
$10$ &$4$	&$67$	&$8^3 9^2 8^3$ \updated{and 14 others}\\
$11$ &$4$	&$75$	&\updated{6 examples}\\
$12$ &$4$	&$83$	&$8^1 11^6 8^1$ \updated{and 6 others}\\
$13$ &$4$	&$97$	&$12^8$, $8^1 12^7 4^1$, $4^1 12^7 8^1$\\
$14$ &$4$	&$99$	&$12^3 13^2 12^3$ \updated{and 20 others}\\
\updated{15} &\updated{4}	&$107$	&9 examples\\
\updated{16} &\updated{4}	&$115$	&7 examples\\
\updated{17} &\updated{4}	&$129$	&$16^8$, $12^1 16^7 4^1$, $8^1 16^7 8^1$, $4^1 16^7 12^1$\\
\updated{18} &\updated{4}	&$131$	&7 examples\\
\updated{19} &\updated{4}	&$139$	&12 examples
\end{tabular}\label{tab:i = 4}\end{table}

\begin{table}\caption{Verified and \updated{updated} version of the table from \cite{L} $\left(i\geq 5\right)$}\centering\linespread{1}\selectfont\begin{tabular}{rrrl}\hline\\[-8pt]
$k$	&$i$	&\multicolumn{2}{l}{$Q\left(k-i,k\right)$\hfill Maximal Valid Colorings\hfill\null}\\[2pt]\hline\hline\\[-8pt]
$5$	&$5$	&\hphantom{$99$}$178$	&\updated{96812 examples}\\
$6$	&$5$	&$67$	&$2^2 5^2 2^2 3^2 2^2 5^2 2^2 3^2 2^2 5^2 2^2$\\
$7$	&$5$	&$73$	&\updated{Four examples}\\
$8$ &$5$	&$93$	&\updated{198 examples}\\
$9$ &$5$	&\updated{115}	&\updated{44 examples}\\
\cline{1-3}
$10$ &$5$	&$83$	&$5^1 9^8 5^1$ \updated{and 25 others}\\
$11$ &$5$	&$101$	&$10^{10}$, $5^1 10^9 5^1$\\
$12$ &$5$	&\updated{103}	&\updated{10 examples}\\
$13$ &$5$	&\updated{115}	&\updated{$5^1 10^4 12^2 10^4 5^1$}\\
$14$ &$5$	&\updated{123}	&$5^1 10^2 13^2 10^2 13^2 10^2 5^1$\\
$15$ &$5$	&\updated{133}	&\updated{26 examples}\\
\updated{16} &\updated{5}	&$151$	&$15^{10}$, $10^1 15^9 5^1$, $5^1 15^9 10^1$\\
\updated{17} &\updated{5}	&$153$	&40 examples\\
\\
$7$ &$6$	&\updated{127}	&$52256466664255246511566432362141$,\\
	&&&$522564666642552466666432362141$\\
\updated{11} &\updated{6}	&$184$	&94 examples (see QPcirc)\\
\cline{1-3}
\updated{12} &\updated{6}	&$123$	&61 examples\\
\updated{15} &\updated{6}	&$\geq 161$	&$6^1 12^5 14^2 12^5 6^1$\\
\\
$8$	&$7$	&\updated{$\geq$ 194}	&$7263634353377422552562575$-\\&&&-$12267472256226572133351$\\
\\
$9$	&$8$	&\updated{$\geq$ 289}	&$82836386822482622723782342252472$-\\&&&-$88578852484422168253442387428231$
\end{tabular}\label{tab:i geq 5}\end{table}

A natural direction for further study is to try to obtain upper bounds similar to \autoref{thm:GeneralUpperBound} for quasi-progressions with smaller diameter. While it seems unlikely that exact results can be obtained, it may be possible to get the correct order of growth. Landman proved $Q\left(k-1,k\right) = 2k-1$ and \autoref{thm:GeneralUpperBound} says in essence that $Q\left(\frac{k}{2},k\right) \leq 2k^2$. Accordingly, we make the following conjecture.

\begin{conjecture}For a fixed positive integer $r$, there exists a constant $c$ such that \[Q\left(\frac{k}{r},k\right) \leq c k^r\]
\end{conjecture}
%!TEX root=../Dissertation.tex
\chapter{PRETENDING: THE ROW SPACE OF AN ADJACENCY MATRIX}

At the $21$st British Combinatorial Conference, Peter Cameron posed this question:

\begin{question}[Cameron, \cite{Cam}]\label{question:hood vector} Let $G$ be a nonempty graph and let $A$ be the adjacency matrix of $G$. Is there always a nonzero $\{0,1\}$-vector in $\Row A$ (over $\mathbb{R}$) that is not a row of $A$?
\end{question}

While Cameron posed the question for vectors over the real numbers, $\mathbb{R}$, note that it suffices to work over the rational numbers, $\mathbb{Q}$.

\section{Introduction}

For basic linear algebra terminology, notation, and results, see \cite{Lay}. For basic graph theory, see \cite{West}. Recall the definition of the adjacency matrix of a graph.

\begin{definition}
	\index{adjacency matrix}
	%\index{n-monoplane@$n$-monoplane|see{monoplane}}
	\index{graph!adjacency matrix}
Let $G$ be a graph with vertex set $V\left(G\right) = \left\{v_1, v_2, \dotsc, v_n\right\}$. The \defn{adjacency matrix} of $G$, denoted $\adjm{G}$, is the $n \times n$ $\left\{0,1\right\}$-matrix with $a_{ij} = 1$ if and only if $v_i$ and $v_j$ are adjacent in $G$.
\end{definition}

\begin{center}\hfill
	\begin{tikzpicture}[baseline={(0,0)}]
		\pgfmathsetmacro{\rad}{0.5/sin(pi/3 r)}
		\node[vertex] (u) at (60:\rad){};			\node[vlab] at ($(u)+(45:0.4)$){$u$};
		\node[vertex] (v) at (180:\rad){};			\node[vlab] at ($(v)+(90:0.4)$){$v$};
		\node[vertex] (w) at (300:\rad){};			\node[vlab] at ($(w)+(315:0.4)$){$w$};
		\node[vertex] (x) at ($(v)!1!150:(u)$){};		\node[vlab] at ($(x)+(90:0.4)$){$x$};
		\draw (v) -- (x) (v) -- (w) (v) -- (u);
		\draw (u) -- (w);
	\end{tikzpicture}\hfill
	{$\linespread{1}\selectfont \adjm{G} = 
	\begin{array}{c}
	\begin{array}{r*{4}{x{1.1em}}}
		u	& v	& w	& x
	\end{array}\\
	\left[ \begin{array}{r*{4}{x{1.1em}}}
		0	&1	&1	&0\\
		1	&0	&1	&1\\
		1	&1	&0	&0\\
		0	&1	&0	&0
	\end{array} \right]\\\null
	\end{array}$}\hfill\null
\end{center}

\begin{definition}
	\index{rank!of a graph}
	\index{graph!rank}
	The \defn{rank} of a graph $G$ is the rank of $\adjm{G}$ (over $\mathbb{R}$).
\end{definition}

In their unpublished manuscript \cite{ACK}, Akbari, Cameron, and Khosrovshahi noticed that many graph parameters have bounds related to the rank of the graph.

\begin{proposition}[\relax{\cite[p. 10--13]{ACK}}] Let $G$ be a graph with rank $r$. Each of these graph parameters is bounded by a function of $r$.
	\begin{itemize}
		\item The number of connected components (aside from isolated vertices) is at most $\left\lfloor\frac{r}{2}\right\rfloor$ with equality if and only if at most one component is complete tripartite and the rest are complete bipartite.
		\item The clique number $\omega\left(G\right) \leq r$, with equality if and only if $G$ is a complete $r$-partite graph (possibly with isolated vertices).
		\item The chromatic number $\chi\left(G\right)$ is bounded by some function of $r$, but Raz and Spieker \cite{RazSpi} have shown that $\chi\left(G\right)$ is not bounded by any polynomial function of $r$.
		\item For fixed $k$, the smallest number of factors in an edge partition into complete $k$-partite graphs is bounded by an unspecified function of $r$.% mention Kotlov result giving o( (4/3)^r )?
		\item If $G$ has no isolated vertices, then the domination number $\gamma\left(G\right) \leq r$ with equality if and only if $G$ is a complete bipartite graph $K_{k,l}$ with $k,l \geq 2$.
		\item If $G$ has no isolated vertices, then the total domination number $\gamma_{t}\left(G\right) \leq r$ with equality if and only if each component of $G$ is complete bipartite.
		\item If $G$ is connected, then the diameter $\diam G \leq r$.
		\item The order of the largest composition factor of the group $\Aut\left(G\right)$ which is not an alternating group is bounded by $r$ but not by a polynomial function of $r$.
	\end{itemize}
\end{proposition}

They are able to say more, but first we need some definitions.

\begin{definition}
	\index{twin}
	\index{vertex!twin}
	Let $G$ be a graph and let $u,v\in V\left(G\right)$. If $N\left(u\right) = N\left(v\right)$ then we say that $u$ and $v$ are \defn{twins} in $G$.
\end{definition}

Isolated vertices will induce $\zerovect$ as a row of $\adjm{G}$. Twin vertices will produce two copies of the same row in $\adjm{G}$. Therefore adding or deleting isolated vertices or twin vertices will not affect the rank of $G$.

\begin{definition}
	\index{reduced graph}
	\index{graph!reduced}
	A simple graph $G$ is said to be a \defn{reduced graph} if and only if $G$ has no isolated vertices and $G$ has no twins.
\end{definition}

Akbari, Cameron, and Khosrovshahi prove that there are only finitely many reduced graphs with a given rank. They give a lower bound on $m\left(r\right)$, the largest order of a reduced graph with rank $r$: \[ m\left(r\right) \geq \begin{cases}2^{(r+2)/2}-1	&\text{if $r$ is even}\\ 5\cdot 2^{(r-3)/2}-1	&\text{if $r$ is odd and $r > 1$}\end{cases}\] and conjecture that their bound is actually the correct value. If every graph has a vector as described in \autoref{question:hood vector}, then $m\left(r\right)$ is an increasing function. Thus we will reformulate \autoref{question:hood vector} as a conjecture.

\begin{conjecture}\label{conj:Adjacency Matrix}If $G$ is a reduced graph then $\Row \adjm{G}$ contains a nonzero $\{0,1\}$-vector that is not a row of the matrix.
\end{conjecture}

\begin{definition}[hood vector]
	\index{hood vector}
	Let $M$ be a $\{0,1\}$-matrix and let $\vect{v} \in \Row M$ be nonzero. We say $\vect{v}$ is a \defn{hood vector} if $\vect{v}$ is a $\left\{0,1\right\}$-vector and $\vect{v}$ is not a row of $M$. If $M = \adjm{G}$ for some graph $G$, we also say that $\vect{v}$ is a hood vector of $G$.
\end{definition}

Rephrased in this language, \autoref{conj:Adjacency Matrix} says that every reduced graph has a hood vector. A hood vector of a graph `looks like' the rows of $\adjm{G}$, which correspond to the neighborhoods of vertices of $G$. Thus the hood vector `looks like' a neighborhood in $G$.

\begin{definition}
	\index{shadow neighborhood}
	\index{neighborhood!shadow}
	Let $G$ be a graph with a hood vector $\vect{v}$. The \defn{shadow neighborhood} of $\vect{v}$ in $G$ is the set of vertices whose corresponding position in $\vect{v}$ is equal to $1$.
\end{definition}

\begin{definition}
	\index{neighborhood vector}
	\index{neighborhood!vector}
	Let $v$ be a vertex of a graph $G$. The \defn{neighborhood vector} of $v$, denoted $\nvect{v}$, is the row of $\adjm{G}$ corresponding to $v$.
\end{definition}
%\begin{align*}
%	\vect{N\left(v\right)}		&&\vect{N}\left(v\right)	&&N\vv{\left(v\right)}
%\end{align*}
%Let's preserve some citations \nocite{*}.

\section{Basic Results}

Any graph whose adjacency matrix has full rank will have $\onevect$ as a hood vector. Costello and Vu~\cite{CoVu} have shown that the adjacency matrix of a random graph will almost surely have full rank. This suggests that a probabilistic approach will not yield the result. It may be fruitful to find some properties that any counterexample to \autoref{conj:Adjacency Matrix} must satisfy.

\begin{lemma}\label{lem:Edge on Triangle}In any counterexample $G$ to \autoref{conj:Adjacency Matrix}, every edge of $G$ must lie on a triangle.
\end{lemma}
\begin{proof}Assume $G$ has an edge $uv$ not on a triangle. This implies $N\left(u\right) \cap N\left(v\right) = \varnothing$. Therefore $\nvect{u}+\nvect{v}$ is a $\left\{0,1\right\}$-vector. Any vertex with this vector as its neighborhood would have to be adjacent to both $u$ and $v$, which is impossible. Thus, the vector produced is a hood vector.
\end{proof}

\begin{proposition}\label{prop:Disjoint Neighborhoods are Covered}Let $G$ be a counterexample to \autoref{conj:Adjacency Matrix}. If $u$ and $v$ are vertices of $G$ with disjoint neighborhoods, then there exists a vertex $w$ such that $N\left(w\right) = N\left(u\right) \cup N\left(v\right)$.
\end{proposition}
\begin{proof}The vector $\nvect{u}+\nvect{v}$ is a $\{0,1\}$-vector that is $1$ exactly in the positions corresponding to $N\left(u\right) \cup N\left(v\right)$.\end{proof}

\begin{corollary}\label{cor:diam G leq 4}If $G$ is a counterexample to \autoref{conj:Adjacency Matrix} then $\diam G \leq 4$.
\end{corollary}

\begin{proposition}\label{prop:diam G leq 3}If $G$ is a counterexample to \autoref{conj:Adjacency Matrix} then $\diam G \leq 3$.
\end{proposition}
\begin{proof}\Autoref{cor:diam G leq 4} tells us that $\diam G \leq 4$, so assume $\diam G = 4$ to obtain a contradiction. Let $u$ and $v$ be vertices of $G$ such that $d\left(u,v\right) = 4$. Not only are the neighborhoods of $u$ and $v$ disjoint, we also know that no edges run between these neighborhoods. Let $w\in N\left(v\right)$.

\begin{center}\begin{tikzpicture}%[baseline={(0,0)}]
	\node[vertex] (u) at (-2.4,0){}; \node[vlab] at (-2.4,0.3){$u$};
	\node[vertex] (v) at (2.4,0){}; \node[vlab] at (2.4,0.3){$v$};
	\foreach \y in {0.8,0.5,0.2,-0.2,-0.5,-0.8}{
		\coordinate (l) at (-1.2,\y){};
		\coordinate (r) at (1.2,\y){};
		\draw (u) -- (l) (r) -- (v);
		\draw[dashed] (l) to (r);
	};
	\draw[fill=white] (-1.2,0) ellipse (0.35 and 0.8);
	\draw[fill=white] (1.2,0) ellipse (0.35 and 0.8);
	\node[vertex] (w) at (1.2,0){}; \node[vlab] at ($(w)+(90:0.3)$){$w$};
	\draw (v) -- (w);

	\extendtopbound
\end{tikzpicture}\end{center}

The vector $\nvect{u}+\nvect{w}$ is a $\left\{0,1\right\}$-vector since no edges run between $N\left(u\right)$ and $N\left(v\right)$. This vector is $1$ on $v$, all vertices in $N\left(u\right)$, and possibly on some other vertices of no consequence. If this vector were the neighborhood vector of a vertex, that vertex would have to be in $N\left(v\right)$ and adjacent to vertices in $N\left(u\right)$, which is impossible.
\end{proof}

\begin{lemma}\label{lem:Triangle on K_4 or lollipop}In any counterexample $G$ to \autoref{conj:Adjacency Matrix}, every triangle of $G$ must be contained in a $K_4$ or an induced lolliop $L_{3,1}$\index{lollipop graph@lollipop graph ($L_{3,1}$)}.
\end{lemma}
%\begin{figure}[htb]
\begin{center}
	\index{lollipop graph@lollipop graph ($L_{3,1}$)}
	\begin{tikzpicture}[every path/.style={line width=0.5pt}]
		\foreach \v in {0,...,2}{ \node[vertex] (v\v) at (\v*120:0.6){}; };
		\node[vertex] (v3) at (1.2,0){};
		\draw (v0) -- (v1) -- (v2) -- (v0) -- (v3);
		\node[vlab] at (0,-1.2){Lollipop $\smash{L_{3,1}}$};
		\extendtopbound
	\end{tikzpicture}
	%\caption{The lollipop graph $L_{3,1}$}
\end{center}
%\end{figure}
\begin{proof}To obtain a contradiction assume that $u$, $v$, and $w$ are vertices in $G$ of a triangle not contained in any $K_4$ or $L_{3,1}$. Thus every vertex of $G$ is adjacent to exactly $0$ or $2$ of $u$, $v$, and $w$. Therefore the vector $\frac{1}{2}\nvect{u} + \frac{1}{2}\nvect{v} + \frac{1}{2}\nvect{w}$ is a $\{0,1\}$-vector. This vector describes a neighborhood containing $u$, $v$, and $w$, which implies the vector is a hood vector since no vertex in $G$ is adjacent to all three of $u$, $v$, and $w$.
\end{proof}

We can use \autoref{lem:Triangle on K_4 or lollipop} to obtain a nice result, but first we need some definitions.

\begin{definition}[From {\cite[p. 281]{West}}] Let $G$ be a simple graph.

	\index{odd triangle}\index{triangle!odd}A triangle $T$ is \defn{odd} if and only if some vertex of $G$ is adjacent to an odd number of vertices of $T$. That is, there exists $v\in V\left(G\right)$ such that $|N(v) \cap V(T)|$ is odd.

	\index{even triangle}\index{triangle!even}A triangle $T$ is \defn{even} if and only if every vertex of $G$ is adjacent to an even number of vertices of $T$. That is, for all $v\in V\left(G\right)$ we have $|N(v) \cap V(T)|$ is even.
\end{definition}

Thus \autoref{lem:Triangle on K_4 or lollipop} says that in a counterexample, every triangle must be odd. Cameron stated that \autoref{conj:Adjacency Matrix} is true for line graphs. We can now prove this result using this characterization of line graphs (\cite{vRW}, paraphrased from \cite[p. 281]{West}):

\begin{theorem}[van Rooij and Wilf, 1965]
	\label{thm:vRW line graph}
	\index{line graph}
	The graph $G$ is a line graph if and only if $G$ is claw-free\index{claw graph@claw graph ($K_{1,3}$)} and no induced diamond\index{diamond graph@diamond graph ($K_4-e$)} of $G$ has two odd triangles.
\end{theorem}

\begin{center}\hfill
	\begin{tikzpicture}
		\pgfmathsetmacro{\rad}{2/3}
		\foreach \v in {0,1,2} \node[vertex] (v\v) at (90+120*\v:\rad){};
		\node[vertex] (c) at (0,0){};
		\foreach \v in {0,1,2} \draw (v\v) -- (c);
		\node[vlab] at (0,-5/6){Claw $\smash{K_{1,3}}$};
		\extendtopbound
	\end{tikzpicture}\hfill\begin{tikzpicture}
		\foreach \u in {0,1}\node[vertex] (u\u) at (2*\u,0.5){};
		\foreach \v in {0,1}\node[vertex] (v\v) at (1,\v){};
		\foreach \u in {0,1}\foreach \v in {0,1} \draw (u\u) -- (v\v);
		\draw (v0) -- (v1);
		\node[vlab] at (1,-0.5){Diamond $\smash{K_4-e}$};
		\extendtopbound
	\end{tikzpicture}\hfill\null
\end{center}

\begin{theorem}\label{thm:Line Graphs work}Every line graph has a hood vector.
\end{theorem}
\begin{proof}
	Let $G$ be a line graph. If $G$ has an induced diamond, then by \autoref{thm:vRW line graph} at least one of the triangles must be even and so $G$ must have a hood vector. If $G$ does not contain an induced diamond then $G$ is claw-free. Every edge of $G$ must be on a triangle by \autoref{lem:Edge on Triangle}.

Take a maximal clique $C$ in $G$. Since every edge of $G$ is on a triangle, $C$ must have at least three vertices. If $C=G$ then $\adjm{G}$ has full rank and so $\onevect$ is a hood vector. Otherwise, choose a vertex $v$ of $G$ which is adjacent to some $u\in V\left(C\right)$ but is not itself in $C$. Since $C$ is maximal, there is some $w \in V\left(C\right)$ such that $v$ is not adjacent to $w$. In fact $v$ cannot be adjacent to any vertex in $C$ except $u$, since if there were some $x\in V\left(C\right)$ adjacent to $v$, $\left\{u,x,w,v\right\}$ would induce a diamond.
\begin{center}\begin{tikzpicture}
	\draw[gray] (0,0) circle (1); \node[vlab,gray] at (-1.2,0.5){$C$};
	\node[vertex] (W) at (180:0.2){}; \node[vlab] at ($(W)+(90:0.3)$){$w$};
	\node[vertex] (U) at (30:1){}; \node[vlab] at ($(U)+(90:0.3)$){$u$};
	\node[vertex] (X) at (330:1){}; \node[vlab] at ($(X)+(-30:0.3)$){$x$};
	\node[vertex] (V) at ($(X)!1!-60:(U)$){}; \node[vlab] at ($(V)+(90:0.3)$){$v$};
	\draw (U) edge (V)
		(U) edge (W)
		(U) edge (X)
		(W) edge (X)
		(V) edge (X);
\end{tikzpicture}\end{center}

Thus, $G$ is composed of maximal cliques (each containing at least three vertices) joined at single vertices. Furthermore, since $G$ is claw-free no vertex can be in more than two such cliques. Call this collection of maximal cliques $\mathcal{M}$. We divide into cases based on $\left|\mathcal{M}\right|$.

\begin{case}
	If $\left|\mathcal{M}\right|=1$ then $G$ is a complete graph, a situation previously handled.
\end{case}

\begin{case}
	If $\left|\mathcal{M}\right|=2$ then $G$ is a pair of cliques $K_k$ and $K_l$ joined at a vertex $v$. We can obtain $\onevect$ as a hood vector $G$ using the linear combination

	\[ \onevect = \sum_{\mathclap{x \in K_k\setminus \left\{v\right\}}} a \nvect{x} + b \nvect{v} + \sum_{\mathclap{x \in K_l\setminus \left\{v\right\}}}c \nvect{x} \]
	%\begin{align*}
	%	a\text{~on~}&K_k\setminus \left\{v\right\}
	%	&b\text{~on~}&v
	%	&c\text{~on~}&K_l\setminus \left\{v\right\}
	%\end{align*}
where $a$, $b$, $c$ satisfy
	\begin{align*}
		\left(k-2\right)a + b &= 1
		&\left(k-1\right)a + \left(l-1\right)c &= 1
		&b + \left(l-2\right)c &= 1
	\end{align*}
This linear system has the solution
	\begin{align*}
		a	&= \frac{2-l}{3l+3k-2kl-4}
		&b	&= \frac{l+k-kl}{3l+3k-2kl-4}
		&c	&= \frac{2-k}{3l+3k-2kl-4}
	\end{align*}
	To see that this solution is always valid, assume $3l+3k-2kl-4 = 0$. Then
	\begin{equation}\label{eq:line graph proof} l = \frac{3k-4}{2k-3} = \frac{2k-3+k-1}{2k-3} = 1+\frac{k-1}{2k-3}\end{equation}
	Since every clique in $\mathcal{M}$ has at least three vertices, $k\geq 3$ and
	\[ \left(2k-3\right)-\left(k-1\right) = k-2 \geq 3-2 = 1 > 0 \]
	Therefore $2k-3 > k-1$. But then \autoref{eq:line graph proof} forces $1 < l < 2$, which cannot possibly be a valid value for $l$. Thus the solution to this linear system is always valid.
\end{case}

\begin{case}If $\left|\mathcal{M}\right| \geq 3$, there are two possibilities.
	\begin{subcase}First, suppose $\mathcal{M}$ contains three cliques $X$, $Y$, $Z$ which all pairwise intersect. They do so at distinct vertices $v_{XY}$, $v_{XZ}$, and $v_{YZ}$. Let $z\in Z\setminus\!\left\{v_{XZ}, v_{YZ}\right\}$. This means $z$ is adjacent to $v_{XZ}$ and $v_{YZ}$ but not $v_{XY}$. Therefore $\left\{ v_{XY}, v_{XZ}, v_{YZ} \right\}$ induces a diamond, contradicting our assumptions.
	\end{subcase}
	\begin{center}\begin{tikzpicture}
		\draw[gray] (0,1) circle (1); \node[vlab,gray] at (-0.3,1.4){$X$};
		\draw[gray] (0,-1) circle (1); \node[vlab,gray] at (-0.3,-1.4){$Y$};
		\draw[gray] ($(0,-1)!1!-60:(0,1)$) circle (1); \node[vlab,gray] at (2.4,0){$Z$};
		\node[vertex] (W) at (0,0){}; \node[vlab] at ($(W)+(225:0.4)$){$v_{XY}$};
		\node[vertex] (U) at ($(0,1)+(-30:1)$){}; \node[vlab] at ($(U)+(150:0.4)$){$v_{XZ}$};
		\node[vertex] (X) at ($(0,-1)+(30:1)$){}; \node[vlab] at ($(X)+(240:0.3)$){$v_{YZ}$};
		\node[vertex] (V) at ($(X)!1!-60:(U)$){}; \node[vlab] at ($(V)+(90:0.3)$){$z$};
		\draw (U) edge (V)
			(U) edge (W)
			(U) edge (X)
			(W) edge (X)
			(V) edge (X);
	\end{tikzpicture}\end{center}
	
	\begin{subcase}On the other hand if $\mathcal{M}$ has no triple that all pairwise intersect, then taking $X, Y, Z\in \mathcal{M}$, we can assume without loss of generality that $X$ and $Z$ do not intersect. If there exist $x\in X$ and $z\in Z$ such that $x$ is not in a clique containing an element of $Z$ and $z$ is not in a clique containing an element of $X$, then the $\left\{0,1\right\}$-vector $\nvect{x}+\nvect{z}$ is a hood vector.
		
		If such elements do not exist, then (without loss of generality) every element of $X$ shares a clique with an element of $Z$. Let $x_1, x_2, x_3\in X$. There exist $z_1, z_2, z_3\in Z$ and $Y_1, Y_2, Y_3\in \mathcal{M}$ such that $x_i, z_i\in\nobreak V(Y_i)$. Since $\mathcal{M}$ has no triple that all pairwise intersect, two of $\left\{Y_1, Y_2, Y_3\right\}$ must not intersect; without loss of generality we will assume $Y_1$ and $Y_2$ do not intersect. Then the $\left\{0,1\right\}$-vector $\frac{1}{2}\nvect{x_1} + \frac{1}{2}\nvect{x_2} + \frac{1}{2}\nvect{z_1} + \frac{1}{2}\nvect{z_2}$ is a hood vector.
	\end{subcase}
	\begin{center}\begin{tikzpicture}
		%\draw[help lines] (-2,-2) grid (5,2);
		\draw[gray,rounded corners] (-1,-1.6) -- (0,-1.6) -- (0,2) -- (-2.5,2) -- (-2.5,-1.6) -- cycle;
		\node[vlab,gray] at (-2,1.4){$X$};
		\draw[gray,rounded corners] (4,-1.6) -- (5.5,-1.6) -- (5.5,2) -- (3,2) -- (3,-1.6) -- cycle;
		\node[vlab,gray] at (5,1.4){$Z$};
		
		\draw[gray] (1.5,1.5) ellipse (1.5 and 1.1);
		\node[vlab,gray] at (1.5,2.2){$Y_1$};
		\draw[gray] (1.5,-1) ellipse (1.5 and 1.1);
		\node[vlab,gray] at (1.5,-1.8){$Y_2$};
	
		\node[vertex] (x1) at (0,1.5){};
		\node[vertex] (x2) at (0,-1){};
		\node[vertex] (x3) at (-0.75,-1.3){};
		\node[vertex] (x4) at (-1.5,-1){};
		\node[vertex] (x5) at (-1.5,1.5){};
		\node[vertex] (x6) at (-0.75,1.8){};
		\node[vertex] (x7) at (-1.9,0.8){};
		\node[vertex] (x8) at (-1.9,-0.3){};
		\foreach \u in {1,...,7}{ \foreach \v in {\u,...,8}{\draw (x\u) edge (x\v); }; };
	
		\node[vertex] (z1) at (3,1.5){};
		\node[vertex] (z2) at (3,-1){};
		\node[vertex] (z3) at (3.75,-1.3){};
		\node[vertex] (z4) at (4.5,-1){};
		\node[vertex] (z5) at (4.5,1.5){};
		\node[vertex] (z6) at (3.75,1.8){};
		\node[vertex] (z7) at (4.9,0.8){};
		\node[vertex] (z8) at (4.9,-0.3){};
		\foreach \u in {1,...,7}{ \foreach \v in {\u,...,8}{\draw (z\u) edge (z\v); }; };
		
		\draw (x1) edge (z1);
		\draw (x2) edge (z2);
	
		\node[vertex] (y2) at ($(x1)+(30:1)$){};
		\node[vertex] (y4) at ($(x1)+(-30:1)$){};
		\node[vertex] (y1) at ($(z1)+(150:1)$){};
		\node[vertex] (y5) at ($(z1)+(210:1)$){};
		\foreach \y in {1,2,4,5}{ \draw (x1) edge (y\y) 	(z1) edge (y\y);	};
		\foreach \u/\v in {1/2,1/4,1/5,2/4,2/5,4/5}{\draw (y\u) edge (y\v); };
	
		\node[vertex] (y2) at ($(x2)+(30:1)$){};
		\node[vertex] (y4) at ($(x2)+(-30:1)$){};
		\node[vertex] (y1) at ($(z2)+(150:1)$){};
		\node[vertex] (y5) at ($(z2)+(210:1)$){};
		\foreach \y in {1,2,4,5}{ \draw (x2) edge (y\y) 	(z2) edge (y\y);	};
		\foreach \u/\v in {1/2,1/4,1/5,2/4,2/5,4/5}{\draw (y\u) edge (y\v); };
	
		\draw[->,shorten >= 3pt,shorten <= 6pt] (-0.5,2.25) node[vlab]{$x_1$} -- (x1);
		\draw[->,shorten >= 3pt,shorten <= 6pt] (-0.5,-2) node[vlab]{$x_2$} -- (x2);
		\draw[->,shorten >= 3pt,shorten <= 6pt] (3.5,2.25) node[vlab]{$z_1$} -- (z1);
		\draw[->,shorten >= 3pt,shorten <= 6pt] (3.5,-2) node[vlab]{$z_2$} -- (z2);
		\extendtopbound
	\end{tikzpicture}\end{center}
\end{case}

All the cases are accounted for and thus we have the result.
\end{proof}

One might be tempted to try to build a counterexample to \autoref{conj:Adjacency Matrix} by starting with some base graph and appending vertices corresponding to hood vectors in hopes of extinguishing the list. This next proposition shows that such an approach is unlikely to succeed.

\begin{proposition}Let $G$ be a graph with a hood vector $\vect{x}$. Let \[ \vect{x} = c_1 \nvect{v_1} + \dotsb  + c_k\nvect{v_k} + c_{k+1}\nvect{v_{k+1}} + \dotsb + c_n \nvect{v_n} \] where $v_1$, $\dotsc$, $v_k$ are the vertices in the shadow neighborhood of $\vect{x}$ and $v_{k+1}$, $\dotsc$, $v_n$ are the other vertices of $G$. Let $G^\prime$ be the graph obtained from $G$ by appending the vertex with neighborhood vector $\vect{x}$. If $c_1 + \dotsb + c_k \neq 0$ then $G^\prime$ also contains a hood vector.
\end{proposition}
\begin{proof}
	Append the vertex $v_{n+1}$ with neighborhood $v_1, v_2, \dotsc, v_k$ and let \[ \Gamma = c_1 + c_2 + \dots + c_k \neq 0 \]
	We can obtain a new hood vector $\vect{y} = \left\langle y_1, \dotsc, y_n\right\rangle$ by letting
	\[ \vect{y} = \frac{c_1}{\Gamma}\nvect{v_1} + \dotsb + \frac{c_n}{\Gamma}\nvect{v_n} + \frac{\Gamma - 1}{\Gamma}\nvect{v_{n+1}} \]
	which will produce
		\begin{align*}y_1 = y_2 = \dots = y_k = y_{n+1} &= 1	&y_{k+1} = \dots = y_n &= 0\end{align*}
which cannot be the neighborhood vector of any vertex since no vertex in $G$ had neighborhood $v_1, v_2, \dotsc, v_k$.
\end{proof}

\section{Generalization}

What is the adjacency matrix of a graph? Ultimately, it is just a symmetric $\left\{0,1\right\}$-matrix with $0$ on the main diagonal. If \autoref{conj:Adjacency Matrix} is true, then it seems reasonable to ask whether  each of these conditions is necessary. Perhaps \autoref{conj:Adjacency Matrix} follows as a corollary to a more general statement.

\begin{question}Does every $\left\{0,1\right\}$-matrix have a hood vector?
\end{question}

The answer is \emph{no}, as this example shows.

\begin{example}[$\{0,1\}$-matrix with no hood vectors]\label{ex:mat without hood}The matrix $M$ has no hood vectors.
\[ \linespread{1}\setlength{\jot}{0pt}\renewcommand{\arraystretch}{1}\selectfont M = \begin{bmatrix}
	0&1&1&0&0&1&1\\
	1&1&0&1&1&0&0\\
	1&0&1&0&1&1&1\\
	1&1&0&1&0&1&1\\
	0&1&1&0&1&0&0\\
	1&0&1&1&0&0&1\\
	1&0&1&1&0&1&0
\end{bmatrix}\]
\end{example}
\begin{proof} In reduced row echelon form $M$ is
\[ \linespread{1}\setlength{\jot}{0pt}\renewcommand{\arraystretch}{1.1}\selectfont\begin{bmatrix}
	1&0&0&0&0&0&\frac{7}{2}\\
	0&1&0&0&0&0&\frac{3}{2}\\
	0&0&1&0&0&0&\frac{1}{2}\\
	0&0&0&1&0&0&\ntv 3\\
	0&0&0&0&1&0&\ntv 2\\
	0&0&0&0&0&1&\ntv 1\\
	0&0&0&0&0&0&0\\
\end{bmatrix}\]

	Thus $\rank M = 6$ and $\vect{n} = \langle 7, 3, 1, \ntv 6, \ntv 4, \ntv 2, \ntv 2\rangle \in \Nul M$.
Moreover $\Span{\vect{n}} = \Nul M$, so vectors in $\Nul M$ have the form $k\cdot \vect{n}$ for some $k \in \mathbb{R}$. Suppose $\vect{a} = \left\langle a_i\right\rangle \in \Row M$. Then
	\[
		\vect{a}\cdot \vect{n} = 7a_1 + 3a_2 + a_3 + \ntv6 a_4 + \ntv 4 a_5 + \ntv 2 a_6 + \ntv 2 a_7 = 0\] and so \[ 7a_1 + 3a_2 + a_3 = 6 a_4 + 4 a_5 + 2 a_6 + 2 a_7 \]
If $\vect{a}$ is a non-zero $\left\{0,1\right\}$-vector, then since all the left-hand coefficients are odd and all the right-hand coefficients are even, $\vect{a}$ must have a $1$ in exactly two of the first three positions. Therefore the following solutions are the only possible.
	\begin{align*}
		7 + 3 + 0 &= 6 + 4 + 0 + 0	&\langle 1,1,0,1,1,0,0\rangle &= \vect{r_2}\\
		7 + 3 + 0 &= 6 + 0 + 2 + 2	&\langle 1,1,0,1,0,1,1\rangle &= \vect{r_4}\\
		7 + 0 + 1 &= 6 + 0 + 2 + 0	&\langle 1,0,1,1,0,1,0\rangle &= \vect{r_7}\\
		7 + 0 + 1 &= 6 + 0 + 0 + 2	&\langle 1,0,1,1,0,0,1\rangle &= \vect{r_6}\\
		7 + 0 + 1 &= 0 + 4 + 2 + 2	&\langle 1,0,1,0,1,1,1\rangle &= \vect{r_3}\\
		0 + 3 + 1 &= 0 + 4 + 0 + 0	&\langle 0,1,1,0,1,0,0\rangle &= \vect{r_5}\\
		0 + 3 + 1 &= 0 + 0 + 2 + 2	&\langle 0,1,1,0,0,1,1\rangle &= \vect{r_1}
	\end{align*}
Thus $\vect{a}$ must be a row of $M$, and so $M$ has no hood vectors.
\end{proof}

Note that $M$ is not symmetric and has some $1$s on the main diagonal. Further, $M^T$ has forty hood vectors. Thus it seems unlikely that a simple modification to $M$ will produce a counterexample to \autoref{conj:Adjacency Matrix}.
\newcommand{\bxl}[1]{\hspace{-\arraycolsep}\begin{array}{|c|}\hline #1\\\hline\end{array}\hspace{-\arraycolsep}}
\[ \linespread{1}\selectfont M = \left[
	\begin{array}{ccccccc}
		0		&1		&1		&\bxl{0}	&0		&1		&1\\
		1		&\bxl{1}	&0		&1		&1		&0		&0\\
		1		&0		&\bxl{1}	&0		&1		&1		&1\\
		\bxl{1}	&1		&0		&\bxl{1}	&0		&1		&1\\
		0		&1		&1		&0		&\bxl{1}	&0		&0\\
		1		&0		&1		&1		&0		&0		&1\\
		1		&0		&1		&1		&0		&1		&0
	\end{array} \right]
\]

That said, \autoref{ex:mat without hood} generalizes to an entire family of $\left\{0,1\right\}$-matrices without hood vectors.

\begin{example}Let $m$ be even. Then the matrix $M$ has no hood vectors.
\[	\linespread{1}\selectfont M = 
	\begin{array}{c@{}c}
		\left[
		\begin{array}{ccccc}
			0	& 1	& 1	& 0	& 0\\
			1	& 1	& 0	& 1	& 1\\
			1	& 0	& 1	& 0	& 1\\
			1	& 1	& 0	& 1	& 0\\
			0	& 1	& 1	& 0	& 1\\\hline
%
			1	& 0	& 1	& 1	& 0\\
			1	& 0	& 1	& 1	& 0\\
			&&&\vdots\\
			1	& 0	& 1	& 1	& 0
		\end{array}\right.
		&\overbrace{\begin{array}{cccc}
			1	& 1	& \ldots & 1\\
			0	& 0	& \ldots	& 0\\
			1	& 1	& \ldots	& 1\\
			1	& 1	& \ldots	& 1\\
			0	& 0	& \ldots	& 0\\\hline
%
			0	& 1	& \ldots	& 1\\
			1	& 0	& 		& 1\\
			\vdots&&\ddots\\
			1	& 1	& \ldots	& 0
		\end{array}}^m
		\left.\vphantom{\begin{array}{c}a\\a\\a\\a\\a\\a\\a\\a\end{array}}\right]
	\end{array}
\]
\end{example}
\begin{proof}We will construct a vector $\vect{n} \in \Nul M$ that is orthogonal only to non-zero $\left\{0,1\right\}$-vectors that are rows of $M$. Let $\vect{n} = \left\langle a, b, c, x, y, w, w, \dotsc, w\right\rangle$. Then if $\vect{n} \in \Nul M$,
\begin{align*}
	b + c + mw &= 0		&a + b + x + mw &= 0\\
	a + b + x + y &= 0		&b + c + y &= 0\\
	a + c + y + mw &= 0	&a + c + x + \left(m-1\right) w &= 0
\end{align*}

Thus
\begin{align*}
	y &= mw		&x &= \left(m+1\right) w
\end{align*}

And so
\begin{align*}
	a + b + \left(2m+1\right)w &= 0		&&			&a &= \ntv\left(3m+1\right)w/2\\
	a + c + 2mw &= 0					&\implies	&	&b &= \ntv\left(m+1\right)w/2\\
	b + c + mw &= 0					&&			&c &= \ntv\left(m-1\right)w/2
\end{align*}

Let $w= \ntv 2$. Then $\vect{n} = \left\langle 3m+1, m+1, m-1, \ntv 2m, \ntv 2m-2, \ntv 2, \ntv 2, \dotsc, \ntv 2 \right\rangle$. By construction $\vect{n} \in \Nul M$. On the other hand, if $\vect{v}\cdot \vect{n} = 0$ for some non-zero $\left\{0,1\right\}$-vector $\vect{v}$, then exactly two of the first three entries in $\vect{v}$ must be $1$, and all possible ways to balance such a sum are already present in $M$.
\end{proof}

Each member of this infinite family of examples has one asymmetry and four non-zero values on the main diagonal. However, the transposes of these matrices contain many hood vectors. \Autoref{ex:logarithmic family} will show another family of $\left\{0,1\right\}$-matrices that avoid hood vectors in the matrix as well as the transpose, but that have a non-constant number of asymmetries and non-zero entries on the main diagonal. Furthermore, \autoref{ex:nonbinary logarithmic family} shows that there is a family of symmetric $\left\{0,\frac{1}{2},1\right\}$-matrices without hood vectors.

These examples show that if \autoref{conj:Adjacency Matrix} is true then it is close to the best possible statement. It is hard to imagine that the main diagonal entries are of actual importance, so it is reasonable to strengthen \autoref{conj:Adjacency Matrix} by removing this requirement.

\begin{conjecture}\label{conj:Symmetric Matrix}Every symmetric $\left\{0,1\right\}$-matrix has a hood vector.
\end{conjecture}

\section{Searching and Verification}

This section investigates algorithmic aspects of the problem of finding hood vectors. By expressing the problem of finding a hood vector as an integer linear program we can enlist powerful computer libraries and toolkits to assist us in our search for a counterexample. We also develop a short certificate that allows us to quickly count the number of hood vectors without explicitly listing them. Linear algebra provides the framework for both aspects. Recall the following well-known theorem:

\begin{theorem}\label{thm:Row orth Nul}Given a basis $\vect{n_1}$, $\dotsc$, $\vect{n_k}$ for $\Nul M$, a vector $\vect{x}\in \Row M$ if and only if $\vect{x}\cdot\vect{n_i}$ for all $i\in \left[k\right]$.
\end{theorem}

Thus, if we have a basis for $\Nul M$ we can easily check if a given vector is in $\Row M$. That test will become the linear program. We will approach the certification problem by producing a substitute for the basis, called a monopole\index{monopole}.

\subsection{Linear Programming}

Let $\vect{r_1},\dotsc,\vect{r_n}$ be the rows of the matrix $M$, let $d_i$ be the weight of row $i$, and let $\vect{n_1},\dotsc,\vect{n_k}$ be a basis for $\Nul M$. Suppose we are given a $\left\{0,1\right\}$-vector $\vect{x}$ and we want to check if it is a hood vector of $M$.

To ensure $\vect{x}$ is not already a row of our matrix, we can check that it differs in some position from each $\vect{r_i}$. Since $\vect{x}$ and $\vect{r_i}$ are both $\{0,1\}$-vectors, $\vect{x}\cdot\vect{r_i}$ will count the number of positions in which $\vect{x}$ and $\vect{r_i}$ are both $1$. Similarly $\left(\onevect - \vect{x}\right) \cdot \left(\onevect - \vect{r_i}\right)$ will count the number of positions in which $\vect{x}$ and $\vect{r_i}$ are both $0$. Thus
{\allowdisplaybreaks\begin{align*}
	n &\geq \vect{x}\cdot\vect{r_i} +\left(\onevect - \vect{x}\right) \cdot\left(\onevect - \vect{r_i}\right)\\
	&= \sum_{j=1}^{n}x_j\cdot r_{ij} + \sum_{j=1}^{n}\left(1 - x_j\right)\cdot \left(1 - r_{ij}\right)\\
	&= \sum_{j=1}^{n}x_j\cdot r_{ij} + 1 - x_j - r_{ij} + x_j\cdot r_{ij}\\
	&= \sum_{j=1}^{n}\left(2 x_j\cdot r_{ij} - x_j\right) + \sum_{j=1}^{n}1 - \sum_{j=1}^{n}r_{ij}\\
	&= \sum_{j=1}^{n}\left(2 x_j\cdot r_{ij} - x_j\right) + n - d_i\\
\intertext{and therefore}
	d_i &\geq \sum_{j=1}^{n}x_j\left(2 r_{ij} - 1\right)
\end{align*}}
where equality holds if and only if $\vect{x} = \vect{r_i}$.

With this in mind we can formulate the problem of finding a hood vector as an integer linear program:

\noindent\hfill\parbox{0.5\textwidth}{Maximize:}\hfill\null
\[ \sum_{i=1}^{n} x_i\]
\hfill\parbox{0.5\textwidth}{Subject to:}\hfill\null
\begin{subequations}\begin{align}
	\sum_{j=1}^{n} x_j \cdot n_{ij} 	&=0\ \text{for all}\ i\in \left[k\right]\label{eq:HV linear program 1}\\
		\sum_{j=1}^{n} x_j\left(2r_{ij}-1\right)&\leq d_i -1\ \text{for all}\ i\in \left[n\right]\label{eq:HV linear program 2}\\
		x_i	\in \left\{0,1\right\}&\ \text{for all}\ i\in \left[n\right]\label{eq:HV linear program 3}%0 \leq x_i	&\leq 1
\end{align}\end{subequations}

This program will return a $\left\{0,1\right\}$-vector of maximum weight in $\Row M$ that is not a row of $M$. Thus, if it does not return $\zerovect$ it will return a hood vector, and if it does return $\zerovect$ then $M$ has no hood vectors.

To find the dual problem we need to first replace the equality \altref[constraint]{eq:HV linear program 1} with a pair of inequality constraints.
\begin{align*}
	\sum_{j=1}^{n} x_j \cdot n_{ij}		&\leq 0\ \text{for all}\ i\in \left[k\right]\tag{\ref*{eq:HV linear program 1}$^+$}\label{eq:HV linear program 1+}\\
	-\sum_{j=1}^{n} x_j \cdot n_{ij}	&\leq 0\ \text{for all}\ i\in \left[k\right]\tag{\ref*{eq:HV linear program 1}$^-$}\label{eq:HV linear program 1-}
\end{align*}

We also need to replace the integer linear program with its fractional relaxation by loosening \altref[constraint]{eq:HV linear program 3}:
\begin{align*}
	x_i &\leq 1\ \text{for all}\ i\in \left[n\right]\tag{\ref*{eq:HV linear program 3}$^\prime$}\label{eq:HV linear program 3'}\\
	x_i &\geq 0\ \text{for all}\ i\in \left[n\right]
\end{align*}

We introduce dual variables: $a_1^{+}$, $\dotsc$, $a_k^{+}$ for \altref[constraint]{eq:HV linear program 1+}; $a_1^{-}$, $\dotsc$, $a_k^{-}$ for \altref[constraint]{eq:HV linear program 1-}; $b_1$, $\dotsc$, $b_n$ for \altref[constraint]{eq:HV linear program 2}; and $c_1$, $\dotsc$, $c_n$ for \altref[constraint]{eq:HV linear program 3'}. The dual problem is given by:

\noindent\hfill\parbox{0.75\textwidth}{Minimize:}\hfill\null
\[ \sum_{i=1}^{n} b_i\left(d_i-1\right) + \sum_{i=1}^{n}c_i\]
\hfill\parbox{0.75\textwidth}{Subject to:}\hfill\null
\begin{align*}
	c_j + \sum_{i=1}^{k}n_{ij}\left(a_i^{+}-a_i^{-}\right) + \sum_{i=1}^{n}b_i\left(2r_{ij}-1\right) &\geq 1\ \text{for all}\ j\in \left[n\right]\\
	a_i^{+}, a_i^{-}&\geq 0\ \text{for all}\ i \in \left[k\right]\\
	b_i, c_i &\geq 0\ \text{for all}\ i \in \left[n\right]
\end{align*}
\comments{% a suggestion from TeX.SX
 \begin{alignat*}{2}
    \text{minimize }   & \sum_{i=1}^m c_i x_i + \sum_{j=1}^n d_j t_j\  \\
    \text{subject to } & \sum_{i=1}^m a_{ij} + e_j t_j \geq g_j &,\ & 1\leq j\leq n\\
                       & f_i x_i + \sum_{j=1}^n b_{ij}t_j \geq h_i\ &,\ & 1\leq i\leq m\\
                       & x\geq 0,\ t_j\geq 0\ &,\ & 1\leq j\leq n,\ 1\leq i\leq m
  \end{alignat*}}


\subsection{Monopoles}
\index{monopole|(}

When an $n\times n$ matrix $M$ has rank $n-1$, we can use row reduction to obtain a nice integer vector $\vect{v}$ such that $\Span{\vect{v}} = \Nul M$. Thus we can check if a given $\left\{0,1\right\}$-vector is in $\Row M$ by checking if it is orthogonal to $\vect{v}$ and we can count how many $\left\{0,1\right\}$-vectors are in $\Row M$ by counting how many subcollections of $\left\{v_1, \dotsc, v_n\right\}$ sum to $0$. If $M$ has rank less than $n-1$ it is not immediately obvious that we can find a vector with the same properties as $\vect{v}$ but we will establish that such a vector always exists.

\begin{definition}[monopole]
	\index{monopole}
	We say that $\vect{v}\in\mathbb{Z}^n$ is a \defn{monopole} for a matrix $M$ if for all $\left\{0,1\right\}$-vectors $\vect{x}$,\[ \vect{x}\in \Row M\ \text{if and only if}\ \vect{x}\cdot \vect{v} = 0\]

	We say $\vect{v}$ is a \defn{monopole for a graph} $G$ if $\vect{v}$ is a monopole for $\adjm{G}$. 
\end{definition}

\begin{theorem}\label{thm:monopoles exist}Every $\left\{0,1\right\}$-matrix has a monopole.
\end{theorem}
\begin{proof}Let $M$ be a $\left\{0,1\right\}$-matrix with $n$ columns. If $M$ has full rank then $\zerovect$ is a monopole for $M$. Otherwise, let $\left\{ \vect{n_1}, \dotsc, \vect{n_k} \right\}$ be a basis for $\Nul M$ and assume (without loss of generality) that $\vect{n_i}\in \mathbb{Z}^n$ for all $i \in \left[k\right]$.

Define $m_1$, $\dotsc$, $m_k$ by 
	\begin{align*}
		m_j &= \max_{\vect{x} \in \left\{0,1\right\}^n}\left\{\left|\vect{x}\cdot\vect{n_j}\right|\right\}\\
			&= \max\left\{\left(\text{sum of the positive entries of $\vect{n_j}$}\right),\left|\text{sum of the negative entries of $\vect{n_j}$}\right|\right\}
	\end{align*}
and define $c_1$, $\dotsc$, $c_k$ recursively by
	\begin{align*}
		c_1 &= 1		&c_j &= 1+ \sum_{i=1}^{j-1}c_i\cdot m_i
	\end{align*}
Now let $\vect{v} \in \mathbb{Z}^n$ be defined as \[ \vect{v} = \sum_{i=1}^{k}c_i\cdot \vect{n_i} \]
We will show that $\vect{v}$ is a monopole for $M$. Let $\vect{x} \in \left\{0,1\right\}^n$.

 Assume $\vect{x}\in \Row M$. By \autoref{thm:Row orth Nul} $\vect{x}\cdot\vect{n_j} = 0$ for all $j\in \left[k\right]$.
	\begin{align*}
		\vect{x}\cdot\vect{v} &= \vect{x}\cdot\left(\sum_{i=1}^{k}c_i\cdot \vect{n_i}\right)\\
		&= \sum_{i=1}^{k}c_i \cdot \vect{x}\cdot \vect{n_i}\\
		&= 0
	\end{align*}

Now assume $\vect{x}\cdot\vect{v} = 0$. By \autoref{thm:Row orth Nul} it suffices to show $\vect{x}\cdot\vect{n_j} = 0$ for all $j\in \left[k\right]$. Suppose not. Let $l = \max\left\{i\mid \vect{x}\cdot \vect{n_i} \neq 0\right\}$.
	\begin{align*}
		0 &= \vect{x}\cdot\vect{v} = \vect{x}\cdot \left(\sum_{i=1}^{k}c_i \cdot \vect{n_i}\right)\\
		0 &=  \sum_{i=1}^{l-1}c_i \cdot\vect{x}\cdot\vect{n_i} + c_l\cdot \vect{x}\cdot\vect{n_l} + 0
	\end{align*}
Therefore
	\begin{align*}
		c_l	&\leq \left| c_l\cdot \vect{x}\cdot \vect{n_l}\right|= \left| \sum_{i=1}^{l-1}c_i\cdot \vect{x}\cdot \vect{n_i}\right|\\
			&\leq \sum_{i=1}^{l-1}c_i\cdot \left|\vect{x}\cdot\vect{n_i}\right|\ \text{by the triangle inequality}\\
			&\leq \sum_{i=1}^{l-1}c_i\cdot m_i\\
			&< c_l
	\end{align*}
	a contradiction. Thus $\vect{x}\cdot\vect{n_j} = 0$ for all $j\in \left[k\right]$.
\end{proof}
\index{monopole|)}

\begin{example}\label{ex:first monopole}Neither $M$ nor $M^T$ has any hood vectors.
\[ \linespread{1}\selectfont M = 
	\begin{array}{|@{}c@{}c@{}|@{}c@{}|}\hline
		\begin{array}{cc|}
			1	& 1\\
			1	& 1\\\hline
		\end{array}
		&\begin{array}{cc|c}
			\AJbf1	& \AJbf0	& 1\\
			\AJbf0	& \AJbf1	& 1\\\hline
		\end{array}
		&\begin{array}{cc}
			1	& 0\\
			0	& 1
		\end{array}\\
		\begin{array}{cc|}
			\AJbf0	& \AJbf1\\
			\AJbf1	& \AJbf0\\\hline
			1	& 1
		\end{array}
		&\begin{array}{ccc}
			0	& 0	& 0\\
			0	& 0	& 0\\
			0	& 0	& 0
		\end{array}
		&\begin{array}{cc}
			1	& 0\\
			0	& 1\\\hline
			1	& 1
		\end{array}\\\hline
		\begin{array}{cc}
			1	& 0\\
			0	& 1
		\end{array}
		&\begin{array}{cc|c}
			1	& 0	& 1\\
			0	& 1	& 1
		\end{array}
		&\begin{array}{cc}
			0	& 0\\
			0	& 0
		\end{array}\\\hline
	\end{array}
\]
\end{example}
\begin{proof}In reduced row echelon form $M$ is

\[	\linespread{1}\selectfont% M = 
	\begin{array}{c}
	\begin{array}{r*{6}{x{1.1em}}}
		a	& b	& c	& d	& e	& f		& g
	\end{array}\\
	\left[ \begin{array}{r*{6}{x{1.1em}}}
		1	& 0	& 0	& 0	& 0	& 0		& 1\\
		0	& 1	& 0	& 0	& 0	& 1		& 0\\
		0	& 0	& 1	& 0	& 1	& 0		& \ntv 1\\
		0	& 0	& 0	& 1	& 1	& \ntv 1	& 0\\
		0	& 0	& 0	& 0	& 0	& 0		& 0\\
		0	& 0	& 0	& 0	& 0	& 0		& 0\\
		0	& 0	& 0	& 0	& 0	& 0		& 0
	\end{array} \right]
	\end{array}
\]
\begin{align*}
	a &= \ntv g	&c &= \ntv e + g\\
	b &= \ntv f	&d &= \ntv e + f
\end{align*}
Where $e,f,g$ are independent variables. By choosing
\begin{align*}
	e &= 3^0 = 1		&f &= 3^1 = 3		&g &= 3^2 = 9
\end{align*}
We obtain the monopole
\[\linespread{1}\selectfont
	\begin{array}{r@{}*{7}{c}@{}l}
		& a		& b			& c			& d	& e		& f		& g\\
	\vect{n} = \big\langle%
		&\ntv 9,	&\ntv 3,	&8,	&2,	&1,	&3,	&9%
	&\big\rangle
	\end{array}
\]
\[ \linespread{1}\selectfont M = \left[
	\begin{array}{*{7}{c}}
		1	& 1	& 1	& 0	& 1	& 1	& 0\\
		1	& 1	& 0	& 1	& 1	& 0	& 1\\
		0	& 1	& 0	& 0	& 0	& 1	& 0\\
		1	& 0	& 0	& 0	& 0	& 0	& 1\\
		1	& 1	& 0	& 0	& 0	& 1	& 1\\
		1	& 0	& 1	& 0	& 1	& 0	& 0\\
		0	& 1	& 0	& 1	& 1	& 0	& 0\\
	\end{array} \right]
\]
It is easy to see from the construction that $\vect{n}$ is orthogonal to every row of $M$.  Suppose $\vect{a} = \left\langle a_i\right\rangle$ is a $\left\{0,1\right\}$-vector orthogonal to $\vect{n}$. Then
	\[
		\vect{a}\cdot \vect{n} = \ntv 9a_1 + \ntv 3a_2 + 8a_3 + 2a_4 + 1a_5 + 3a_6 + 9a_7 = 0\]
	and so \begin{equation}\label{eq:logarithmic example 1}%
		8a_3 + 2a_4 + 1a_5 + 3a_6 + 9a_7 = 9a_1 + 3a_2 \end{equation}
	If we can show that \autoref{eq:logarithmic example 1} has exactly seven non-zero solutions then it follows that $M$ has no hood vectors. We do not count the solution where both sides sum to zero, since that represents $\zerovect\cdot \vect{v} = 0$. The right-hand side can sum to $0$, $3$, $9$, or $12$ in exactly one way each.
	\begin{center}\linespread{1}\selectfont\par\null\par\begin{tabular}{r|r}
		\multicolumn{2}{c}{Right}\\\hline
		sum	& ways\\\hline
		0	& 1\\
		3	& 1\\
		9	& 1\\
		12	& 1
	\end{tabular}\\\null\end{center}
	
 We will construct a similar table for the left-hand side by considering each term in sequence. At first, the only possible sum is $0$. After considering $8$, we have one way to get $0$ and one way to get $8$. When considering $x$, to determine the number of ways to get $i$, we just need to consider how many ways we could get $i$ or $i-x$ in the previous step. \Autoref{fig:dynamic sum} illustrates this process. After considering all the terms on the left-hand side we will have a table similar to the one we obtained for the right-hand side.

\newcommand{\drawdynprog}[3]{% \drawdynprog[step][max sum][entry list]
	
	\pgfmathsetmacro{\yskip}{\ys*#1}
	%\foreach \v in {0,...,23}\node[vlab] at (\xs*\v+0.5*\xs,0.75-\yskip){$\v$};
	\draw (0,-\yskip) grid[xstep=\xs,ystep=0.5] (\xs+\xs*#2,1-\yskip);
	\node[vlab] at (-0.5,0.7-\yskip){sum};
	\node[vlab] at (-0.5,0.2-\yskip){ways};
	\foreach \v/\w in {#3}{
		\node[vlab] (s#1_\v) at (\xs*\v+0.5*\xs,0.75-\yskip){$\v$};
		\node[vlab] (w#1_\v) at (\xs*\v+0.5*\xs,0.25-\yskip){$\w$};
	}
}
\begin{figure}[htb]%
\begin{ctikzpicture}
	\pgfmathsetmacro{\xs}{0.5}
	\pgfmathsetmacro{\ys}{3}
	\pgfmathsetmacro{\rmin}{0.15}
	\pgfmathsetmacro{\rmax}{0.75}
	
	\drawdynprog{0}{0}{0/1};

	\drawdynprog{1}{8}{%
		 0/1,  1/0,  2/0,  3/0,  4/0,  5/0,  6/0,  7/0,  8/1%
	};

	\pgfmathsetmacro{\nmax}{1}
	\pgfmathtruncatemacro{\ast}{0}\pgfmathtruncatemacro{\asp}{\ast+1}
	\foreach \u in {0}{
		\pgfmathtruncatemacro{\up}{\u+8}
		\pgfmathsetmacro{\r}{\rmax-0.5*(\rmax-\rmin)}
		\draw[-latex,semithick] ($(w\ast_\u.south)+(0.25*\xs,0)$) %
			|- ($(w\ast_\u.south)+(0.5*\xs,0)+(s\asp_0.north)-(s\asp_0.north)!\r!(w\ast_0.south)$) %
			-| ($(s\asp_\up.north)+(-0.25*\xs,0)$);
		\draw[-latex,semithick] (w\ast_\u.south) -- (s\asp_\u.north);
	};

	\drawdynprog{2}{10}{%
		 0/1,  1/0,  2/1,  3/0,  4/0,  5/0,  6/0,  7/0,  8/1,  9/0, 10/1%
	};

	\pgfmathsetmacro{\nmax}{3}
	\pgfmathtruncatemacro{\ast}{1}\pgfmathtruncatemacro{\asp}{\ast+1}
	\foreach \u in {0,8}{
		\pgfmathtruncatemacro{\up}{\u+2}
		\pgfmathsetmacro{\r}{\rmax-0.5*(\rmax-\rmin)}
		\draw[-latex,semithick] ($(w\ast_\u.south)+(0.25*\xs,0)$) %
			|- ($(w\ast_\u.south)+(0.5*\xs,0)+(s\asp_1.north)-(s\asp_1.north)!\r!(w\ast_1.south)$) %
			-| ($(s\asp_\up.north)+(-0.25*\xs,0)$);
		\draw[-latex,semithick] (w\ast_\u.south) -- (s\asp_\u.north);
	};

	\drawdynprog{3}{11}{%
		 0/1,  1/1,  2/1,  3/1,  4/0,  5/0,  6/0,  7/0,  8/1,  9/1, 10/1, 11/1%
	};

	\pgfmathsetmacro{\nmax}{3}
	\pgfmathtruncatemacro{\ast}{2}\pgfmathtruncatemacro{\asp}{\ast+1}
	\foreach \u in {0,2,8,10}{
		\pgfmathtruncatemacro{\up}{\u+1}
		\pgfmathsetmacro{\r}{\rmax-0.5*(\rmax-\rmin)}
		\draw[-latex,semithick] ($(w\ast_\u.south)+(0.25*\xs,0)$) %
			|- ($(w\ast_\u.south)+(0.5*\xs,0)+(s\asp_1.north)-(s\asp_1.north)!\r!(w\ast_1.south)$) %
			-| ($(s\asp_\up.north)+(-0.25*\xs,0)$);
		\draw[-latex,semithick] (w\ast_\u.south) -- (s\asp_\u.north);
	};

	\drawdynprog{4}{14}{%
		0/1,  1/1,  2/1,  3/2,  4/1,  5/1,  6/1,  7/0,  8/1,  9/1, 10/1, 11/2,%
		12/1, 13/1, 14/1%
	};

	
	\pgfmathsetmacro{\nmax}{4}
	\pgfmathtruncatemacro{\ast}{3}\pgfmathtruncatemacro{\asp}{\ast+1}
	\foreach \u in {0,1,2,3,8,9,10,11}{
		\pgfmathtruncatemacro{\up}{\u+3}
		\pgfmathsetmacro{\r}{\rmax-(\rmax-\rmin)/(\nmax-1)*mod(\u,4)}%{0.94-0.08*\u}
		\draw[-latex,semithick] ($(w\ast_\u.south)+(0.25*\xs,0)$) %
			|- ($(w\ast_\u.south)+(0.5*\xs,0)+(s\asp_1.north)-(s\asp_1.north)!\r!(w\ast_1.south)$) %
			-| ($(s\asp_\up.north)+(-0.25*\xs,0)$);
		\draw[-latex,semithick] (w\ast_\u.south) -- (s\asp_\u.north);
	};

	\drawdynprog{5}{23}{%
		0/1,  1/1,  2/1,  3/2,  4/1,  5/1,  6/1,  7/0,  8/1,  9/2, 10/2, 11/3,%
		12/3, 13/2, 14/2, 15/1, 16/0, 17/1, 18/1, 19/1, 20/2, 21/1, 22/1, 23/1%
	};

	\pgfmathsetmacro{\nmax}{15}
	\pgfmathtruncatemacro{\ast}{4}\pgfmathtruncatemacro{\asp}{\ast+1}
	\foreach \u in {0,1,2,3,4,5,6,8,9,10,11,12,13,14}{
		\pgfmathtruncatemacro{\up}{\u+9}
		\pgfmathsetmacro{\r}{\rmax-(\rmax-\rmin)/(\nmax-1)*\u}
		\draw[-latex,semithick] ($(w\ast_\u.south)+(0.25*\xs,0)$) %
			|- ($(w\ast_\u.south)+(0.5*\xs,0)+(s\asp_1.north)-(s\asp_1.north)!\r!(w\ast_1.south)$) %
			-| ($(s\asp_\up.north)+(-0.25*\xs,0)$);
		\draw[-latex,semithick] (w\ast_\u.south) -- (s\asp_\u.north);
	};

	\node[vlab,anchor=east] at (0,-0.15+\ys*0.5){start};
	\foreach \a/\b in {1/8, 2/2, 3/1, 4/3, 5/9} \node[vlab,anchor=east] at (0,{0.5+\ys*(0.5-\a)}){consider $\b$};
\end{ctikzpicture}
\caption{Building the table of possible sums using dynamic programming}\label{fig:dynamic sum}
\end{figure}

For each possible sum $i$, let $l_i$ be the number of ways to obtain the sum $i$ on the left and $r_i$ be the number of ways to obtain the sum $i$ on the right. The total number of solutions to \autoref{eq:logarithmic example 1} will then be \[ \sum_{i}r_i\cdot l_i \]

\begin{center}\begin{tikzpicture}
	\pgfmathsetmacro{\xs}{0.5}
	\pgfmathsetmacro{\ys}{2}
	\pgfmathsetmacro{\rmin}{0.15}
	\pgfmathsetmacro{\rmax}{0.75}

	\drawdynprog{0}{23}{%
		0/1,  1/1,  2/1,  3/2,  4/1,  5/1,  6/1,  7/0,  8/1,  9/2, 10/2, 11/3,%
		12/3, 13/2, 14/2, 15/1, 16/0, 17/1, 18/1, 19/1, 20/2, 21/1, 22/1, 23/1%
	};
	\node[vlab,anchor=east] at (0,1.2){left};

	\drawdynprog{1}{12}{%
		0/1,  1/0,  2/0,  3/1,  4/0,  5/0,  6/0,  7/0,  8/0,  9/1, 10/0, 11/0, 12/1%
	};

	\foreach \v in {3,9,12}{
		\draw[very thick] (w1_\v.south west) rectangle ($(s0_\v.north)+(0.5*\xs,0)$);
	};
	\node[vlab,anchor=east] at (0,-0.8){right};
\end{tikzpicture}\end{center}
Thus in this case the number of valid solutions is $2\cdot 1 + 2\cdot 1 + 3\cdot 1 = 7$. Therefore $M$ has no hood vectors. The argument for $M^T$ is similar.
\end{proof}

We can describe the general version of the technique we used in \autoref{ex:first monopole} to count vectors orthogonal to the monopole, but first let us establish some simple facts about the structure of monopoles for counterexamples to \autoref{conj:Adjacency Matrix} and \autoref{conj:Symmetric Matrix}.

\begin{lemma}\label{lem:symmetric row wt 1}Let $n \geq 2$ and let $M$ be a $n\times n$ symmetric $\left\{0,1\right\}$-matrix. If $M$ has a row of weight $1$, then $M$ has a hood vector.
\end{lemma}
\begin{proof}
	To obtain a contradiction assume $M$ has no hood vectors. Let $\vect{v}$ be the row of weight $1$. For each row $\vect{u}\neq \vect{v}$, either $\vect{u}+\vect{v}$ or $\vect{u}-\vect{v}$ is also a row of $M$. Thus every column of $M$ must have weight $0$ or at least $2$. But $M$ has a row of weight $1$, contradicting that $M$ is symmetric.
\end{proof}

\begin{proposition}\label{prop:monopole nonzero}
	Let $n \geq 2$ and let $M$ be an $n\times n$ symmetric $\left\{0,1\right\}$-matrix. Let $\vect{v}=\left\langle v_1, \ldots, v_n\right\rangle$ be a monopole for $M$. If $v_i = 0$ for some $i \in \left[n\right]$ then $M$ has a hood vector.
\end{proposition}
\begin{proof}%[Proof of \autoref{prop:monopole nonzero}]
	Without loss of generality we may assume $v_n=0$. Since $\vect{v}\cdot \left\langle 0,0,\ldots, 0,1\right\rangle = 0$, $M$ has a row of weight $1$. Thus $M$ has a hood vector by \autoref{lem:symmetric row wt 1}.
\end{proof}

Therefore the monopoles of any potential counterexample to \autoref{conj:Adjacency Matrix} or \autoref{conj:Symmetric Matrix} cannot have a $0$ in any position.

\begin{proposition}\label{prop:monopole more than one negative}Let $M$ be a $\left\{0,1\right\}$-matrix with no row equal to $\zerovect$ and let $\vect{v}$ be a monopole for $M$. If $\vect{v}$ does not have $0$ in any position and has exactly one negative (or exactly one positive) entry then $M$ has $\onevect$ as a column.
\end{proposition}
\begin{proof}Without loss of generality assume $v_n$ is the only negative entry of $\vect{v}$. Let $\vect{r}$ be a row of $M$. We need to show $r_n = 1$.
	\begin{align}
		r_1\cdot v_1 + \dotsb + r_{n-1}\cdot v_{n-1} + r_n\cdot v_n &= 0\nonumber\\
		r_1\cdot v_1 + \dotsb + r_{n-1}\cdot v_{n-1} &= \ntv r_n\cdot v_n\label{eq:one negative}
	\end{align}
	Since $\vect{r}\neq \zerovect$, both sides of \autoref{eq:one negative} must be nonzero. But the only way the right-hand side is nonzero is if $r_n = 1$. Therefore the $n$th entry of every row of $M$ is a $1$ and so the $n$th column of $M$ is $\onevect$.
\end{proof}

Since $\onevect$ cannot be a column of the adjacency matrix of a graph, any monopole for a counterexample to \autoref{conj:Adjacency Matrix} must have at least two negative (and symmetrically, at least two positive) entries.

Returning to the problem of counting hood vectors, suppose we have a monopole $\vect{v} = \left\langle v_1, \dotsc, v_k, \ntv v_{k+1}, \dotsc, \ntv v_n\right\rangle$. Without loss of generality suppose $v_1 \geq v_2 \geq \dotsm \geq v_k > 0$ and $v_{k+1} \geq v_{k+2} \geq \dotsm \geq v_n > 0$ (recall that \autoref{prop:monopole nonzero} guarantees $v_i \neq 0$). Given $\vect{a} = \left\langle a_i\right\rangle$ is a $\left\{0,1\right\}$-vector orthogonal to $\vect{n}$, we have%
\begin{equation}\label{eq:knapsack}
	v_1 a_1 + \dotsm + v_k a_k = v_{k+1} a_{k+1} + \dotsm + v_n a_n
\end{equation}

Thus the number of solutions to \autoref{eq:knapsack} is equal to the number of ways in which a subcollection of $\left\{ v_1, \dotsc, v_k \right\}$ sums to the same value as a subcollection of $\left\{ v_{k+1}, \dotsc, v_n \right\}$. Phrased this way we have the enumeration version of the \emph{subset sum problem}, a variant of the well-studied \emph{knapsack problem}. This problem admits a pseudo-polynomial time algorithm via dynamic programming---the very algorithm we used in \autoref{ex:first monopole} and illustrated in \autoref{fig:dynamic sum}.

\begin{example}\label{ex:logarithmic}Neither $M$ nor $M^T$ has any hood vectors. The bolded entries are the only asymmetries in $M$.
\[ \linespread{1}\selectfont M = 
	\begin{array}{|@{}c@{}c@{}|@{}c@{}|}\hline
		\begin{array}{cccc|}
			1	&1	&0	&0\\
			1	&1	&0	&0\\
			0	&0	&1	&1\\
			0	&0	&1	&1\\\hline
		\end{array}
		&\begin{array}{cc|ccc}
			\AJbf1	&\AJbf0	&1	&0	&1\\
			\AJbf0	&\AJbf1	&1	&0	&1\\
			\AJbf1	&\AJbf0	&0	&1	&1\\
			\AJbf0	&\AJbf1	&0	&1	&1\\\hline
		\end{array}&\begin{array}{cccc}
			1	&0	&1	&0\\
			0	&1	&0	&1\\
			1	&0	&1	&0\\
			0	&1	&0	&1\\
		\end{array}\\
		\begin{array}{cccc|}
			\AJbf0	&\AJbf1	&\AJbf0	&\AJbf1\\
			\AJbf1	&\AJbf0	&\AJbf1	&\AJbf0\\\hline
			1	&1	&0	&0\\
			0	&0	&1	&1\\
			1	&1	&1	&1\\
		\end{array}&\begin{array}{ccccc}
			0	&0	&0	&0	&0\\
			0	&0	&0	&0	&0\\
			0	&0	&0	&0	&0\\
			0	&0	&0	&0	&0\\
			0	&0	&0	&0	&0\\
		\end{array}&\begin{array}{cccc}
			1	&0	&1	&0\\
			0	&1	&0	&1\\\hline 
			1	&1	&0	&0\\
			0	&0	&1	&1\\
			1	&1	&1	&1\\
		\end{array}\\\hline
		\begin{array}{cccc}
			1	&0	&1	&0\\
			0	&1	&0	&1\\
			1	&0	&1	&0\\
			0	&1	&0	&1\\
		\end{array}&\begin{array}{cc|ccc}
			1	&0	&1	&0	&1\\
			0	&1	&1	&0	&1\\
			1	&0	&0	&1	&1\\
			0	&1	&0	&1	&1\\
		\end{array}&\begin{array}{cccc}
			0	&0	&1	&1\\
			0	&0	&1	&1\\
			1	&1	&0	&0\\
			1	&1	&0	&0\\
		\end{array}\\\hline
		\multicolumn{1}{@{}c@{}}{\begin{array}{cccc}a&b&c&d\end{array}}
		&\multicolumn{1}{@{}c@{}}{\begin{array}{ccccc}e&f&g&h&i\end{array}}
		&\multicolumn{1}{@{}c@{}}{\begin{array}{cccc}j&k&l&m\end{array}}
	\end{array}
\]
\end{example}
\begin{proof}In reduced row echelon form $M$ is

\[	\linespread{1}\selectfont% M = 
	\begin{array}{c}
	\begin{array}{r*{12}{x{1.1em}}}
		a	& b	& c	& d		& e	& f	& g	& h		& i	& j		& k		& l		& m
	\end{array}\\
	\left[ \begin{array}{r*{12}{x{1.1em}}}
		1	& 0	& 0	& \ntv 1	& 0	& 0	& 0	& 0		& 0	& 0		& 1		& \ntv 1	& 0\\
		0	& 1	& 0	& 1		& 0	& 0	& 0	& 0		& 0	& 1		& 0		& 1		& 0\\
		0	& 0	& 1	& 1		& 0	& 0	& 0	& 0		& 0	& 0		& 0		& 1		& 1\\
		0	& 0	& 0	& 0		& 1	& 0	& 0	& 1		& 1	& 1		& 0		& 0		& \ntv 1\\
		0	& 0	& 0	& 0		& 0	& 1	& 0	& 1		& 1	& 0		& 1		& \ntv 1	& 0\\
		0	& 0	& 0	& 0		& 0	& 0	& 1	& \ntv 1	& 0	& \ntv 1	& \ntv 1	& 1		& 1\\
		0	& 0	& 0	& 0		& 0	& 0	& 0	& 0		& 0	& 0		& 0		& 0		& 0\\
		0	& 0	& 0	& 0		& 0	& 0	& 0	& 0		& 0	& 0		& 0		& 0		& 0\\
		0	& 0	& 0	& 0		& 0	& 0	& 0	& 0		& 0	& 0		& 0		& 0		& 0\\
		0	& 0	& 0	& 0		& 0	& 0	& 0	& 0		& 0	& 0		& 0		& 0		& 0\\
		0	& 0	& 0	& 0		& 0	& 0	& 0	& 0		& 0	& 0		& 0		& 0		& 0\\
		0	& 0	& 0	& 0		& 0	& 0	& 0	& 0		& 0	& 0		& 0		& 0		& 0\\
		0	& 0	& 0	& 0		& 0	& 0	& 0	& 0		& 0	& 0		& 0		& 0		& 0
	\end{array} \right]
	\end{array}
\]
This corresponds to 
\begin{align*}
	a	&= \pntv d - k + l		&e	&= \ntv h - i - j + m\\
	b	&= \ntv d - j - l		&f	&= \ntv h - i - k + l\\
	c	&= \ntv d - l - m		&g	&= \pntv h + j + k - l - m
\end{align*}
Where $d, h, i, j, k, l, m$ are independent variables. By choosing
\begin{align*}
	d &= 3^0 = 1		&j &= 3^3 = 27	&m &= 3^6 = 729\\
	h &= 3^1 = 3		&k &= 3^4 = 81\\
	i &= 3^2 = 9		&l &= 3^5 = 243
\end{align*}
We obtain the monopole
\[\linespread{1}\selectfont
	\begin{array}{r@{}*{13}{c}@{}l}
& a		& b			& c			& d	& e		& f		& g			& h	& i	& j		& k	& l			& m\\
	\big\langle%
& 163,	& \ntv 271,	& \ntv 973,	& 1,	& 690,	& 150,	& \ntv 861,	& 3,	& 9,	& 27,	& 81,	& 243,	& 729%
	&\big\rangle
	\end{array}
\]
Verifying that there are exactly thirteen $\left\{0,1\right\}$-vectors orthogonal to this monopole is left as an exercise to the reader.
\end{proof}

The matrices in \autoref{ex:first monopole} and \autoref{ex:logarithmic} both come from a larger family of matrices without hood vectors.

\begin{example}\label{ex:logarithmic family}Neither $M_k$ nor $M_k^T$ has any hood vectors.

	Let $D_{k}$ be the $2k \times 2k$ matrix with $2\times 2$ blocks of $1$s along the main diagonal and $0$ elsewhere. For example,
\[ \linespread{1}\selectfont D_3 =
	\left[\begin{array}{cccccc}
		1	&1	&0	&0	&0	&0\\
		1	&1	&0	&0	&0	&0\\
		0	&0	&1	&1	&0	&0\\
		0	&0	&1	&1	&0	&0\\
		0	&0	&0	&0	&1	&1\\
		0	&0	&0	&0	&1	&1
	\end{array}\right]
\]

	Let $A_k$ be the $2k \times 2^k -1$ matrix whose columns are indicator vectors for a nonempty subset of $\left[k\right]$ with each entry repeated twice. For example,
\[ \linespread{1}\selectfont A_3 =
	\begin{array}{c}
		\begin{array}{@{}*{7}{w{3.5em}@{}}}
			\left\{1\right\}	&\left\{2\right\}	&\left\{3\right\}	&\left\{1,2\right\}	&\left\{1,3\right\}	&\left\{2,3\right\}	&\left\{1,2,3\right\}
		\end{array}\\
		\left[\begin{array}{@{}*{7}{w{3.5em}@{}}}
			1	&0	&0	&1	&1	&0	&1\\
			1	&0	&0	&1	&1	&0	&1\\
			0	&1	&0	&1	&0	&1	&1\\
			0	&1	&0	&1	&0	&1	&1\\
			0	&0	&1	&0	&1	&1	&1\\
			0	&0	&1	&0	&1	&1	&1\\
		\end{array}\right]
	\end{array}
\]

	Let $01_{m\times n}$ be the $m\times n$ matrix whose entries alternate $0$ and $1$ in both row and column and with $0$ in the upper-left entry. Let $10_{m\times n}$ be the same except with $1$ in the upper-left entry. For example,
\[ \linespread{1}\selectfont 01_{3\times 4} = 
	\left[ \begin{array}{cccc}
		0	&1	&0	&1\\
		1	&0	&1	&0\\
		0	&1	&0	&1
	\end{array}\right]
\]

Recall that $J$ is the all-ones matrix. For a given $k$,
\[ M_k = \begin{tikzpicture}[baseline=(c)]
	\foreach \xl/\xv in {a/0, b/2, c/3.25, d/5, e/7}{
		\foreach \yl\yv in {a/0, b/2, c/4, d/5, e/7}{
			\coordinate (\xl\yl) at (\xv,\yv){};
		};
	};
	%\draw[help lines] (aa) grid (ee);
	\draw (aa) rectangle (ee);
	\coordinate (c) at ($(aa)!0.5!(ee)+(0,-6 pt)$){};
	\draw (ad) -- (dd);
	\draw (ac) -- (bc) (dc) -- (ec);
	\draw (ab) -- (eb);
	\draw (bb) -- (be);
	\draw (ca) -- (cb) (cd) -- (ce);
	\draw (da) -- (de);
	

	\node[vlab] at ($(ad)!0.5!(be)$){$D_k$};
	\node[vlab] at ($(bd)!0.5!(ce)$){$10_{2k\times 2}$};
	\node[vlab] at ($(cd)!0.5!(de)$){$A_k$};
	\node[vlab] at ($(dc)!0.5!(ee)$){$10_{2k+2\times 2k}$};
	\node[vlab] at ($(ac)!0.5!(bd)$){$01_{2\times 2k}$};
	\node[vlab] at ($(ab)!0.5!(bc)$){$A_k^T$};
	\node[vlab] at ($(bb)!0.5!(dd)$){\Large $0$};
	\node[vlab] at ($(db)!0.5!(ec)$){$A_k^T$};
	\node[vlab] at ($(aa)!0.5!(cb)$){$10_{2k\times 2k+2}$};
	\node[vlab] at ($(ca)!0.5!(db)$){$A_k$};
	\node[vlab] at ($(da)!0.5!(eb)$){$J-D_k$};
	\end{tikzpicture}
\]
\end{example}

\begin{example}\label{ex:nonbinary logarithmic family}For each member of the family described in \autoref{ex:logarithmic family}, the matrix obtained by replacing the upper-left and lower-right blocks with $\frac{1}{2}J$ and swapping columns $2k+1$ and $2k+2$ yields a symmetric matrix with no hood vectors.
\end{example}

\section{Conclusion}

Thus far we have been unable to transform the families of matrices described in \autoref{ex:logarithmic family} and \autoref{ex:nonbinary logarithmic family} into a counterexample to \autoref{conj:Symmetric Matrix}. However we believe that \autoref{conj:Symmetric Matrix} is false and that some variation of these families should provide the counterexample, and possibly even a counterexample to \autoref{conj:Adjacency Matrix}. If \autoref{conj:Adjacency Matrix} or even \autoref{conj:Symmetric Matrix} is true, \autoref{ex:logarithmic family} and \autoref{ex:nonbinary logarithmic family} demonstrate that it is true by a very narrow margin. Coupled with Costello and Vu's result \cite{CoVu} about random graphs, it is understandable that this problem is so resistant to attack.

Monopoles provide an interesting side channel for attack. Given a monopole, we can count the number of $\left\{0,1\right\}$-vectors orthogonal to the monopole using the dynamic programming approach. In practice this algorithm can test hundreds of millions of monopoles per second on a desktop computer. With enough information about the structure of the monopoles of a potential counterexample we may be able to reduce the search space to a feasible size for a brute-force search. \Autoref{prop:monopole nonzero} and \autoref{prop:monopole more than one negative} are first steps in this direction. If a monopole has very few negative (or symmetrically, very few positive) entries then the search algorithm can be made even more efficient by limiting the growth of the dynamic programming table. \Autoref{prop:monopole more than one negative} showed us that we cannot hope to have just a single negative entry, but will two suffice? Or perhaps three? In general a constant bound is unlikely, but perhaps there is a reasonable bound if we restrict ourselves to graphs of order at most $n$.
%!TEX root=../Dissertation.tex
\newcommand*{\key}[2]{\ensuremath{K_{#1}^{#2}}}%
\tikzstyle{cvertex}=[draw,circle,fill=white,text=black,inner sep=1pt]%
\tikzstyle{oriented}=[line width=1pt]%
\chapter{QUERYING: YAO'S PROBLEM}

Many fundamental problems in computer science involve storage and retrieval of information. Given tables of data,  for instance the names of all the employees of a business, a storage scheme is devised to record the data in some manner. The retrieval aspect may be concerned with whether a certain entry is present in the table; call this question the \index{membership question}\defn{membership question}. Some algorithm is invoked to inspect the data and return a yes or no answer to the membership question.

The na\"{i}ve approach to answering the membership question is to inspect, or \index{query}\defn{query}, every cell of the table to see if it contains the data in question. If the table has $n$ cells and we ask the membership question for an element that is not in the table, then this algorithm will require $n$ queries to decide the membership question. But remember, we also have control over how data is stored. Suppose the table is sorted upon storage. Now a binary search can answer the membership question in at most $\left\lceil\lg\left(n+1\right)\right\rceil$ queries. The extra time required to sort the table will be easily eclipsed by the time saved over multiple searches.

\section{Introduction}

In 1981, Andrew Yao asked, `Should tables be sorted?' \cite{Yao}. In particular, he wondered if some restriction on the data could allow us to answer the membership question in fewer queries than the traditional sort/search described above.\footnote{Binary search on a sorted table is not uniquely the fastest approach. There are other approaches, notably \emph{hashing} or \emph{interpolation search} that will perform better for certain applications but the worst-case scenarios still require a number of queries that is at least logarithmic in $n$.} He supposed that the table contained $n$ elements chosen from a set of $m$ elements (called the key space and denoted $M$), rather than from a potentially infinite set. Let \index{f(n,m)@$f(n,m)$}$f\left(n,m\right)$ be the number of queries needed in the worst case to answer this membership question. He showed that if the key space is significantly larger than the table then the traditional approach is optimal. That is, \[ m \gg n\ \implies\ f\left(n,m\right) = \left\lceil\lg\left(n+1\right)\right\rceil \] But when $m$ and $n$ are relatively close in size another store/query strategy can yield improved performance.

He then looked at the question from a different angle: for a fixed $k$ and fixed $n$, what is the largest $m$ such that we can answer the membership question using at most $k$ queries? Let \index{g(n,k)@$g(n,k)$}$g\left(n,k\right)$ be the largest $m$ such that $f\left(n,m\right) = k$.

To obtain a simple lower bound, consider the case when $m = n+1$; i.e., the table contains all but one element of the key space. Picture the key space as a sorted list wrapping around at the ends. Choose the element that is the successor to the missing element and place it in the first cell in the table. Now we can answer the membership question in one query by querying the first cell in the table. That will tell us which element is missing and by extension, which elements are present. Thus $g\left(n,1\right) \geq n+1$.

However this simple approach does not fully take advantage of the tools at our disposal; we just blindly query the first entry of the table. More sophisticated techniques will yield improvements. Yao answered this question for $k=1$ and so determined $g\left(n,1\right)$.

\begin{theorem}[Yao, \cite{Yao}]\label{thm:Yao one query}~\phantom{hi.} \[ g\left(n,1\right) = \begin{cases}3&\text{if}\ n=2\\ 2n-2&\text{if}\ n>2\end{cases}\]
\end{theorem}
\begin{proof}We will establish that for $n>2$, $g\left(n,1\right) \geq 2n-2$ by providing a store/query strategy. Proving that $2n-2$ is best possible takes up several pages in \cite{Yao} so the argument is omitted, but \autoref{thm:Yao minus 1} will establish that $2n$ is impossible.

Let $m = 2n-2$. To establish a labeling scheme, suppose that we could force more than two keys to fit into a cell of the table. We could then fit the entire key space into the table by placing two keys in each of the first $n-2$ cells, and one key in cells $n-1$ and $n$. Label the keys in cell $i$ by choosing one to be $\key{i}{L}$ (the `lower key') and the other to be $\key{i}{U}$ (the `upper key'). Treat the lone keys in cells $n-1$ and $n$ as lower keys; these cells have no upper key. This gives our labeling for the key space: $M = \left\{\key{1}{L}, \dotsc, \key{n}{L}, \key{1}{U}, \dotsc, \key{n-2}{U}\right\}$.

\begin{ctikzpicture}[scale=1.2]
	\draw (0,0) grid (2,2);
	\foreach \v in {1,2}{
		\node[vlab] at (\v-0.5,-0.3){\small$\v$};
		\node[vlab] at (\v-0.5,0.5){$\key{\v}{L}$};
		\node[vlab] at (\v-0.5,1.5){$\key{\v}{U}$};
	};
	\pgfmathsetmacro{\xs}{3.5}
	\draw (\xs,0) -- +(3,0);
	\draw (\xs,1) -- +(3,0);
	\draw (\xs,2) -- +(1,0);
	\draw (\xs,0) -- +(0,2);
	\draw (\xs+1,0) -- +(0,2);
	\draw (\xs+2,0) -- +(0,1);
	\draw (\xs+3,0) -- +(0,1);

	%\node[vlab] at (1+0.5*\xs,1){\makebox[0pt]{\huge$\dotsb$}};
	\foreach \d in {0,-0.3,0.3}\fill (1+0.5*\xs+\d,0.5) circle (0.05);
	\node[vlab] at (\xs+0.5,-0.3){$n-2$};
	\node[vlab] at (\xs+0.5,0.5){$\key{n-2}{L}$};
	\node[vlab] at (\xs+0.5,1.5){$\key{n-2}{U}$};

	\node[vlab] at (\xs+1.5,-0.3){$n-1$};
	\node[vlab] at (\xs+1.5,0.5){$\key{n-1}{L}$};

	\node[vlab] at (\xs+2.5,-0.3){$n$};
	\node[vlab] at (\xs+2.5,0.5){$\key{n}{L}$};
	\extendtopbound
\end{ctikzpicture}

Given a set $N$ of $n$ keys with which to fill the table, the storage scheme proceeds as follows:
	\begin{enumerate}
		\item\label{en:2n-2 empty cell} For any cell $i$ such that no key with cell label $i$ is present in $N$, mark the cell as available for other use.
		\item\label{en:2n-2 end cell} If the keys with cell label $n-1$ or $n$ are present in $N$ they are added to a list called the shuffle list and the cell is marked as a shuffle cell.
		\item For the first $n-2$ cells:
			\begin{enumerate}
				\item\label{en:2n-2 one cell} If only one of the two keys with that cell label is present in $N$ then that key is stored in the cell.
				\item\label{en:2n-2 two cell} If both keys with that cell label are present in $N$ then by the pigeonhole principle another cell must be marked as available via \ref{en:2n-2 empty cell}. The upper key for the current cell is stored in the available cell. The lower key for the current cell is added to the shuffle list. The current cell is marked as a shuffle cell.
			\end{enumerate}
		\item At this stage, every cell marked as available in \ref{en:2n-2 empty cell} is filled with an upper key with a different cell label. The other cells are either filled with a key with the corresponding cell label or are marked as a shuffle cell.
		\item\label{en:2n-2 shuffle} The shuffle list must have at least two keys, since by the pigeonhole principle for each of $\key{n-1}{L}$ and $\key{n}{L}$ that are not in $N$ there must have been a cell for which both keys with that cell label are present in $N$. Note that every key on the shuffle list is a lower key and that there are the same number of shuffle cells as keys on the shuffle list. Complete the table by assigning the keys from the shuffle list to shuffle cells in such a way that no shuffle cell is assigned a key with that cell label (a cyclic shift will suffice).
	\end{enumerate}
To answer the membership question for a key with cell label $i$, we query cell $i$.
	\begin{itemize}
		\item If cell $i$ holds $\key{i}{L}$ or $\key{i}{U}$ then the key was assigned by \ref{en:2n-2 one cell} and so the other key with that cell label is not in the table.
		\item If cell $i$ holds $\key{j}{U}$, $j \neq i$ then the key was assigned by \ref{en:2n-2 two cell}. Therefore cell $i$ was marked as available by \ref{en:2n-2 empty cell} and so neither $\key{i}{L}$ nor $\key{i}{U}$ is in the table.
		\item If cell $i$ holds $\key{j}{L}$, $j \neq i$ then the key was assigned by \ref{en:2n-2 shuffle}. Therefore cell $i$ was marked as a shuffle cell by \ref{en:2n-2 end cell} or by \ref{en:2n-2 two cell} and so both $\key{i}{L}$ and $\key{i}{U}$ (if $i < n-1$) are present in the table.
	\end{itemize}
Thus $f\left(n,2n-2\right) = 1$ and so $g\left(n,1\right) \geq 2n-2$.
\end{proof}

Although Yao takes several pages to prove the upper bound half of \autoref{thm:Yao one query}, the start of his argument can be adapted into a quick proof leaving a gap of $1$.

\begin{theorem}\label{thm:Yao minus 1}$g\left(n,1\right) \leq 2n-1$
\end{theorem}
\begin{proof}Suppose we have a store/search strategy for a key space containing at least $2n$ keys. By the pigeonhole principle, the search strategies for at least two keys will query the same cell. Without loss of generality, the search for keys $1$ and $2$ both query cell $1$.

Let $Y_1$ (the \defn{`yes set'} for key $1$) be the set of keys whose appearance in cell $1$ resolve the membership question for key $1$ in the affirmative. All keys not in $Y_1$ are in $N_1$, the \defn{`no set'} for key $1$. Define $Y_2$ and $N_2$ similarly.

\begin{claim}$\left|Y_1\right| \leq n$.

If $\left|Y_1\right| > n$, then there is a subset $S$ of $Y_1$ of cardinality $n$ that does not contain key $1$. But if the table is populated with the keys from $S$, then the first cell will contain a key that will falsely imply that key $1$ is in the table, contradicting that this store/search strategy is valid.
\end{claim}

Therefore $\left|N_1\right| \geq n$. Since key $1$ is clearly in $Y_1$, we can populate the table with key $1$ and  $n-1$ keys chosen from $N_1$ without choosing key $2$. The storage scheme must place key $1$ in cell $1$, and so key $1$ must belong to $N_2$. Similarly, key $2$ must belong to $N_1$.

Now populate the table using key $1$, key $2$, and $n-2$ keys chosen from $N_1$. The storage scheme must place key $1$ in cell $1$ since all other keys belong to $N_1$. But since key $1$ is in $N_2$, when we attempt to answer the membership question for key $2$ we will see key $1$ in cell $1$ and falsely imply that key $2$ is not in the table. Thus, this search/store strategy cannot be valid.
\end{proof}

\section{Two queries\label{sec:Yao two queries}}

At his 2010 REGS, Doug West proposed \cite{WestYao} that we try to determine the maximum size of the key space when we are permitted two queries.

If the table has only three cells then we can accommodate a key space of any size by storing the keys in the table in sorted order. With the first query we query the middle cell. If that is the key we are looking for then great, we are done. If that key is larger than the key we are looking for then we query the first cell. Otherwise we query the last cell. Therefore $g\left(3,2\right) = \infty$ and more generally, using a similar sort/search strategy, $g\left(2^k-1,k\right) = \infty$. Thus we only need to consider tables with at least four cells.

\begin{proposition}\label{prop:two queries 3n-4}For $n \geq 4$, $g\left(n,2\right) \geq 3n-4$.
\end{proposition}
\begin{proof}Start by labeling all the keys in $M$ using a technique similar to Yao's from \autoref{thm:Yao one query}. For $i \in \left[n-2\right]$, three keys will have cell label $i$: a `lower key' $\key{i}{L}$, a `middle key' $\key{i}{M}$, and an `upper key' $\key{i}{U}$. For $i\in \left\{n-1,n\right\}$ only one key will have cell label $i$: $\key{n-1}{U}$ and $\key{n}{M}$.
\begin{ctikzpicture}[scale=1.2]
	\draw (0,0) grid (2,3);
	\foreach \v in {1,2}{
		\node[vlab] at (\v-0.5,-0.3){\small$\v$};
		\node[vlab] at (\v-0.5,0.5){$\key{\v}{L}$};
		\node[vlab] at (\v-0.5,1.5){$\key{\v}{M}$};
		\node[vlab] at (\v-0.5,2.5){$\key{\v}{U}$};
	};
	\pgfmathsetmacro{\xs}{3.5}
	\foreach \y in {0,1,2,3}\draw (\xs,\y) -- +(1,0);
	\draw (\xs+2,1) -- +(1,0);
	\draw (\xs+1,2) -- +(2,0);
	\draw (\xs+1,3) -- +(1,0);
	\draw (\xs,0) -- +(0,3);
	\draw (\xs+1,0) -- +(0,3);
	\draw (\xs+2,1) -- +(0,2);
	\draw (\xs+3,1) -- +(0,1);

	\foreach \d in {0,-0.3,0.3}\fill (1+0.5*\xs+\d,0.5) circle (0.05);
	\node[vlab] at (\xs+0.5,-0.3){$n-2$};
	\node[vlab] at (\xs+0.5,0.5){$\key{n-2}{L}$};
	\node[vlab] at (\xs+0.5,1.5){$\key{n-2}{M}$};
	\node[vlab] at (\xs+0.5,2.5){$\key{n-2}{U}$};

	\node[vlab] at (\xs+1.5,-0.3){$n-1$};
	\node[vlab] at (\xs+1.5,2.5){$\key{n-1}{U}$};

	\node[vlab] at (\xs+2.5,-0.3){$n$};
	\node[vlab] at (\xs+2.5,1.5){$\key{n}{M}$};
	\extendtopbound
\end{ctikzpicture}

Given a set $N$ of $n$ keys from the key space, the storage scheme works as follows:
\begin{enumerate}
	\item Examine the first $n-2$ cells.
	\begin{enumerate}
		\item If no keys in $N$ have this cell label, then the cell is flagged as available.
		\item\label{en:3n-4 only key} If only one key in $N$ has this cell label then that key is assigned to the cell.
		\item If two of more keys in $N$ have this cell label then we classify the cell by type:
		\begin{ctikzpicture}%[scale=1.2]
			\foreach \x in {0,2,4,6}\draw (\x,0) grid +(1,3);
			\foreach \x/\y/\lab in {0/0/L,0/1/M,0/2/U,2/0/L,2/2/U,4/0/L,4/1/M,6/1/M,6/2/U}\node[vlab] at (\x+0.5,\y+0.5){$\key{i}{\lab}$};
			\foreach \x/\ty in {0/A,2/B,4/C,6/D}\node[vlab] at (\x+0.5,3.3){Type $\ty$};
			%\extendtopbound
		\end{ctikzpicture}
	\end{enumerate}
	\item Pair up cells of type $A$ and $B$ with other cells of type $A$ and $B$.
		\begin{ctikzpicture}%[scale=1.2]
			\foreach \x in {0,2,5,7,10,12}{
				\draw (\x,0) grid +(1,3);
				\draw (\x,-2) rectangle +(1,1);
			};
			\foreach \x/\y/\lab in {0/0/L,0/1/M,0/2/U	,2/-2/L,%
				5/0/L,5/2/U,7/-2/U,%
				10/0/L,10/1/M,10/2/U,12/-2/L%
			}\node[vlab] at (\x+0.5,\y+0.5){$\key{i}{\lab}$};
			\foreach \x/\y/\lab in {2/0/L,2/1/M,2/2/U,0/-2/L,%
				7/0/L,7/2/U,5/-2/U,%
				12/0/L,12/2/U,10/-2/U%
			}\node[vlab] at (\x+0.5,\y+0.5){$\key{j}{\lab}$};
			\foreach \x/\ty in {0/A,2/A,5/B,7/B,10/A,12/B}\node[vlab] at (\x+0.5,3.3){Type $\ty$};
			\foreach \xl/\yl/\xr/\yr in {0/0/2/0,5/2/7/2,10/0/12/2}{
				\draw[very thick,->] (\xl+0.75,\yl+0.25) -- (\xr+0.25,-1.25);
				\draw[very thick,->] (\xr+0.25,\yr+0.25) -- (\xl+0.75,-1.25);
			};
			%\extendtopbound
		\end{ctikzpicture}
	\item We may have one extra cell of type either $A$ or $B$. Now consider that extra cell $i$ (if it exists) and cell $n-1$.
		\begin{ctikzpicture}%[scale=1.2]
			\foreach \x in {0,6}{
				\draw (\x,0) grid +(1,3);
				\draw (\x+2,2) rectangle +(1,1);
				\draw (\x,-2) rectangle +(1,1);
				\draw (\x+2,-2) rectangle +(1,1);
			};
			\foreach \x/\y/\lab in {0/0/L,0/1/M,0/2/U	,2/-2/M,0/-2/U,%
				6/0/L,6/1/M,6/2/U,8/-2/M%
			}\node[vlab] at (\x+0.5,\y+0.5){$\key{i}{\lab}$};
			\node[vlab] at (6.5,-1.5){$\key{n-1}{U}$};
			\node[vlab] at (8.5,2.5){$\key{n-1}{U}$};
		
			\foreach \x/\ty in {0/A,6/A}\node[vlab] at (\x+0.5,3.3){Type $\ty$};
			\foreach \x in {2,8}\node[vlab] at (\x+0.5,3.3){$n-1$};
			\foreach \xl/\yl/\xr/\yr in {6/1/8/2}{
				\draw[very thick,->] (\xl+0.75,\yl+0.25) -- (\xr+0.25,-1.25);
				\draw[very thick,->] (\xr+0.25,\yr+0.25) -- (\xl+0.75,-1.25);
			};
			\draw[very thick,->] (0.75,1.25) -- (2.25,-1.25);
			\draw[very thick,->] (0.125,2.25) -- (0.125,-1.25);
			%\extendtopbound
		\end{ctikzpicture}
		\begin{ctikzpicture}%[scale=1.2]
			\foreach \x in {0,6}{
				\draw (\x,0) grid +(1,3);
				\draw (\x+2,2) rectangle +(1,1);
				\draw (\x,-2) rectangle +(1,1);
				\draw (\x+2,-2) rectangle +(1,1);
			};
			\foreach \x/\y/\lab in {0/0/L,0/2/U,2/-2/L,0/-2/U,%
				6/0/L,6/2/U,8/-2/L%
			}\node[vlab] at (\x+0.5,\y+0.5){$\key{i}{\lab}$};
			\node[vlab] at (6.5,-1.5){$\key{n-1}{U}$};
			\node[vlab] at (8.5,2.5){$\key{n-1}{U}$};
		
			\foreach \x/\ty in {0/B,6/B}\node[vlab] at (\x+0.5,3.3){Type $\ty$};
			\foreach \x in {2,8}\node[vlab] at (\x+0.5,3.3){$n-1$};
			\foreach \xl/\yl/\xr/\yr in {6/0/8/2}{
				\draw[very thick,->] (\xl+0.75,\yl+0.25) -- (\xr+0.25,-1.25);
				\draw[very thick,->] (\xr+0.25,\yr+0.25) -- (\xl+0.75,-1.25);
			};
			\draw[very thick,->] (0.75,0.25) -- (2.25,-1.25);
			\draw[very thick,->] (0.125,2.25) -- (0.125,-1.25);
			%\extendtopbound
		\end{ctikzpicture}
	\item Pair up cells of type $C$ and $D$ with other cells of type $C$ and $D$.
		\begin{ctikzpicture}%[scale=1.2]
			\foreach \x in {0,2,5,7,10,12}{
				\draw (\x,0) grid +(1,3);
				\draw (\x,-2) rectangle +(1,1);
			};
			\foreach \x/\y/\lab in {0/0/L,0/1/M,2/-2/M,%
				5/1/M,5/2/U,7/-2/U,%
				10/0/L,10/1/M,12/-2/L%
			}\node[vlab] at (\x+0.5,\y+0.5){$\key{i}{\lab}$};
			\foreach \x/\y/\lab in {2/0/L,2/1/M,0/-2/M,%
				7/1/M,7/2/U,5/-2/M,%
				12/1/M,12/2/U,10/-2/M%
			}\node[vlab] at (\x+0.5,\y+0.5){$\key{j}{\lab}$};
			\foreach \x/\ty in {0/C,2/C,5/D,7/D,10/C,12/D}\node[vlab] at (\x+0.5,3.3){Type $\ty$};
			\foreach \xl/\yl/\xr/\yr in {0/1/2/1,5/2/7/1,10/0/12/1}{
				\draw[very thick,->] (\xl+0.75,\yl+0.25) -- (\xr+0.25,-1.25);
				\draw[very thick,->] (\xr+0.25,\yr+0.25) -- (\xl+0.75,-1.25);
			};
			%\extendtopbound
		\end{ctikzpicture}
	\item We may have one extra cell of type either $C$ or $D$. Now consider that extra cell $i$ (if it exists) and cell $n$.
		\begin{ctikzpicture}%[scale=1.2]
			\foreach \x in {0,6}{
				\draw (\x,0) grid +(1,3);
				\draw (\x+2,1) rectangle +(1,1);
				\draw (\x,-2) rectangle +(1,1);
				\draw (\x+2,-2) rectangle +(1,1);
			};
			\foreach \x/\y/\lab in {0/0/L,0/1/M,2/-2/L,0/-2/M,%
				6/0/L,6/1/M,8/-2/L%
			}\node[vlab] at (\x+0.5,\y+0.5){$\key{i}{\lab}$};
			\node[vlab] at (6.5,-1.5){$\key{n}{M}$};
			\node[vlab] at (8.5,1.5){$\key{n}{M}$};
		
			\foreach \x/\ty in {0/C,6/C}\node[vlab] at (\x+0.5,3.3){Type $\ty$};
			\foreach \x in {2,8}\node[vlab] at (\x+0.5,3.3){$n$};
			\foreach \xl/\yl/\xr/\yr in {6/0/8/1}{
				\draw[very thick,->] (\xl+0.75,\yl+0.25) -- (\xr+0.25,-1.25);
				\draw[very thick,->] (\xr+0.25,\yr+0.25) -- (\xl+0.75,-1.25);
			};
			\draw[very thick,->] (0.75,0.25) -- (2.25,-1.25);
			\draw[very thick,->] (0.125,1.25) -- (0.125,-1.25);
			%\extendtopbound
		\end{ctikzpicture}
		\begin{ctikzpicture}%[scale=1.2]
			\foreach \x in {0,6}{
				\draw (\x,0) grid +(1,3);
				\draw (\x+2,1) rectangle +(1,1);
				\draw (\x,-2) rectangle +(1,1);
				\draw (\x+2,-2) rectangle +(1,1);
			};
			\foreach \x/\y/\lab in {0/1/M,0/2/U,2/-2/U,0/-2/M,%
				6/1/M,6/2/U,8/-2/U%
			}\node[vlab] at (\x+0.5,\y+0.5){$\key{i}{\lab}$};
			\node[vlab] at (6.5,-1.5){$\key{n}{M}$};
			\node[vlab] at (8.5,1.5){$\key{n}{M}$};
		
			\foreach \x/\ty in {0/D,6/D}\node[vlab] at (\x+0.5,3.3){Type $\ty$};
			\foreach \x in {2,8}\node[vlab] at (\x+0.5,3.3){$n$};
			\foreach \xl/\yl/\xr/\yr in {6/2/8/1}{
				\draw[very thick,->] (\xl+0.75,\yl+0.25) -- (\xr+0.25,-1.25);
				\draw[very thick,->] (\xr+0.25,\yr+0.25) -- (\xl+0.75,-1.25);
			};
			\draw[very thick,->] (0.75,2.25) -- (2.25,-1.25);
			\draw[very thick,->] (0.125,1.25) -- (0.125,-1.25);
			%\extendtopbound
		\end{ctikzpicture}
	\item\label{en:3n-4 fill available} Fill any cells flagged as available with any remaining keys.
\end{enumerate}
	Thus we are able to answer the membership question in at most two queries for a table of $n$ keys chosen from a key space of size $3n-4$ and so $g\left(n,2\right) \geq 3n-4$.
\end{proof}

When we were permitted only a single query, we had to know in advance which cell we would choose to query when presented with the membership question for a specific key. With two queries this is no longer necessarily the case. While we still need to know the first cell to query, the strategy given in \autoref{prop:two queries 3n-4} is \defn{adaptive}: it uses information gained from the first query to decide which other cell to query. We could mandate that the strategy be \defn{non-adaptive}, meaning that when we are presented with the membership question for a specific key the algorithm decides which cells to query before making any queries to the table. The best-known upper bound for the two query problem was established by David Howard but is an upper bound only for the non-adaptive case.

\begin{proposition}[Howard, \cite{Howard}] If restricted to non-adaptive strategies,\[ g\left(n,2\right) \leq \bigO\left(n^2 \log n\right) \]
\end{proposition}

Yao's original question was motivated by situations in which the key space was restricted. One approach to the two query problem is to impose further restrictions on the key space. If we can solve that problem, we may be able to solve the original problem by tiling the original key space with copies of the restricted space.

\section{Graph Queryability\label{sec:Yao graph queryability}}

Yao restricted the key space by supposing that it contained only $m$ keys. However, any subset of size $n$ is permitted. Suppose we further mandate that only certain subsets of the key space are permitted. If we consider subsets of size $2$ we can represent the restrictions as a graph with the key space as vertices and with edges between pairs of keys that are permitted to appear together.

\begin{example}Let $\left\{a,b,c,d\right\}$ be the key space and suppose that we permit all pairs except $\left\{b,d\right\}$. This can be represented as the diamond graph\index{diamond graph@diamond graph ($K_4-e$)} $K_4-e$, where $bd$ is the deleted edge.
	\begin{ctikzpicture}
		\foreach \u in {0,1}\node[vertex] (u\u) at (2*\u,0.5){};
		\foreach \v in {0,1}\node[vertex] (v\v) at (1,\v){};
		\foreach \u in {0,1}\foreach \v in {0,1} \draw (u\u) -- (v\v);
		\draw (v0) -- (v1);
		\node[vlab] at ($(v1)+(90:0.3)$){$a$};
		\node[vlab] at ($(v0)+(270:0.3)$){$c$};
		\node[vlab] at ($(u0)+(180:0.3)$){$b$};
		\node[vlab] at ($(u1)+(0:0.3)$){$d$};
		
		\node[vlab] at (1,-0.75){Diamond $\smash{K_4-e}$};
		\extendtopbound
	\end{ctikzpicture}
\end{example}

\begin{definition}A graph $G$ is \defn{$1$-queryable}\index{graph!1-queryable@$1$-queryable} means there exists a one-query sort/search strategy that will answer the membership question for any element of $V\left(G\right)$ on any table formed from an edge of $G$.
\end{definition}

\begin{example}The complete graph on three vertices, $K_3$, is $1$-queryable.
\end{example}
\begin{proof}We will use this example to demonstrate how the sort/search strategy can be \defn{encoded} into the graph. The table contains two cells. Every permissible subset of size $2$ is an edge of the graph. To indicate how a given pair of keys will be stored in the table, we orient the corresponding edge such that the orientation $u\rightarrow v$ indicates that $u$ will be stored in the first cell of the table and $v$ will be stored in the second cell. So if we have the orientation
	\begin{ctikzpicture}
		\foreach \v in {0,1,2}\node[vertex] (v\v) at (90+120*\v:1){};
		\foreach \u/\v in {0/1,1/2,2/0}\draw[-latex,oriented] (v\u) -- (v\v);
		\node[vlab] at ($(v0)+(180:0.3)$){$c$};
		\node[vlab] at ($(v1)+(180:0.3)$){$a$};
		\node[vlab] at ($(v2)+(0:0.3)$){$b$};
		\extendtopbound
	\end{ctikzpicture}
and we are presented with $\left\{a,c\right\}$ to store in the table, we will store $c$ in the first cell and $a$ in the second cell.

When we are asked to answer the membership question for a key $v$, we must decide which cell of the table to query. Suppose we decide to query the first cell. We can indicate this decision in the graph by coloring vertex $v$ color $0$. So if we have the coloring
	\begin{ctikzpicture}
		\foreach \v/\c in {0/0,1/0,2/1}\node[cvertex] (v\v) at (90+120*\v:1){$\c$};
		\foreach \u/\v in {0/1,1/2,2/0}\draw[-latex,oriented] (v\u) -- (v\v);
		\node[vlab] at ($(v0)+(180:0.4)$){$c$};
		\node[vlab] at ($(v1)+(180:0.4)$){$a$};
		\node[vlab] at ($(v2)+(0:0.4)$){$b$};
		\extendtopbound
	\end{ctikzpicture}
then we will query the first cell when asked the membership question for $a$ or $c$ and we will query the second cell when asked the membership question for $b$.

Under the encoding given, opening either door will determine exactly which edge is stored in the table. Thus $K_3$ is $1$-queryable and the given encoding is \defn{valid}\index{graph!1-queryable@$1$-queryable!valid encoding}\index{valid encoding}.
\end{proof}

It is refreshingly easy to characterize valid encodings. Notice that given a valid encoding, if we reverse the orientation of every edge and the color of every vertex then we have another valid encoding. We will use this fact on multiple occasions to fix the color of a vertex without any loss of generality.

\begin{lemma}\label{lem:reminder}An encoding of $G$ is valid if and only if for every edge $u\rightarrow v$:
	\begin{enumerate}
		\item\label{en:reminder 0} if $v$ has color $0$ then $\outdegree{u} = 1$
		\item\label{en:reminder 1} if $u$ has color $1$ then $\indegree{v} = 1$
	\end{enumerate}
	\begin{ctikzpicture}
		\node[vertex] (u) at (0,0){};
		\node[cvertex] (v) at (1.5,0){$0$};
		\node[vlab] at ($(u)+(180:0.3)$){$u$};
		\node[vlab] at ($(v)+(0:0.5)$){$v$};
		\draw[-latex,oriented] (u) -- (v);
		\draw[-latex,oriented] (u) -- (1.5,-0.8);
		\path (u) -- (1.5,-0.8) node[cross out,sloped,draw,pos=0.5]{};

		\pgfmathsetmacro{\xs}{4}
		\node[cvertex] (u) at (\xs,-0.8){$1$};
		\node[vertex] (v) at (\xs+1.5,-0.8){};
		\node[vlab] at ($(u)+(180:0.5)$){$u$};
		\node[vlab] at ($(v)+(0:0.3)$){$v$};
		\draw[-latex,oriented] (u) -- (v);
		\draw[-latex,oriented] (\xs,0) -- (v);
		\path (\xs,0) -- (v) node[cross out,sloped,draw,pos=0.5]{};
		\extendtopbound
	\end{ctikzpicture}
\end{lemma}
\begin{proof}
	Suppose every edge satisfies \ref{en:reminder 0} and \ref{en:reminder 1} and that we are asked to answer the membership question for vertex $a$. Without loss of generality, assume $a$ has color $0$. If we query the first cell and find $a$ then we are done; if not, let $b$ be the vertex found in the first cell. If $ab$ is not an edge of $G$ or if the edge is encoded as $a \rightarrow b$ then we conclude that $a$ is not in the table. Otherwise the edge $ab$ is encoded as $b \rightarrow a$. Since $a$ has color $0$, we know that $\outdegree{b} = 1$ by \ref{en:reminder 0}. Therefore $a$ must be in the second cell of the table. Thus we can decide the membership question in one query with this strategy and so the encoding is valid.

	Now suppose that some oriented edge $c\rightarrow d$ does not satisfy both \ref{en:reminder 0} and \ref{en:reminder 1}. If it does not satisfy \ref{en:reminder 0} then $d$ has color $0$ and $\outdegree{c} > 1$. When the table contains the edge $cd$ and we are asked to solve the membership question for $d$, we will query the first cell and find $c$. But since $\outdegree{c} > 1$, there is another vertex $x$ that is oriented $c\rightarrow x$. Thus we are unable to decide the membership question for $d$ in one query using this encoding, and so the encoding is not valid. The argument is similar in the case that \ref{en:reminder 1} is not satisfied.
\end{proof}

If $G$ is a bipartite graph with partite sets $X$ and $Y$, consder the encoding given by coloring all vertices in $X$ color $0$, all vertices in $Y$ color $1$, and orienting all edges to point from $X$ to $Y$. This encoding will vacuously satisfy the requirements of \autoref{lem:reminder}.

\begin{corollary}Every bipartite graph is $1$-queryable.
\end{corollary}

\begin{lemma}\label{lem:triangle directed}In a valid encoding of a $1$-queryable graph, every triangle must be oriented as a directed cycle.
\end{lemma}
\begin{proof}
	Let $a$, $b$, and $c$ be the vertices of the triangle. Without loss of generality $c$ is colored $0$ and the edge $ab$ is oriented $a \rightarrow b$. It follows by \autoref{lem:reminder} that $ac$ must be oriented $c\rightarrow a$. To obtain a contradiction, assume $bc$ is oriented $c\rightarrow b$. Both $a$ and $b$ must be colored $1$ by \autoref{lem:reminder}. This is a contradiction, since $\outdegree{b} \geq 2$.
	\begin{ctikzpicture}
		\node[vertex] (a) at (0,0){};				\node[vlab] at ($(a)+(180:0.3)$){$a$};
		\node[vertex] (b) at (2,0){};				\node[vlab] at ($(b)+(0:0.3)$){$b$};
		\node[cvertex] (c) at ($(a)!1!60:(b)$){$0$};		\node[vlab] at ($(c)+(180:0.5)$){$c$};
		\draw[oriented,-latex] (a) -- (b);
		\draw[oriented] (a) -- (c);
		\draw[oriented] (b) -- (c);
	\end{tikzpicture}\hspace{1cm}\begin{tikzpicture}
		\node[vertex] (a) at (0,0){};				\node[vlab] at ($(a)+(180:0.3)$){$a$};
		\node[vertex] (b) at (2,0){};				\node[vlab] at ($(b)+(0:0.3)$){$b$};
		\node[cvertex] (c) at ($(a)!1!60:(b)$){$0$};		\node[vlab] at ($(c)+(180:0.5)$){$c$};
		\draw[oriented,-latex] (a) -- (b);
		\draw[oriented,latex-] (a) -- (c);
		\draw[oriented,latex-] (b) -- (c);
	\end{tikzpicture}\hspace{1cm}\begin{tikzpicture}
		\node[cvertex] (a) at (0,0){$1$};				\node[vlab] at ($(a)+(180:0.5)$){$a$};
		\node[cvertex] (b) at (2,0){$1$};				\node[vlab] at ($(b)+(0:0.5)$){$b$};
		\node[cvertex] (c) at ($(a)!1!60:(b)$){$0$};		\node[vlab] at ($(c)+(180:0.5)$){$c$};
		\draw[oriented,-latex] (a) -- (b);
		\draw[oriented,latex-] (a) -- (c);
		\draw[oriented,latex-] (b) -- (c);
		\extendtopbound
	\end{ctikzpicture}
\end{proof}

\begin{lemma}[3-chaining]\label{lem:3-chaining}Let $v_0, v_1, \dotsc, v_n$ be a path in a graph $G$ with $d(v_i) \geq 3$ for $i=1,\dotsc n$. If $v_0$ and $v_1$ have color $0$ and the edge $v_{0} v_{1}$ is oriented $v_{1} \rightarrow v_{0}$ then:
	\begin{enumerate}
		\item $v_1$, $\dotsc$, $v_n$ all have color $0$.
		\item The edges $v_{i-1} v_{i}$ are oriented $v_{i} \rightarrow v_{i-1}$ for $i = 1,\dotsc, n$.
		\item Every neighbor $u$ of $v_i$ has color $0$ and $uv_{i}$ is oriented $u \rightarrow v_{i}$ (except when $u = v_{i-1}$) for $i = 1, \dotsc, n$.
		\item For any $1 \leq i < j \leq n$, if $j \neq i+1$ then $v_i$ is not adjacent to $v_j$.
	\end{enumerate}
\end{lemma}
\begin{proof}
	The hypotheses, together with \autoref{lem:reminder} imply that $\outdegree{v_1} = 1$, and so all edges incident to $v_1$, aside from $v_{0} v_{1}$, must flow into $v_{1}$. Thus $\indegree{v_1} \geq 2$, so all vertices adjacent to $v_1$ must have color $0$. This provides the base case for the claim, and the rest follows by induction on $n$.
	\begin{ctikzpicture}
		\pgfmathsetmacro{\gap}{1.5}
		\node[cvertex] (v0) at (0,0){$0$};				\node[vlab] at ($(v0)+(270:0.5)$){$v_0$};
		\node[cvertex] (v1) at (\gap,0){$0$};			\node[vlab] at ($(v1)+(270:0.5)$){$v_1$};
		\node[vertex] (v2) at ($(v1)+(\gap,0)$){};		\node[vlab] at ($(v2)+(270:0.5)$){$v_2$};
		\node[vertex] (vn) at ($(v2)+(1.5*\gap,0)$){};	\node[vlab] at ($(vn)+(270:0.5)$){$v_n$};

		\node[vertex] (u1) at ($(v1)+(0,\gap)$){};	%\node[vlab] at ($(v1)+(270:0.5)$){$v_1$};
		\node[vertex] (u2) at ($(v2)+(0,\gap)$){};		%\node[vlab] at ($(v2)+(270:0.5)$){$v_2$};
		\node[vertex] (un) at ($(vn)+(0,\gap)$){};		%\node[vlab] at ($(v2)+(270:0.5)$){$v_2$};

		\draw[oriented,latex-] (v0) -- (v1);
		\draw[oriented] (v1) -- (v2);
		\draw[oriented] (v2) -- +(0.25*\gap,0);
		\draw[oriented, dashed] (v2)++(0.25*\gap,0) -- (vn)++(-0.25*\gap,0);
		\draw[oriented] (vn)++(-0.25*\gap,0) -- (vn);
		\draw[oriented] (vn) -- +(0.25*\gap,0);
		\draw[oriented, dashed] (vn)++(0.25*\gap,0) -- +(0.5*\gap,0);

		\draw[oriented] (u1) -- (v1);
		\draw[oriented] (u2) -- (v2);
		\draw[oriented] (un) -- (vn);
	\end{tikzpicture}\hspace{1cm}\begin{tikzpicture}
		\pgfmathsetmacro{\gap}{1.5}
		\node[cvertex] (v0) at (0,0){$0$};				\node[vlab] at ($(v0)+(270:0.5)$){$v_0$};
		\node[cvertex] (v1) at (\gap,0){$0$};			\node[vlab] at ($(v1)+(270:0.5)$){$v_1$};
		\node[cvertex] (v2) at ($(v1)+(\gap,0)$){$0$};		\node[vlab] at ($(v2)+(270:0.5)$){$v_2$};
		\node[vertex] (vn) at ($(v2)+(1.5*\gap,0)$){};	\node[vlab] at ($(vn)+(270:0.5)$){$v_n$};

		\node[cvertex] (u1) at ($(v1)+(0,\gap)$){$0$};	%\node[vlab] at ($(v1)+(270:0.5)$){$v_1$};
		\node[vertex] (u2) at ($(v2)+(0,\gap)$){};		%\node[vlab] at ($(v2)+(270:0.5)$){$v_2$};
		\node[vertex] (un) at ($(vn)+(0,\gap)$){};		%\node[vlab] at ($(v2)+(270:0.5)$){$v_2$};

		\draw[oriented,latex-] (v0) -- (v1);
		\draw[oriented,latex-] (v1) -- (v2);
		\draw[oriented] (v2) -- +(0.25*\gap,0);
		\draw[oriented, dashed] (v2)++(0.25*\gap,0) -- (vn)++(-0.25*\gap,0);
		\draw[oriented] (vn)++(-0.25*\gap,0) -- (vn);
		\draw[oriented] (vn) -- +(0.25*\gap,0);
		\draw[oriented, dashed] (vn)++(0.25*\gap,0) -- +(0.5*\gap,0);

		\draw[oriented,-latex] (u1) -- (v1);
		\draw[oriented] (u2) -- (v2);
		\draw[oriented] (un) -- (vn);
		\extendtopbound
	\end{ctikzpicture}
\end{proof}

\begin{corollary}\label{cor:odd cycles}Let $C$ be an odd cycle in a graph $G$ such that every vertex of $C$ has degree at least $3$.
	\begin{itemize}
		\item $C$ must be oriented as a directed cycle and must be monochromatic.
		\item Any vertex in $G-C$ that is adjacent to a vertex in $C$ must also have that same color.
		\item $C$ acts as either a source or a sink for the vertices of $G-C$ that are adjacent to a vertex of $C$, according to whether the color is a $0$ or a $1$, respectively.
	\end{itemize}
\end{corollary}

\begin{lemma}\label{lem:unicyclic}Let $G$ be a $1$-queryable graph with no vertices of degree two. Either $G$ is bipartite, or $G$ is unicyclic.
\end{lemma}
\begin{proof}Suppose $G$ is neither bipartite nor unicyclic. $G$ has an odd cycle $C$ with minimum degree three, so by \autoref{cor:odd cycles}, $C$ is monochromatic. Without loss of generality, $C$ has color $0$ and is a sink. $C$ cannot have a chord, as a chord would not be orientable.

If another cycle shares two vertices with $C$ then there is a path between two vertices $u$ and $v$ of $C$. Call the interior vertices of the path $p_1, p_2, \dotsc, p_k$. Starting at $v$ and applying \autoref{lem:3-chaining}, every $p_i$ must be colored $0$, and all edges must be oriented $p_i \rightarrow p_{i+1}$. But now $\outdegree{p_1} = 2$, contradicting \autoref{lem:reminder}.
	\begin{ctikzpicture}
		\pgfmathsetmacro{\rad}{3}
		\pgfmathsetmacro{\fa}{1}
		\pgfmathsetmacro{\sa}{15}
		\pgfmathsetmacro{\ea}{30}
		\draw (\rad,0) arc (0:\sa:\rad);
		\draw (\rad,0) arc (0:-\sa:\rad);
		\draw (-\rad,0) arc (180:180-\sa:\rad);
		\draw (-\rad,0) arc (180:180+\sa:\rad);
		\draw[dashed] (\sa:\rad) arc (\sa:\ea:\rad);
		\draw[dashed] (-\sa:\rad) arc (-\sa:-\ea:\rad);
		\draw[dashed] (180-\sa:\rad) arc (180-\sa:180-\ea:\rad);
		\draw[dashed] (180+\sa:\rad) arc (180+\sa:180+\ea:\rad);
		%\draw (0,0) circle (\rad);
		\node[cvertex] (c1) at (-\rad,0){$0$};			\node[vlab] at ($(c1)+(180:0.5)$){$u$};
		\node[cvertex] (c2) at (\rad,0){$0$};			\node[vlab] at ($(c2)+(0:0.5)$){$v$};
		\node[cvertex] (p1) at ($(c1)+(\fa,0)$){$0$};		\node[vlab] at ($(p1)+(270:0.5)$){$p_1$};
		\node[cvertex] (p1n) at ($(p1)+(0,\fa)$){$0$};
		\node[cvertex] (p2) at ($(p1)+(\fa,0)$){$0$};		\node[vlab] at ($(p2)+(270:0.5)$){$p_2$};
		\node[cvertex] (p2n) at ($(p2)+(0,\fa)$){$0$};
		\coordinate (p3) at ($(p2)+(\fa,0)$){};
		\node[cvertex] (pk) at ($(c2)+(-\fa,0)$){$0$};	\node[vlab] at ($(pk)+(270:0.5)$){$p_k$};
		\node[cvertex] (pkn) at ($(pk)+(0,\fa)$){$0$};
		\coordinate (pj) at ($(pk)+(-\fa,0)$){};

		\foreach \u/\v in {p1/p2, p2/p3, pj/pk}\draw[oriented,-latex] (\u) -- (\v);
		\foreach \u/\v in {c1/p1, p1/p1n, p2/p2n, c2/pk, pkn/pk}\draw[oriented,latex-] (\u) -- (\v);
		\draw[dashed] (p3)++(0.125*\fa,0) -- ($(pj)+(-0.125*\fa,0)$);
		\extendtopbound
	\end{ctikzpicture}

Thus any other cycle of $G$ must be connected to $C$ by a path. This is a contradiction to \autoref{lem:3-chaining}, completing the proof.
\end{proof}

\begin{lemma}\label{lem:components}Let $G$ be a $1$-queryable graph and let $S$ be the set of all vertices of $G$ of degree two. Every component of $G-S$ is either bipartite or unicyclic.
\end{lemma}
\begin{proof}Suppose $G-S$ has a component that is neither bipartite nor unicyclic. This component has at least one odd cycle, and since we removed all vertices of $G$ of degree two, every vertex of this odd cycle has minimum degree at least three. Thus, by \autoref{cor:odd cycles} it is monochromatic. The component also contains another cycle, which either shares vertices with the odd cycle or is connected to the cycle by a path. Either way, in $G$ all the relevant vertices have minimum degree three, and so by \autoref{lem:unicyclic}, $G$ has a subgraph that is not $1$-queryable, contradicting that $G$ is $1$-queryable.
\end{proof}

\begin{theorem}Every $1$-queryable graph has chromatic number at most three.
\end{theorem}
\begin{proof}Let $G$ be a $1$-queryable graph and let $S$ be the set of all vertices of $G$ of degree two. By \autoref{lem:components}, every component of $G-S$ is either bipartite or unicyclic and so can be colored with at most three colors. Hence $G-S$ has a valid coloring using colors chosen from the set $\left\{a,b,c\right\}$. This coloring can be extended to a valid coloring of $G$ by coloring each vertex $v$ of $S$ in turn. Since every vertex in $S$ has degree two, at least one of the colors from $\left\{a,b,c\right\}$ is unused by the neighbors of $v$. Thus the coloring can be extended to $v$ and hence to all of $G$.
\end{proof}

\section{Conclusion}

This chapter studied a storage problem introduced by Yao and resurrected by Doug West in his 2010 REGS \cite{WestYao}. \Autoref{sec:Yao two queries} gives a lower bound for the $2$-query, adaptive version of this problem. No known upper bound even comes close to matching this lower bound, so \autoref{sec:Yao graph queryability} revisits the structure of extremal $1$-queryable graphs in the hopes that this will provide a tiling argument leading to a better upper bound for the $2$-query case. While this section does give a near-complete characterization of $1$-queryable graphs, a generalization to $1$-queryable $k$-uniform hypergraphs\nocite{Berge} is required to power this tiling which reduces the $2$-queryable problem to a $1$-queryable problem with an open cell. An `open cell' whose contents are always known. With one query and one open cell it is possible to accommodate a key space of $2n-1$ keys, one more than Yao's bound for the basic $1$-query case; we conjecture that this is the best possible. Another variant is to allow for `closed' cells: ones that cannot be queried. The bounds for this problem are very close to the bounds for the original problem.

There has been some progress on characterizing $1$-queryable $3$-uniform hypergraphs. If for every pair of vertices the co-degree $0$ or at least $7$ then the hypergraph must be tripartite. This is the $3$-uniform analogue to the fact that if a $1$-queryable graph has minimum degree at least $3$ then the graph must be bipartite. A complete characterization of $1$-queryable graphs is not yet known, nor is a generalization to graphs of higher uniformity. The co-degree restriction imposes a sort of edge-density requirement on the hypergraph. It remains an open problem to determine the maximum number of edges that can be contained in a $k$-uniform $1$-queryable hypergraph, but it is known that the extremal hypergraphs will not be $k$-partite.
%\input{Conclusion/Conclusion}

\backmatter
\printbibliography
\printglossary
\printindex

\newcommand{\event}[3][]{%
	\begin{minipage}[t]{0.15\textwidth}\scshape #3\end{minipage}
	\begin{minipage}[t]{0.75\textwidth}\textbf{#2}%
		\def\temptyA{#1}%
		\ifx\temptyA\empty\else {\\ \vphantom{.}\hspace{12pt} #1}\fi%
	\end{minipage}\vspace{12pt}
}

\begin{vita}
{\large Education}\vspace{12pt}

\event[Ph.D. in Applied and Industrial Mathematics]{University of Louisville}{2011}

\event[M.S. in Mathematics]{University of Louisville}{2006}

\event[B.S. in Mathematics]{University of Louisville}{2004}

\noindent{\large Publications}\vspace{12pt}

\textit{The Hub Number of a Graph} (with T. Grauman, S. G. Hartke, B. Kinnersley, D. B. West, L. Wiglesworth, P. Worah, H. Wu), \textit{Inform. Process. Lett.} \textbf{108} (2008), 226--228\vspace{12pt}

\textit{On Hamiltonian colorings and hc-stable graphs} (with B. Allgeier, G. Chartrand, L. Nebesk{\'y}, P. Zhang),  \textit{Proceedings of the Thirty-Ninth Southeastern International Conference on Combinatorics, Graph Theory and Computing} (2008) 65--76\vspace{12pt}

\textit{Stack Domination Density} (with T. Brauch, P. Horn, D. J. Wildstrom), submitted.\vspace{12pt}

\textit{Ramsey Functions for Quasi-Progressions with Large Diameter} (with A. K\'ezdy, H. Snevily, S. White), submitted.%\vspace{12pt}
\end{vita}
\end{document}