\begin{proposition}For $n \geq 4$, $g\left(n,2\right) \geq 3n-2$.
\end{proposition}
\begin{proof}We will start by labeling all the keys in $M$ using a technique similar to Yao's from \autoref{thm:Yao one query}. For $i \in \left[n-2\right]$, three keys will have cell label $i$: a `lower key' $K_i^L$, a `middle key' $K_i^M$, and an `upper key' $K_i^U$. For $i\in \left\{n-1,n\right\}$ two keys will have cell label $i$: an `X key' $K_i^X$ and a `Y Key' $K_i^Y$.
\begin{ctikzpicture}[scale=1.2]
	\draw (0,0) grid (2,3);
	\foreach \v in {1,2}{
		\node[vlab] at (\v-0.5,-0.3){\small$\v$};
		\node[vlab] at (\v-0.5,0.5){$K_{\v}^L$};
		\node[vlab] at (\v-0.5,1.5){$K_{\v}^M$};
		\node[vlab] at (\v-0.5,2.5){$K_{\v}^U$};
	};
	\pgfmathsetmacro{\xs}{3.5}
	\foreach \y in {0,1,2,3}\draw (\xs,\y) -- +(1,0);
	\foreach \y in {0,1,2}\draw (\xs+1,\y+0.5) -- +(2,0);
	\draw (\xs,0) -- +(0,3);
	\draw (\xs+1,0) -- +(0,3);
	\draw (\xs+2,0.5) -- +(0,2);
	\draw (\xs+3,0.5) -- +(0,2);

	%\node[vlab] at (1+0.5*\xs,1){\makebox[0pt]{\huge$\dotsb$}};
	\foreach \d in {0,-0.3,0.3}\fill (1+0.5*\xs+\d,0.5) circle (0.05);
	\node[vlab] at (\xs+0.5,-0.3){$n-2$};
	\node[vlab] at (\xs+0.5,0.5){$K_{n-2}^L$};
	\node[vlab] at (\xs+0.5,1.5){$K_{n-2}^M$};
	\node[vlab] at (\xs+0.5,2.5){$K_{n-2}^U$};

	\node[vlab] at (\xs+1.5,-0.3){$n-1$};
	\node[vlab] at (\xs+1.5,1){$K_{n-1}^X$};
	\node[vlab] at (\xs+1.5,2){$K_{n-1}^Y$};

	\node[vlab] at (\xs+2.5,-0.3){$n$};
	\node[vlab] at (\xs+2.5,1){$K_n^X$};
	\node[vlab] at (\xs+2.5,2){$K_n^Y$};
	\extendtopbound
\end{ctikzpicture}

Given a set $N$ of $n$ keys from the key space, the storage scheme begins by examining the first $n-2$ cells:
\begin{enumerate}
	\item If no keys in $N$ have this cell label, then the cell is flagged as available.
	\item If only one key in $N$ has this cell label then that key is assigned to the cell.
	\item If two of more keys in $N$ have this cell label then we classify the cell by type:
	\begin{ctikzpicture}[scale=1.2]
		\foreach \x in {0,2,4,6}\draw (\x,0) grid +(1,3);
		\foreach \x/\y/\lab in {0/0/L,0/1/M,0/2/U,2/0/L,2/2/U,4/0/L,4/1/M,6/1/M,6/2/U}\node[vlab] at (\x+0.5,\y+0.5){$K_{i}^{\lab}$};
		\foreach \x/\ty in {0/A,2/B,4/C,6/D}\node[vlab] at (\x+0.5,-0.3){Type $\ty$};
		\extendtopbound
	\end{ctikzpicture}
	\item Pair up cells of type $A$ and $B$ with other cells of type $A$ and $B$.
		\begin{enumerate}
			\item If cell $i$ and cell $j$ are both type $A$ then fill cell $i$ with $K_j^L$ and fill cell $j$ with $K_i^L$.
			\item If cell $i$ and cell $j$ are both type $B$ then fill cell $i$ with $K_j^U$ and fill cell $j$ with $K_i^U$.
			\item If cell $i$ is type $A$ and cell $j$ is type $B$ then fill cell $i$ with $K_j^U$ and fill cell $j$ with $K_i^L$.
		\end{enumerate}
	\item Pair up cells of type $C$ and $D$ with other cells of type $C$ and $D$.
		\begin{enumerate}
			\item If cell $i$ and cell $j$ are both type $C$ then fill cell $i$ with $K_j^M$ and fill cell $j$ with $K_i^M$.
			\item If cell $i$ and cell $j$ are both type $D$ then fill cell $i$ with $K_j^M$ and fill cell $j$ with $K_i^U$.
			\item If cell $i$ is type $C$ and cell $j$ is type $D$ then fill cell $i$ with $K_j^M$ and fill cell $j$ with $K_i^L$.
		\end{enumerate}
	\item We may have one extra cell of type either $A$ or $B$ and one extra cell of type either $C$ or $D$.
\end{enumerate}
Now we classify cells $n-1$ and $n$ by type:
	\begin{ctikzpicture}
		\foreach \x in {0,2,4,6}{\begin{scope}[yshift=0.5 cm]\draw (\x,0) grid +(1,2);\end{scope}};
		\foreach \x/\y/\lab in {2/0/X,4/1/Y,6/0/X,6/1/U}\node[vlab] at (\x+0.5,\y+1){$K_{i}^{\lab}$};
		\foreach \x/\ty in {0/E,2/F,4/G,6/H}\node[vlab] at (\x+0.5,-0.3){Type $\ty$};
		\extendtopbound
	\end{ctikzpicture}
\begin{enumerate}[resume]
	\item Pair cell $n-1$ with the extra cell $i$ of type $A$ or $B$.
		\par\begin{tabular}{c|c c c|c}
			\multicolumn{2}{c}{Cell}	&	&\multicolumn{2}{c}{Cell}\\
			$i$	&$n-1$	&		&$i$		&$n-1$\\\cline{1-2}\cline{4-5}
			A	&E		&$\rightarrow$	&$K_i^L$		&$K_i^M$\\
			A	&F		&$\rightarrow$	&$K_{n-1}^X$		&$K_i^M$\\
			A	&G		&$\rightarrow$	&$K_{n-1}^Y$		&$K_i^M$\\
			A	&H		&$\rightarrow$	&$K_{n-1}^X$		&$K_i^U$
		\end{tabular}\hfill\begin{tabular}{c|c c c|c}
			\multicolumn{2}{c}{Cell}	&	&\multicolumn{2}{c}{Cell}\\
			$i$	&$n-1$	&		&$i$		&$n-1$\\\cline{1-2}\cline{4-5}
			B	&E		&$\rightarrow$	&$K_i^L$			&$K_i^U$\\
			B	&F		&$\rightarrow$	&$K_{n-1}^X$		&$K_i^L$\\
			B	&G		&$\rightarrow$	&$K_{n-1}^Y$		&$K_i^L$\\
			B	&H		&$\rightarrow$	&$K_{n-1}^Y$		&$K_i^U$
		\end{tabular}\hfill\null\par\null
	\item Pair cell $n$ with the extra cell $i$ of type $C$ or $D$.
		\par\begin{tabular}{c|c c c|c}
			\multicolumn{2}{c}{Cell}	&	&\multicolumn{2}{c}{Cell}\\
			$i$	&$n$	&		&$i$			&$n$\\\cline{1-2}\cline{4-5}
			C	&E		&$\rightarrow$	&$K_i^M$			&$K_i^L$\\
			C	&F		&$\rightarrow$	&$K_{n}^X$		&$K_i^L$\\
			C	&G		&$\rightarrow$	&$K_{n}^Y$		&$K_i^L$\\
			C	&H		&$\rightarrow$	&$K_{n}^X$		&$K_i^M$
		\end{tabular}\hfill\begin{tabular}{c|c c c|c}
			\multicolumn{2}{c}{Cell}	&	&\multicolumn{2}{c}{Cell}\\
			$i$	&$n$	&		&$i$			&$n$\\\cline{1-2}\cline{4-5}
			D	&E		&$\rightarrow$	&$K_i^M$			&$K_i^U$\\
			D	&F		&$\rightarrow$	&$K_{n}^X$		&$K_i^U$\\
			D	&G		&$\rightarrow$	&$K_{n}^Y$		&$K_i^U$\\
			D	&H		&$\rightarrow$	&$K_{n}^Y$		&$K_i^M$
		\end{tabular}\hfill\null\par\null
	\item If there isn't an extra cell of that type then... yeah that's a problem.
\end{enumerate}
\end{proof}


\begin{proposition}\label{prop:two queries 3n-4}For $n \geq 4$, $g\left(n,2\right) \geq 3n-4$.
\end{proposition}
\begin{proof}We will start by labeling all the keys in $M$ using a technique similar to Yao's from \autoref{thm:Yao one query}. For $i \in \left[n-2\right]$, three keys will have cell label $i$: a `lower key' $\key{i}{L}$, a `middle key' $\key{i}{M}$, and an `upper key' $\key{i}{U}$. For $i\in \left\{n-1,n\right\}$ only one key will have cell label $i$: $\key{n-1}{U}$ and $\key{n}{M}$.
\begin{ctikzpicture}[scale=1.2]
	\draw (0,0) grid (2,3);
	\foreach \v in {1,2}{
		\node[vlab] at (\v-0.5,-0.3){\small$\v$};
		\node[vlab] at (\v-0.5,0.5){$\key{\v}{L}$};
		\node[vlab] at (\v-0.5,1.5){$\key{\v}{M}$};
		\node[vlab] at (\v-0.5,2.5){$\key{\v}{U}$};
	};
	\pgfmathsetmacro{\xs}{3.5}
	\foreach \y in {0,1,2,3}\draw (\xs,\y) -- +(1,0);
	\draw (\xs+2,1) -- +(1,0);
	\draw (\xs+1,2) -- +(2,0);
	\draw (\xs+1,3) -- +(1,0);
	\draw (\xs,0) -- +(0,3);
	\draw (\xs+1,0) -- +(0,3);
	\draw (\xs+2,1) -- +(0,2);
	\draw (\xs+3,1) -- +(0,1);

	\foreach \d in {0,-0.3,0.3}\fill (1+0.5*\xs+\d,0.5) circle (0.05);
	\node[vlab] at (\xs+0.5,-0.3){$n-2$};
	\node[vlab] at (\xs+0.5,0.5){$\key{n-2}{L}$};
	\node[vlab] at (\xs+0.5,1.5){$\key{n-2}{M}$};
	\node[vlab] at (\xs+0.5,2.5){$\key{n-2}{U}$};

	\node[vlab] at (\xs+1.5,-0.3){$n-1$};
	\node[vlab] at (\xs+1.5,2.5){$\key{n-1}{U}$};

	\node[vlab] at (\xs+2.5,-0.3){$n$};
	\node[vlab] at (\xs+2.5,1.5){$\key{n}{M}$};
	\extendtopbound
\end{ctikzpicture}

Given a set $N$ of $n$ keys from the key space, the storage scheme works as follows:
\begin{enumerate}
	\item Examine the first $n-2$ cells.
	\begin{enumerate}
		\item If no keys in $N$ have this cell label, then the cell is flagged as available.
		\item\label{en:3n-4 only key} If only one key in $N$ has this cell label then that key is assigned to the cell.
		\item If two of more keys in $N$ have this cell label then we classify the cell by type:
		\begin{ctikzpicture}[scale=1.2]
			\foreach \x in {0,2,4,6}\draw (\x,0) grid +(1,3);
			\foreach \x/\y/\lab in {0/0/L,0/1/M,0/2/U,2/0/L,2/2/U,4/0/L,4/1/M,6/1/M,6/2/U}\node[vlab] at (\x+0.5,\y+0.5){$\key{i}{\lab}$};
			\foreach \x/\ty in {0/A,2/B,4/C,6/D}\node[vlab] at (\x+0.5,-0.3){Type $\ty$};
			\extendtopbound
		\end{ctikzpicture}
	\end{enumerate}
	\item Pair up cells of type $A$ and $B$ with other cells of type $A$ and $B$.
		\begin{ctikzpicture}%[scale=1.2]
			\foreach \x in {0,2,5,7,10,12}{
				\draw (\x,0) grid +(1,3);
				\draw (\x,-2) rectangle +(1,1);
			};
			\foreach \x/\y/\lab in {0/0/L,0/1/M,0/2/U	,2/-2/L,%
				5/0/L,5/2/U,7/-2/U,%
				10/0/L,10/1/M,10/2/U,12/-2/L%
			}\node[vlab] at (\x+0.5,\y+0.5){$\key{i}{\lab}$};
			\foreach \x/\y/\lab in {2/0/L,2/1/M,2/2/U,0/-2/L,%
				7/0/L,7/2/U,5/-2/U,%
				12/0/L,12/2/U,10/-2/U%
			}\node[vlab] at (\x+0.5,\y+0.5){$\key{j}{\lab}$};
			\foreach \x/\ty in {0/A,2/A,5/B,7/B,10/A,12/B}\node[vlab] at (\x+0.5,3.3){Type $\ty$};
			\foreach \xl/\yl/\xr/\yr in {0/0/2/0,5/2/7/2,10/0/12/2}{
				\draw[very thick,->] (\xl+0.75,\yl+0.25) -- (\xr+0.25,-1.25);
				\draw[very thick,->] (\xr+0.25,\yr+0.25) -- (\xl+0.75,-1.25);
			};
			\extendtopbound
		\end{ctikzpicture}
		\begin{enumerate}
			\item\label{en:3n-4 AA} If cell $i$ and cell $j$ are both type $A$ then fill cell $i$ with $\key{j}{L}$ and fill cell $j$ with $\key{i}{L}$.
			\item\label{en:3n-4 BB} If cell $i$ and cell $j$ are both type $B$ then fill cell $i$ with $\key{j}{U}$ and fill cell $j$ with $\key{i}{U}$.
			\item\label{en:3n-4 AB} If cell $i$ is type $A$ and cell $j$ is type $B$ then fill cell $i$ with $\key{j}{U}$ and fill cell $j$ with $\key{i}{L}$.
		\end{enumerate}
	\item We may have one extra cell of type either $A$ or $B$. Now consider that extra cell $i$ (if it exists) and cell $n-1$.
		\begin{ctikzpicture}%[scale=1.2]
			\foreach \x in {0,6}{
				\draw (\x,0) grid +(1,3);
				\draw (\x+2,2) rectangle +(1,1);
				\draw (\x,-2) rectangle +(1,1);
				\draw (\x+2,-2) rectangle +(1,1);
			};
			\foreach \x/\y/\lab in {0/0/L,0/1/M,0/2/U	,2/-2/M,0/-2/U,%
				6/0/L,6/1/M,6/2/U,8/-2/M%
			}\node[vlab] at (\x+0.5,\y+0.5){$\key{i}{\lab}$};
			\node[vlab] at (6.5,-1.5){$\key{n-1}{U}$};
			\node[vlab] at (8.5,2.5){$\key{n-1}{U}$};
		
			\foreach \x/\ty in {0/A,6/A}\node[vlab] at (\x+0.5,3.3){Type $\ty$};
			\foreach \x in {2,8}\node[vlab] at (\x+0.5,3.3){$n-1$};
			\foreach \xl/\yl/\xr/\yr in {6/1/8/2}{
				\draw[very thick,->] (\xl+0.75,\yl+0.25) -- (\xr+0.25,-1.25);
				\draw[very thick,->] (\xr+0.25,\yr+0.25) -- (\xl+0.75,-1.25);
			};
			\draw[very thick,->] (0.75,1.25) -- (2.25,-1.25);
			\draw[very thick,->] (0.125,2.25) -- (0.125,-1.25);
			\extendtopbound
		\end{ctikzpicture}
		\begin{ctikzpicture}%[scale=1.2]
			\foreach \x in {0,6}{
				\draw (\x,0) grid +(1,3);
				\draw (\x+2,2) rectangle +(1,1);
				\draw (\x,-2) rectangle +(1,1);
				\draw (\x+2,-2) rectangle +(1,1);
			};
			\foreach \x/\y/\lab in {0/0/L,0/2/U,2/-2/L,0/-2/U,%
				6/0/L,6/2/U,8/-2/L%
			}\node[vlab] at (\x+0.5,\y+0.5){$\key{i}{\lab}$};
			\node[vlab] at (6.5,-1.5){$\key{n-1}{U}$};
			\node[vlab] at (8.5,2.5){$\key{n-1}{U}$};
		
			\foreach \x/\ty in {0/B,6/B}\node[vlab] at (\x+0.5,3.3){Type $\ty$};
			\foreach \x in {2,8}\node[vlab] at (\x+0.5,3.3){$n-1$};
			\foreach \xl/\yl/\xr/\yr in {6/0/8/2}{
				\draw[very thick,->] (\xl+0.75,\yl+0.25) -- (\xr+0.25,-1.25);
				\draw[very thick,->] (\xr+0.25,\yr+0.25) -- (\xl+0.75,-1.25);
			};
			\draw[very thick,->] (0.75,0.25) -- (2.25,-1.25);
			\draw[very thick,->] (0.125,2.25) -- (0.125,-1.25);
			\extendtopbound
		\end{ctikzpicture}
		\begin{enumerate}
			\item If there is no extra cell and $\key{n-1}{U} \notin N$ then cell $n-1$ is flagged as available.
			\item\label{en:3n-4 n-1 no extra} If there is no extra cell and $\key{n-1}{U} \in N$ then fill cell $n-1$ with $\key{n-1}{U}$.
			\item\label{en:3n-4 n-1 empty A} If cell $i$ is type $A$ and $\key{n-1}{U} \notin N$ then fill cell $i$ with $\key{i}{U}$ and fill cell $n-1$ with $\key{i}{M}$.
			\item\label{en:3n-4 n-1 nonempty A} If cell $i$ is type $A$ and $\key{n-1}{U} \in N$ then fill cell $i$ with $\key{n-1}{U}$ and fill cell $n-1$ with $\key{i}{M}$.
			\item\label{en:3n-4 n-1 empty B} If cell $i$ is type $B$ and $\key{n-1}{U} \notin N$ then fill cell $i$ with $\key{i}{U}$ and fill cell $n-1$ with $\key{i}{L}$.
			\item\label{en:3n-4 n-1 nonempty B} If cell $i$ is type $B$ and $\key{n-1}{U} \in N$ then fill cell $i$ with $\key{n-1}{U}$ and fill cell $n-1$ with $\key{i}{L}$.
		\end{enumerate}
	\item Pair up cells of type $C$ and $D$ with other cells of type $C$ and $D$.
		\begin{enumerate}
			\item\label{en:3n-4 CC} If cell $i$ and cell $j$ are both type $C$ then fill cell $i$ with $\key{j}{M}$ and fill cell $j$ with $\key{i}{M}$.
			\item\label{en:3n-4 DD} If cell $i$ and cell $j$ are both type $D$ then fill cell $i$ with $\key{j}{M}$ and fill cell $j$ with $\key{i}{U}$.
			\item\label{en:3n-4 CD} If cell $i$ is type $C$ and cell $j$ is type $D$ then fill cell $i$ with $\key{j}{M}$ and fill cell $j$ with $\key{i}{L}$.
		\end{enumerate}
	\item We may have one extra cell of type either $C$ or $D$. Now consider that extra cell $i$ (if it exists) and cell $n$.
		\begin{enumerate}
			\item\label{en:3n-4 flag available} If there is no extra cell and $\key{n}{M} \notin N$ then cell $n$ is flagged as available.
			\item\label{en:3n-4 n no extra} If there is no extra cell and $\key{n}{M} \in N$ then fill cell $n$ with $\key{n}{M}$.
			\item\label{en:3n-4 n empty C} If cell $i$ is type $C$ and $\key{n}{M}  \notin N$ then fill cell $i$ with $\key{i}{M}$ and fill cell $n$ with $\key{i}{L}$.
			\item\label{en:3n-4 n nonempty C} If cell $i$ is type $C$ and $\key{n}{M}  \in N$ then fill cell $i$ with $\key{n}{M}$ and fill cell $n$ with $\key{i}{L}$.
			\item\label{en:3n-4 n empty D} If cell $i$ is type $D$ and $\key{n}{M}  \notin N$ then fill cell $i$ with $\key{i}{M}$ and fill cell $n$ with $\key{i}{U}$.
			\item\label{en:3n-4 n nonempty D} If cell $i$ is type $D$ and $\key{n}{M}  \in N$ then fill cell $i$ with $\key{n}{M}$ and fill cell $n$ with $\key{i}{U}$.
		\end{enumerate}
	\item\label{en:3n-4 fill available} Fill any cells flagged as available with any remaining keys.
\end{enumerate}
To answer the membership question for a key with cell label $i$, we first query cell $i$.
	\begin{itemize}
		\item If cell $i$ holds a key with cell label $i$ it must have been assigned by \ref{en:3n-4 only key}, \ref{en:3n-4 n-1 no extra}, \ref{en:3n-4 n-1 empty A}, \ref{en:3n-4 n-1 empty B}, \ref{en:3n-4 n no extra}, \ref{en:3n-4 n empty C}, or \ref{en:3n-4 n empty D}. If this is not the key in question,
			\begin{itemize}
				\item If cell $i$ holds key $\key{i}{U}$, then query cell $n-1$.
					\begin{itemize}
						\item If cell $n-1$ holds a key with cell label other than $i$ then cell $i$ was assigned key $\key{i}{U}$ by \ref{en:3n-4 only key} and thus $\key{i}{L}$ and $\key{i}{M}$ are not in the table.
						\item If cell $n-1$ holds key $\key{i}{L}$ then it was assigned by \ref{en:3n-4 n-1 empty B} so cell $i$ is type $B$. Therefore $\key{i}{M}$ is not in the table.
						\item If cell $n-1$ holds key $\key{i}{M}$ then it was assigned by \ref{en:3n-4 n-1 empty A} so cell $i$ is type $A$. Therefore $\key{i}{L}$ is in the table.
					\end{itemize}
				\item If cell $i$ holds key $\key{i}{M}$, then query cell $n$.
					\begin{itemize}
						\item If cell $n$ holds a key with cell label other than $i$ then cell $i$ was assigned key $\key{i}{M}$ by \ref{en:3n-4 only key} and thus $\key{i}{L}$ and $\key{i}{U}$ are not in the table.
						\item If cell $n$ holds key $\key{i}{L}$ then it was assigned by \ref{en:3n-4 n empty C} so cell $i$ is type $C$. Therefore $\key{i}{U}$ is not in the table.
						\item If cell $n$ holds key $\key{i}{U}$ then it was assigned by \ref{en:3n-4 n empty D} so cell $i$ is type $D$. Therefore $\key{i}{L}$ is not in the table.
					\end{itemize}
				\item If cell $i$ holds key $\key{i}{L}$ then it was assigned by \ref{en:3n-4 only key} and thus $\key{i}{M}$ and $\key{i}{U}$ are not in the table.
			\end{itemize}
		\item If cell $i$ holds a key with cell label $j$ then query cell $j$.
			\begin{itemize}
				\item If cell $j$ holds a key with cell label other than $i$, then cell $i$ was assigned its key by \ref{en:3n-4 fill available} after being flagged available by \ref{en:3n-4 flag available} and so none of the keys with cell label $i$ are in the table.
				\item If $j=n-1$ or $j=n$:
					\begin{itemize}
						\item If $j=n-1$ and cell $j$ holds $\key{i}{M}$ then cell it was assigned by \ref{en:3n-4 n-1 nonempty A} so cell $i$ is type $A$ and therefore both $\key{i}{L}$ and $\key{i}{U}$ are in the table.
						\item If $j=n-1$ and cell $j$ holds $\key{i}{L}$ then cell it was assigned by \ref{en:3n-4 n-1 nonempty B} so cell $i$ is type $B$ and therefore $\key{i}{U}$ is in the table.
						\item If $j=n$ and cell $j$ holds $\key{i}{L}$ then cell it was assigned by \ref{en:3n-4 n nonempty C} so cell $i$ is type $C$ and therefore $\key{i}{M}$ is in the table.
						\item If $j=n$ and cell $j$ holds $\key{i}{U}$ then cell it was assigned by \ref{en:3n-4 n nonempty D} so cell $i$ is type $D$ and therefore $\key{i}{M}$ is in the table.
					\end{itemize}
				\item If cell $j$ holds a key with cell label $i$:
					\begin{itemize}
						\item If $i=n-1$ or $i=n$ we are done since only one key has each of those cell labels.
						\item If cell $j$ holds $\key{i}{L}$ and cell $i$ holds $\key{j}{L}$ or $\key{j}{U}$ then they were assigned by \ref{en:3n-4 AA} or \ref{en:3n-4 AB}, respectively. Therefore cell $i$ is type $A$ and thus $\key{i}{M}$ and $\key{i}{U}$ are in the table.
						\item If cell $j$ holds $\key{i}{L}$ and cell $i$ holds $\key{j}{M}$ then they were assigned by \ref{en:3n-4 CD} and so cell $i$ is type $C$. Therefore $\key{i}{M}$ is in the table and $\key{i}{U}$ is not in the table.
						\item If cell $j$ holds $\key{i}{M}$ and cell $i$ holds $\key{j}{L}$ or $\key{j}{U}$ then they were assigned by \ref{en:3n-4 CD} or \ref{en:3n-4 DD}, respectively. Therefore cell $i$ is type $D$ and thus $\key{i}{U}$ is in the table and $\key{i}{L}$ is not in the table.
						\item If cell $j$ holds $\key{i}{M}$ and cell $i$ holds $\key{j}{M}$ then they were assigned by \ref{en:3n-4 CC} and so cell $i$ is type $C$. Therefore $\key{i}{M}$ is in the table and $\key{i}{U}$ is not in the table.
						\item If cell $j$ holds $\key{i}{U}$ and cell $i$ holds $\key{j}{L}$ or $\key{j}{U}$ then they were assigned by \ref{en:3n-4 AB} or \ref{en:3n-4 BB}, respectively. Therefore cell $i$ is type $B$ and thus $\key{i}{L}$ is in the table and $\key{i}{M}$ is not in the table.
						\item If cell $j$ holds $\key{i}{U}$ and cell $i$ holds $\key{j}{M}$ then they were assigned by \ref{en:3n-4 DD} and so cell $i$ is type $D$. Therefore $\key{i}{M}$ is in the table and $\key{i}{L}$ is not in the table.
					\end{itemize}
			\end{itemize}
	\end{itemize}
	Thus we are able to answer the membership question in at most two queries for a table of $n$ keys chosen from a key space of size $3n-4$ and so $g\left(n,2\right) \geq 3n-4$.
\end{proof}

