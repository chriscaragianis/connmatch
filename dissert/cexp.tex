\documentclass[12pt]{article}

\usepackage{amsthm, amsrefs,xspace}

\newcommand{\kfree}{$\overline{K_3}$ free\xspace}
\linespread{1.5}


\begin{document}
	\noindent Let $G$ be a \kfree graph with $n$ vertices and no connected matching of size $n/c$.
	\begin{enumerate}
		\item $\omega(G) \leq n/c$
			\begin{enumerate}
				\item $\delta(G) \geq (c-1)n/c-1$
				\item Try to determine $k$ so that $\omega(G) \leq \frac{(c-1)^k}{c^{k+1}}n$
			\end{enumerate}    
		\item For $n$ sufficiently large, $|E(R)|\geq n^3/k$ where  $\frac{k}{4k+4} < \frac{1}{c}$ 
			\begin{enumerate}
			\item For $c > 4$, $b >0$, and $n \geq n_{b,c}$ sufficiently large, $\omega(G) > b\sqrt{n\log n}$
			\item Looks like it works for $\omega(G) > bn^{3/4}$
			\end{enumerate}
		\item No edge dominates $n-c$ vertices. (If $G$ is a vertex minimal counterexample)
		\item $\kappa(G) \geq (c-1)n/c$
		\item If $G$ is edge critical, both $G$ and $\overline{G}$ have diameter 2.
		\item $G$ has a connected matching of size $\lfloor n/c \rfloor -1 .$
	\end{enumerate}

\begin{proof}[Proof of 1, 1a]
	Let $S$ be a set of $n/c$ vertices inducung a clique.  For any subset of $S'\subseteq S$, the intersection $\displaystyle I = \bigcap_{s\in S'} \{v \in V(G): sv \notin E(G)\}$ induces a clique.  If for any $S'\subseteq S$, $|I| > n/2$, then there is a $cn$-connected matching in the clique induced by $I$.  Otherwise, $|N(S')| \geq |S'|$ for all $S'\subseteq S$, and there is an $(S,V(G)-S)$ matching saturating $S$.  This matching must be connected, since $S$ induces a clique. Since $G$ is \kfree, $\delta(G) \geq n-\omega{G}-1$ (This appears implicitly in \cite{FGS})
\end{proof}

\begin{proof}[Proof of 2, 2a]
Let $T$ be the number of edges in the T\`uran graph $T(m. n/c-1)$.  Then if 
\[R+B \leq {m\choose 2} - T\]
$\mathcal{G}$ has a green $n/c$-clique and $G$ has a $n/c$-cconnected matching. Some calculation shows \[{m\choose 2} - T >\sim \frac{cn^3}{8}\]
$B$ is certainly smaller than the number of edges in the line graph of $K_n$, so 
\[B < \sim \frac{n^3}{2}\]
Hence, if $R$ is less than $n^3/k$ where  $k > \frac{c-4}{8}$, and $n$ is sufficiently large, $G$ has a $n/c$-connected matching.  2a follows from calculating the number of induced $K_2$s in a \kfree graph with minimum degree high enough to allow $\omega(G) \leq b\sqrt{n\log n}$.
\end{proof}

If 3 fails, we can carry out an inductive step.

\begin{proof}[Proof of 4]
	We will show that $\nu_c(G) \geq n-\kappa(G)$. The proof begins similarly to lemma 2.1 of \cite{blas}. Let $S$ be a minimum cut set of $G$.  Then let $L$, $R$ be a partition of $V(G) - S$ so that $L$ and $R$ do not touch.  Since $G$ is \kfree, $L$ and $R$ are cliques, and every vertex of $S$ is complete to $L$ or complete to $R$.  Let $S_L$ be the set of vertices complete to $L$ and $S_R$ the set of vertices complete to $R$.    We claim that between any $A \subseteq S_L$ with $|A| \leq |R|$ and $R$ ($S_R$ and $L$ resp.) there is a matching that saturates $A$.  Suppose there is no $(A, R)$ matching saturating $A$.  Hall's condition then implies that there is $T\subseteq A$ such that $|N(T)\cap R| < |T|$. But then $(S-T)\cup (N(T)\cap R)$ is a cut set separating $L\cup T$ and $R-N(T)$. This set is smaller than $S$, yielding a contradiction.   

Let $M$ be the largest possible matching obtained with $(S_L,R)$ edges and $(S_R,L)$ edges.  This matching is connected, and $|M| = \min\{|S_L|, |R|\} + \min\{|S_R|, |L|\}$.  If both $|R| \leq S_L$ and $|L| \leq S_R$, then we are done.  If, WLOG,  $|R| > S_L$, then let $U_R$ denote the set of vertices of $R$ unmatched by $M$.  Since $|S|\geq n/2$, there are at least $|U_R|$ unmatched vertices of $S_R$ (denoted $U_{S_R}$).  Augment $M$ with any $(U_R , U_{S_R})$ matching saturating $U_R$ to yield $M'$.  These new edges are mutually connected, and connected to any $(S_R,L)$ or $(S_L, R)$ edges via $R$.  Hence, $M'$ is a $(S,S^c)$ connected matching saturating $S^c = R\cup L$, and $|R\cup L| = n-\kappa(G)$. 
\end{proof}

\begin{proof}[Proof of 5]
 Obviously $\overline{G}$ has diameter 2, or else $G$ would not be edge-critical.  Conversely, if two vertices of $G$ have neighborhoods that do not intersect, then $G$ is partitionable into two cliques, and $\nu_c(G) \geq n/4$.
\end{proof}

\input{references}
\end{document}


