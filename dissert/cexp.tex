\documentclass[12pt]{article}

\usepackage{amsthm, amsrefs,xspace}

\newcommand{\kfree}{$\overline{K_3}$ free\xspace}
\linespread{1.5}


\begin{document}
	\noindent Let $G$ be a \kfree graph with $n$ vertices and no connected matching of size $n/c$.
	\begin{enumerate}
		\item $\omega(G) \leq n/c$
			\begin{enumerate}
				\item $\delta(G) \geq (c-1)n/c-1$
				\item Try to determine $k$ so that $\omega(G) \leq \frac{(c-1)^k}{c^{k+1}}n$
			\end{enumerate}    
		\item For $n$ sufficiently large, $|E(R)|\geq n^3/k$ where  $\frac{k}{4k+4} < \frac{1}{c}$ 
			\begin{enumerate}
			\item For $c > 4$, $b >0$, and $n \geq n_{b,c}$ sufficiently large, $\omega(G) > b\sqrt{n\log n}$
			\item Looks like it works for $\omega(G) > bn^{3/4}$
			\end{enumerate}
		\item No edge dominates $n-c$ vertices. (If $G$ is a vertex minimal counterexample)
		\item $\kappa(G) \geq (c-1)n/c$
		\item If $G$ is edge critical, both $G$ and $\overline{G}$ have diameter 2.
		\item $G$ has a connected matching of size $\lfloor n/c \rfloor -1 .$
	\end{enumerate}

\begin{proof}[Proof of 1, 1a]
	Let $S$ be a set of $n/c$ vertices inducung a clique.  For any subset of $S'\subseteq S$, the intersection $\displaystyle I = \bigcap_{s\in S'} \{v \in V(G): sv \notin E(G)\}$ induces a clique.  If for any $S'\subseteq S$, $|I| > n/2$, then there is a $cn$-connected matching in the clique induced by $I$.  Otherwise, $|N(S')| \geq |S'|$ for all $S'\subseteq S$, and there is an $(S,V(G)-S)$ matching saturating $S$.  This matching must be connected, since $S$ induces a clique. Since $G$ is \kfree, $\delta(G) \geq n-\omega{G}-1$ (This appears implicitly in \cite{FGS})
\end{proof}

\begin{proof}[Proof of 2, 2a]
Let $T$ be the number of edges in the T\`uran graph $T(m. n/c-1)$.  Then if 
\[R+B \leq {m\choose 2} - T\]
$\mathcal{G}$ has a green $n/c$-clique and $G$ has a $n/c$-cconnected matching. Some calculation shows \[{m\choose 2} - T >\sim \frac{cn^3}{8}\]
$B$ is certainly smaller than the number of edges in the line graph of $K_n$, so 
\[B < \sim \frac{n^3}{2}\]
Hence, if $R$ is less than $n^3/k$ where  $k > \frac{c-4}{8}$, and $n$ is sufficiently large, $G$ has a $n/c$-connected matching.  2a follows from calculating the number of induced $K_2$s in a \kfree graph with minimum degree high enough to allow $\omega(G) \leq b\sqrt{n\log n}$.
\end{proof}

If 3 fails, we can carry out an inductive step.

\begin{proof}[Proof of 4]
	We will show that $\nu_c(G) \geq n-\kappa(G)$. The proof begins similarly to lemma 2.1 of \cite{blas}. Let $S$ be a minimum cut set of $G$.  Then let $L$, $R$ be a partition of $V(G) - S$ so that $L$ and $R$ do not touch.  Since $G$ is \kfree, $L$ and $R$ are cliques, and every vertex of $S$ is complete to $L$ or complete to $R$.  Let $S_L$ be the set of vertices complete to $L$ and $S_R$ the set of vertices complete to $R$.    We claim that between any $A \subseteq S_L$ with $|A| \leq |R|$ and $R$ ($S_R$ and $L$ resp.) there is a matching that saturates $A$.  Suppose there is no $(A, R)$ matching saturating $A$.  Hall's condition then implies that there is $T\subseteq A$ such that $|N(T)\cap R| < |T|$. But then $(S-T)\cup (N(T)\cap R)$ is a cut set separating $L\cup T$ and $R-N(T)$. This set is smaller than $S$, yielding a contradiction.   

Let $M$ be the largest possible matching obtained with $(S_L,R)$ edges and $(S_R,L)$ edges.  This matching is connected, and $|M| = \min\{|S_L|, |R|\} + \min\{|S_R|, |L|\}$.  If both $|R| \leq S_L$ and $|L| \leq S_R$, then we are done.  If, WLOG,  $|R| > S_L$, then let $U_R$ denote the set of vertices of $R$ unmatched by $M$.  Since $|S|\geq n/2$, there are at least $|U_R|$ unmatched vertices of $S_R$ (denoted $U_{S_R}$).  Augment $M$ with any $(U_R , U_{S_R})$ matching saturating $U_R$ to yield $M'$.  These new edges are mutually connected, and connected to any $(S_R,L)$ or $(S_L, R)$ edges via $R$.  Hence, $M'$ is a $(S,S^c)$ connected matching saturating $S^c = R\cup L$, and $|R\cup L| = n-\kappa(G)$. 
\end{proof}

\begin{proof}[Proof of 5]
 Obviously $\overline{G}$ has diameter 2, or else $G$ would not be edge-critical.  Conversely, if two vertices of $G$ have neighborhoods that do not intersect, then $G$ is partitionable into two cliques, and $\nu_c(G) \geq n/4$.
\end{proof}

{\linespread{1}
\begin{bibsection}[References]\vspace{-\parskip} % This is the start of the bibliography. 
	\begin{biblist}[\normalsize] % Replace the \bib entries with ones relevant to your problem.
							% The bulk of each entry can be copied and pasted from MathSciNet
							% ( http://www.ams.org/mathscinet/ ). When viewing the review of an
							% item you want to site, open the "Select alternative format" pull-down
							% and select AMSrefs.
							% Most likely, the only part you will need to change is the first parameter
							% after \bib. This is the internal name you use to cite the reference
							% with \cite. By default it will be the Mathematical Reviews number
							% (for example MR1375315). To make my life easier when I merge these all
							% into the summary document, please choose a name that begins with your
							% initials, followed by the number of problems you have submitted
							% (including this one).
							% For example, since this problem was submitted by Leonhard Euler and
							% since this is the first problem he is presenting, all citation names
							% begin with "le1". If this was his third problem, they would begin with
							% "le3".
%Z. Füredi, A. Gyárfás, G. Simonyi,  Connected matchings and Hadwiger's conjecture,  Combin. Probab. Comput., Problem Section, 14 (2005), 435--438.
%M. Kriesell, On seymour's strengthening of Hadwidger's conjecture for graphs with certain forbidden subgraphs 



\bib{dwest}{book}{
   author={West, Douglas B.},
   title={Introduction to graph theory},
   publisher={Prentice Hall Inc.},
   place={Upper Saddle River, NJ},
   date={1996},
   pages={xvi+512},
   isbn={0-13-227828-6},
   review={\MR{1367739 (96i:05001)}},
}
\bib{FGS}{article}{
   author={F\"{u}redi, Zolt\'{a}n},
   author={Gy{\'a}rf{\'a}s, Andr{\'a}s},
   author={Simonyi, G\'{a}bor},
   title={Connected matchings and Hadwiger's conjecture},
   journal={Combin. Probab. Comput.},
   part={Problem Section},
   volume={14},
   date={2005},
   pages={435--438},
}

\bib{sqrtofgraph}{article}{
   author={Mukhopadhyay, A.},
   title={The square root of a graph},
   journal={J. Combinatorial Theory},
   volume={2},
   date={1967},
   pages={290--295},
   review={\MR{0210616 (35 \#1502)}},
}
\bib{DandM}{article}{
   author={Duchet, P.},
   author={Meyniel, H.},
   title={On Hadwiger's number and the stability number},
   conference={
      title={Graph theory},
      address={Cambridge},
      date={1981},
   },
   book={
      series={North-Holland Math. Stud.},
      volume={62},
      publisher={North-Holland},
      place={Amsterdam},
   },
   date={1982},
   pages={71--73},
   review={\MR{671905 (84h:05074)}},
}

\bib{MR882610}{article}{
   author={Maffray, F.},
   author={Meyniel, H.},
   title={On a relationship between Hadwiger and stability numbers},
   journal={Discrete Math.},
   volume={64},
   date={1987},
   number={1},
   pages={39--42},
   issn={0012-365X},
   review={\MR{882610 (88g:05076)}},
   doi={10.1016/0012-365X(87)90238-X},
}


\bib{MR1411244}{article}{
   author={Toft, Bjarne},
   title={A survey of Hadwiger's conjecture},
   note={Surveys in graph theory (San Francisco, CA, 1995)},
   journal={Congr. Numer.},
   volume={115},
   date={1996},
   pages={249--283},
   issn={0384-9864},
   review={\MR{1411244 (97i:05048)}},
}
\bib{MR1654153}{article}{
   author={Reed, Bruce},
   author={Seymour, Paul},
   title={Fractional colouring and Hadwiger's conjecture},
   journal={J. Combin. Theory Ser. B},
   volume={74},
   date={1998},
   number={2},
   pages={147--152},
   issn={0095-8956},
   review={\MR{1654153 (99k:05079)}},
   doi={10.1006/jctb.1998.1835},
}
\bib{MR1844036}{article}{
   author={Kotlov, Andre{\u\i}},
   title={Matchings and Hadwiger's conjecture},
   note={Algebraic and topological methods in graph theory (Lake Bled,
   1999)},
   journal={Discrete Math.},
   volume={244},
   date={2002},
   number={1-3},
   pages={241--252},
   issn={0012-365X},
   review={\MR{1844036 (2002k:05087)}},
   doi={10.1016/S0012-365X(01)00087-5},
}



\bib{K_Cam}{article}{
   author={Cameron, Kathie},
   title={Connected matchings},
   conference={
      title={Combinatorial optimization---Eureka, you shrink!},
   },
   book={
      series={Lecture Notes in Comput. Sci.},
      volume={2570},
      publisher={Springer},
      place={Berlin},
   },
   date={2003},
   pages={34--38},
   review={\MR{2163948 (2006c:90072)}},
   %doi={10.1007/3-540-36478-1_5},
}

\bib{MR1979786}{article}{
   author={Klazar, Martin},
   title={Non-$P$-recursiveness of numbers of matchings or linear chord
   diagrams with many crossings},
   note={Formal power series and algebraic combinatorics (Scottsdale, AZ,
   2001)},
   journal={Adv. in Appl. Math.},
   volume={30},
   date={2003},
   number={1-2},
   pages={126--136},
   issn={0196-8858},
   review={\MR{1979786 (2004h:05006)}},
   doi={10.1016/S0196-8858(02)00528-6},
}



\bib{Spec_case}{article}{
   author={Plummer, Michael D.},
   author={Stiebitz, Michael},
   author={Toft, Bjarne},
   title={On a special case of Hadwiger's conjecture},
   journal={Discuss. Math. Graph Theory},
   volume={23},
   date={2003},
   number={2},
   pages={333--363},
   issn={1234-3099},
   review={\MR{2070161 (2005e:05055)}},
}



\bib{DandMimprove}{article}{
   author={Kawarabayashi, Ken-ichi},
   author={Plummer, Michael D.},
   author={Toft, Bjarne},
   title={Improvements of the theorem of Duchet and Meyniel on Hadwiger's
   conjecture},
   journal={J. Combin. Theory Ser. B},
   volume={95},
   date={2005},
   number={1},
   pages={152--167},
   issn={0095-8956},
   review={\MR{2156345 (2006b:05118)}},
   doi={10.1016/j.jctb.2005.04.001},
}


\bib{MR2249267}{article}{
   author={Gy{\'a}rf{\'a}s, Andr{\'a}s},
   author={Ruszink{\'o}, Mikl{\'o}s},
   author={S{\'a}rk{\"o}zy, G{\'a}bor N.},
   author={Szemer{\'e}di, Endre},
   title={One-sided coverings of colored complete bipartite graphs},
   conference={
      title={Topics in discrete mathematics},
   },
   book={
      series={Algorithms Combin.},
      volume={26},
      publisher={Springer},
      place={Berlin},
   },
   date={2006},
   pages={133--144},
   review={\MR{2249267 (2008c:05120)}},
   %doi={10.1007/3-540-33700-8_8},
}

\bib{M_Krie}{article}{
   author={Kriesell, Matthias},
   title={On Seymour's strengthening of Hadwiger's conjecture for graphs with certain forbidden subgraphs},
   journal={Discrete Mathematics},
   %volume={},
   date={2010},
   pages={435--438},
}

\bib{MR0297600}{article}{
   author={Chv{\'a}tal, V.},
   author={Erd{\H{o}}s, P.},
   title={A note on Hamiltonian circuits},
   journal={Discrete Math.},
   volume={2},
   date={1972},
   pages={111--113},
   issn={0012-365X},
   review={\MR{0297600 (45 \#6654)}},
}

\bib{MR1369063}{article}{
   author={Kim, Jeong Han},
   title={The Ramsey number $R(3,t)$ has order of magnitude $t\sp 2/\log t$},
   journal={Random Structures Algorithms},
   volume={7},
   date={1995},
   number={3},
   pages={173--207},
   issn={1042-9832},
   review={\MR{1369063 (96m:05140)}},
   doi={10.1002/rsa.3240070302},
}

\bib{MR0284366}{article}{
   author={Nash-Williams, C. St. J. A.},
   title={Edge-disjoint Hamiltonian circuits in graphs with vertices of
   large valency},
   conference={
      title={Studies in Pure Mathematics (Presented to Richard Rado)},
   },
   book={
      publisher={Academic Press},
      place={London},
   },
   date={1971},
   pages={157--183},
   review={\MR{0284366 (44 \#1594)}},
}

\bib{blas}{article}{
   author={Blasiak, Jonah},
   title={A special case of Hadwiger's conjecture},
   journal={J. Combin. Theory Ser. B},
   volume={97},
   date={2007},
   number={6},
   pages={1056--1073},
   issn={0095-8956},
   review={\MR{2354718 (2009a:05064)}},
   doi={10.1016/j.jctb.2007.04.003},
}

\bib{edhc}{article}{
	author={Christofides, Demetres},
	author={K\"{u}hn, Daniela},
	author={Osthus, Deryk},
	title={Edge-disjoint Hamilton cycles in graphs},
	date={31 Aug 2009},
	eprint={arXiv:0908.4572v1 [math.CO]},
	url={http://arxiv.org/abs/0908.4572},
}


	\end{biblist}
\end{bibsection}}

\end{document}


