\documentclass[12pt]{article}




\usepackage{amsrefs,array,amsthm,amsmath,setspace,tikz}
%\usepackage[top=1 in, bottom=1in, left=1.5 in, right=1in]{geometry}



\renewcommand{\arraystretch}{1.5}
\doublespacing
\linespread{2}

\newtheorem{conj}{Conjecture}
	\newcommand{\bconj}[1]{\begin{conj}#1\end{conj}}
\newtheorem{mconj}{Metaconjecture}

\newtheorem{prop}{Proposition}
	\newcommand{\bprop}[1]{\begin{prop}#1\end{prop}}
\newtheorem{lem}{Lemma}
	\newcommand{\blem}[1]{\begin{lem}#1\end{lem}}
\newtheorem{theorem}{Theorem}
	\newcommand{\bthm}[1]{\begin{theorem}#1\end{theorem}}


\newtheorem{guess}{Guess}
	\newcommand{\bguess}[1]{\begin{guess}#1\end{guess}}

\theoremstyle{definition}
\newtheorem{mydef}{Definition}




\begin{document}
\bthm{A graph $G$ with $n$ vertices $v_1, v_2, \ldots , v_n$ is the distance 2 graph of some triangle-free graph $H$ if and only if some set of $n$ complete subgraphs of $G$ whose union is $G$ can be labeled $C_1, C_2, \ldots, C_n$ so that $v_a \notin C_a$ and whenever $v_i, v_j \in C_k$, $v_i \notin C_j$ and $v_j \notin C_i$.}
\begin{proof}
Suppose we have a triangle free graph $H$ on $n$ vertices.  Two vertices of $H$ are at distance two from each other if and only if they are both in the neighborhood of a third vertex.  With this in mind, we label the $n$ complete subgraphs of $d_2(H)$ corresponding to the neighborhoods of the vertices of $H$ as $C_1, C_2, \ldots, C_n$.  The union of these does indeed equal $d_2(G)$.  Furthermore, if two vertices appear in $C_k$, there is no edge between them , and neither is in the neighborhood of the other.  Hence the second condition is satisfied.

To prove neccesity, we suppose that we have a graph $G$ on $v_1, v_2, \ldots, v_n$, and a collection of cliques $C_1, C_2, \ldots, C_n$ that satisfy the given conditions.  Construct a graph $H$ on the same vertices as $G$ by adding an edge from $v_i$ to the vertices of $C_i$ for all $i = 1, 2, \ldots, n$.  

\noindent\textit{Claim.} $H$ is triangle free.

Suppose there is a triangle $v_iv_jv_k$ in $H$.  Then $v_i, v_j \in C_k$, but also $v_i \in C_j$, presenting a contradiction.

\noindent\textit{Claim.}  $G$ is the distance 2 graph of $H$.

Let $v_i, v_j \in V(H)$ be such that $d_H(v_i,v_j) = 2$.  Then there is a third vertex $v_k$ with the property that $v_i, v_j \in N_H(v_k)$.  Thus $v_i, v_j \in E(G)$.  Now suppose $v_iv_j \in E(G)$.  This edge is covered by some clique $C_k$, and so edges $v_iv_k$ and $v_jv_k$ are added in the construction, impying $d_H(v_i, v_k) \leq 2$.  The previously established fact that $H$ is trangle free then implies that $d_H(v_i, v_k) =2$.
\end{proof}

\end{document}
