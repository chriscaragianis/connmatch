\section{The connected matching graph}

We define the connected matching graph $CM(H)$ of a simple graph $H$ to be the distance-2 graph of the line graph of $H$.  $CM(H)$ then has the property that cliques correspond to connected matchings in $H$.  In what follows, if $S$ is an edge set, we will denote by $[S]$ the vertex set induced by $S$.  If $\mathcal{S}$ is a collection of edge sets, we will denote by $[\mathcal{S}]$ the collection of vertex sets induced by the members of $\mathcal{S}$.

\subsection{Characterizing connected matching graphs}

To characterize connected matching graphs, we will make use of Krausz's characterization of line graphs in terms of an edge partition.  

\bthm{[Krausz, 1943 CITE] \label{krausz} A simple graph $G$ is the line graph of a simple graph $H$ if and only if there is a partition $\mathcal{P} = \{P_1, P_2, \ldots, P_k\}$ of the edges of $G$ with the property that each $P_i \in \mathcal{P}$ is a clique and each vertex of $G$ is used by at most two elements of $\mathcal{P}$. }
\begin{proof}
	Let $H$ be a simple graph and take $L(H)$ in the following manner.  For each vertex $v\in V(H)$, let $P(v)$ be the set of edges incident to $v$.  Then $P(v)$ is a set of vertices inducing a clique in $L(H)$.  Every edge of $L(H)$ indicates two edges of $H$ sharing some endpoint, so the collection $\mathcal(P) = \bigcup_{v\in V(H)} \{L(H)[(P(v)]\}$ is a collection of cliques covering the edges of $L(H)$.  If there were adjacent vertices $x,y \in P(u)\cap P(v)$, then there would be two edges incident to $u$ and $v$ in $H$.  Since $H$ is simple, this cannot be the case.  Finally suppose there were a vertex used by three elements of $\mathcal{P}$.  This would correspond to an edge in $H$ incident to three vertices.  Hence, $\mathcal{P}$ has the desired properties. 
	Now suppose that $G$ has a suitable edge partition $\mathcal{P}$.  We will construct a simple graph $H$ with the property that $G = L(H)$.  For each ....FINISH
\end{proof}
Note that the partition $\mathcal{P}$ has the additional property that any two elements of $[\mathcal{P}]$ intersect in at most one vertex.  If this were not the case, two elements of $\mathcal{P}$ would cover the same edge.  

Our characterization of connected matching graphs relies on Krausz's characterization of line graphs, and similarly uses an edge covering by structurally similar subgraphs with restricted intersections.  We will use the term \textit{augmented biclique} to refer to the union of a complete bipartite graph and an isolated vertex.  Suppose that two vertex sets $S_1$ and $S_2$ each induce augmented bicliques.  Denote the partite sets of $S_i$ by $S_i^1$ and $S_i^2$.  Denote the isolated vertex of $S_i$ by $s_i$.  We say that $S_1$ and $S_2$ \textit{cross} if $S_1^a \cup \{s_1\}= S_2^b\cup \{s_2\}$ and $S_1\cap S_2 = S_1^a \cup S_2^b \cup \{s_1,s_2\}$ for some $a,b \in \{1,2\}$.  We say that a clique $K$ is \textit{small} if $K =K_1$ or $K_2$.


\documentclass[12pt]{article}




\usepackage{amsrefs,array,amsthm,amsmath,setspace,tikz}
%\usepackage[top=1 in, bottom=1in, left=1.5 in, right=1in]{geometry}



\renewcommand{\arraystretch}{1.5}
\doublespacing
\linespread{2}

\newtheorem{conj}{Conjecture}
	\newcommand{\bconj}[1]{\begin{conj}#1\end{conj}}
\newtheorem{mconj}{Metaconjecture}

\newtheorem{prop}{Proposition}
	\newcommand{\bprop}[1]{\begin{prop}#1\end{prop}}
\newtheorem{lem}{Lemma}
	\newcommand{\blem}[1]{\begin{lem}#1\end{lem}}
\newtheorem{theorem}{Theorem}
	\newcommand{\bthm}[1]{\begin{theorem}#1\end{theorem}}


\newtheorem{guess}{Guess}
	\newcommand{\bguess}[1]{\begin{guess}#1\end{guess}}

\theoremstyle{definition}
\newtheorem{mydef}{Definition}




\begin{document}
\begin{theorem}
$G$ is the connected matching graph of some graph $H$ if and only if $G$ has a edge cover $\mathcal{Q} = \{Q_1, Q_2, \ldots, Q_3\}$ by augmented bicliques so that the following hold:
\begin{description}
{\setlength\itemindent{0.5cm}
	\item[1.] If a vertex $v$ is in three blocks, then two of them cross in the parts containing $v$.
	\item[2.] If two parts intersect in more than one vertex, then the containing blocks cross in these parts.
	\item[3.] If $q_i$ is in $[Q_j^b]$, then $Q_i$ and $Q_j$ cross in part $b$ of $Q_j$.
	\item[4.] Take parts $a$, $b$, and $c$.  If $a$ and $b$ cross. and $b$ and $c$ cross, then $a$ and $c$ cross.
	\item[5.] }
\end{description}
\end{theorem}
\begin{proof}
First we show that for any simple graph $H$ with $G = CM(H)$ we can construct a suitable decomposition $\mathcal{Q}$.
%
Consider a Krausz partition $\mathcal{P} = \{P_1, P_2, \ldots, P_l\}$ of the edges of $L(H)$.
%
Define an index set $\mathcal{I} \subset [l]\times[l]$ where $(x,y) \in \mathcal{I}$ if and only if $x < y$ and $[P_x] \cap [P_y] \neq \emptyset$.
%
For each $a = (a_1, a_2) \in \mathcal{I}$, define $S_a = [P_{a_1}] \cup [P_{a_2}]$, and let $\mathcal{S} = \{S_a: a \in \mathcal{I}\}$.
%
Now every edge of $L(H)$ is induced by some element of $\mathcal{S}$.
%
For each $S_a \in \mathcal{S}$, if there are $u \in [P_{a_1}]$ and $v \in [P_{a_2}]$ so that $u,v \neq p_{a_1a_2}$ and $u,v \in [P_k]$ for $a_1,a_2\neq k$, then remove $u$ and $v$ from $S_a$.  
%
(Note that the edges of $L(H)$ are still induced. Edges incident to $u$ previously indcued by $S_{a}$, for example, are still induced some $S_{b}$, with $b = (a_1, k)$ or $(k, a_1)$.)
%
For each $a \in \mathcal{I}$, let $Q_a$ be the complement of the graph induced by  $S_{a}$, and let $\mathcal{Q} = \{Q_a: a \in \mathcal{I}\}$.
%
Each $Q_{a}$ is an augmented biclique with the folowing properties: 
\begin{description}
	{\setlength\itemindent{25pt}
	\item[a.] $Q_{a}^1 \subseteq [P_{a_1}]$
	\item[b.] $Q_{a}^2 \subseteq [P_{a_2}]$
	\item[c.] the isolated vertex $q_{ij} = p_{a_1a_2}$.}
\end{description}
%
If at the end of this construction $Q_{a}$ induces no edges,  discard $Q_{a}$ from $\mathcal{Q}$.   

We claim that $\mathcal{Q}$ is an edge cover of $G$.
%
We will show that an edge $uv$ is in some $Q_a$ if and only if $d_{L(H)}(u,v) = 2$.
%
Suppose $uv \in E(Q_a)$.
%
This means that $u \in [P_{a_1}]$, $v \in [P_{a_2}]$, and these cliques intersect.
%
Hence, $d_{L(H)}(u,v) \leq 2$.
%
We also know that neither $u$ nor $v$ is $p_{a_1a_2}$, and there is no element of $[\mathcal{P}]$ containing both $u$ and $v$ (as otherwise they would have been removed from $Q_{a}$).
%
Thus $d_{L(H)}(u,v) \geq 2$.
%
Now suppose we have $d_{L(H)}(u',v') = 2$.  The edge $u'v' \notin E(L(H))$, so there is no element of $[\mathcal{P}]$ containing both.
%
There exists some vertex $w$ so that $u'w, v'w \in E(L(H))$, and these edges are covered by distinct, intersecting elements $[P_i]$ and $[P_j]$ of $[\mathcal{P}]$.  
%
Hence, $u'v' \in E(Q_{(i,j)})$, with $(i,j) \in \mathcal{I}$.

Next we will show that $\mathcal{Q}$ satisfies \textbf{i}, \textbf{ii} and \textbf{ii} above. 
%
Suppose $q_a$ is the isolated vertex of $Q_a$, and $q_a \in [Q_a] \cap [Q_b] = I$ for some $a \neq b \in \mathcal{I}$.
%
$q_a$ is in $[P_{a_1}]$ and $[P_{a_2}]$, as well as one of $[P_{b_1}]$ and $[P_{b_2}]$.  
%
Any vertex is used by at most two cliques of the Krausz partition, so it must be the case that $\{a_1, a_2\} \cap \{b_1, b_2\} \neq \emptyset$.  
%
WLOG, say $a_1 = b_1$ and $a_2 \neq b_2$.  
%
If $q_b \notin I$, then $q_b$ must have been removed from $Q_a$ in the maximalization step.
%
This means there must be $c \neq a_1, a_2, b_2$ so that $q_b \in [P_c]$.
%
This is impossible, as $q_b$ would then appear in three cliques of $\mathcal{P}$. 
%
So $\{q_a, q_b\} \subseteq I$.
%
So far we have that $[Q_a^1], [Q_b^1] \subseteq [P_{a_1}]$.
%
Suppose then, that there is $v \in Q_a^2 \cap Q_b^2$.
%
If this were true, $q_a$ and $q_b$ would have been removed from $Q_a^2$ and $Q_b^2$ repectively.
%

For the converse, suppose we have a graph $F$ that satisfies the above hypotheses.  
%
We will show that there exists $F'$ so that $F$ is the distance-2 graph of $F'$, and $F'$ is a line graph.
%
Take the vertex set of $F$, and for all $Q_i \in \mathcal{Q}$ add an edge $uv$ whenever $u,v \in [Q_i]$ are non-adjacent in $Q_i$.  
%
Furthermore, add all edges between $Q_i^a$ and $Q_j^b$ whenever $Q_i$ and $Q_j$ cross in $(a,b)$.
%
Call the resulting graph $F'$.
%

First we show that $F'$ is a line graph.
%
Each element of $[\mathcal{Q}]$ induces a pair of cliques intersecting in a single vertex.
%
Each of these cliques $Q_i^a$ is contained in a larger clique with vertices from partite sets crossing $Q_i^a$.  
%
This set $\mathcal{P}$ of maximal cliques form a Krausz partition of the edges of $F'$.
%
That they cover the edges of $F'$ is clear.
%
An edge added by the first step can only be covered once by hypothesis \textbf{i}.
%
An edge added by the second step is covered only once by maximality.
%
A vertex in three cliques means that some two of them cross (BY ZERO).
%
Hence the edges are properly partitioned.

Now we show that $F$ is the distance-2 graph of $F'$.  
%
If $uv \in E(F)$, then $u$ and $v$ are in intersecting cliques of $F'$, implying that $d_{F'}(u,v)\leq 2$.
%
Also, edges in $F'$ are not present in $F$,(?????) so $uv \in E(F) \implies d_{F'}(u,v) \geq 2$. 
%
Now let $d_{F'}(u',v') = 2$.  There is a vertex $w$ so that $u'w, v;w \in E(F')$.  

\end{proof}

\end{document}
