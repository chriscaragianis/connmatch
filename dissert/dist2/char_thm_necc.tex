\documentclass[12pt]{article}




\usepackage{amsrefs,array,amsthm,amsmath,setspace,tikz}
%\usepackage[top=1 in, bottom=1in, left=1.5 in, right=1in]{geometry}



\renewcommand{\arraystretch}{1.5}
\doublespacing
\linespread{2}

\newtheorem{conj}{Conjecture}
	\newcommand{\bconj}[1]{\begin{conj}#1\end{conj}}
\newtheorem{mconj}{Metaconjecture}

\newtheorem{prop}{Proposition}
	\newcommand{\bprop}[1]{\begin{prop}#1\end{prop}}
\newtheorem{lem}{Lemma}
	\newcommand{\blem}[1]{\begin{lem}#1\end{lem}}
\newtheorem{theorem}{Theorem}
	\newcommand{\bthm}[1]{\begin{theorem}#1\end{theorem}}


\newtheorem{guess}{Guess}
	\newcommand{\bguess}[1]{\begin{guess}#1\end{guess}}

\theoremstyle{definition}
\newtheorem{mydef}{Definition}




\begin{document}

\bprop{$Q'= Q(L(H))$ has an edge covering $\mathcal{Q} = \{Q_1, Q_2, \ldots, Q_k\}$ by augmented bicliques.}
\begin{proof}
	The vertex sets $\mathcal{S}$ in the construction of $Q'$ induce augmented bicliques that cover the edges of $Q'$. So $\mathcal{Q}= \{Q'[S_a]: S_a \in \mathcal{S}\}$.  These sets inherit the indexing by $\mathcal{I}$.
\end{proof}

\bprop{For any $i \neq j \in \mathcal{I}$, if $|[Q_i] \cap [Q_j]| \geq 2$, then $Q_i$ and $Q_j$ cross if and only if $i \cap j \neq \emptyset$.}
\begin{proof}
	Suppose $i_1 = j_1$. If there is a vertex $v \in [Q_i] \cap [Q_j]$ not contained in $[P_{i_1}]$, then $v \in [P_{i_2}] \cap [P_{j_2}]$ and $v \neq q_i, q_j$.
%
	Then, $q_i, v \in [P_{i_2}]$, so $v$ would have been removed from $Q_j$ in step 4 of the construction.  
%
	So $[Q_i] \cap [Q_j] \subseteq [Q_i^1]$.  
%
	Now suppose $|[Q_i] \cap [Q_j]| < 2$.
%
	Then $q_i$ has been removed from $[Q_j^1]$ in step 4.  
%
	As $q_i$ is in at most two elements of $[\mathcal{P}]$, we must have $u \in [P_{j_2}] \cap [P_{i_2}]$.
%
	Then $k = (i_2. j_2) \in \mathcal{I}$. 
%
		
	
	Conversely, the intersection of any two elements of $[\mathcal{P}]$ is at most one vertex.  If all parts are subsets of different elements of $\mathcal{P}$, then no intersection is large enough to contain two isolated vertices.  Thus $Q_i$ and $Q_j$ cannot cross.
\end{proof}

\bprop{If a vertex $v$ is in three elements of $[\mathcal{Q}]$, then two of them cross in the parts containing $v$.}
\begin{proof}
	V CAN BE IN ONLY THREE Ps, USE PROP 2
\end{proof}

\bprop{If two parts intersect in more than one vertex, then the containing $\mathcal{Q}$ elements cross in these parts.}
\begin{proof}
	MORE THAN TWO MEANS THEY CAME FROM THE SAME P, USE PROP 2
\end{proof}

\bprop{If $q_i \in [Q^b_j]$, then $Q_i$ and $Q_j$ cross in part $b$ of $Q_j$. }
\begin{proof}
	OTHERWISE VERTEX REMOVED BY STEP 4
\end{proof}

\bprop{Take three parts $a$, $b$, and $c$ each from a different element of $\mathcal{Q}$.  If $a$ crosses $b$, and $b$ crosses $c$, then $a$ crosses $c$.}
\begin{proof}
	Follows immediately from prop 2
\end{proof}

\bprop{If $u,v \in [Q_i^a] \triangle [Q_j^b]$ and $Q_i^a$ and $Q_j^b$ cross, then $uv \notin E(Q')$. }
\begin{proof}
	SUCH VERTICES REMOVED IN STEP 4
\end{proof}

\bprop{If $u \in [Q_i^a]$ is not an isolated vertex, then $u$ is in every part that crosses $Q_i^a$.}
\begin{proof}
	No removal without an intersecting $\mathcal{P}$-clique, then $u$ is an isolated vertex.
\end{proof}

\end{document}
