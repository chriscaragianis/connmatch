\documentclass[12 pt]{article}

\usepackage{tikz, amsmath, amsfonts, amsthm,xspace,cancel,ulem}
\newtheorem{theorem}{Theorem}
\newtheorem{lemma}{Lemma}
\newtheorem{proposition}{Proposition}
\tikzstyle{every node}=[circle, draw, fill=black,
                        inner sep=0pt, minimum width=4pt]
\tikzstyle{plain}=[fill = none, draw = white]
\newcommand{\kfree}{$\overline{K_3}$-free\xspace}
\newcommand{\St}[1]{\mathcal{S}_#1}
\newcommand{\claim}[1]{\vskip 0.5 cm \noindent{\it #1}}
\begin{document}
\linespread{1.5}
\begin{theorem}
	Let $G = (A,B,E)$ is a chordal bipartite graph with $|A| = k$.  $G$ has a connected matching saturating $A$ if and only if there is an ordering $a_1, a_2, \ldots a_k$ of $A$ and distinct vertices $b_1, b_2, \ldots b_k$ of $B$ so that $\{a_i : i \leq j\} \subset N(b_j)$ for each $j\in [k]$.
\end{theorem}
%
\begin{proof}

	The condition is clearly sufficient for bipartite graphs in general.  The matching $\displaystyle\bigcup_{i \in [k]} a_ib_i$ is a connected matching saturating $A$.

For the converse, it suffices to show that if a connected matching saturates $A$,  then a vertex from $B$ dominates $A$.  We proceed by induction on the order of $A$, noting that the base case of $|A| = 1$ is obvious.
We assume that our claim holds when $|A|\leq k-1$.

Suppose there is a connected matching $M$ saturating $A$.  Let $G_1$ be the subgraph of $G$ induced $M$.  Since $G_1$ is chordal bipartite, it has a bisimplicial edge $xy$, with $x\in A$.

\noindent{\it Case 1}: $xy \notin M$.

If $xy \notin M$,  then there is an edge $x'y$ of $M$, with $x \neq x'$.  If we remove this edge, then $M-x'y$ is a connected matching saturating $A-x'$, and by the induction hypothesis there remains a vertex $y' \in B-y$ such that $A-x' \subset N(y')$.  Since $xy$ is bisimplicial, $x' \in N(y)$, and $y' \in N(x)$ (by virtue of $y'$ dominating $A-x' \ni x$), the edge $y'x'$ must be present, and $y'$ dominates $A$.

\noindent{\it Case 2}: $xy \in M$.

Remove $xy$.  The remaining connected matching has a vertex $y'$ dominating $A-x$.  If $y' \in N(x)$, then $y'$ dominates $A$ and we are done.  If not, its matched partner $x'$ is in $N(y)$, as otherwise $xy$ and $x'y'$ are separable.  Now remove $x'y'$ and continue removing edges in this way until either $M$ is empty (in which case $y$ dominates $A$) or $\overline{N(x)} \cap B$ is exhausted, and some edges of $M$ remain.  What remains of $A$ is now dominated by some vertex $y'' \in N(x)$.  Note that all vertices removed from $A$ were from $N(y)$.  Since $xy$ is bisimplicial, $y''$ dominates both the remaining and the removed vertices fo $A$.
\end{proof}

\end{document}